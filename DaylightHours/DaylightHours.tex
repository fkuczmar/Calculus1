\documentclass{ximera}
\title{Hours of Daylight}

\newcommand{\pskip}{\vskip 0.1 in}

\begin{document}
\begin{abstract}
Modeling the number of hours of daylight.
\end{abstract}
\maketitle


\begin{question}  \label{Qdf4ppp3455}
The function
\[
    h = f(\delta) = \frac{24}{\pi} \arccos(-\tan\phi \tan\delta) \, , \, -\pi/2+\phi < \delta < \pi/2-\phi ,
\]
expresses the number of hours of daylight per day at latitude $\phi$ in terms of the declination of the sun. The declination of the sun ($\delta$) is the angle the sun's rays make with the plane of the equator, taken to be positive between the spring and fall equinoxes in the northern hemisphere. The latitude $\phi$, $-\pi/2 \leq \phi \leq \pi/2$, is positive at points in the northern hemipshere. The graph of the function $h=f(\delta)$ for latitude $\phi\sim 1.1$ is shown below.

Don't be confused by the angle $\delta$ (delta). It's just the lower case Greek letter for $\Delta$.

\begin{onlineOnly}
    \begin{center}
\desmos{ifomatkcta}{900}{600}
\end{center}
\end{onlineOnly}

Desmos activity available at \href{https://www.desmos.com/calculator/ifomatkcta}{151: Length of Day 1}

\begin{enumerate}
\item Use the graph to approximate the number of hours of daylight/day at a latitude of $\phi=1.1$ radians on March 1, when the sun is about $8^\circ$ below the plane of the equator. What about at our latitude? At the equator?

\item Evaluate the derivative %On March 1, the line of sight to the sun sun is about $8^\circ$ below the equator. Find the value of the derivative
\[
  \frac{dh}{d\delta} \Big|_{\delta = 0} .
\]

\item What are the units of the derivative above? Interpret its meaning.

\item Suppose now that each month has 30 days so that there are 360 days in one year. Suppose also that the declination of the sun varies sinusoidally as a function of time, that the maximum declination of $\delta = 23.5^\circ$ occurs on the summer solstice (say June 21st) and the minimum declination $\delta = -23.5^\circ$ occurs on the winter solstice (December 21st). 

\begin{enumerate}
\item Find an expression for a function
\[
     \delta = k(t) \, , \, t\geq 0,
\]
that gives the declination of the sun (measured in radians) in terms of the number of days since the spring equinox. 

\item Evaluate the derivative
\[
  \frac{d}{dt}\left( f(k(t))  \right)\Big|_{t= 0} =    \frac{dh}{dt} \Big|_{t= 0} .
\]

\item What are the units of the derivative above? Interpret its meaning. 

\end{enumerate}
\end{enumerate}


\end{question}








\begin{question}  \label{Q:er43455}
The function
\[
    h = f(\delta) = \frac{24}{\pi} \arccos(-\tan\phi \tan\delta) \, , \, -\pi/2+\phi < \delta < \pi/2-\phi ,
\]
expresses the number of hours of daylight per day at latitude $\phi$ in terms of the declination of the sun. The declination of the sun ($\delta$) is the angle the sun's rays make with the plane of the equator, taken to be positive between the spring and fall equinoxes in the northern hemisphere. The latitude $\phi$, $-\pi/2 \leq \phi \leq \pi/2$, is positive at points in the northern hemipshere.

\begin{onlineOnly}
    \begin{center}
\geogebra{vnhrutwu}{900}{600}
\end{center}
\end{onlineOnly}

Geogebra activity available at \href{https://www.geogebra.org/classic/vnhrutwu}{151: Declination of Sun 2}

\begin{enumerate}
\item Use the graph of the function $f$ below (at latitude $\phi \sim 1.1$) to sketch a graph of the derivative
\[
   \frac{dh}{d\delta} = f^\prime(\delta) .
\]

\item Suppose the latitude $\phi$ is held constant and find an expression for the derivative 
\[
   \frac{dh}{d\delta} %= \frac{d}{d\delta} \left(  -\tan\phi \tan \delta    \right).
\]

\item What are the units of the derivative in part (b)?

\item Input your expression for the derivative in Line 4 of the worksheet below (follow the directions there). Then vary the slider $\phi$ to see how the function $f$ and its derivative vary with latitude. Summarize your observations.

\item Find an expression for the derivative 
\[
   \frac{dh}{d\delta}\Big|_{\delta = 0} %=  \frac{d}{d\delta} \left(  -\tan\phi \tan \delta    \right)\Big|_{\delta =0}
\]
in terms of the latitude $\phi$. 

\item Evaluate the derivative in part (e) at a latitude of $\phi =\pi/4$. Interpret its meaning in terms of small changes.

\begin{onlineOnly}
    \begin{center}
\desmos{ifomatkcta}{900}{600}
\end{center}
\end{onlineOnly}

Desmos activity available at \href{https://www.desmos.com/calculator/ifomatkcta}{151: Length of Day 1}

\pskip 

\item Suppose now that each month has 30 days so that there are 360 days in one year. Suppose also that the declination of the sun varies sinusoidally as a function of time, that the maximum declination of $\delta = 23.5^\circ$ occurs on the summer solstice (say June 21st) and the minimum declination $\delta = -23.5^\circ$ occurs on the winter solstice (December 21st). 

Find an expression for a function
\[
     \delta = k(t) \, , \, t\geq 0,
\]
that gives the declination of the sun (measured in radians) in terms of the number of days since the spring equinox. 

\item Find a function
\[
   r = g(\phi) \, , \, -\pi/2 < \phi < \pi ,
\]
that expresses the rate of change in the number of hours of daylight per day (measured in (hours of daylight/day)/day)) on the spring equinox in terms of the latitude $\phi$. Input this function in Line 1 of the demonstration below.

\item On the spring equinox, all latitudes receive $12$ hours of daylight/day. Use the result of part (h) to approximate how many extra minutes of daylight we would get on the following day living in 

\begin{enumerate}
\item Shoreline, latitude $47.75^\circ$N

\item Fairbanks, Alaska, latitude $64.8^\circ$N

\end{enumerate}

\begin{onlineOnly}
    \begin{center}
\desmos{nf8n5uphhl}{900}{600}
\end{center}
\end{onlineOnly}

Desmos activity available at \href{https://www.desmos.com/calculator/nf8n5uphhl}{151: Length of Day 2}

\end{enumerate}

\end{question}

\end{document}
