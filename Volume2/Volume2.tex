\documentclass{ximera}
\title{Volumes, Part 2}

\newcommand{\pskip}{\vskip 0.1 in}

\begin{document}
\begin{abstract}
Volume.
\end{abstract}
\maketitle

\section{Paraboloid of Revolution}

\begin{question} \label{QPkdf9fM}

\begin{enumerate}

\item Find an expression for the area of the region below bounded by the parabola and the line $y=h$ in terms of the height $h$ and base $b$ of the region.

\begin{onlineOnly}
    \begin{center}
\desmos{ftsereyrj4}{900}{600}
\end{center}
\end{onlineOnly}

\href{https://www.desmos.com/calculator/ftsereyrj4}{152: Parabola 23B}

\item What fraction of the area of the surrounding rectangle does the parabolic region occupy?

\item Rotate the region above about the $y$ axis and you get a paraboloid of revolution as illustrated below.

\begin{onlineOnly}
    \begin{center}
\desmosThreeD{1bculfxxtt}{900}{600}
\end{center}
\end{onlineOnly}

\href{https://www.desmos.com/3d/1bculfxxtt}{152: Paraboloid}

\item Find an expression for the volume of the paraboloid in terms of its height $h$ and the radius $a$ of its base.

\item Click the wrench in the upper right corner of the worksheet above and check the \emph{Translucent Surfaces} box. Then activate the folder \emph{Cylinder} in Line 3.

\item What fraction of the volume of the surrounding cylinder does the paraboloid occupy? Does this look correct?

\item How can you explain the difference between the fractions in parts (b) and (f)?

\end{enumerate}
\end{question}

\section{Napkin Ring}

\begin{question} \label{QOERerr3nnn}

\begin{enumerate}
\item Start with a solid sphere of radius $a$ as shown below.

\begin{onlineOnly}
    \begin{center}
\desmosThreeD{dw5gjr0es4}{900}{600}
\end{center}
\end{onlineOnly}

\href{https://www.desmos.com/3d/dw5gjr0es4}{152: Napkin Ring}

\item Click the wrench in the upper right corner of the worksheet above and check the \emph{Translucent Surfaces} box.

\item Now drill a circular hole of height $2h$ through the center of the sphere. To see this:

\begin{enumerate}
\item Activate the folder \emph{Cylinder} in Line 3 above.

\item Then activate the folder \emph{Napkin Ring} in Line 5.

\item Then turn off the folders \emph{Sphere} and \emph{Cylinder} in Lines 1 and 3.
\end{enumerate}

\item We now go about expressing the volume of the remaining solid (a napkin ring) in terms of $a$ and $h$.

\begin{enumerate}

\item Not sure yet.

\begin{onlineOnly}
    \begin{center}
\desmosThreeD{ubyhmxtccz}{900}{600}
\end{center}
\end{onlineOnly}

\href{https://www.desmos.com/3d/ubyhmxtccz}{152: Napkin Ring 2}


\end{enumerate}

\item Drag the slider $a$ in Line 12 to change the radius of the original sphere. How does the volume of the napkin ring change as you vary $a$?

\end{enumerate}



\end{question}

\end{document}