\documentclass{ximera}
\title{The Ladder and the Tree, Part 2}

\newcommand{\pskip}{\vskip 0.1 in}

\begin{document}
\begin{abstract}
Implicit differentiation.
\end{abstract}
\maketitle

\begin{question} \label{Q43hhbdsfdsfdszz}

A tree leans precariously with its trunk making an angle of $\phi = \pi/3$ radians with the ground. One end of a ladder leans against the trunk, the other rests on the horizontal ground. We analyze how a small change in the distance between the bottom of the ladder and the base of the trunk changes the distance between the top of the ladder and the base of the trunk.


\pdfOnly{
Access Desmos interactives through the online version of this text at
 
\href{https://www.desmos.com/calculator/bscbpblji1}.
}
 
\begin{onlineOnly}
   \begin{center}
\desmos{bscbpblji1}{900}{600}
\end{center}
\end{onlineOnly}

\href{https://www.desmos.com/calculator/bscbpblji1}{151: Ladder and Tree 21}

%We'll let $u$ be the distance $OT$ between the top of the ladder and the base of the trunk (measured in feet) and  $s$ the distance $OB$ between the bottom of the ladder and the base of the trunk (also measured in feet).

\begin{enumerate}
\item Find a function
\[
    v_B = f(\theta) \, , \,  0\leq \theta \leq 2\pi/3 ,
\]
that expresses the speed of the ladder's bottom end in terms of the angle $\theta = \angle OTP$ the ladder makes with the tree and the ladder's rotation rate $\omega$ (measured in rad/sec). Do not assume the rotation rate is constant.

\item Check that your expression for the speed has the correct units.

\item Similarly find a function 
\[
    v_T = g(\theta) \, , \,  0\leq \theta \leq 2\pi/3 ,
\]
that expresses the speed of the ladder's top end in terms of $\theta$ and $\omega$.

\item Play the slider in Line 2 of the worksheet above. Then use the animation to sketch by hand graphs of the speed functions in parts (a) and (c) assuming the ladder rotates at a constant rate.

\item Activate the \emph{speed functions} folder in Line 41 to check your graphs.

\item Find the ratio $v_B/v_T$ of the speeds when the bottom end of the ladder is five feet from the tree's base.

\item What angle does the ladder make with the tree when the ladder's bottom end is moving twice as fast as its top end? Find all possibilities.

\item Answer part (g) when the tree is perpendicular to the ground instead.

\end{enumerate}

\end{question}

\begin{question}  \label{Q34hghnbnnh}
Solve the equation
\[
        9 - 3 \sin \theta = 11 .
\]

\begin{enumerate}
\item Explain your reasoning thoroughly, in complete sentences.

\item Include a picture as in class to help with your explanation.

\item End with a concluding sentence that expresses the solution as a set.
\end{enumerate}

\end{question}


\end{document}