\documentclass{ximera}
\title{The Ladder and the Tree, Part 2}

\newcommand{\pskip}{\vskip 0.1 in}

\begin{document}
\begin{abstract}
Implicit differentiation.
\end{abstract}
\maketitle

\begin{question}  \label{Qhfhghllllgg}
The bottom end of a seven-foot ladder slides across a horizontal floor as its top end slides down a vertical wall.

\begin{enumerate}

\item Suppose the bottom of the ladder is sliding toward the wall at a speed of $5$ ft/sec when the bottom is $3$ feet from the wall. At what rate is the ladder rotating at this instant?

\item Drag the slider $s$ in Line 1 of the worksheet below to approximate the angle the ladder makes with the ground when its bottom end is moving twice as fast as its top end.

\item Find the exact angle in part (b).
\end{enumerate}

\pdfOnly{
Access Desmos interactives through the online version of this text at
 
\href{https://www.desmos.com/calculator/mlulonqhoa}.
}
 
\begin{onlineOnly}
   \begin{center}
\desmos{mlulonqhoa}{900}{600}
\end{center}
\end{onlineOnly}

\href{https://www.desmos.com/calculator/mlulonqhoa}{151: Ladder and Tree 23}

\end{question}



\begin{question}  \label{Qvgggbbhhhggg}
A tree leans precariously with its trunk making an angle of $\phi = \pi/3$ radians with the ground. One end of a ladder leans against the trunk, the other rests on the horizontal ground.

At the moment the bottom end of the ladder is $8$ feet from the tree's base (in the position shown below), its bottom end is moving toward the tree's base at a speed of $5$ ft/sec. At this same moment the top end of the ladder is $3$ feet from the tree's base.

\pdfOnly{
Access Desmos interactives through the online version of this text at
 
\href{https://www.desmos.com/calculator/oftz4vb9qj}.
}
 
\begin{onlineOnly}
   \begin{center}
\desmos{oftz4vb9qj}{900}{600}
\end{center}
\end{onlineOnly}

\href{https://www.desmos.com/calculator/oftz4vb9qj}{151: Ladder and Tree 22}

\begin{enumerate}
\item Is the ladder rotating clockwise or counterclockwise at this moment?

\item At what rate? Do \emph{not} assume that the bottom end moves at a constant speed. Solve this problem twice. First, with implicit differentiation and then again without. Do \emph{not} use a calculator. 
\end{enumerate}

%\item Play the slider $\phi_2$ in Line 2. Then find the exact angle(s) the ladder makes with the ground when its bottom end is moving twice as fast as its top end. Find all possibilities. Do not use a calculator.

\end{question}




\begin{question}  \label{Qsdafdsfhggg}
A tree leans precariously with its trunk making an angle of $\phi = \pi/3$ radians with the ground. One end of a seven-foot ladder leans against the trunk, the other rests on the horizontal ground.

\begin{enumerate}
\item Find all possible angles the ladder makes with the ground when its bottom end is moving twice as fast as its top end.

\item Drag the slider $\phi_2$ in Line 2 of the worksheet below to check your answer. 
\end{enumerate}

\pdfOnly{
Access Desmos interactives through the online version of this text at
 
\href{https://www.desmos.com/calculator/oftz4vb9qj}.
}
 
\begin{onlineOnly}
   \begin{center}
\desmos{oftz4vb9qj}{900}{600}
\end{center}
\end{onlineOnly}

\href{https://www.desmos.com/calculator/oftz4vb9qj}{151: Ladder and Tree 22}

\end{question}














\begin{question} \label{Q43hhbdsfdsfdszz}

A tree leans precariously with its trunk making an angle of $\phi = \pi/3$ radians with the ground. One end of a ladder leans against the trunk, the other rests on the horizontal ground. %We analyze how a small change in the distance between the bottom of the ladder and the base of the trunk changes the distance between the top of the ladder and the base of the trunk.


\pdfOnly{
Access Desmos interactives through the online version of this text at
 
\href{https://www.desmos.com/calculator/bscbpblji1}.
}
 
\begin{onlineOnly}
   \begin{center}
\desmos{bscbpblji1}{900}{600}
\end{center}
\end{onlineOnly}

\href{https://www.desmos.com/calculator/bscbpblji1}{151: Ladder and Tree 21}

%We'll let $u$ be the distance $OT$ between the top of the ladder and the base of the trunk (measured in feet) and  $s$ the distance $OB$ between the bottom of the ladder and the base of the trunk (also measured in feet).

\begin{enumerate}
\item Find a function
\[
    v_B = f(\theta) \, , \,  0\leq \theta \leq 2\pi/3 ,
\]
that expresses the speed of the ladder's bottom end in terms of the angle $\theta = \angle OTP$ the ladder makes with the tree and the ladder's rotation rate $\omega$ (measured in rad/sec). Do not assume the rotation rate is constant.

\item Check that your expression for the speed has the correct units.

\item Similarly find a function 
\[
    v_T = g(\theta) \, , \,  0\leq \theta \leq 2\pi/3 ,
\]
that expresses the speed of the ladder's top end in terms of $\theta$ and $\omega$.

\item Play the slider in Line 2 of the worksheet above. Then use the animation to sketch by hand graphs of the speed functions in parts (a) and (c) assuming the ladder rotates at a constant rate.

\item Activate the \emph{speed functions} folder in Line 41 to check your graphs.

\item Find the ratio $v_B/v_T$ of the speeds when the bottom end of the ladder is five feet from the tree's base.

\item What angle does the ladder make with the tree when the ladder's bottom end is moving twice as fast as its top end? Find all possibilities.

\item Answer part (g) when the tree is perpendicular to the ground instead.

\end{enumerate}

\end{question}





\end{document}