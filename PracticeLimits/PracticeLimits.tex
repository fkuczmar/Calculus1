\documentclass{ximera}
\title{Practice with Limits and Derivatives}

\newcommand{\pskip}{\vskip 0.1 in}

\begin{document}
\begin{abstract}
Practice with limits and derivatives.
\end{abstract}
\maketitle


\section{More Practice with Rational Functions}

\begin{question} \label{QPODFDgdfgfg}
The function
\[
        W = f(r) = \frac{2000}{r^2} , r\geq 4 ,
\]
expresses the weight of an astronaut (measured in pounds) in terms of her distance from the center of the earth (measured in thousands of miles).

\begin{enumerate}
\item  Find an expression for the average rate of change in the astronaut's weight with respect to her distance from the earth's center between distances $r=b$ and $r=c$ thousands of miles from the center. Assume $b,c\geq 4$ and that $b\neq c$.

\item Use your \emph{expression from part (a) directly and the definition of the derivative} to find an expression (fully simplified) for the derivative
\[
   \frac{dW}{dr}\Big|_{r=b} .
\]

\item Use part (b) to evaluate the derivative
\[
   \frac{dW}{dr}\Big|_{r=25} .
\]

\item What are the units of the derivative above?

\item Explain the meaning of the derivative in part (c) using the language of small changes.
\end{enumerate}

\end{question}

\begin{question} \label{QOidIdtgt4n}
The function
\[
   g = f(v) = \frac{1}{2}v + 10 \, , \, 10\leq v \leq 55 ,
\]
expresses the gas mileage (in miles/gal) of a car in terms of its speed.

\begin{enumerate}
\item At what rate (in gal/hr) does the car burn gas at a speed of $40$ miles/hour?

\item Find a function
\[
    r =g(v)\, , \, 10\leq v \leq 55 ,
\]
that expresses the rate (in gal/hr) at which the car burns gas in terms of its speed.

Answer: The function is 
\[
     r = g(v) = \answer{\frac{v}{\frac{1}{2}v + 15}} \, , \, 10\leq v \leq 55 .
\]

\item Find an expression for the average rate of change in the rate at which the car burns gas with respect to its speed between speeds of $v=w$ and $v=b$ miles/hour. Then use this expression and the algebra of limits to find an expression for the derivative
\[
   \frac{dr}{dv}\Big|_{v=b} .
\]

\item Evaluate the derivative
\[
   \frac{dr}{dv}\Big|_{v=40} .
\]

\item What are the units of the derivative above. Explain its meaning in terms of small changes.


\end{enumerate}
\end{question}

\section{The Squaring Function and Its Inverse}

\begin{question}  \label{QPodfoitte43}
The function
\[
         A = f(s) \, , \, s\geq 0 ,
\]
expresses the area of a square (measured in square feet) in terms of its side length (in feet).

\begin{enumerate}
\item Find an expression for the average rate of change in the area of a square with respect to its side length between side lengths $s=b$ and $s=w$ feet.

\item Use your expression from part (a) and the algebra of limits to find an expression for the derivative
\[
      \frac{dA}{ds}\Big|_{s=b}.
\]

\item Use the result of part (b) to evaluate the derivative
\[
   \frac{dA}{ds}\Big|_{s=5}.
\]

\item What are the units of the derivative in part (c).

\item Explain the meaning of the derivative (c) in terms of small changes.

\begin{explanation}
You should have found that
\[
    \frac{dA}{ds}\Big|_{s=5} = 10 \text{ ft}^2/\text{ft} .
\]
This means that if the change
\[
    \Delta s = s-5 \sim 0,
\]
in the side length is small, and  
\[
   \Delta A = f(s) - f(5) = f(s)-25
\]
is the resulting change in the area, the derivative is a good approximation to the average rate of change. That is,
\[
  \frac{\Delta A}{\Delta s} \sim \frac{dA}{ds}\Big|_{s=5} = 10 \text{ ft}^2/\text{ft},
\]
and
\[
   \Delta A \sim 10 \Delta s. 
\]

For example, if 
\[
 \Delta s = 0.1 ,
\]
then 
\[
    \Delta A \sim (10 \text{ ft} ) (0.1 \text{ ft}) = 1 \text{ ft}^2.
\]
Compare this approximation with the actual change
\[
   \Delta A = f(5.1) - f(5) = 1.01 \text{ ft}^2
\]
in the area. Pretty good.
\end{explanation}

\end{enumerate}

\end{question}

\begin{question} \label{QDfg5rtghbcvdf}
The function 
\[
   s = g(A) =\sqrt{A} \, , \, A\geq 0 ,
\]
expresses the side length (in feet) of a square in terms of its area (in square feet).

\begin{enumerate}
%\item Find an expression for the average rate of change in the side length with respect to area between areas $A=a$ and $A=w$ square feet.

\item Use the algebra of limits to find an expression for the derivative
\[
      \frac{ds}{dA}\Big|_{A=a}.
\]

\item Evaluate the derivative
\[
     \frac{ds}{dA}\Big|_{A=25}.
\]
What are its units? Explain its meaning in terms of small changes.

\begin{explanation}
The standard approach goes something like this.
\begin{align*}
    \frac{ds}{dA}\Big|_{A=a} &= \lim_{w\to a} \frac{g(w)-g(a)}{w-a} \\
                                             &= \lim_{w\to a} \frac{\sqrt{w}-\sqrt{a}}{w-a} \\
                                            &= \lim_{w\to a} \frac{\sqrt{w}-\sqrt{a}}{(\sqrt{w}-\sqrt{a})(\sqrt{w}+\sqrt{a})} \\
                                           &= \lim_{w\to a} \frac{1}{\sqrt{w}+\sqrt{a}} \\
                                            &= \frac{1}{2\sqrt{a}} .
\end{align*}

This approach obscures a key point. Namely, that the function $g$ is the \emph{inverse} of the function $f$ in Question 1. We can exploit this by instead rewriting the equation
\[
       s = \sqrt{A} = g(A)
\]
as 
\[
    A = s^2 = f(s).
\]
Now we'll find an expression for the derivative
\[
     \frac{ds}{dA}\Big|_{s=b}
\]
in terms of the side length $s=b$. This gives
\begin{align*}
        \frac{ds}{dA}\Big|_{s=b} &= \lim_{w\to b}\frac{w-b}{f(w) - f(b)} \\
                                              &=\lim_{w\to b}\frac{w-b}{w^2 - b^2)}\\
                                              &=\lim_{w\to b}\frac{1}{w+b)} \\
                                               &= \frac{1}{2b} .
\end{align*}

Letting $a = b^2$ be the area, we have $b=\sqrt{a}$, and
\[
  \frac{ds}{dA}\Big|_{A=a} = \frac{1}{2\sqrt{A}}.
\]

This makes it clear that the derivative of the inverse of a function is the \emph{reciprocal} of the function's derivative.

\end{explanation}


\item Use the result of part (b) to evaluate the derivative
\[
   \frac{dA}{ds}\Big|_{s=5}.
\]

\item What are the units of the derivative in part (c).

\item Explain the meaning of the derivative (c) in terms of small changes.


\end{enumerate}
\end{question}

\section{The Cubing Function and its Inverse}

\begin{question} \label{QRDFGGERE}
Repeat the two questions of the previous section with the function
\[
    V = f(s) = s^3 \, , \, s\geq 0 ,
\]
that expresses the volume of a cube in terms of its side length.
\end{question}




\section{More on the Squaring Function}

Check back later.


\end{document}