\documentclass{ximera}
\title{Practice Quiz 2}

\newcommand{\pskip}{\vskip 0.1 in}

\begin{document}
\begin{abstract}
Practice quiz, Weeks 4-6
\end{abstract}
\maketitle

%\section{Directions}
\emph{Directions:}

\begin{enumerate}
\item Show all work.

\item Give brief explanations for each problem. Include these explanations in the flow of the solution.

\item Show all units in all computations.

\item No calculators.

\item Show each step when using the chain, product, and quotient rules and use the Leibniz notation when doing so. Here is an example.

\end{enumerate}

\begin{example}
Find an expression for the derivative
\[
    \frac{d}{d\theta}\left( 4 + 5\cos^3(2\theta) \right) .
\]

\begin{explanation}
With $y=4 + 5\cos^3(2\theta)$, the derivative $dy/d\theta$ is 
\begin{align*}
\frac{dy}{d\theta} &= \frac{d}{d\theta}\left(   4 + 5\cos^3(2\theta)   \right) \\
                          & = \frac{d}{d\theta}\left( 4 \right) + 5\frac{d}{d\theta}\left( \left( \cos(2\theta)  \right)^3  \right) \\
                           &=  5(3) (\cos(2\theta))^2 \frac{d}{d\theta}\left(  \cos (2\theta) \right) \\
                          &= 15(\cos(2\theta))^2 (-\sin(2\theta))\frac{d}{d\theta}(2\theta) \\
                           &= -30(\cos(2\theta))^2\sin(2\theta) .
\end{align*}


\end{explanation}
\end{example}



\begin{question}  \label{Qujn5tyh6uu}
Find expressions for each of the following derivatives.

\begin{enumerate}
\item 
\[
\frac{d}{dt}\left( 4 - t^4 e^{-5t})  \right)
\]

\item 
\[
\frac{d}{d\theta}\left(  \frac{12}{4 + 5\tan (4\theta)}   \right)
\]

\item
\[
   \frac{d}{dw}\left(   \sqrt{40-5w^2}   \right)
\]

\item
\[
   \frac{d}{dt}\left(   2^{t/8}   \right)
\]

\end{enumerate}

\end{question}

\begin{question}
\begin{enumerate}

\item Find an equation of the tangent line to the curve 
\[
  y =  4 - t^4 e^{-5t}
\]
at the point on the curve with $t$-coordinate $t=0$.

\item Find an equation of the tangent line to the curve 
\[
  y =   \frac{12}{4 + 5\tan (4\theta)}
\]
at the point on the curve with $\theta$-coordinate $\theta=0$.
\end{enumerate}
\end{question}

\begin{question} \label{Q000d9ggdgbgh}
Assume for this question that each month has 30 days and that the number of hours of daylight/day in Seattle is a sinusoidal function of time. Assume also that on June 21, Seattle gets a maximum of $16$ hours of daylight/day and that on December 21, Seattle gets a minimu of $8$ hours of daylight/day.

\begin{enumerate}
\item Find a function 
\[
   H = f(t) \, , \, 0\leq t\leq 12,
\]
that expresses the number of daylight hours/day in Seattle in terms of the number of months since June 21. Use the cosine function. Start by sketching a graph. Explain your reasoning.

\item Evaluate the derivative 
\[
   \frac{dH}{dt}\Big|_{t=2} .
\]

\item What are the units of the derivative above?

\item Explain the meaning of the above derivative.
\end{enumerate}

\end{question}

\begin{question}  \label{Q33eerrggg}
The function
\[
   T = f(m) = 20+70e^{-m/20}\, , \, 0\leq m \leq 60 ,
\]
expresses the temperature (in Celsius degress) of a cup of coffee in terms of the number of minutes past noon.

\begin{enumerate}
\item At what rate is the temperature changing at 12:40pm?

\item Does the temperature change at a constant relative rate? Justify your assertion.

\item Find a function $r=g(T)$ that expresses the rate of change in the temperature (measured in $^\circ$C/min) in terms of the temperature (measured in Celsius degrees).

\item Find the temperature when it is decreasing at the rate of $3^\circ$C/min.

\item Find a function $s=h(T)$ that expresses the relative rate of change in the temperature in terms of the temperature (measured in Celsius degrees).

\item Find the temperature when it is decreasing at the rate of $3\%$/min.
\end{enumerate}
\end{question}


\begin{question} \label{Qggey5ghhtt}
Assume for this problem that over the course of a 24-hour period beginning at midnight of July 21, the temperature in Shoreline is a sinusoidal function of time. Assume also that the temperature reaches its minimum of $50^\circ$F at 5am and its maximum of $80^\circ$ at 5pm.

At what rate (with respect to time) is the temperature changing when it is $68^\circ$F for the second time? Start by finding a function that expresses the temperature in terms of time. Be sure to define your variables.
\end{question}

\end{document}