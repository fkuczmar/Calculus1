\documentclass{ximera}
\title{Limits}


\newcommand{\pskip}{\vskip 0.1 in}

\begin{document}
\begin{abstract}
Limits in context.
\end{abstract}
\maketitle


\pskip

\section{Limits and Tangent Lines}

\begin{example}  \label{Ex:9uudfg9}
Let
\[
     g(x) = \frac{x^2-9}{3x-9} .
\]
(a) Evaluate each of the following expressions.

\pskip

(i) $g(7)$

(ii) $\lim_{x\to 7} g(x)$

(iii) $g(3)$

(iv) $\lim_{x\to 3} g(x)$

\pskip

(b) Simplify and then graph the function $g(x)$.

(c) Interpret the expressions in part (a) geometrically by considering the graph of the function $f(x)=x^2/3$ as in the demonstration below.

\begin{onlineOnly}
    \begin{center}
\desmos{u0uvuchnrk}{900}{600}
\end{center}
\end{onlineOnly}

Desmos activity available at
\href{https://www.desmos.com/calculator/u0uvuchnrk}{151: Parabola Basic}

\end{example}

\section{Limits and Gas Mielage}
\begin{example} \label{Ex:9sd8gfs}
The function 
\[
   G = f(s) = \frac{2}{5} +\frac{1}{5000}(40s^2 - s^3) , \, 2\leq s \leq 25 ,
\]
expresses the number of gallons of gas in your car in terms of your distance from home. The distance is measured in miles along your route. 

\begin{onlineOnly}
    \begin{center}
\desmos{e2uxt3o4sr}{900}{600}
\end{center}
\end{onlineOnly}

Desmos activity available at
\href{https://www.desmos.com/calculator/e2uxt3o4sr}{151: Gas as a Function of Distance}


(a) Use the graph of the function $f$ shown above to determine if you are driving toward or away from home. Explain your reasoning.

(b) Use numerical methods to approximate your gas mileage at the moment you are $20$ miles from home.

(c) Use the algebra of limits to determine your exact gas mileage when you are $20$ miles from home.

(d) Use the graph to approximate your distance from home when your car was getting the best mileage. Explain your reasoning. 

(e) Use the graph to approximate your distance from home when your car was getting the least mileage. Explain your reasoning. 

(f) 

\end{example}



\section{Limits, Gas Mileage and Speed}

\begin{example}  \label{Ex:34t45rtg}
Suppose that between speeds of $30$ miles/hour and $70$ miles/hour the gas mileage of a car is a quadratic function of its speed. 
Suppose also that the car gets a maximum of $42$ miles/gal at a speed of $50$ miles/hour and that the car gets $38$ miles/gallon at a speed of $40$ miles/hour.

(a) Find an expression for the function
\[
    G = f(v) \, , \, 30\leq v \leq 70 ,
\]
that gives the gas mileage (in miles/gal) in terms of the speed (in miles/hour).

(b) Give numerical  and graphical evidence that either supports or refutes the claim that a small change in the car's speed at $60$ miles/hour gives an approximately proportional change in its gas mileage.

(c) Use the results of part (b) to approximate the proportionality constant. What are its units?

(d) Use the algebra of limits to find the exact value of the proportionality constant.

(e) Explain the meaning of the proportionality constant.

(f) Approximate the change
\[
    \Delta G = g - f(60)
\]
in gas mileage in terms of a small change
\[
  \Delta v = v - 60
\]
in the car's speed.

(g) Use part (f) to approximate the speed at which the car gets $36$ miles/gallon. 

(h) Would you expect your approximation in part (g) to be greater or less than the actual speed? Explain your reasoning with a graph.

(i) Simplify the units of the proportionality constant. What might these units suggest about a way to interpret the constant?

(j) At what rate (in gal/hr) does the car burn gas at a speed of $60$ miles/hour? 

(k) How is the rate in part (j) related to the proportionality constant?

\end{example}


\section{Limits, Speed and Altitude}
\begin{example}  \label{Ex:deft4}
A rock dropped from a height of $100$ feet falls to the surface of Planet Krypton without air resistance.

(a) By considering only the physical situation and \emph{without} doing any computations, sketch a graph of the function
\[
    v =g(h) \, , 0\leq h \leq 100
\]
that expresses the rock's speed (in ft/sec) in terms of its height (in feet).

(b) \begin{question} \label{Qerfr45rt}

Use the results from part (a) to choose a reasonable expression for the function $g$ from the list below.

\begin{multipleChoice}
\choice{$g(t)=100 - 9t^2$, $0\leq t \leq 10/3$}
\choice{$g(h)=100 - 9h^2$, $0\leq h \leq 100$}
\choice{$g(h)=0.005(100-h)^2$, $0\leq h \leq 100$}
\choice[correct]{$g(h)=6\sqrt{100-h}$, $0\leq h \leq 100$}
\end{multipleChoice}
\end{question}

(c)  Give numerical  and graphical evidence that either supports or refutes the claim that a small change in the rocks height from $64$ feet gives an approximately proportional change in its speed.
  
(d) Use the results of part (c) to approximate the proportionality constant. What are its units?

(e) Use the algebra of limits to find the exact value of the proportionality constant.

(f) Explain the meaning of the proportionality constant.

(g) Approximate the change
\[
    \Delta v = v - g(64)
\]
in the rock's speed in terms of a small change
\[
  \Delta h = h - 64
\]
in its height.

(h) Use part (g) to approximate the rock's speed at a height of $63$ feet.

(i) Would you expect your approximation in part (h) to be greater or less than the actual speed? Explain your reasoning with a graph.

(j) Simplify the units of the proportionality constant. Does this simplification help to understand or obscure the meaning of the proportionality constant?
\end{example}


\section{Limits and Gas Mileage}


\section{Limits and Purchasing Power}


\end{document}