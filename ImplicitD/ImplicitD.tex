\documentclass{ximera}
\title{Implicit Differentiation}

\newcommand{\pskip}{\vskip 0.1 in}

\begin{document}
\begin{abstract}
An introduction to implicit differentiation.
\end{abstract}
\maketitle

\section*{Discussion Questions}

\begin{question}  \label{Q:dsftr4tr656t}
Which of the following equations define $y$ implicty as a function of $x$ in a sufficiently small neighborhood of the given point? Supplement your reasoning with a graph of each relation.

(a) $x^2 + y^2 = 25$ near the point $(4,-3)$

(b) $x^2 + y^2 = 25$ near the point $(0,-5)$

(c) $x^2 + y^2 = 25$ near the point $(-5,0)$

(d) $x^2 - xy + y^2 = 1$ near the point $(1,1)$.

(e) $x^2 - xy + y^2 = 3$ near the point $(1,2)$.

(f) $x^2 - xy + y^2 = 3$ near the point $(2,1)$.

\end{question}

\section*{Some Problems}

\begin{question}  \label{Q:4bbfdellk}
(a) Use implicit differentiation to find an equation of the tangent line to the ellipse 
\[
   \frac{x^2}{a^2} + \frac{y^2}{b^2} = 1
\]
at the point $P$ with coordinates $(x_0, y_0)$. Here $a$, $b$ are positive constants.

(b) Input your equation from part (a) on Line 3 of the desmos worksheet below. Drag point $P$ to check your equation is correct.

(c) Solve part (c) again \emph{without} calculus by considering a composition of transformations that takes the circle
\[
       x^2 + y^2 = 1
\]
to the ellipse in part (a).

\begin{onlineOnly}
    \begin{center}
\desmos{t1e9v7ncpv}{900}{600}
\end{center}
\end{onlineOnly}

Desmos activity available at \href{https://www.desmos.com/calculator/t1e9v7ncpv}{151: Tangents to Ellipse}

\end{question}





\begin{question}  \label{Q:LLKKKMM}
(a) Use implicit differentiation to show that segments cut by the coordinate axes from the tangent lines to the astroid
\[
   x^{2/3} + y^{2/3} = a^{2/3}
\]
all have the same length. Here $a>0$ is a constant.

(b) Prove the same result by using trigonometric functions to parameterize the astroid.

\begin{onlineOnly}
    \begin{center}
\desmos{vrythrvjuc}{900}{600}
\end{center}
\end{onlineOnly}

Desmos activity available at \href{https://www.desmos.com/calculator/vrythrvjuc}{151: Astroid}


\end{question}


\begin{question} \label{Q:ergbbfrr}
An ellipse through the point $P(0,b)$ has focal points $F_1$ at the origin and $F_2$ at the point $A(a,0)$.

(a) Use the definition of an ellipse as the set of points whose distances to the foci have a constant sum to find an equation of the ellipse.

(b) Use implicit differentiation to find the slope of the tangent line to the ellipse at $P$.

(c) Find an equation of the line normal to the ellipse at $Q$.

(d) Find the coordinates of the point $Q$ where the normal line intersects the $x$-axis.

(e) Express the ratio $F_1 Q : Q F_2$ in terms of $a$ and $b$. Interpret the ratio geometrically.

The ratio is
\[
    \frac{F_1 Q}{Q F_2} = \answer{\frac{b}{\sqrt{a^2+b^2}}} . 
\]

(f) Check your work in the demonstration below.

\begin{onlineOnly}
    \begin{center}
\desmos{xqywotxxf9}{900}{600}
\end{center}
\end{onlineOnly}

Desmos activity available at \href{https://www.desmos.com/calculator/xqywotxxf9}{151:Normal to an Ellipse}



\end{question}

\begin{question}  \label{Q:JDJHDHD}
Let $F_1$ and $F_2$ be respectively the origin and the point with coordinates $(a,0)$. The curve ${\cal C}$ passes through the point $P(0,b)$. The curve is defined by the property that the sum of the distance from a point $Q$ of ${\cal C}$ to $F_1$ and $k$ times its distance to $F_2$ is a constant.

(a) Find an equation of the curve ${\cal C}$.

(b) Find an equation of the tangent line to ${\cal C}$ at $P$.

(c) Find an equation of the line normal to ${\cal C}$ at $P$.

(d) Find the coordinates of the point $Q$ where the normal line intersects the $x$-axis.

(e) Express the ratio $F_1 Q : Q F_2$ in terms of $a$ and $b$. Interpret the ratio geometrically.

The ratio is
\[
    \frac{F_1 Q}{Q F_2} = \answer{\frac{bk}{\sqrt{a^2+b^2}}} . 
\]

(f) Check your work in the demonstration below.

\begin{onlineOnly}
    \begin{center}
\desmos{uz7w5uh0vl}{900}{600}
\end{center}
\end{onlineOnly}

Desmos activity available at \href{https://www.desmos.com/calculator/uz7w5uh0vl}{151: Generalized Ellipse}


\end{question}

\section*{Waves}

\begin{question}   \label{Q:DFLDFDFGggg}
The function
\begin{equation}  \label{Eq:Wave1}
      y= f(x,t) = a \sin (kx - \omega t) , t\geq 0,
\end{equation}
describes a wave on a string. The functions expresses the displacement (in meters) of a point on the string in terms of the position $x$ (in meters) of the point and time $t$, measured in seconds since the motion began.

(a) Experiment with the sliders $k$, $\omega$, in the demonstration below, playing the slider $u$ ($u$ is just another name for $t$). Summarize your observations. In particular, be sure to turn off Line 1 to be better able to see the motion of the individual points of the string.

(b) What are the units of $k$ and $\omega$? How do you know?

(c) Find an expression for the wavelength $lambda$ in terms of $k$, $\omega$.

(d) Find an expression for the period of oscillation $T$ in terms of $k$, $\omega$.
 
(e) Hold $y$ constant and differentiate each side of equation (\ref{Eq:Wave1}) with respect to $t$ to find an expression for the speed of the wave. Turn on the graph in Line 9. Explain the logic behind the computation. 

\begin{onlineOnly}
    \begin{center}
\desmos{9xmkig9hwi}{900}{600}
\end{center}
\end{onlineOnly}

Desmos activity available at \href{https://www.desmos.com/calculator/9xmkig9hwi}{151: Traveling Wave 1}

\end{question}


\end{document}