\documentclass{ximera}
\title{Derivatives of Inverse Functions, Short Version}

\newcommand{\pskip}{\vskip 0.1 in}

\begin{document}
\begin{abstract}
Derivatives of inverse functions.
\end{abstract}
\maketitle

\section{The Main Idea}

The main idea here is simple. That the derivative of the \emph{inverse} of a function is the \emph{reciprocal} of the function's derivative. Or expressed more succinctly, \emph{the derivative of the inverse is the reciprocal of the derivative}.

We don't really need to know much about calculus to understand why. 

\begin{question} \label {Qtyhhdhfdgfd}
Take for example, the function
\[
      G = f(s)  \, , \, 0\leq s\leq 20, 
\]
that expresses the number of gallons of gas in your car in terms of your distance from home. The distance is measured in miles along your route and the graph of the function is shown below.

\begin{onlineOnly}
    \begin{center}
\desmos{oltntpzth9}{900}{600}  %    pxsmo04nmg  8swp20zond
\end{center}
\end{onlineOnly}

\href{https://www.desmos.com/calculator/oltntpzth9}{151: Inverse Function 1}

\begin{enumerate}
\item Use the slider $s$ above to approximate (with units) the derivative
\[
   \frac{dG}{ds}\Big|_{s=2} .
\]
Explain the derivative's meaning.

\item Use the slider $s$ above to approximate (with units) the derivative
\[
   \frac{ds}{dG}\Big|_{G=f(2)} .
\]
Explain the derivative's meaning.

\item Write an equation that expresses the relationship between the derivatives in parts (a) and (b). Explain why this relationship holds.

\end{enumerate}
\end{question}

\begin{question}  \label{Qttynmmbbb}
This question is a continuation of Question \ref{Qtyhhdhfdgfd}. Now we are given an expression
\[
   G = f(s)  = \frac{s^2}{2000} + \frac{23s}{1000} + \frac{1}{10} \, , \, 0\leq s\leq 20, 
\]
for the function $f$.

\begin{enumerate}
\item Use calculus to find an expression for the derivative $dG/ds$ and to compute the exact value of the derivative
\[
        \frac{dG}{ds}\Big|_{s=2} .
\]

\item Use the idea of Question \ref{Qtyhhdhfdgfd} to find an expression for the derivative $ds/dG$ and the exact value of the derivative
\[
   \frac{ds}{dG}\Big|_{G=f(2)} .
\]

\item For a more algebraic way to find an expression for the derivative $ds/dG$, we'll differentiate both sides of the equation
\[
    G =  \frac{s^2}{2000} + \frac{23s}{1000} + \frac{1}{10}
\]
with respect to $G$. Keeping in mind that this equation, for values of $s$ between $0$ and $20$, defines $s$ implicitly as a function of $G$ (why?). Then using the chain rule tells us that
\[
     \frac{dG}{dG} = \left( \frac{s+23}{1000} \right)  \left(\frac{ds}{dG} \right) .
\]
And solving for the derivative $ds/dG$ gives
\[
     \frac{ds}{dG} = \answer{\frac{1000}{s+23}} .
\]

\end{enumerate}
\end{question}


\section{Derivatives of Inverse Power Functions}
Let's use what we just learned to verify something we already know, but in a more meaningful context.

\begin{example}  \label{ExLDKr3eDFRER}

The function
\[
         A = f(s) = s^2\, , \, s\geq 0 ,
\]
expresses the area of a square (measured in square feet) in terms of its side length (in feet).

Since
\[
   \frac{dA}{ds} = 2s ,
\]
we know that the derivative of the inverse function
\[
    s = f^{-1}(A)  = \sqrt{A}\, , \, A\geq 0,
\]
expressing the side length of a square in terms of its area is
\begin{align*}
 \frac{ds}{dA} &= \frac{1}{dA/ds} \\   
                      &= \frac{1}{2s} \\
                      &= \frac{1}{2\sqrt{A}} .
\end{align*}

We could verify this using the power rule:
\begin{align*}
 \frac{ds}{dA} &= \frac{d}{dA} \left( \sqrt{A} \right) \\   
                        &= \frac{1}{2\sqrt{A}} .
\end{align*}

To make this more meaningful, let's look at the derivative 
\begin{align*}
     \frac{dA}{ds}\Big|_{s=5} &= (2s)\Big|_{s=5} \\
                                           &= 10 \frac{\text{ft}^2}{\text{ft}} \\
                                           &= 10 \text{ ft}.
\end{align*}
Although it's almost never a good idea to simplify the units of the derivative, it is here. The derivative is $10$ feet and is equal to \emph{half} the perimeter of our square with side length $s=5$ feet. To get an idea why this is so, we'll interpret the derivative in terms of small changes.

Suppose we make a small change (measured in feet) 
\[
  \Delta s = s - 5 \sim 0 
\]
in the side length of the square. What can we say about the change 
\[
 \Delta A = f(s) - f(5) = A-25
\] 
in its area?
 
Well since $\Delta s \sim 0$,
\[
      \frac{\Delta A}{\Delta s} \sim   \frac{dA}{ds}\Big|_{s=5} = 10,
\]
we know that
\[
     \Delta A \sim  \left( \frac{dA}{ds}\Big|_{s=5}\right)\Delta s = 10 \Delta s .
\]
\emph{To approximate the change in the area we multiply the change in the side length by $10$ feet}.

\begin{enumerate}
\item The picture below suggests why this is a good approximation. Explain how. 

\item Do some algebra and compare the actual change 
\[
    \Delta A = f(5+s) - f(5) = (5+\Delta s)^2 - 25
\]
in the area with the approximate change. Reconcile the difference with the picture.

\end{enumerate}
\begin{freeResponse}
\end{freeResponse}

\begin{onlineOnly}
    \begin{center}
\desmos{h2fm6mm8ua}{900}{600}     % ruysb5sjsy
\end{center}
\end{onlineOnly}


\href{https://www.desmos.com/calculator/h2fm6mm8ua}{151: Square Error}


To interpret the derivative 
\[
    \frac{ds}{dA}\Big|_{A=25} = \frac{1}{10} text{ ft}^{-1}
\]
in much the same way by rewriting the above approximation as
\[
    \Delta s \sim \left( \frac{ds}{dA}\Big|_{A=25}\right)\Delta A = \left(\frac{1}{10}\right) \Delta A.
\]
\emph{To approximate the change in the side length we divide the change in the area by $10$ feet}.

This interpretation of the derivative is useful in starting to approximate square roots. For example,  thes ide length of a square with an area of $25.2 \text{ ft}^2$ is 
\begin{align*}
              \sqrt{25.2} &=      5 \text{ ft} + \Delta s  \text{ ft}  \\
                               & \sim 5 \text{ ft} +   \left(\frac{1}{10 \text{ ft}}\right)(0.2 \text{ ft}^2) \\
                                      &= 5.02 \text{ ft},
\end{align*}
giving us a pretty good approximation to 
\[
    \sqrt{25.2} \sim 5.01996 .
\]
\end{example}


\begin{exercise}  \label{ExLDFDFDdare}
Carry out a similar analysis with the function
\[
    V = s^3 \, , \, s\geq 0,
\]
expressing the volume of a cube (in $m^3$) in terms of its edge length (in meters). Try to include a picture as well.
\end{exercise}



\section{Derivatives of Inverse Trig Functions}

Our goal here is to find an expression for the derivative
\[
       \frac{d\theta}{dy} = \frac{d}{dy} \left(   \arcsin y \right)
\]
of the inverse sine function
\[
       \theta = \arcsin y = \sin^{-1}(y) .
\]

The first step is to recognize that the inverse sine function is \emph{not} the inverse of the sine function. Because the sine function is not one-to-one, its inverse is not a function. But the inverse sine function $\theta = \arcsin y$ is the inverse of the function
\[
  y = f(\theta) = \sin\theta \, , \, -\pi/2 \leq \theta \leq \pi.
\]

\begin{example} \label{ExLKDFrsfD}
Before computing the derivative of the inverse sine function in general, we'll look at a specific example and evaluate the derivative
\[
   \frac{d}{dy} \left(  \arcsin y \right) \Big|_{y=4/5} .
\]

\begin{onlineOnly}
    \begin{center}
\desmos{alp2mnxqhc}{900}{600}     % ruysb5sjsy
\end{center}
\end{onlineOnly}

\href{https://www.desmos.com/calculator/alp2mnxqhc}{151: Inverse Sine Function}

The idea to computing the above derivative is to first find the slope of the the tangent line to the curve $y=\sin\theta$, $-\pi/2 \leq \theta \leq \pi/2$, at the point $P$ on the graph above. Since $d(\sin\theta)/d\theta = \cos\theta$, the slope is the derivative
\begin{align*}
    \frac{d}{d\theta} \left(\sin\theta \right) \Big|_{\sin\theta = 4/5} &=  \frac{d}{d\theta} \left(\sin\theta \right) \Big|_{\theta = \arcsin(4/5)}  \\
                         &= \cos\theta        \Big|_{\theta = \arcsin(4/5)}  \\                                                       
 & = \cos (\arcsin(4/5).
\end{align*}

To evaluate this composition, we need to find the cosine of the angle 
\[
    \theta_0 = \arcsin(4/5)
\]
between $-\pi/2$ and $\pi/2$ whose sine is $4/5$. But since 
\[
  \cos^2\theta_0 + \sin^2\theta_0 = 1 ,
\] 
\begin{align*}
  \cos\theta_0 &= \pm \sqrt{1-\sin^2\theta_0} \\
                      &= \pm \sqrt{1-\left(\frac{4}{5}\right)^2} \\
                      &= \pm 3/5 .
\end{align*}

To choose the correct sine, we use the fact that $-\pi/2 \leq \theta_0 \leq \pi/2$. This tells us $\cos\theta_0 \geq 0$, so $\cos\theta_0 = 3/5$.

We now know that for the function
\[
  y = f(\theta) = \sin\theta \, , \, -\pi/2 \leq \theta \leq \pi
\]
\[
    \frac{dy}{d\theta}\Big|_{y=4/5} = 3/5.
\]
So for the inverse function
\[
     \theta = \arcsin y ,
\]
\begin{align*}
 \frac{d\theta}{dy}\Big|_{y=4/5} &= \left( \frac{dy}{d\theta}\Big|_{y=4/5} \right)^{-1} \\
                                                & =\frac{1}{\sqrt{1-(4/5)^2}}  \\
                                                & = 5/3 .
\end{align*}

\end{example}

Next we'll compute the derivative of the arcsine function for a general input.



%The idea of computing the derivative of the inverse of a one-to-one function as illustrated above is simple. But if we require that the derivative of the inverse be expressed in terms of \emph{its independent variable}, the algebra can get a bit tricky.

\begin{example} \label{Ecbfghnbng}
Take for example the function
\[
     y = g(\theta) = \sin\theta.
\]

Now this function is \emph{not} one-to-one and so its inverse is not a function. But we can restrict the domain of $g$ to get a new function
\begin{equation}  
     y =f(\theta) = \sin\theta \, , \, -\pi/2 \leq \theta \leq \pi/2 , \label{Eq:ArcSine}
\end{equation}
that is one-to-one and that has the same range $\{y | -1 \leq y \leq 1\}$ as $g$. The inverse of this new function $f$ is the inverse sine function
\[
    \theta = f^{-1}(y) = \arcsin(y) = \sin^{-1}(y) .
\]

The function $\arcsin (y)$ gives the angle between $-\pi/2$ and $\pi/2$ whose sine is equal to $y$. The key to expressing its derivative
\[
  \frac{d\theta}{dy} = \frac{d}{dy}\left(  \arcsin y \right)
\]
in terms of $y$, is to express the derivative of the original function $y=f(\theta)$ in terms of its output $y=\sin\theta$.

Now
\[
     \frac{dy}{d\theta} = \frac{d}{d\theta} \left( \sin\theta \right) = \cos\theta .
\]
To express this derivative in terms of $y=\sin\theta$, we use the fact that 
\[
 \cos^2\theta + \sin^2\theta = 1 .
\]
So 
\[
      \frac{dy}{d\theta} = \cos\theta = \pm \sqrt{1-\sin^2\theta} = \pm \sqrt{1-y^2}.
\]

To choose between $\pm$, we must remember that by the definition of the arcsine function (see equation (1)), 
\[
   \pi/2 \leq \theta \leq \pi/2 .
\]
This tells us that $\cos\theta \geq 0$. So we conclude that
\[
\frac{d}{dy}\left(  \arcsin y \right)  =\cos\theta = \sqrt{1-y^2}.
\]
\end{example}

\section{Exercises}

\begin{exercise}  \label{Eggghhybvbxzzz}
Express the derivative of each of the following functions in terms of its \emph{output}.
\begin{enumerate}
\item 
\[
       x = \cos\theta \, , \, 0\leq \theta \leq \pi
\]


\item 
\[
       x = \cos\theta \, , \, \pi\leq \theta \leq 2\pi
\]

\item 
\[
       y = \tan\theta \, , \, -\pi/2 < \theta < \pi/2
\]

\item
\[
     P = e^{t/5} \, , \, t\in \mathbb{R} 
\]

\item 
\[
  y = \frac{e^x - e^{-x}}{2} \, , \, x\in \mathbb{R} 
\]

\item 
\[
  y = \frac{e^x - e^{-x}}{e^x + e^{-x}} \, , \, x\in \mathbb{R} 
\]

\item 
\[
  y = \frac{e^x+ e^{-x}}{2} \, , \, x \leq 0 
\]


\item
\[
   G = \frac{s^2}{2000} + \frac{23s}{1000} + \frac{1}{10} \, , \, 0\leq s\leq 20, 
\]

\end{enumerate}

\end{exercise}

\begin{exercise} \label{E898gbfgg}
Use the results of Exercise 5 to express the derivative of the inverse of each of the functions in that exercise in terms of its \emph{independent} variable.
\end{exercise}

\begin{exercise}  \label{Ehglkhfgjsdafn}
\begin{itemize}
\item Find simplified expressions for each of the following derivatives. 

\item Do \emph{not} simplify a function before taking its derivative. 

\item Show all steps in using the chain rule. Use the Leibniz notation. 

\item Use Desmos to graph each function (before taking the derivative).

\item Explain how you could have found an expression for each derivative directly from the graph of the (original) function. 

\end{itemize}

\begin{enumerate}
\item
\[
       \frac{d}{dx}\left( \sin(\arcsin x ) \right)
\]

\item 
\[
       \frac{d}{d\theta}\left( \arcsin(\sin \theta ) \right)
\]

\item 
\[
  \frac{d}{dx}\left( \arctan x + \arctan(1/x)  \right)
\]

\item
\[
    \frac{d}{dx}\left( \arccos x + \arccos\left(\sqrt{1-x^2}\right)  \right)
\] 
\end{enumerate}

\end{exercise}


\end{document}