\documentclass{ximera}
\title{Derivatives of Inverse Functions, Short Version}

\newcommand{\pskip}{\vskip 0.1 in}

\begin{document}
\begin{abstract}
Derivatives of inverse functions.
\end{abstract}
\maketitle

\section{The Main Idea}

The main idea here is simple. That the derivative of the \emph{inverse} of a function is the \emph{reciprocal} of the function's derivative. Or expressed more succinctly, \emph{the derivative of the inverse is the reciprocal of the derivative}.

We don't really need to know much about calculus to understand why. 

\begin{question} \label {Qtyhhdhfdgfd}
Take for example, the function
\[
      G = f(s)  \, , \, 0\leq s\leq 20, 
\]
that expresses the number of gallons of gas in your car in terms of your distance from home. The distance is measured in miles along your route and the graph of the function is shown below.

\begin{onlineOnly}
    \begin{center}
\desmos{oltntpzth9}{900}{600}  %    pxsmo04nmg  8swp20zond
\end{center}
\end{onlineOnly}

\href{https://www.desmos.com/calculator/oltntpzth9}{151: Inverse Function 1}

\begin{enumerate}
\item Use the slider $s$ above to approximate (with units) the derivative
\[
   \frac{dG}{ds}\Big|_{s=2} .
\]
Explain the derivative's meaning.

\item Use the slider $s$ above to approximate (with units) the derivative
\[
   \frac{ds}{dG}\Big|_{G=f(2)} .
\]
Explain the derivative's meaning.

\item Write an equation that expresses the relationship between the derivatives in parts (a) and (b). Explain why this relationship holds.

\end{enumerate}
\end{question}




\section{Derivatives of Inverse Trig Functions}

The idea of computing the derivative of the inverse of a one-to-one function should be simple. But if we require that the derivative of a function be expressed in terms of its independent variable, the algebra can get a bit tricky.

\begin{example} \label{Ecbfghnbng}
Take for example the function
\[
     y = g(\theta) = \sin\theta.
\]

Now this function is \emph{not} one-to-one and so its inverse is not a function. But we can restrict the domain of $g$ to get a new function
\begin{equation}  
     y =f(\theta) = \sin\theta \, , \, -\pi/2 \leq \theta \leq \pi/2 , \label{Eq:ArcSine}
\end{equation}
that is one-to-one and that has the same range $\{y | -1 \leq y \leq 1\}$ as $g$. The inverse of this new function $f$ is the inverse sine function
\[
    \theta = f^{-1}(y) = \arcsin(y) = \sin^{-1}(y) .
\]

The function $\arcsin (y)$ gives the angle between $-\pi/2$ and $\pi/2$ whose sine is equal to $y$. The key to expressing its derivative
\[
  \frac{d\theta}{dy} = \frac{d}{dy}\left(  \arcsin y \right)
\]
in terms of $y$, is to express the derivative of the original function $y=f(\theta)$ in terms of its output $y=\sin\theta$.

We know that
\[
     \frac{dy}{d\theta} = \frac{d}{d\theta} \left( \sin\theta \right) = \cos\theta .
\]
To express this derivative in terms of $y=\sin\theta$, we use the fact that 
\[
 \cos^2\theta + \sin^2\theta = 1 .
\]
So 
\[
      \frac{dy}{d\theta} = \cos\theta = \pm \sqrt{1-\sin^2\theta} = \pm \sqrt{1-y^2}.
\]

To choose between $\pm$, we must remember that by the definition of the arcsine function (see equation (\ref{Eq:ArcSine})), 
\[
   \pi/2 \leq \theta \leq \pi/2 .
\]
This tells us that $\cos\theta \geq 0$. So we conclude that
\[
\frac{dy}{d\theta} = \cos\theta = \sqrt{1-y^2}.
\]
\end{example}

\end{document}