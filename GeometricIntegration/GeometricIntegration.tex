\documentclass{ximera}
\title{Geometric Integration}

\newcommand{\pskip}{\vskip 0.1 in}

\begin{document}
\begin{abstract}
Evaluating or approximating definite integrals geometrically.
\end{abstract}
\maketitle


\emph{Directions:} Do \emph{not} use the Fundamental Theorem of Calculus for these problems. Instead evaluate an integral geometrically or approximate an integral with a Riemann sum with $100$ equally-spaced intervals.

\begin{question} \label{QOOOREfbxx3er}
For each definite integral, draw a graph of the appropriate function and shade the region of integration. Then use geometry to evaluate the integral.

\begin{enumerate}

\item $\int_3^8 12 \, du$

\item $\int_8^3 12 \, dt$

\item $\int_a^b k \, dx$, where $a,b,k$ are constants

\item $\int_4^9 \frac{w}{2} \, dw$

\item $\int_a^b \frac{z}{k} \, dz$,where $a,b,k$ are constants

\item $\int_0^4 \sqrt{25-x^2}\, dx$

\item $\int_{-5}^5 \left( 2-\sqrt{25-x^2} \right)\, dx$

\end{enumerate} 
\end{question}

\begin{question}  \label{QPPlDlfe343421}
Make up a question in an applied scenario whose answer is the integral in Question 1(b) above and explain why the answer makes sense.
\end{question}

\begin{question} \label{EUERr3rDR}
The function 
\[
    r = f(t) \, , \, 0\leq t \leq  60 , 
\]
expresses a balloon's rate of ascent (measured in ft/min) in terms of the number of minutes past noon. Its graph is shown below.

\begin{onlineOnly}
    \begin{center}
\desmos{tgi5yiuzab}{450}{600}  
\end{center}
\end{onlineOnly}

\href{https://www.desmos.com/calculator/tgi5yiuzab}{152:Balloon Introduction}

\pskip

The balloon is $1900$ feet high at 12:10pm.

\begin{enumerate}
\item Write an expression for the balloon's height at 12:40pm.Then use the graph to approximate this height.

\item Write an expression for the balloon's height at 12:04pm.Then use the graph to approximate this height.

\item Find an expression for the function 
\[
         h=g(t) \, , \, 0\leq t \leq 60, 
\]
that gives the balloon's height at time $t$ minutes past noon.
\end{enumerate}
\end{question}




\begin{question} \label{EdfdhUERr3rDR}
The function 
\[
    r = f(t) \, , \, 0\leq t \leq  60 , 
\]
expresses a balloon's rate of ascent (measured in ft/min) in terms of the number of minutes past noon. Its graph is shown below.

\begin{onlineOnly}
    \begin{center}
\desmos{vfk9toeayo}{450}{600}  
\end{center}
\end{onlineOnly}

\href{https://www.desmos.com/calculator/vfk9toeayo}{152:Balloon 53}

\pskip

The balloon is $3200$ feet high at 12:40pm.

\begin{enumerate}
\item Ignore the graph and write an expression with a for the balloon's height at 12:30pm.Then use the graph to find this height.

\item Ignore the graph and write an expression for the balloon's height at 12:10pm.Then use the graph to find this height.

\item When is the balloon the same height as it is at 12:09pm?

\item When is the balloon $20$ feet higher than it is at 12:18pm?


\item Ignore the graph and find an expression for the function 
\[
         h=g(t) \, , \, 0\leq t \leq 60, 
\]
that gives the balloon's height at time $t$ minutes past noon. Then use the graph and find an explicit expression for this piece-wise defined function.
\end{enumerate}
\end{question}









\end{document}