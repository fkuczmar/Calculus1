\documentclass{ximera}
\title{The Tangent Function and its Inverse}

\newcommand{\pskip}{\vskip 0.1 in}

\begin{document}
\begin{abstract}
The tangent function and its inverse.
\end{abstract}
\maketitle


\section{The Tangent Function}

The most common way to find the derivative of the function $y=\tan(\theta)$ is to write 
\[
  \tan\theta = \frac{\sin\theta}{\cos \theta}
\]
and use the quotient rule. But this is not all that insightful, and we have not learned the quotient rule yet. Here are two other approaches.

\begin{question}  \label{QPDeredsRMNFR}
df
\end{question}




\begin{question}  \label{Qhfhghllllgg}
The bottom end of a seven-foot ladder slides across a horizontal floor as its top end slides down a vertical wall.

\begin{onlineOnly}
   \begin{center}
\desmos{4nmxshey0e}{900}{600}
\end{center}
\end{onlineOnly}

\href{https://www.desmos.com/calculator/4nmxshey0e}{151: Ladder and Tree 23}  

\end{question}


\end{document}
