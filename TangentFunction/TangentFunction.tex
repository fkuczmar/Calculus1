\documentclass{ximera}
\title{The Tangent Function and its Inverse}

\newcommand{\pskip}{\vskip 0.1 in}

\begin{document}
\begin{abstract}
The tangent function and its inverse.
\end{abstract}
\maketitle


\section{The Tangent Function}

The most common way to find the derivative of the function $y=\tan(\theta)$ is to write 
\[
  \tan\theta = \frac{\sin\theta}{\cos \theta}
\]
and use the quotient rule. But this is not all that insightful, and we have not learned the quotient rule yet. Here are two other approaches.

\begin{question}  \label{QPDeredsRMNFR}

We'll use the figure below to find an expression for the derivative
\[
   \frac{d}{d\theta} \left( \tan\theta \right)
\]
as follows.

\begin{onlineOnly}
   \begin{center}
\desmos{2la5tvxn56}{900}{600}
\end{center}
\end{onlineOnly}

\href{https://www.desmos.com/calculator/2la5tvxn56}{151: Tangent Derivative}  

\begin{enumerate}
\item First convince yourself that the marked angles $\theta = \angle BOA$ and $\angle QBC$ are congruent. Well almost, assuming $d\, \theta \sim 0$.

\item Express the length $OB$ in terms of $\theta$.

\item Express the length $QB$ in terms of $\theta$ and $\Delta \theta$.

\item Express the length $BC$ in terms of $\theta$ and $\Delta \theta$.

\item Then find an expression for 
\[
    dy/d\theta = \frac{d}{d\theta} \left(  \frac{\sin\theta}{\cos \theta}  \right) .
\]
\end{enumerate}


\end{question}






\end{document}
