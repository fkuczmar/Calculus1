\documentclass{ximera}
\title{Addition}

\newcommand{\pskip}{\vskip 0.1 in}

\begin{document}
\begin{abstract}
When do we need to add?
\end{abstract}
\maketitle


\begin{exercise}  \label{EXKDmFMERE}
The table below shows your speed at several times as you drive north on I-5. 

\begin{tabular}{|c|c|}
\hline
Time & Speed (miles/hour) \\
\hline
12:30 & 42 \\  \hline
12:40 & 50 \\ \hline
12:44 & 55 \\ \hline
12:50 & 40 \\
\hline
\end{tabular}


\begin{enumerate}
\item USe the data to give two approximations to the distance travelled between 12:30pm and 12:50pm. Explain your reasoning.

\item Letting
\end{enumerate}

\begin{onlineOnly}
    \begin{center}
\desmos{jjobuctzsa}{450}{600}  
\end{center}
\end{onlineOnly}

\href{https://www.desmos.com/calculator/jjobuctzsa}{152: Distance Travelled}



\end{exercise}



\begin{exercise}  \label{ExY6r3erMMdvb}

The table below shows your gas mileage on a car trip as you pass different exits while driving north on I-5. The exit numbers record the distance to the Columbia river along I-5, measured in miles.


\begin{tabular}{|c|c|}
\hline
Exit Number & Mileage (miles/gallon) \\
\hline
176 & 40 \\  \hline
200 & 32 \\ \hline
248 & 30 \\
\hline
\end{tabular}

\begin{enumerate}

\item Is this enough information to compute upper and lower bounds for the number of gallons of gas your car burned between exits 176 and 248? If so, explain why. If not, what other information do you need?

\item Using the given information and your additional hypothesis from part (a), if needed, find the best possible upper and lower and upper bounds for the number of gallons of your car burned between exits 176 and 248. 

Include two copies of a reasonable graph of the appropriate function (that you need to define) with your explanation. Label the coordinates of the key points. Sketch (and explain) the geometric representation (as an area) of your estimate for the lower bound on one graph, and a geometric representation (and explanation) of your estimate for the upper bound on the other. 

Explain your reasoning throughly in the form of a letter to a friend who has no knowledge of calculus or the ideas of Section 5.1. Include units in all steps of your computations.

\end{enumerate}
\end{exercise}


\begin{exercise}  \label{EXKjdDMFne3}

The table below shows several speeds of a roller coaster as it slides down a cicular loop-the-loop with a radius of 80 feet. The bottom of the loop is 20 feet above the ground. 

\begin{tabular}{|c|c|}
\hline
Height above ground & Speed (ft/sec) \\
\hline
100 & 40 \\  \hline
80 & 48 \\ \hline
50 & 64 \\ \hline
20 & 80 \\
\hline
\end{tabular}

\begin{enumerate}
\item Is this enough information to compute upper and lower bounds for the time it takes for the roller coaster to fall from a height of 100 feet to a height of 20 feet? If so, explain why. If not, what other information do you need? Do not assume any particular values for the speed at other heights.

\item Using the given information and your additional hypothesis from part (a) if needed, find the best possible upper and lower and upper bounds for the time it takes for the roller coaster to fall from a height of 100 feet to a height of 20 feet. Do not use a calculator for this part. But your bounds should be in a form that could be input directly into a calculator.

\item Use a calculator to find approximations to your bounds from part (b). Then give your best estimate of the time together with bounds for the error in your estimate.

\end{enumerate}

Explain your reasoning throughly in the form of a letter to a friend with no knowledge of calculus or the ideas of Section 5.1. Include units in all steps of your computations. Be careful. In particular, take note that the roller coaster moves along a circular arc. It does not fall straight down.

\end{exercise}

\begin{exercise}  \label{EX:JJndre344adf}
The table below shows your speed and gas mileage at several times during the course of a two-hour trip.

\begin{tabular}{|c|c|c|}
\hline
Time & Speed (miles/hr) & Mileage (miles/gal) \\
\hline
1:00pm & 40 & 30 \\  \hline
1:45pm & 48 & 36 \\  \hline
2:15pm & 60 & 40 \\ \hline
3:00pm & 70 & 28 \\
\hline
\end{tabular}

\begin{enumerate}

\item Give two estimates for the distance travelled during the trip. Explain your reasoning.

\item Give two estimates for the number of gallons of gas your car burned during the trip. Explain your reasoning.

\end{enumerate}
\end{exercise}

\begin{exercise}  \label{EX:MnVWeDF3x}
The table below shows the inclination angle of the Mt. Washington Cog Railway at several distances from Waumbek Station (the departure point, elevation 4000 feet) on its way to the summit (elevation 6288 feet).

\begin{tabular}{|c|c|}
\hline
Distance (miles) & Inclination Angle (radians) \\
\hline
0.5 & 0.25 \\  \hline
1.0 & 0.30 \\ \hline
1.5 & 0.32 \\ \hline
2.0 & 0.38  \\ \hline
%2.5 & 0.22 \\ \hline
\hline
\end{tabular}

Use the data to find two estimates for the elevation change from mile marker $0.5$ to mile marker $2.0$. Explain your reasoning.


\end{exercise}

\end{document}