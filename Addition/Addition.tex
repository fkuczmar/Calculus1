\documentclass{ximera}
\title{Addition}

\newcommand{\pskip}{\vskip 0.1 in}

\begin{document}
\begin{abstract}
When do we need to add?
\end{abstract}
\maketitle


\section{Time, Distance, and Mileage Problems}

\begin{exercise}  \label{EXKDmFMERE}
The table below shows your speed at several times as you drive north on I-5. 

\begin{tabular}{|c|c|}
\hline
Time & Speed (miles/hour) \\
\hline
12:30 & 42 \\  \hline
12:40 & 50 \\ \hline
12:44 & 55 \\ \hline
12:50 & 40 \\
\hline
\end{tabular}


\begin{enumerate}
\item Use the data to give two approximations to the distance travelled between 12:30pm and 12:50pm. Once assuming a constant speed over each subinterval equal to the speed at the start of the interval. And again assuming a constant speed over each subinterval equal to the speed at the end of the interval. Explain your reasoning.

\item Activate the folder in Line 5 of the worksheet below and explain how what you see is related to part (a). Then turn the folder off and activate the folder in Line 11. Explain. 

\item To practice summation notation, let $t_i$, $i=1,2,3,4$ be the times, in order, in the above table measured in hours past 12:30pm and let $v_i$ be the correponding speeds. Then using the speed at start of each subinterval , the distance travelled is approximately 
\[
   \sum_{i=1}^{\answer{4}}(t_{\answer{i+1}} - t_{\answer{i}})v_{\answer{i}}. 
\]
Using the speed at the end of each subinterval, the distance is approximately
\[
    \sum_{i=1}^{\answer{4}}(t_{\answer{i+1}} - t_{\answer{i}})v_{\answer{i+1}}.
\]

\item Input the sums from part (c) in Lines 16 and 18 of the worksheet below to check your arithmetic. You'll need to write $t[i]$ in place of $t_i$ and so on.
\end{enumerate}

\begin{onlineOnly}
    \begin{center}
\desmos{jjobuctzsa}{450}{600}  
\end{center}
\end{onlineOnly}

\href{https://www.desmos.com/calculator/jjobuctzsa}{152: Distance Travelled}
\end{exercise}



\begin{exercise}  \label{ExY6r3erMMdvb}

The table below shows your gas mileage on a car trip as you pass different exits while driving north on I-5. The exit numbers record the distance to the Columbia river along I-5, measured in miles.


\begin{tabular}{|c|c|}
\hline
Exit Number & Mileage (miles/gallon) \\
\hline
176 & 40 \\  \hline
200 & 32 \\ \hline
248 & 30 \\
\hline
\end{tabular}

\begin{enumerate}

\item Is this enough information to compute upper and lower bounds for the number of gallons of gas your car burned between Exits 176 and 248? If so, explain why. If not, what other information do you need?

\item Using the given information and your additional hypothesis from part (a), if needed, find the best possible upper and lower and upper bounds for the number of gallons of your car burned between Exits 176 and 248. 

\item Let $e_i$ and $g_i$, $i=1,2,3$ denote the respective exit numbers and gas mileages in the table above. Use summation notation to write your estimates in part (b) above.

The lower bound for the number of gallons of gas is
\[
  \sum_{i=1}^3 \left( e_{i+1} - e_{i}  \right)\left( \answer{\frac{1}{g_i}}  \right) .
\]
The upper bound is
\[
  \sum_{i=1}^3 \left( e_{i+1} - e_{i}  \right)\left( \answer{\frac{1}{g_{i+1}}}  \right) .
\]


\item Represent the sums in parts (a) and (b) geometrically with two copies of a of the appropriate function (that you need to define) with your explanation. Label the coordinates of the key points. Sketch (and explain) the geometric representation (as an area) of your estimate for the lower bound on one graph, and a geometric representation (and explanation) of your estimate for the upper bound on the other. 

\end{enumerate}
\end{exercise}


\begin{exercise}  \label{EX:JJndre344adf}
The table below shows your speed and gas mileage at several times during the course of a two-hour trip.

\begin{tabular}{|c|c|c|}
\hline
Time & Speed (miles/hr) & Mileage (miles/gal) \\
\hline
1:00pm & 40 & 30 \\  \hline
1:45pm & 48 & 36 \\  \hline
2:15pm & 60 & 40 \\ \hline
3:00pm & 70 & 28 \\
\hline
\end{tabular}

\begin{enumerate}

\item Give two estimates for the distance travelled during the trip. Explain your reasoning.

\item Give two estimates for the number of gallons of gas your car burned during the trip. Explain your reasoning.

\end{enumerate}
\end{exercise}


\section{On Trails}

\begin{exercise}  \label{EX:MnVWeDF3x}
The table below shows the inclination angle of the Mt. Washington Cog Railway at several distances from Waumbek Station (the departure point, elevation 4000 feet) on its way to the summit (elevation 6288 feet).

\begin{tabular}{|c|c|}
\hline
Distance (miles) & Inclination Angle (radians) \\
\hline
0.5 & 0.25 \\  \hline
1.0 & 0.30 \\ \hline
1.5 & 0.32 \\ \hline
2.0 & 0.38  \\ \hline
%2.5 & 0.22 \\ \hline
\hline
\end{tabular}

Use the data to find two estimates for the elevation change from mile marker $0.5$ to mile marker $2.0$. Explain your reasoning.
\end{exercise}



\begin{exercise}  \label{EX:MnV443fgDF3x}
The table below shows the slope along a section of a trail at several altitudes.


\begin{tabular}{|c|c|}
\hline
Altitude (feet) & Slope (ft/ft) \\
\hline
1000 & 0.05 \\  \hline
1200 & 0.10 \\ \hline
1350 & 0.12 \\ \hline
1500 & 0.15  \\ \hline
\hline
\end{tabular}


\begin{enumerate}
\item Use the data to find two estimates for the length of the trail between altitudes $1000$ ft and $1500$ feet. Explain your reasoning.

\item Let $h_i$ and $m_i$, $i=1,2,3,4$ denote the respective altitudes and slopes in the table above. Use summation notation to write your estimates in part (b) above.

Assuming the steepness of the trail increases between altitudes $1000$ feet and $1500$ feet, a lower bound for the trail's length is
\[
    s_1 = \sum_{i=1}^4 (h_{i+1}-h_{i}) \left( \answer{\sqrt{1+1/m_{i+1}^2}} \right)  .
\]
An upper bound is
\[
    s_1 = \sum_{i=1}^4 (h_{i+1}-h_{i}) \left( \answer{\sqrt{1+1/m_{i}^2}} \right)  .
\]

\item Input your expressions for the lower and upper bounds in the worksheet below.

\begin{onlineOnly}
    \begin{center}
\desmos{mmi4rx2rqp}{450}{600}  
\end{center}
\end{onlineOnly}

\href{https://www.desmos.com/calculator/mmi4rx2rqp}{152: Length of Trail}

 

\end{enumerate}

\end{exercise}









\section{A Rollercoaster Problem}


\begin{exercise}  \label{EXKjdDMFne3}

The table below shows several speeds of a roller coaster as it slides down a cicular loop-the-loop with a radius of 80 feet. The bottom of the loop is 20 feet above the ground. 

\begin{tabular}{|c|c|}
\hline
Height above ground & Speed (ft/sec) \\
\hline
100 & 40 \\  \hline
80 & 48 \\ \hline
50 & 64 \\ \hline
20 & 80 \\
\hline
\end{tabular}

\begin{enumerate}
\item Is this enough information to compute upper and lower bounds for the time it takes for the roller coaster to fall from a height of 100 feet to a height of 20 feet? If so, explain why. If not, what other information do you need? Do not assume any particular values for the speed at other heights.

\item Using the given information and your additional hypothesis from part (a) if needed, find the best possible upper and lower and upper bounds for the time it takes for the roller coaster to fall from a height of 100 feet to a height of 20 feet. Do not use a calculator for this part. But your bounds should be in a form that could be input directly into a calculator.

\item Use a calculator to find approximations to your bounds from part (b). Then give your best estimate of the time together with bounds for the error in your estimate.

\end{enumerate}

Explain your reasoning throughly in the form of a letter to a friend with no knowledge of calculus or the ideas of Section 5.1. Include units in all steps of your computations. Be careful. In particular, take note that the roller coaster moves along a circular arc. It does not fall straight down.

\end{exercise}




\begin{exercise} \label{EXIIU4349se9}
Assume for this problem that each month has 30 days and that the number of hours of daylight/day in Seattle throughout the course of a year is a sinusoidal function of time. Suppose also that Seattle receives a maximum of $16$ hours of daylight/day on June 21 and a minimum of $8$ hours of daylight/day on December 21.

\begin{enumerate}
\item Find an expression for a function
\[
        h = f(t) \, , \, 0\leq t \leq 360 ,
\]
that expresses the number of hours of daylight/day in terms of the number of days since June 21. Use the \emph{cosine} function.

The function is
\[
    h = f(t) = \answer{12} + \answer{4}\cos \left(  \answer{\frac{\pi}{180}t} \right) .
\]

\item Use summation notation to write an expression for the total number of hours of daylight Seattle gets during the summer. 

The total number of hours of daylight during the summer is
\[
    \sum_{i=0}^{\answer{89}} \answer{f(i)} .
\]

\item Enter the sum in Line 1 of the worksheet below.

\begin{onlineOnly}
    \begin{center}
\desmos{jvwyqn0vbj}{450}{600}  
\end{center}
\end{onlineOnly}

\href{https://www.desmos.com/calculator/jvwyqn0vbj}{152: Hour of Daylight 1}

\item Compute the average number of hours of daylight/day in Seattle during the summer?

\item Is the number of hours of daylight/day in Seattle increasing or decreasing on March 21st? At what rate?

\item Does the previous question make sense? Explain.


\end{enumerate}


\end{exercise}

\end{document}