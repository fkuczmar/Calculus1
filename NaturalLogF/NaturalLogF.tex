\documentclass{ximera}
\title{The Natural Log Function}

\newcommand{\pskip}{\vskip 0.1 in}

\begin{document}
\begin{abstract}
Log functions and their derivatives.
\end{abstract}
\maketitle


\section{The Derivative of the Natural Log Function}
\begin{question} \label{Qdfggghghhfll}

\begin{enumerate}
\item Explain the meaning of 
\[
  \ln 5 = \log_e 5 .
\]

\item Simplify the function
\[
     f(x) = e^{\ln x} .
\]
Include the appropriate domain.

\item Use the result of part (b) to find an expression for the derivative
\[
 \frac{d}{dx} \left(  \ln x \right ) .
\]
\end{enumerate}

%\begin{expandable}
\begin{explanation}
To compute the derivative in part (c), we know that since the functions $f(x)=\ln x$ and $g(x)=e^x$ are inverses of one another,
\[
    e^{\ln x} = x .
\]
Then differentiate both sides of this equation with respect to $x$ to get
\[
        \frac{d}{dx} \left(  e^{\ln x} \right) = \frac{dx}{dx} .
\]
And by the chain rule we can rewrite this equation as
\[
         \left(  e^{\ln x} \right)   \frac{\answer{d}}{dx} \left(  \answer{\ln x} \right) = \answer{1} . 
\]
And since $e^{\ln x} = \answer{x}$, 
\[
         \frac{d}{dx} \left(  \ln x  \right) = \answer{1/x}.
\]
%\end{expandable}
\end{explanation}

\end{question}

\begin{question} \label{Qhhghgfgfdgghhfll}
Find simplified expressions for each of the following derivatives. Include an appropriate domain for each.

\begin{enumerate}
\item 
\[
   \frac{d}{dx} \left( \ln \left( x/2 \right) \right)
\]

\item 
\[
\frac{d}{dx} \left( \ln \left( x^2 \right) \right)
\]
\end{enumerate}

\end{question}

\begin{question} \label{Q89fghvbnm}
Find an equation of the tangent line to the curve
\[
    y = 3\ln \left( \frac{x^2+1}{x^3+1} \right)
\]
at the point on the curve with $x$-coordinate $x=1$.
\end{question}


\begin{question}  \label{Qggfdghbfrgbhrdtgr}
\begin{enumerate}
\item Graph the function 
\[
    y = f(x) = \ln |x|
\]
by hand.

\item Use the graph above to sketch (by hand) a graph of the derivative
\[
  y = \frac{d}{dx} \left( \ln |x|  \right) .
\]

\item Use the chain rule to compute the derivative
\[
   \frac{d}{dx} \left( \ln |x| \right) .
\]
Do this by noting that
\[
|x| = 
\begin{cases}
    x , \text{ if } x\geq 0 \\
   -x ,  \text{ if } x<0
\end{cases}
\]
and using the chain rule to compute 
\[
  \frac{d}{dx}  \left(  -x \right).
\]

\item A reflection about the $y$-axis takes the graph of a function $y=f(x)$ to the graph of a function $y=g(x)$. Describe the transformation that takes the graph of $y=f^\prime(x)$ to the graph of $y=g^\prime(x)$. Explain your reasoning.

\end{enumerate}

\end{question}

\begin{question} \label{Qhjklghvbnm}
Find an equation of the tangent line to the curve
\[
    y = 3\ln \Big| \frac{x^2+1}{x^3-1} \Big|
\]
at the point on the curve with $x$-coordinate $x=-1$.
\end{question}

\begin{question}  \label{Q435rtrtgfgfg}
Let ${\cal L}$ be the tangent line to the curve
\[
    y = \ln \left| \sec \theta  \right| \, , \, -\pi/2 < \theta < \pi/2 ,
\]
at the point on the curve with $\theta=\pi/7$.

Find the acute angle ${\cal L}$ makes with the $x$-axis.
\end{question}


\section{Relative Rates of Change}

\begin{question}  \label{Qghlvmnvngng}
The function $P=f(t)$, $0\leq t \leq 12$, expresses the population of a colony of bacteria in terms of the number of hours past noon.

Suppose that
\[
   \frac{d}{dt} \left(  \ln P \right)\Big|_{t=4} = \frac{d}{dt} \left(  \ln (f(t)) \right)\Big|_{t=4}  = \frac{3}{10} .
\]

\begin{enumerate}
\item What are the units of this derivative?

\item Explain the meaning of the derivative with and without the language of small changes.
\end{enumerate}
\end{question}

\begin{question} \label{Q34vcvnhh}
At a certain instant the population of a country is decreasing at the rate of $3\%$/yr. At this same instant the country's national debt is increasing at the rate of $5\%$/yr.

\begin{enumerate}
\item At what relative rate is the per capita (ie. per person) share of the national debt changing at this time? \emph{Hint:} Use the idea of the previous question.

\item If the per capita share of the national debt is $\$200,000$/person at this time, at what (absolute) rate is the per-capita share of the debt changing at this instant?

\end{enumerate}
\end{question}


\begin{question}  \label{Qddfdsf4thmm}
The function 
\[ 
      G = f(v) = 40-0.08\left(\frac{v}{2}-25\right)^{2}\, \ , \, 25\leq v\leq 65 ,
\]
expresses the gas mileage of a car (in miles/gallon) in terms of its speed (in miles/hr).

(a) Explain the meaning of the derivative 
\[
     \frac{d}{dv} \left( \ln (f(v)) \right) .
\]
Include units in your explanation. Also, what are the units of $25$ in the above expression? How do you know?

(b) Evaluate the above derivative at $v=30$ and explain its meaning in terms of small changes.

\end{question}




\end{document}
