\documentclass{ximera}
\title{The Rythm of a Function}

\newcommand{\pskip}{\vskip 0.1 in}

\begin{document}
\begin{abstract}
Listening to the beat of a function.
\end{abstract}
\maketitle

Have you ever sat by the train tracks listening to the trains roll by? If so, and you're near a joint in a rather old track, you'll hear a regular beat of the train as it passes by. The video below (especially recommended for engineers), explains how modern technonlogy gets around these types of joints.

\href{https://practical.engineering/blog/2023/12/5/why-railroads-dont-need-expansion-joints}{Expansion Joints}



\begin{exploration}  \label{EP9ppdfgdggt}

But we can use this idea, the old-fashioned clickity-clack rythm of a train, to listen to the beat of a function.

We'll imagine shrinking our train to a point that moves upward along a straight track (the $s$-axis in the demonstration below). And we'll start listening to our train at time $t=0$ seconds as it passes the point $s=0$. For the demonstration below, the joints are spaced just $0.1$ meters apart.

\begin{onlineOnly}
    \begin{center}
\desmos{yghwlfxyjo}{450}{600}  %qvk0mzy26u
\end{center}
\end{onlineOnly}

\href{https://www.desmos.com/calculator/yghwlfxyjo}{151: Sound Squaring Function 3}

Follow the directions in the exploration above and then do the following.

\begin{enumerate}
\item Describe how the train's speed varies.

\item Sketch a rough graph of the function $v=f(t)$, $0\leq t \leq 1$, that expresses the train's speed (in meters/sec) as fuction of the number of seconds since the train passed the point $y=0$. Label the axes with the correct variable names and units.

\item Use your graph from part (b) to sketch a graph of the function $s=g(t)$, that expresses the train's distance (in meters) from the point $s=0$ as a function of time (measured in seconds as in part (b)).

\item Finally, use your graphs from part (b) and (c) to sketch a graph of the function $v=h(s)$, that expresses the train's speed (in meters/sec) in terms of its distance (in meters) from $s=0$.  
\end{enumerate}


\end{exploration}



\end{document}