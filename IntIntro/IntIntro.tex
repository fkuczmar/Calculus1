\documentclass{ximera}
\title{Introduction to Integration}

\newcommand{\pskip}{\vskip 0.1 in}

\begin{document}
\begin{abstract}
Introduction to integral calculus.
\end{abstract}
\maketitle

The \emph{differential} in Differential Calculus shares the same root as \emph{difference}, and \emph{calculus} the same root as \emph{compute}. Differential calculus is all about \emph{computing differences}. Expressed more simply, it is about \emph{subtraction}.

%On the other hand, \emph{Integral Calculus} is about addition. \emph{Integral} shares the same root as \emph{integer} and comes from the Latin \emph{integratus}, the past participle of \emph{integrare}, to make whole. 

On the other hand, \emph{Integral Calculus} is about addition. \emph{Integral} shares the same root as \emph{integer} and comes from the Latin \emph{integrare}, to make whole. 

The following example illustrates the main idea of differential calculus.

\begin{example}  \label{Ex:98dfrghha}

The function 
\[
    r = f(t) \, , \, 0\leq t \leq  60 , 
\]
expresses a balloon's rate of ascent (measured in ft/min) in terms of the number of minutes past noon. Its graph is shown below.

\begin{onlineOnly}
    \begin{center}
\desmos{tgi5yiuzab}{450}{600}  
\end{center}
\end{onlineOnly}

\href{https://www.desmos.com/calculator/tgi5yiuzab}{152:Balloon Introduction}

Suppose the balloon is $2500$ feet above the ground at 12:34pm. 

\begin{enumerate}

\item Use the graph (and nothing else) to approximate the balloon's height $30$ seconds later. Explain your reasoning.

\item Is your estimate greater or less than the actual height? Explain your reasoning.

\end{enumerate}
\end{example}

We can turn the previous question into an addition problem of integral calculus by trying to estimate the change in the balloon's height over a longer time interval.

\begin{example} \label{ExLKDrDEfRE9}

\begin{onlineOnly}
    \begin{center}
\desmos{h6cworakdw}{450}{600}  
\end{center}
\end{onlineOnly}

\href{https://www.desmos.com/calculator/h6cworakdw}{152:Balloon Introduction 2}

\end{example} 



\end{document}