\documentclass{ximera}
\title{Introduction to Integration, Part 2}

\newcommand{\pskip}{\vskip 0.1 in}

\begin{document}
\begin{abstract}
Introduction to numerical integration.
\end{abstract}
\maketitle

This chapter is a continuation of the introduction where we saw how to approximate a balloon's change in height over an interval of time with a sum of small approximate changes. We develop that theme here with summation notation and take advantage of  technology to get accurate approximations.

But first a question for review.

\section{Review}

\begin{example}  \label{Ex:IUDFr3f3fgl}
Between 12:14pm and 1:00pm the temperature of a beaker of water increases at a constant rate of $\frac{33}{13}^\circ$C/sec. The temperature is $23^\circ$C at 12:23pm.

\begin{enumerate}
\item Find an expression for the function 
\[
       C=f(t)\; , \; 14\leq t \leq 60, 
\]
that expresses the temperature of the water (in Celsius degrees) in terms of the number of minutes past noon.

\item Explain the logic behind your expression.

\end{enumerate}

\begin{freeResponse}
\end{freeResponse}
\end{example}

\section{Summation Notation and Technology}
Back to the problem from the introduction. But first a few conceptual questions.

\begin{example}  \label{Ex:98dfrghha}

The function 
\[
    r = f(t) \, , \, 0\leq t \leq  60 , 
\]
expresses a balloon's rate of ascent (measured in ft/min) in terms of the number of minutes past noon. Its graph is shown below.

\begin{onlineOnly}
    \begin{center}
\desmos{tgi5yiuzab}{450}{600}  
\end{center}
\end{onlineOnly}

\href{https://www.desmos.com/calculator/tgi5yiuzab}{152:Balloon Introduction}

\begin{enumerate}

\item When is the balloon ascending? Descending? Give all times for each and explain your reasoning.

\item Do you think the balloon is higher at noon or at 1:00pm? Explain your reasoning. Try to prove your assertion using only arithmetic and simple reasoning.

\begin{freeResponse}
\end{freeResponse}

\end{enumerate}


Now suppose the balloon is $1900$ feet above the ground at 12:10pm. 

\begin{enumerate}

\item Use the graph to approximate the balloon's height $30$ seconds later. Explain your reasoning.

\item Is your estimate greater or less than the actual height? Explain your reasoning.
\end{enumerate}

\begin{freeResponse}
\end{freeResponse}

\end{example}

We can turn the previous question of differential calculus into an addition problem of integral calculus by estimating the balloon's height at a time farther removed from 12:10. 

\begin{example} \label{ExLKDrDEfRE9}

We'll assume as before that the balloon is $1900$ feet above the ground at 12:10pm, and we'll approximate its height at 12:50pm.

\begin{onlineOnly}
    \begin{center}
\desmos{h6cworakdw}{450}{600}  
\end{center}
\end{onlineOnly}

\href{https://www.desmos.com/calculator/h6cworakdw}{152:Balloon Introduction 2}

We'll start by computing a rough approximation of the change in height between 12:10pm and 12:50pm. For this, we'll partition the $40$-minute time interval into five equal subintervals and suppose that during each subinterval the balloon ascends at a constant rate. 

What rate we choose for each interval doesn't really matter as long as we choose the actual rate of ascent at some time during that interval. To make things simple, we'll choose the rate at the start of each subinterval as illustrated in the worksheet above. 

\begin{enumerate}

\item Use summation notation to express the approximate change in height from 12:10pm to 12:50pm. Type this expression in Line 5 of the worksheet above. Compare this value with your estimate from part (a).

Using the rate of ascent at the start of each $8$-minute time interval, the approximate change in height from 12:10pm to 12:50pm is
\[
   \sum_{i=0}^{\answer{4}}8 f(\answer{10 + 8i})  
\]
and the approximate height at 12:50pm is
\[
     \answer{1900} + \sum_{i=0}^{\answer{4}}8 f(\answer{10 + 8i})  .
\]

\item Repeat part (a) using the rate of ascent at the \emph{end} of each subinterval instead. Turn off the folder in Line 11 and activate the folder in Line 18 above. As a check, type your expression for the approximate height in Line 5 above.

The approximate change in height from 12:10pm to 12:50pm is
\[
   \sum_{i=1}^{\answer{5}}8 f(\answer{10 + 8i})  
\]
and the approximate height at 12:50pm is
\[
     \answer{1900} + \sum_{i=1}^{\answer{5}} 8 f(\answer{10 + 8i})  .
\]

\item Repeat part (a)using the rate of ascent at the \emph{midpoint} of each subinterval instead. Turn off the folder in Line 18 and activate the folder in Line 24 above. As a check, type your expression for the approximate height in Line 5 above.

The approximate change in height from 12:10pm to 12:50pm is
\[
  \sum_{i=0}^{\answer{4}}8 f(\answer{14+ 8i})  
\]
and the approximate height at 12:50pm is
\[
     \answer{1900} + \sum_{i=0}^{\answer{4}}8 f(\answer{14+ 8i})  .
\]

\item Compare your three estimates for the change in height. Which do you think is most accurate? Explain why.

\end{enumerate}
\end{example} 

\begin{example}  \label{Ex:IjdRJrehreDF}
\begin{enumerate}
\item  Drag the slider in Line 4 of Example 3 to $n=20$ and repeat parts (a) - (c) of Example 4 with $n=20$ equal subintervals. 

\item Try parts (c) - (e) of Example 3 in general, with $n$-equal subintervals.

Using left endpoints, the approximate height at 12:50pm is
\[
        f(50) \sim   \answer{1900} +  \answer{\frac{40}{n}} \sum_{i=0}^{\answer{n-1}} f\left(\answer{10+ \frac{40}{n}i}\right)  .
\]
With right endpoints,
\[
        f(50) \sim   \answer{1900} +  \answer{\frac{40}{n}} \sum_{i=1}^{\answer{n}} f\left(\answer{10+ \frac{40}{n}i}\right)  .
\]
And with midpoints,
\[
        f(50) \sim   \answer{1900} +  \answer{\frac{40}{n}} \sum_{i=0}^{\answer{n-1}} f\left(\answer{10+\frac{20}{n} + \frac{40}{n}i}\right)  .
\]

\end{enumerate}
\end{example}


\end{document}