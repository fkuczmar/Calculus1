\documentclass{ximera}
\title{Stretching Factors}

\newcommand{\pskip}{\vskip 0.1 in}

\begin{document}
\begin{abstract}
Interpreting the derivative as a local stretching/scaling factor.
\end{abstract}
\maketitle

\section{Rubber Band Calculus}

Think of a function and the first thing that might come to mind is its graph. But we can also think of a function's domain as an elastic band and the function as acting on that band by stretching or compressing it. %For example, the function $f(x)= 2x$ stretches the band about the origin by a factor of two. As illustrated in Figure1, the stretching transforms a set of uniformly spaced marks to a set of marks spaced twice as far apart. Reversing the arrows gives a representation of the inverse function

For example, imagine a thin elastic band of length $4$ meters running along the horizontal $L$-axis from $L=0$ meters to $L=4$ meters.  Now hold the left end fixed at $L=0$ and stretch the band by moving right end an additional four meters to the right . Then the function
\[
      H = g(L) = 2L \, , \, 0\leq L \leq 4 ,
\]
describes this stretching action. It takes as an input the distance (in meters) of a point on the band from the origin ($L=0$) and returns as an output the distance between the origin and the corresponding point on the stretched band. The exploration below shows this stretching action.

\begin{exploration} \label{Ex:98f3rgafgbb}
Drag the slider $k$ in Line 2 below to illustrate the stretching action.

\begin{onlineOnly}
    \begin{center}
\desmos{qejivz36ui}{450}{600}  %qvk0mzy26u
\end{center}
\end{onlineOnly}

\href{https://www.desmos.com/calculator/qejivz36ui}{151: Rubber Band 1}

\end{exploration}

The \emph{global stretching factor} for a linear function like the one above, %The average rate of change of the function,
\[
      H = g(L) = 2L \, , \, 0\leq L \leq 4 ,
\]
is the slope of its graph. We can calculate this factor as an average stretching factor (ie. an average rate of change) between the points $L=a$  meters and $L=b$ meters from the origin. For the function $H=f(L) = 2L$, the stretching factor is
\begin{align*}
  \frac{\Delta H}{\Delta L} &= \frac{f(b) - f(a)}{b-a} \\
                                       &= \frac{\answer{2(b-a)}}{b-a} \\
                                       &= \answer{2},
\end{align*}
This tells us that any two points on the stretched band are twice as far apart as they were on the unstretched band.

\begin{question}  \label{Q:LDJJNMDesd}
\begin{enumerate}
\item What are the units of the stretching factor?
\begin{freeResponse}
\end{freeResponse}

\item Find an expression for the inverse function
\[
      L = f^{-1}(H).
\]
Include a domain. 

\item Interpret the inverse function as a deformation of a thin elastic band. What is the global stretching factor for this function?
\begin{freeResponse}
\end{freeResponse}
\end{enumerate}
\end{question}



\section{Non-Linear Stretching Functions}

\begin{example} \label{Ex:JDJFHDtet434t}
Here's an example
\[
      H = f(L) = 10 - \sqrt{100-L^2} \, , \, 0\leq L \leq 10,
\] 
of a non-linear stretching function (where $L$ and $H$ are measured in meters as before). Like most functions in this class, it acts like a linear function near most points in its domain. 

To stretch the elastic band in the demonstration below, drag the slider $u$ in Line 2 from $u=0$ to $u=1$. Then zoom in close enough to the point $H=f(2)$ in the stretched band (highlighted in black) to make the stretching function look linear. %the band looks like it's stretched by a constant factor. 


\begin{onlineOnly}
    \begin{center}
\desmos{nyd60dbezj}{450}{600}  %qvk0mzy26u
\end{center}
\end{onlineOnly}

\href{https://www.desmos.com/calculator/nyd60dbezj}{151: Rubber Band Ladder 5}
%\end{exploration}


\begin{enumerate}

\item Use the close-up view of the stretched band to approximate the stretching factor at the input $L=2$.

\item Drag the slider $m$ in Line 4 to $m=95$ and repeat part (a) to approximate the stretching factor at the input $L=9.5$ meters.

\item Parts of the elastic band get stretched, others compressed. Identify these.
\end{enumerate}

We can also approximate the stretching factors at different inputs by zooming in near enough to the graph of the function $H=f(L)$ (shown below) to make the graph look like a line. %To see the graph of the function, zoom back out and then activate the folder in Line 22 of the worksheet above.

\begin{onlineOnly}
    \begin{center}
\desmos{jfwvbbrts1}{450}{600}  %qvk0mzy26u
\end{center}
\end{onlineOnly}

\href{https://www.desmos.com/calculator/jfwvbbrts1}{151: Rubber Band Ladder 6}

\begin{enumerate}
\item Zoom in close enough to the point $P(2,f(2))$ to make the graph above look like a straight line. Then click on the graph to get the coordinates of $P$ and a point close to $P$. Use these to approximate the stretching factor at $L=2$.

%\item Drag the slider $m$ in Line 4 to $m=95$ and repeat part (a) to approximate the stretching factor at $L=9.5$.

\item Zoom back out and use the graph of the function $H=f(L)$ to determine the stretching factors at $L=0$ and $L=10$.
\end{enumerate}

We can also approximate the stretching factors numerically using the expression
\[
      H = f(L) = 10- \sqrt{100-L^2} \, , \, 0\leq L \leq 10,
\] 
for the stretching function.

For example, at the input $L=2$, we'll to first compute the \emph{average stretching factor} over the interval between lengths $L=2$ and $L=v$. This average factor is 
\begin{align*}
 k(v) &= \frac{\Delta H}{\Delta L} \\
              &= \frac{f(v) - f(2)}{v-2} \\
              &=\frac{10 - \sqrt{100-v^2} - (10 - \sqrt{\answer{96}})}{v-2}   \\
              &= \frac{\sqrt{100-v^2} - \answer{\sqrt{96}}}{v-2} \, , \, v \neq 2.
\end{align*}

\begin{enumerate}
\item The idea to approximate the stretching factor at $L=2$ is to see if these average stretching factors appear to approach some number as $v\to 2$ (ie. as $v$ approaches $2$). To do this we'll use the Table in Line 3 of the worksheet above. But first you'll need to input the correct expression for the average stretching function in Line 2. Then use the table to approximate the stretching factor at the input $L=2$.

\item Repeat part (a) to approximate the stretching factor at the input $L=9.5$ and compare this with your earlier estimate.

\item Repeat part (a) to approximate the stretching factor at the right end where $L=10$. Anything different here?

\item Use the geometry of the graph of the function $H=f(L)$ to compute the \emph{exact} stretching factors at $L=2, 9.5, 10$.
\end{enumerate}
 
\end{example}



\begin{example} \label{ExLdfdfthyhhhf}
Here's another example
\[
      H = f(L) = 4 - 4\cos (L/2) \, , \, 0\leq L \leq 2\pi.
\]
of a non-linear stretching function. We still need to talk about a \emph{local} stretching factor, one that varies from point to point. 

\begin{exploration} \label{ExLdfdfthyhhhf}
Drag the slider $u$ in Line 2 below from $u=0$ to $u=1$ to illustrate the stretching action.
\begin{onlineOnly}
    \begin{center}
\desmos{hqvyhormhf}{450}{600}  %qvk0mzy26u
\end{center}
\end{onlineOnly}

\href{https://www.desmos.com/calculator/hqvyhormhf}{151: Rubber Band Cosine}
\end{exploration}

%\begin{question} \label{Q:9887dfsdfdsf}
\begin{enumerate}
\item Use the worksheet above to approximate the local stretching factor at the point $L = \pi \sim 3.1$ meters from the origin on the unstretched band. Explain your reasoning.

\item Drag slider $m$ in Line 4 above to approximate the input(s) at which the local stretching factor is equal to 1. (ie. where the stretched band is neither in tension or in compression). Explain your reasoning.

%\end{question}


\item Use the graph of the stretching function below to approximate the local stretching factor at the input $L=\pi$.  Start by 
dragging the slider $u$ in Line 2 below from $u=0$ to $u=1$ to illustrate the stretching action and change the graph from $H=L$ to $H=f(L)$. Then zoom in close to the point $P(\pi,4)$.

\begin{onlineOnly}
    \begin{center}
\desmos{aczfty35qj}{450}{600}  %qvk0mzy26u
\end{center}
\end{onlineOnly}

\href{https://www.desmos.com/calculator/aczfty35qj}{151: Rubber Band Cosine 3}

\item Finally, approximate the local stretching factor at $L=\pi$ numerically by finding entering an expression for the average stretching factor $k(v)$ between inputs $L=2$ and $L=v$ in Line 3 of the worksheet above. Then use the table in Line 4.

\end{enumerate}

\end{example}

%\section{The Sliding Ladder}

%\begin{exploration} \label{Ex:LKDMNDFDSAdsa}
%Here's an example of a decreasing stretching function.

%Drag the slider $u$ in Line 2 below from $u=0$ to $u=1$ to illustrate the stretching action and change the graph from $H=10-L$ to $H=f(L)$.
%\begin{onlineOnly}
%    \begin{center}
%\desmos{04kfcfukfy}{450}{600}  %qvk0mzy26u
%\end{center}
%\end{onlineOnly}

%\href{https://www.desmos.com/calculator/04kfcfukfy}{151: Rubber Band Cosine 2}


%\end{exploration}

\section{A Hanging Slinky}
\begin{example} \label{Ex:MDH4955gfg}
A hanging slinky stretched under its own weight gives a example of a stretching function. 


\begin{onlineOnly}
    \begin{center}
\desmos{zqjjgael5j}{450}{600}  %qvk0mzy26u
\end{center}
\end{onlineOnly}

\href{https://www.desmos.com/calculator/zqjjgael5j}{151: Slinky Photo}

We can model the stretching of an ideal spring stretched under its own weight with the function
\[
 H = f(L) = L + \frac{g\rho}{2kL_0}L^2 \, , \, 0\leq L \leq L_0 ,
\]
where
\begin{itemize}
\item $L_0$ is the length (in meters) of the relaxed (unstretched) spring, 

\item $g$ is the magnitude of the gravitational acceleration (measured in meters/sec/sec),

\item $\rho$ is the linear density (in kg/meter) of the spring,

\item $k$ (measured in Netwons/meter) is the spring constant,

\item $L$ is the distance of a point on the relaxed spring from the spring's bottom end, and

\item $H=f(L)$ is the distance from the corresponding point on the stretched spring to the spring's bottom end.

\end{itemize}

\begin{freeResponse}
Use the above information to check that the above expression for $H=f(L)$ has the correct units. You will need to know that Newtons (a measure of force) have units $\text{kg}\cdot \text{m/sec}^2$. 
\end{freeResponse}



Drag the slider $g$ in Line 5 of the worksheet below from $g=0$ to $g=g_0 = 0.5 \text{m/sec}^2$ to turn on the gravitational field and stretch the slinky. 

\begin{onlineOnly}
    \begin{center}
\desmos{vjjibjkdrz}{450}{600}  %qvk0mzy26u
\end{center}
\end{onlineOnly}

\href{https://www.desmos.com/calculator/vjjibjkdrz}{151: Slinky 3}

We'll now work with the stretching function in the workhseet above,
\[
      H = f(L) = L + \frac{1}{4}L^2 \, , \, 0\leq L \leq 4.
\]

\begin{enumerate}
\item Use the pictures of the springs at the left of the graph to estimate the local stretching factor
\[
  \frac{dH}{dL}\Big|_{L=3}
\]
at the point in the relaxed spring $L=3$ meters from its bottom end.

\item Find an expression for the average stretching factor between inputs $L=3$ and $L=v$ meters. This factor is
\begin{align*}
  a= m(v) &= \frac{\Delta H}{\Delta L} \\
              &= \frac{f(v) - f(\answer{3})}{\answer{v-3}} \\
              &=\frac{v^2-9}{\answer{4(v-3)}}  
\end{align*}
             
\item Enter this expression in Line 1 of the worksheet below. Then
\begin{enumerate}
\item Use the table in Line 2 to guess the exact value of the local stretching factor at $L=3$. Compare this with your estimate.

\item Drag the slider $v$ in Line 11 and use the output in Line 12 ($m(v))$ to guess the exact value of the stretching factor at $L=3$.

\item Describe how the line $PQ$ is related to the average stretching factor $a=m(v)$. 
\end{enumerate}

\begin{onlineOnly}
    \begin{center}
\desmos{zsbupxubm6}{450}{600}  %qvk0mzy26u
\end{center}
\end{onlineOnly}

\href{https://www.desmos.com/calculator/zsbupxubm6}{151: Slinky 4}


\item Now we'll compute the local stretching factor at $L=3$ algebraically. This factor is the derivative
\[
  \frac{dH}{dL}\Big|_{L=3} 
\]
of $H$ with respect to $L$ evaluated at $L=3$. It is equal to the limit, as $v$ approaches $3$, of the average stretching factor $m(v)$ between lengths $L=3$ and $L=v$. Here's the computation with some algebra left for you near the end:

\begin{align*}
  \frac{dH}{dL}\Big|_{L=3} &= \lim_{\Delta L \to 0} \frac{\Delta H}{\Delta L} \\
              &= \lim_{v\to 3}\frac{f(v) - f(3)}{v-3} \\
              &= \lim_{v\to 3}\frac{v^2-9}{4(v-3)}   \\
              &= \lim_{v\to 3} \frac{(\answer{v-3})(v+3)}{4(v-3)} \\
              &=  \lim_{v\to 3} \frac{\answer{v+3}}{4} \\
              &= \frac{\answer{3}+3}{4}  \\
              &= \answer{1.5} .
\end{align*}

\item So our conclusion is that the local stretching factor of the stretching function
\[
      H = f(L) = L + \frac{1}{4}L^2 \, , \, 0\leq L \leq 4,
\]
at $L=3$ is equal to $1.5$ meters/meter. This is a dimensionless scaling factor. It means that a small interval of length $\Delta L\sim 0$ meters around the point $L=3$ on the relaxed spring gets stretched to a small interval approximately $1.5$ times as long. Symbolically, we can write this as
\[
          \Delta H \sim 1.5\Delta L,
\]
where
\[
   \Delta H = f(L) - f(3)
\]
and 
\[
    \Delta L = L - 3 \sim 0 .
\]

\item What about the inverse function $L=f^{-1}(H)$ that relaxes the stretched band? What is the local scaling factor
\[
   \frac{dL}{dH}\Big|_{L=3}
\]
of the inverse at $L=3$? Explain your reasoning.

\item Use the algebra of limits to find the local stretching factor
\[
  \frac{dH}{dL}\Big|_{L=b}
\]
of this same function at the point $L=b$ meters from the bottom end of the relaxed spring by evaluating the limit (as $v\to b$) of the average stretching factor between $L=v$ and $L=b$. Then check your result by substituting $b=3$. Check also with the above worksheets at a few other values of $b$.
\end{enumerate}

\end{example}

\section{Gas Consumption}
Most of our applications will not be about springs or elastic bands. Nevertheless, we can still think of the derivative as a local stretching factor. But iit might be better to talk about a local \emph{scaling factor} instead. The derivative scales (ie. multiplies) a small change in the input to give an approximate corresponding change in the output. Here's an example.

\begin{example} \label{Ex:987GHEjerf}
The function
\[
    G = f(s) = \frac{1}{2000}(s-100)^2 \, , \, 0\leq s \leq 80,
\]
expresses the number of gallons of gas in a car in terms of the trip odometer reading in miles. Its graph is shown below.

\begin{onlineOnly}
    \begin{center}
\desmos{wdnmaszvgb}{450}{600}  
\end{center}
\end{onlineOnly}

\href{https://www.desmos.com/calculator/wdnmaszvgb}{151: Gas Consumption A}

\begin{enumerate}
\item What are the units of the factor $1/2000$ in the function $f$? How do you know?

\item Find the average rate (in gal/mile) at which the car burned gas during the trip. Then find the average gas mileage (in miles/gal) over the entire trip.

\item Use the graph above to approximate the rate (in gal/mile) at which the car burns gas when the odometer reads $60$ miles. Then approximate the gas mileage (in miles/gal) at this odometer reading.

\item Enter the correct expression in Line 5 of the worksheet for the average rate at which the car burns gas between odometer readings $s=v$ and $s=60$ miles. Then use the table to guess the exact rate at which the car burns gas at the odometer reading $s=60$ miles. %Find also the exact gas mileage at this odometer reading.

\item Use the algebra of limits to compute the exact rate (in gal/mile) at which the car burns gas when the odometer reads $60$ miles.

\item Approximate the change 
\[
  \Delta G = f(v) - f(60)
\]
in the number of gallons of gas in the tank from odometer reading $s=60$ miles to odometer reading $s=v$ miles in terms of the change
\[
     \Delta v = v - 60
\] 
in the odometer reading. Assume $\Delta v \sim 0$.

\item Use the algebra of limits as above to compute the exact rate (in gal/mile) at which the car burns gas at the odometer reading $s=u$ miles.

\end{enumerate}

\end{example}


\section{A Falling Slinky}

\href{https://www.youtube.com/watch?v=eCMmmEEyOO0}{Falling Slinky}


\section{Atmospheric Pressure}

\href{https://projects.iq.harvard.edu/files/acmg/files/intro_atmo_chem_bookchap2.pdf}{Atmospheric Pressure} 


\end{document}