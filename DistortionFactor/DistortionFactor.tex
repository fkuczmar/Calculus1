\documentclass{ximera}
\title{Stretching Factors}

\newcommand{\pskip}{\vskip 0.1 in}

\begin{document}
\begin{abstract}
Interpreting the derivative as a local stretching factor.
\end{abstract}
\maketitle

\section{Rubber Band Calculus}

Think of the $x$-axis as an elastic band that can be stretched or compressed, and a function $f : \mathbb{R}\to\mathbb{R}$ as a deformation of this band. %For example, the function $f(x)= 2x$ stretches the band about the origin by a factor of two. As illustrated in Figure1, the stretching transforms a set of uniformly spaced marks to a set of marks spaced twice as far apart. Reversing the arrows gives a representation of the inverse function

For example, imagine a thin elastic band of length $4$ meters running along the horizontal $L$-axis from $L=0$ meters to $L=4$ meters.  Now hold the left end fixed and $L=0$ and stretch the right end to $L=8$. Then the function
\[
      H = g(L) = 2L \, , \, 0\leq L \leq 4 ,
\]
describes that stretching action. It takes as an input the distance (in meters) of a point on the band from the origin and returns as an output the distance between the origin and the corresponding point on the stretched band. The exploration below shows this stretching action.

\begin{exploration} \label{Ex:98f3rgafgbb}
Drag the slider $k$ in Line 2 below to illustrate the stretching action.

\begin{onlineOnly}
    \begin{center}
\desmos{qejivz36ui}{450}{600}  %qvk0mzy26u
\end{center}
\end{onlineOnly}

\href{https://www.desmos.com/calculator/qejivz36ui}{151: Rubber Band 1}

\end{exploration}

The average rate of change of the function,
\[
      H = g(L) = 2L \, , \, 0\leq L \leq 4 ,
\]
between points $L=a$ and $L=b$ meters from the origin,
\begin{align*}
  \frac{\Delta H}{\Delta L} &= \frac{f(b) - f(a)}{b-a} \\
                                       &= \frac{\answer{2(b-a)}}{b-a} \\
                                       &= \answer{2},
\end{align*}
is what we'll call the \emph{global stretching factor}. Any two points on the stretched band are twice as far apart as they were on the unstretched band.

\begin{question}  \label{Q:LDJJNMDesd}
\begin{enumerate}
\item What are the units of the stretching factor?
\begin{freeResponse}
\end{freeResponse}

\item Find an expression for the inverse function
\[
      L = f^{-1}(H).
\]
Include a domain. 

\item Interpret the inverse function as a deformation of a thin elastic band. What is the global stretching factor for this function?
\begin{freeResponse}
\end{freeResponse}
\end{enumerate}
\end{question}



\section{A Non-Linear Streching Function}
Here's an example of a non-linear stretching function. Now the stretching factor is \emph{not} constant. Instead, we need to talk about a \emph{local} stretching factor, one that varies from point to point. 

\begin{exploration} \label{ExLdfdfthyhhhf}

\begin{onlineOnly}
    \begin{center}
\desmos{vxlginlunu}{450}{600}  %qvk0mzy26u
\end{center}
\end{onlineOnly}

\href{https://www.desmos.com/calculator/vxlginlunu}{151: Rubber Band Cosine}



\end{exploration}






\section{A Hanging Slinky}

\begin{onlineOnly}
    \begin{center}
\desmos{zqjjgael5j}{450}{600}  %qvk0mzy26u
\end{center}
\end{onlineOnly}

\href{https://www.desmos.com/calculator/zqjjgael5j}{151: Slinky Photo}



\begin{onlineOnly}
    \begin{center}
\desmos{ew24lplwqf}{450}{600}  %qvk0mzy26u
\end{center}
\end{onlineOnly}

\href{https://www.desmos.com/calculator/ew24lplwqf}{151: Slinky 3}



\section{A Falling Slinky}

\href{https://www.youtube.com/watch?v=eCMmmEEyOO0}{Falling Slinky}


\section{Atmospheric Pressure}

\href{https://projects.iq.harvard.edu/files/acmg/files/intro_atmo_chem_bookchap2.pdf}{Atmospheric Pressure} 


\end{document}