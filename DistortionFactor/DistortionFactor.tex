\documentclass{ximera}
\title{Stretching Factors}

\newcommand{\pskip}{\vskip 0.1 in}

\begin{document}
\begin{abstract}
Interpreting the derivative as a local stretching factor.
\end{abstract}
\maketitle

\section{Rubber Band Calculus}

Think of a function and the first thing that might come to mind is its graph. But there are other interpretations, and one that will prove particularly useful for us is to think of a function's domain as an elastic band and the function as stretching or compressing that band. %For example, the function $f(x)= 2x$ stretches the band about the origin by a factor of two. As illustrated in Figure1, the stretching transforms a set of uniformly spaced marks to a set of marks spaced twice as far apart. Reversing the arrows gives a representation of the inverse function

For example, imagine a thin elastic band of length $4$ meters running along the horizontal $L$-axis from $L=0$ meters to $L=4$ meters.  Now hold the left end fixed at $L=0$ and stretch the band by moving right end an additional four meters to the right . Then the function
\[
      H = g(L) = 2L \, , \, 0\leq L \leq 4 ,
\]
describes this stretching action. It takes as an input the distance (in meters) of a point on the band from the origin ($L=0$) and returns as an output the distance between the origin and the corresponding point on the stretched band. The exploration below shows this stretching action.

\begin{exploration} \label{Ex:98f3rgafgbb}
Drag the slider $k$ in Line 2 below to illustrate the stretching action.

\begin{onlineOnly}
    \begin{center}
\desmos{qejivz36ui}{450}{600}  %qvk0mzy26u
\end{center}
\end{onlineOnly}

\href{https://www.desmos.com/calculator/qejivz36ui}{151: Rubber Band 1}

\end{exploration}

The \emph{global stretching factor} for a linear function like the one above, %The average rate of change of the function,
\[
      H = g(L) = 2L \, , \, 0\leq L \leq 4 ,
\]
is the slope of its graph. We can calculate this factor as an average stretching factor (ie. an average rate of change) between the points $L=a$  meters and $L=b$ meters from the origin. For the function $H=f(L) = 2L$, the stretching factor is
\begin{align*}
  \frac{\Delta H}{\Delta L} &= \frac{f(b) - f(a)}{b-a} \\
                                       &= \frac{\answer{2(b-a)}}{b-a} \\
                                       &= \answer{2},
\end{align*}
This tells us that any two points on the stretched band are twice as far apart as they were on the unstretched band.

\begin{question}  \label{Q:LDJJNMDesd}
\begin{enumerate}
\item What are the units of the stretching factor?
\begin{freeResponse}
\end{freeResponse}

\item Find an expression for the inverse function
\[
      L = f^{-1}(H).
\]
Include a domain. 

\item Interpret the inverse function as a deformation of a thin elastic band. What is the global stretching factor for this function?
\begin{freeResponse}
\end{freeResponse}
\end{enumerate}
\end{question}



\section{Non-Linear Stretching Functions}

\begin{example} \label{Ex:JDJFHDtet434t}
Here's an example
\[
      H = f(L) = 10 - \sqrt{100-L^2} \, , \, 0\leq L \leq 10,
\] 
of a non-linear stretching function (where $L$ and $H$ are measured in meters as before). Like most functions in this class, it acts like a linear function near most points in its domain. 

To stretch the elastic band in the demonstration below, drag the slider $u$ in Line 2 from $u=0$ to $u=1$. Then zoom in close enough to the point $H=f(2)$ in the stretched band (highlighted in black) to make the stretching function look linear. %the band looks like it's stretched by a constant factor. 


\begin{onlineOnly}
    \begin{center}
\desmos{nyd60dbezj}{450}{600}  %qvk0mzy26u
\end{center}
\end{onlineOnly}

\href{https://www.desmos.com/calculator/nyd60dbezj}{151: Rubber Band Ladder 5}
%\end{exploration}


\begin{enumerate}

\item Use the close-up view of the stretched band to approximate the stretching factor at the input $L=2$.

\item Drag the slider $m$ in Line 4 to $m=95$ and repeat part (a) to approximate the stretching factor at the input $L=9.5$ meters.

\item Parts of the elastic band get stretched, others compressed. Identify these.
\end{enumerate}

We can also approximate the stretching factors at different inputs by zooming in near enough to the graph of the function $H=f(L)$ to make the graph look like a line. To see the graph of the function, zoom back out and then activate the folder in Line 22 of the worksheet above.

\begin{enumerate}
\item Now zoom in close enough to the point $P(2,f(2))$ to make the graph look like a straight line. Then click on the graph to get the coordinates of $P$ and a point close to $P$. Use these to approximate the stretching factor at $L=2$.

\item Drag the slider $m$ in Line 4 to $m=95$ and repeat part (a) to approximate the stretching factor at $L=9.5$.

\item Zoom back out and use the graph of the function $H=f(L)$ to determine the stretching factors at $L=0$ and $L=10$.
\end{enumerate}

We can also approximate the stretching factors numerically using the expression
\[
      H = f(L) = 10- \sqrt{100-L^2} \, , \, 0\leq L \leq 10,
\] 
for the stretching function.

For example, at the input $L=2$, we'll to first compute the \emph{average stretching factor} over the interval $v\leq L \leq 2$ (when $v<L$) or the interval $2 \leq L \leq v$ (when $v>2$). This average rate of change is 
\begin{align*}
 g(v) &= \frac{\Delta H}{\Delta L} \\
              &= \frac{f(v) - f(2)}{v-2} \\
              &=\frac{10 - \sqrt{100-L^2} - (10 - \sqrt{\answer{96}})}{v-2}   \\
              &= \frac{\sqrt{100-L^2} - \answer{\sqrt{96}}}{v-2} \, , \, v \neq 2.
\end{align*}

Next we'll check to see if these average stretching factors appear to approach some value as $v\to 2$ (ie. as $v$ approaches $2$).

The quickest way to do this is  
\end{example}



\begin{example} \label{ExLdfdfthyhhhf}
Here's another example of a non-linear stretching function. Now the stretching factor is \emph{not} constant. Instead, we need to talk about a \emph{local} stretching factor, one that varies from point to point. 

\begin{exploration} \label{ExLdfdfthyhhhf}
Drag the slider $u$ in Line 2 below from $u=0$ to $u=1$ to illustrate the stretching action.
\begin{onlineOnly}
    \begin{center}
\desmos{hqvyhormhf}{450}{600}  %qvk0mzy26u
\end{center}
\end{onlineOnly}

\href{https://www.desmos.com/calculator/hqvyhormhf}{151: Rubber Band Cosine}
\end{exploration}

\begin{question} \label{Q:9887dfsdfdsf}
\begin{enumerate}
\item Use the animation above to approximate the local stretching factor at the point $L\sim 3.1$ meters from the origin on the unstretched band. Explain your reasoning.

\item Drag slider $m$ in Line 4 above to approximate the input at which the local stretching factor is equal to 1. (ie. where the stretched band is neither in tension or in compression). Explain your reasoning.
\end{enumerate}
\end{question}


We can also approximate the local stretching factor at $L=3.1$ meters from the graph of the function $H=f(L)$. The stretching function in Exploration 3 is
\[
      H = f(L) = 4 - 4\cos (L/2) \, , \, 0\leq L \leq 2\pi.
\]

\begin{exploration} \label{ExLtegfgfgyhhhf}
Drag the slider $u$ in Line 2 below from $u=0$ to $u=1$ to illustrate the stretching action and change the graph from $H=L$ to $H=f(L)$.
\begin{onlineOnly}
    \begin{center}
\desmos{dpxbfdwiag}{450}{600}  %qvk0mzy26u
\end{center}
\end{onlineOnly}

\href{https://www.desmos.com/calculator/dpxbfdwiag}{151: Rubber Band Cosine 2}

\end{exploration}

\end{example}

\section{The Sliding Ladder}

\begin{exploration} \label{Ex:LKDMNDFDSAdsa}
Here's an example of a decreasing stretching function.

Drag the slider $u$ in Line 2 below from $u=0$ to $u=1$ to illustrate the stretching action and change the graph from $H=10-L$ to $H=f(L)$.
\begin{onlineOnly}
    \begin{center}
\desmos{04kfcfukfy}{450}{600}  %qvk0mzy26u
\end{center}
\end{onlineOnly}

\href{https://www.desmos.com/calculator/04kfcfukfy}{151: Rubber Band Cosine 2}


\end{exploration}

\section{A Hanging Slinky}

\begin{onlineOnly}
    \begin{center}
\desmos{zqjjgael5j}{450}{600}  %qvk0mzy26u
\end{center}
\end{onlineOnly}

\href{https://www.desmos.com/calculator/zqjjgael5j}{151: Slinky Photo}



\begin{onlineOnly}
    \begin{center}
\desmos{ew24lplwqf}{450}{600}  %qvk0mzy26u
\end{center}
\end{onlineOnly}

\href{https://www.desmos.com/calculator/ew24lplwqf}{151: Slinky 3}



\section{A Falling Slinky}

\href{https://www.youtube.com/watch?v=eCMmmEEyOO0}{Falling Slinky}


\section{Atmospheric Pressure}

\href{https://projects.iq.harvard.edu/files/acmg/files/intro_atmo_chem_bookchap2.pdf}{Atmospheric Pressure} 


\end{document}