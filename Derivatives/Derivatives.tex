\documentclass{ximera}
\title{The Derivative}

\newcommand{\pskip}{\vskip 0.1 in}

\begin{document}
\begin{abstract}
Computing derivative with limits.
\end{abstract}
\maketitle
  


\begin{example}  \label{Ex:43tggt5t5}
Suppose 
\[
    y = f(x) = x^2
\]
and let's use limits to evaluate
\[
    f^\prime(3) =   \frac{dy}{dx}\Big|_{x=3} = \frac{d(x^2)}{dx}\Big|_{x=3} .
\]

We have
\begin{align*}
 f^\prime(3) = \frac{d(x^2)}{dx}\Big|_{x=3} &= \lim_{v\to 3} \frac{f(v)-f(3)}{v-3} \\
                                              &= \lim_{v\to 3} \frac{v^2 - 9}{v-3} \\
                                              &= \lim_{v\to 3} \frac{(v-3)(v+3)}{v-3} \\
                                             &= \lim_{v\to 3} (v+3) \\
                                           &= (3+3)  \\
                                           &= 6 .
\end{align*}

Next let's do almost the same thing and compute
\[
   f^\prime(x) =   \frac{dy}{dx}
\] 
for the function
\[
   y = f(x) = x^2 
\]
by replacing $3$ in the above computation with $x$. 

We get
\begin{align*}
 \frac{dy}{dx}= \frac{d(x^2)}{dx}&= \lim_{v\to x} \frac{f(v)-f(x)}{v-x} \\
                                              &= \lim_{v\to x} \frac{v^2 - x^2}{v-x} \\
                                              &= \lim_{v\to x} \frac{(v-x)(v+x)}{v-x} \\
                                             &= \lim_{v\to x} (v+x) \\
                                           &= (x+x)  \\
                                           &= 2x .
\end{align*}

Just as a check, when $x=3$,
\[
   \frac{dy}{dx}\Big|_{x=3} = (2x)\Big|_{x=3} = 2(3) = 6.
\]
\end{example}


\pskip

\begin{example} \label{Ex:3245trg}
Use the result of the previous example to solve each of the following problems. Do \emph{not} use limits.

(a) Find an equation of the tangent line to the parabola $y=x^2$ at the point $(-4,16)$.

(b) Find an equation of the tangent line to the parabola perpendicular to the tangent line in part (a).

(c) Find the coordinates of the point where the lines in parts (a) and (b) intersect.

(d) Let ${\cal L}$ be the line through the points of tangency of the lines in parts (a) and (b). Find the coordinates of the point where ${\cal L}$ intersects the $y$-axis.




\end{example}



\end{document}