\documentclass{ximera}
\title{The Derivative}

\newcommand{\pskip}{\vskip 0.1 in}

\begin{document}
\begin{abstract}
Computing derivative with limits.
\end{abstract}
\maketitle
  


\begin{example}  \label{Ex:43tggt5t5}
Suppose 
\[
    y = f(x) = x^2
\]
and let's use limits to evaluate
\[
    f^\prime(3) =   \frac{dy}{dx}\Big|_{x=3} = \frac{d(x^2)}{dx}\Big|_{x=3} .
\]

We have
\begin{align*}
 f^\prime(3) = \frac{d(x^2)}{dx}\Big|_{x=3} &= \lim_{v\to 3} \frac{f(v)-f(3)}{v-3} \\
                                              &= \lim_{v\to 3} \frac{v^2 - 9}{v-3} \\
                                              &= \lim_{v\to 3} \frac{(v-3)(v+3)}{v-3} \\
                                             &= \lim_{v\to 3} (v+3) \\
                                           &= (3+3)  \\
                                           &= 6 .
\end{align*}

Next let's do almost the same thing and compute
\[
   f^\prime(x) =   \frac{dy}{dx}
\] 
for the function
\[
   y = f(x) = x^2 
\]
by replacing $3$ in the above computation with $x$. 

We get
\begin{align*}
 \frac{dy}{dx}= \frac{d(x^2)}{dx}&= \lim_{v\to x} \frac{f(v)-f(x)}{v-x} \\
                                              &= \lim_{v\to x} \frac{v^2 - x^2}{v-x} \\
                                              &= \lim_{v\to x} \frac{(v-x)(v+x)}{v-x} \\
                                             &= \lim_{v\to x} (v+x) \\
                                           &= (x+x)  \\
                                           &= 2x .
\end{align*}

Just as a check, when $x=3$,
\[
   \frac{dy}{dx}\Big|_{x=3} = (2x)\Big|_{x=3} = 2(3) = 6.
\]
\end{example}


\pskip

\begin{question} \label{Ex:3245trg}
Use the result of the previous example to solve each of the following problems. Do \emph{not} use limits.

(a) Find an equation of the tangent line to the parabola $y=x^2$ at the point $(-4,16)$.

(b) Find an equation of the tangent line to the parabola perpendicular to the tangent line in part (a).

(c) Find the coordinates of the point where the lines in parts (a) and (b) intersect.

(d) Let ${\cal L}$ be the line through the points of tangency of the lines in parts (a) and (b). Find the coordinates of the point where ${\cal L}$ intersects the $y$-axis.

(e) Repeat parts (a)-(d) above for the tangent line to the parabola $y=x^2$ at the general point $(a,a^2)$. What do you notice?

\end{question}

\begin{question}   \label{Ex:3ereftg4t}
(a) Use the method of Example 1 for the function
\[
   y = g(x) = 1/x^2
\]
to compute 
\[
    g^\prime(3) =   \frac{dy}{dx}\Big|_{x=3} = \frac{d(1/x^2)}{dx}\Big|_{x=3} 
\]
and
\[
     g^\prime(x) = \frac{dy}{dx} .
\]


(b) Use the result of part (a) to find an equation of the tangent line to the curve $y=1/x^2$ at the point $(3,1/9)$.
\end{question}


\begin{question}  \label{Q4324}
(a) Use numerical methods to estimate the slope of the tangent line to the curve
\[
     y = f(x) = x^3
\]
at the point $(2,8)$. Include enough data to suggest a progression toward a limit.

(b) Use the algebra of limits to find the exact slope of the tangent line in part (a).

(c) Use algebra to find the coordinates of the all pointw where the tangent line in part (a) intersects the curve $y=x^3$.

(d) Suppose you measure the edge length of a cube to be $2$cm and then use this measurement to compute the volume of the cube. Use the result of part (b) to approximate your error in computing the volume in terms of your error in measuring the edge length. Assume the latter error is small. 

Then compare your exact error in computing the volume with your approximation for some specific edge length near $2$cm. You should start this problem defining a function with meaningful variable names (do not use $x$ and $y$).

\end{question}



\begin{question}  \label{Q354rertb}
Suppose that between speeds of $60$ miles/hr and $72$ miles/hr, the gas mileage of a car is a linear function of its speed. Suppose also that the car gets $36$ miles/gallon at a speed of $60$ miles/hour and $32$ miles/gallon at a speed of $72$ miles/hour.

(a) Find a function
\[
    r = f(v) , 60\leq v \leq 72 ,
\]
that expresses the rate (measured in gal/hr) at which the car burns gas in terms of its speed (measured in miles/hour). Explain your reasoning. This function is \emph{not} linear.
\begin{hint}
\[
        r = f(t) = \answer{\frac{3v}{168-v}} , 60\leq v \leq 72 .
\]
\end{hint}

(b) Use numerical methods to estimate the value of the derivative
\[
    \frac{dr}{dt}\Big|_{v=63} .
\]
Make a table that shows enough data to suggest a progression toward a limit. Include units in all column headings.

(c) Use the algebra of limits to find the exact value of the derivative in part (b). 

(d) Explain the meaning of the derivative in terms of small changes.




\end{question}


\end{document}