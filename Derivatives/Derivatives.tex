\documentclass{ximera}
\title{The Derivative}

\newcommand{\pskip}{\vskip 0.1 in}

\begin{document}
\begin{abstract}
Computing derivatives with limits.
\end{abstract}
\maketitle
  


\begin{example}  \label{Ex:43tggt5t5}
Suppose 
\[
    y = f(x) = x^2
\]
and let's use limits to evaluate
\[
    f^\prime(3) =   \frac{dy}{dx}\Big|_{x=3} = \frac{d(x^2)}{dx}\Big|_{x=3} .
\]

We have
\begin{align*}
 f^\prime(3) = \frac{d(x^2)}{dx}\Big|_{x=3} &= \lim_{v\to 3} \frac{f(v)-f(3)}{v-3} \\
                                              &= \lim_{v\to 3} \frac{v^2 - 9}{v-3} \\
                                              &= \lim_{v\to 3} \frac{(v-3)(v+3)}{v-3} \\
                                             &= \lim_{v\to 3} (v+3) \\
                                           &= (3+3)  \\
                                           &= 6 .
\end{align*}

Next let's do almost the same thing and compute
\[
   f^\prime(x) =   \frac{dy}{dx}
\] 
for the function
\[
   y = f(x) = x^2 
\]
by replacing $3$ in the above computation with $x$. 

We get
\begin{align*}
 \frac{dy}{dx}= \frac{d(x^2)}{dx}&= \lim_{v\to x} \frac{f(v)-f(x)}{v-x} \\
                                              &= \lim_{v\to x} \frac{v^2 - x^2}{v-x} \\
                                              &= \lim_{v\to x} \frac{(v-x)(v+x)}{v-x} \\
                                             &= \lim_{v\to x} (v+x) \\
                                           &= (x+x)  \\
                                           &= 2x .
\end{align*}

Just as a check, when $x=3$,
\[
   \frac{dy}{dx}\Big|_{x=3} = (2x)\Big|_{x=3} = 2(3) = 6.
\]
\end{example}


\pskip

\begin{question} \label{Ex:3245trg}
Use the result of the previous example to solve each of the following problems. Do \emph{not} use limits.

(a) Find an equation of the tangent line to the parabola $y=x^2$ at the point $(-4,16)$.

(b) Find an equation of the tangent line to the parabola perpendicular to the tangent line in part (a).

(c) Find the coordinates of the point where the lines in parts (a) and (b) intersect.

(d) Let ${\cal L}$ be the line through the points of tangency of the lines in parts (a) and (b). Find the coordinates of the point where ${\cal L}$ intersects the $y$-axis.

(e) Repeat parts (a)-(d) above for the tangent line to the parabola $y=x^2$ at the general point $(a,a^2)$. What do you notice?

\end{question}

\begin{question}   \label{Ex:3ereftg4t}
(a) Use the method of Example 1 for the function
\[
   y = g(x) = 1/x^2
\]
to compute 
\[
    g^\prime(3) =   \frac{dy}{dx}\Big|_{x=3} = \frac{d(1/x^2)}{dx}\Big|_{x=3} 
\]
and
\[
     g^\prime(x) = \frac{dy}{dx} .
\]


(b) Use the result of part (a) to find an equation of the tangent line to the curve $y=1/x^2$ at the point $(3,1/9)$.
\end{question}


\begin{question}  \label{Q4324}
(a) Use numerical methods to estimate the slope of the tangent line to the curve
\[
     y = f(x) = x^3
\]
at the point $(2,8)$. Include enough data to suggest a progression toward a limit.

(b) Use the algebra of limits to find the exact slope of the tangent line in part (a).

(c) Use algebra to find the coordinates of the all pointw where the tangent line in part (a) intersects the curve $y=x^3$.

(d) Suppose you measure the edge length of a cube to be $2$cm and then use this measurement to compute the volume of the cube. Use the result of part (b) to approximate your error in computing the volume in terms of your error in measuring the edge length. Assume the latter error is small. 

Then compare your exact error in computing the volume with your approximation for some specific edge length near $2$cm. You should start this problem defining a function with meaningful variable names (do not use $x$ and $y$).

\end{question}

\begin{question}  \label{Qd9fsd0g}
(a) Find a function 
\[
   s = g(V) , s\geq 0
\]
that expresses the edge length (measured in cm) of a cube in terms of its volume (measured in cubic centimeters).

(b) Use the algebra of limits to evaluate the derivative
\[
       g^\prime(V_0) = \frac{ds}{dV}\Big|_{V=V_0}.
\]

(c) What are the units of the derivative in part (b)? Explain how you know.

(d) Suppose you submerge the cube in water and measure its volume to be $8\text{ cm}^3$. You then use this measurement to compute the edge length of the cube. Use the result of part (b) to approximate your error in computing the edge length in terms of your error in measuring the volume. Assume the latter error is small. 

Start this problem by defining the errors $\Delta s$ and $\Delta V$ in terms of the actual edge length $s$ of the cube.

(e) How is this question related to the previous question? Explain.

\end{question}




\begin{question}  \label{Q354rertb}
Suppose that between speeds of $60$ miles/hr and $72$ miles/hr, the gas mileage of a car is a linear function of its speed. Suppose also that the car gets $36$ miles/gallon at a speed of $60$ miles/hour and $32$ miles/gallon at a speed of $72$ miles/hour.

(a) Find a function
\[
    r = f(v) , 60\leq v \leq 72 ,
\]
that expresses the rate (measured in gal/hr) at which the car burns gas in terms of its speed (measured in miles/hour). Explain your reasoning. This function is \emph{not} linear.
\begin{hint}
\[
        r = f(t) = \answer{\frac{3v}{168-v}} , 60\leq v \leq 72 .
\]
\end{hint}

(b) Use numerical methods to estimate the value of the derivative
\[
    f^\prime(63) = \frac{dr}{dv}\Big|_{v=63} .
\]
Make a table that shows enough data to suggest a progression toward a limit. Include units in all column headings.

(c) Use the algebra of limits to find an expression for the derivative
\[
   f^\prime(v) = \frac{dr}{dv} .
\]
Then use this expression to find the exact value of the derivative in part (b). 

(d) What are the units of the derivative in part (b)? Explain its meaning.

(e) Use the result of part (c) to approximate the change 
\[
       \Delta r = f(v) - f(63)
\]
in the rate at which your car burns gas in terms of the change
\[
     \Delta v = v - 63
\]
of the car's speed. Assume $\Delta v \sim 0$.

\end{question}


\begin{question}  \label{Q:ewrtggg}
Suppose that between speeds of $30$ miles/hour and $70$ miles/hour the gas mileage of a car is a quadratic function of its speed. 
Suppose also that the car gets a maximum of $42$ miles/gal at a speed of $50$ miles/hour and $34$ miles/gallon at a speed of $30$ miles/hour.

(a) Find a function
\[
    r = h(v) \, , \, 30\leq v \leq 70 ,
\]
that expresses the rate (in gal/hr) at which the car burns gas in terms of its speed (in miles/hour).
\begin{hint}

(i) At what rate does the car burn gas at a speed of $50$ miles/hour? $\answer{25/21}$ gal/mile

(ii) Find a function that expresses the gas mileage $G$ (measured in miles/gallon) in terms of the speed (measured in miles/hr).
\[
         G = \answer{42 - 0.02(v-50)^2} \, , \, 30\leq v \leq 70 .
\]

(ii) The rate (in gal/hr) at which the car burns gas as a function of its speed (in miles/hr) is
\[
         r = h(v) = \answer{\frac{v}{42 - 0.02(v-50)^2}} \, , \, 30\leq v \leq 70  .
\]
\end{hint}

(b) Use the algebra of limits to evaluate the derivative
\[
    h^\prime(40) = \frac{dr}{dv}\Big|_{v=40} .
\]

(c) What are the units of the above derivative? How do you know?

(d) Express the meaning of the derivative in the context of small changes.

\end{question}


\begin{question}  \label{Qersadefgt4}
At 10:00am on April 18, the wholesale price of Cosmic Crisp apples is $\$2.00$/lb and is decreasing at the rate of $\$0.10/lb/hour$. 

Use the algebra of limits to determine the rate (in pounds/hour) at which the number of pounds of apples a store 
can purchase with $\$1000$ is changing at this time.

Start this question by defining a function that expresses the number of pounds of apples the store can buy with $\$1000$ in terms of the price (in $\$$/lb). Choose meaningful variable names (not $x$ and $y$). Do \emph{not} assume the price is decreasing at a constant rate.
\end{question}


\section{Limits, Speed and Altitude}
\begin{question}  \label{Ex:defdftd}
A rock dropped from a height of $100$ feet falls to the surface of Planet Krypton without air resistance.

(a) By considering only the physical situation and \emph{without} doing any computations, sketch a graph of the function
\[
    v =g(h) \, , 0\leq h \leq 100
\]
that expresses the rock's speed (in ft/sec) in terms of its height (in feet). Explain your reasoning.

(b) Use the results from part (a) to choose a reasonable expression for the function $g$ from the list below.

\begin{multipleChoice}
\choice{$g(t)=100 - 9t^2$, $0\leq t \leq 10/3$}
\choice{$g(h)=100 - 9h^2$, $0\leq h \leq 100$}
\choice{$g(h)=0.005(100-h)^2$, $0\leq h \leq 100$}
\choice[correct]{$g(h)=6\sqrt{100-h}$, $0\leq h \leq 100$}
\end{multipleChoice}


(c)  Give numerical  and graphical evidence that either supports or refutes the claim that a small change in the rocks height from $64$ feet gives an approximately proportional change in its speed. Then approximate the proportionality constant. What are its units?
  
(d) Use the algebra of limits to find an expression for the derivative
\[
     g^\prime(h) = \frac{dv}{dh} .
\]
Then use this expression to find the exact value of the proportionality constant in part (c).

(e) Explain the meaning of the proportionality constant.

(f) Approximate the change
\[
    \Delta v = v - g(64)
\]
in the rock's speed in terms of a small change
\[
  \Delta h = h - 64
\]
in its height.

(g) Use part (g) to approximate the rock's speed at a height of $63$ feet.

(h) Would you expect your approximation in part (h) to be greater or less than the actual speed? Explain your reasoning with a graph.

(i) Simplify the units of the proportionality constant. Does this simplification help to understand or obscure the meaning of the proportionality constant?
\end{question}




\end{document}