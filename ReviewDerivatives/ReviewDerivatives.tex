\documentclass{ximera}
\title{Review of Differentiation}

\newcommand{\pskip}{\vskip 0.1 in}

\begin{document}
\begin{abstract}
Derivative Review.
\end{abstract}
\maketitle

\section*{Examples}

\begin{example}  \label{Ex:GDsdfFGff}
Find an equation of the tangent line to the curve
\[
  y = f(x) = \left( 2x^3 +1  \right)^2
\]
at the point $(1,9)$.

\begin{explanation}
Let 
\[
       y = \left( 2x^3 +1  \right)^2
\]
and 
\[
      u = 2x^3 + 1 .
\]
Then
\[
     y = u^2
\]
and
\begin{align*}
\frac{dy}{dx} &= \frac{dy}{du} \cdot \frac{du}{dx}  \\
                     &= \frac{d}{du} \left( u^2 \right)  \cdot \frac{d}{dx}\left(  2x^3 + 1 \right)  \\
                     &= 2u (6x^2)  \\
                     &= 2(2x^3+1)(6x^2) .
\end{align*}

Then the slope of the tangent line to the curve $y=(2x^3+1)^2$ at the point $(1,9)$ is 
\[
              \frac{dy}{dx}\Big|_{x=1} = 2(3)(6) = 36 ,
\]
and an equation of the tangent line is
\[
   y - 9 = 36(x-1) .
\]
\end{explanation}
\end{example}

\begin{example}  \label{E"MMDNBRR}
Find an expression for the derivative 
\[
  \frac{d}{d\theta} \left( \theta \cos (5\theta) \right) .
\]

\begin{explanation}

We use the product rule first to get
\begin{align*}
       \frac{d}{d\theta} \left( \theta \cos (5\theta) \right) &= \frac{d}{d\theta} \left(  \theta \right) \cos (5\theta) + \theta \frac{d}{d\theta} \left(  \cos (5\theta) \right)  \\ 
                          &= \cos(5\theta) + \theta  \frac{d}{d\theta} \left(  \cos (5\theta) \right) .
\end{align*}

Now we use the chain rule to compute 
\[
     \frac{d}{d\theta} \left(  \cos (5\theta) \right) .
\]
For this we hide the composition by letting
\[
   y  = \cos (5\theta)
\]  
and 
\[
   u = 5\theta.
\]
Then 
\[
     y = \cos u
\]
and by the chain rule 
\begin{align*}
  \frac{dy}{d\theta}  &= \frac{dy}{du} \cdot \frac{du}{d\theta} \\
                              &= \frac{d}{du}\left( \cos u \right) \cdot \frac{d}{d\theta} \left( 5\theta \right)  \\
                              &= (-\sin u)(5) \\
                              & = -5 \sin (5\theta) . 
\end{align*}


The final result is that 
\begin{align*}
               \frac{d}{d\theta} \left( \theta \cos (5\theta) \right) &=  \cos(5\theta) + \theta  \frac{d}{d\theta} \left( \cos (5\theta) \right) \\
           & = \cos(5\theta) - 5\theta \sin (5\theta) .
\end{align*}


\pskip

Here's a shorter version of the same solution.

We use the product rule first to get
\begin{align*}
       \frac{d}{d\theta} \left( \theta \cos (5\theta) \right) &= \frac{d}{d\theta} \left(  \theta \right) \cos (5\theta) + \theta \frac{d}{d\theta} \left(  \cos (5\theta) \right)  \\ 
                          &= \cos(5\theta) + \theta  \frac{d}{d\theta} \left(  \cos (5\theta) \right) .
\end{align*}

Then we use the chain rule to differentiate $\cos(5\theta)$, giving
\begin{align*}
       \frac{d}{d\theta} \left( \theta \cos (5\theta) \right) &= \frac{d}{d\theta} \left(  \theta \right) \cos (5\theta) + \theta \frac{d}{d\theta} \left(  \cos (5\theta) \right)  \\ 
                          &= \cos(5\theta) + \theta  \frac{d}{d\theta} \left(  \cos (5\theta) \right)  \\
                         &= \cos(5\theta) + \theta (-\sin (5\theta)) \frac{d}{d\theta} \left( 5\theta  \right)  \\
                         &= \cos (5\theta) - 5 \theta \sin (5\theta) .
\end{align*}


\end{explanation}


\end{example}



\section*{Exercises}

\emph{Directions:} Follow the method of Example 1 \emph{exactly} for each of the following problems.


\begin{question}  \label{Q:MMMDNFD}
Find expressions for each of the following derivatives. Simplify your expressions for the derivatives. Do \emph{not} simplify the function being differentiated.

Part 1:

(a)  $\frac{d}{dw} \left( \arctan w + \arctan (1/w) \right)$ 

(b) $\frac{d}{dy} \left(  \arcsin\left(  \sqrt{1-y^2} \right) \right)$

\pskip

Part 2:

Any comments or observations on the derivatives above?


\end{question}


\begin{question}  \label{Q:Dfrg4ttgg}

\begin{onlineOnly}
    \begin{center}
\desmos{zjqsdhvz4p}{900}{600}
\end{center}
\end{onlineOnly}

Desmos activity available at \href{https://www.desmos.com/calculator/zjqsdhvz4p}{151: Building Temperature}

\end{question}

\begin{question}  \label{Q:MXCXV}
Water is poured into a cylindrical tank at a constant rate. At the same time, water flows out of a small hole in the bottom of the tank. The tank is empty at noon.

The function
\[
      t = f(h) = - \frac{2}{k_2}  \left( \sqrt{h}+ \frac{k_1}{k_2} \ln \Bigg| \frac{k_1 - k_2 \sqrt{h}}{k_1}   \Bigg| \right)\, , \, t\geq 0 ,
\]
expresses the time (measured in minutes past noon) in terms of the depth (measured in cm) of water in the tank. Here $k_1,k_2$ are postive constants.

(a) Use the above function to verify that the tank is empty at noon.

(b) Find a simplified expression for the derivative $dt/dh$.

(c) Use the result of part (b) to show that
\[
   \frac{dh}{dt} = k_1 - k_2 \sqrt{h}.
\]

(d) What are the units of $k_1$, $k_2$? How do you know?

(e) For much more on this problem, see Questions 23 and 24 of the chapter \emph{Derivatives of Inverse Functions}.

\end{question} 


\begin{question}  \label{Q:LDKVMdecvd}
The function
\[
      s = f(t) = Ae^{-k_1t} \cos(k_2t) \, , \, t\geq 0,
\]
expresses the displacement (in meters) from equilibrium of an oscillating mass on a spring in terms of the number of seconds since the mass was released from rest.

(a) What are the units of the constants $A$, $k_1$, and $k_2$? Explain how you know.

(b) Find an expression for the velocity $ds/dt$ of the mass.

(c) Find an expression for the acceleration
\[
   \frac{d}{dt} \left( \frac{ds}{dt}  \right)  = \frac{d^2s}{dt^2}
\]
of the mass.

(d) Use algebra to show that
\[
     \frac{d^2s}{dt^2} = - \left( 2k_1 \frac{ds}{dt} + (k_1^2 + k_2^2) s \right) .
\]


\begin{onlineOnly}
    \begin{center}
\desmos{ygikqgj7af}{900}{600}
\end{center}
\end{onlineOnly}

Desmos activity available at \href{https://www.desmos.com/calculator/ygikqgj7af}{151: Damped Harmonic Oscillator}

\end{question}

\begin{question}  \label{Q:LLKDKFG}
(a) Find a function
\[
     s = f(h) , h\geq 0 ,
\]
that expresses the distance (in miles) to the horizon in terms of your altitude (in miles). We'll suppose the earth to be a perfect sphere of radius $4000$ miles. The distance to the horizon is the arclength $AT$ below, meaured along the surface of the earth (you can think of this distance as the radius of the spherical disk visible to us). Our height is the distance $AP$. 
 
The function is 
\[
   s = f(h) = \answer{4000 \arccos\left( \frac{4000}{4000+h} \right) } \, , \, h \geq 0 .
\]

\begin{onlineOnly}
    \begin{center}
\desmos{ewowig5sgk}{900}{600}
\end{center}
\end{onlineOnly}

(b) \emph{Without} using technology sketch by hand a graph of the function $s=f(h)$.

(c) \emph{Without} using technology sketch by hand a graph of the derivative
\[
   y = \frac{ds}{dh} = f^\prime(h) .
\] 

(d) Evaluate the derivative 
\[
      \frac{ds}{dh}\Big|_{h=10} .
\]

(e) Interpret the meaning of the derivative in part (d) using the language of small changes. Be specific.

(f) You take a ride on Blue Origin and in a little more than two minutes are boosted straight up to an altitude of 32 miles. Suppose that the function
\[
      h = g(t) = 9t^2 + t^3 \, , \, 0\leq t \leq 1.5 ,
\]
expresses your altitude (in miles) in terms of the number of minutes since launch.

At what rate (with respect to time) is your distance to the horizon changing when you are one minute into the flight?
\end{question}


\end{document}