\documentclass{ximera}
\title{Review of Differentiation}

\newcommand{\pskip}{\vskip 0.1 in}

\begin{document}
\begin{abstract}
Derivative Review.
\end{abstract}
\maketitle

\section*{Examples}

\begin{example}  \label{Ex:GDsdfFGff}
Find an equation of the tangent line to the curve
\[
  y = f(x) = \left( 2x^3 +1  \right)^2
\]
at the point $(1,9)$.

\begin{explanation}
Let 
\[
       y = \left( 2x^3 +1  \right)^2
\]
and 
\[
      u = 2x^3 + 1 .
\]
Then
\[
     y = u^2
\]
and
\begin{align*}
\frac{dy}{dx} &= \frac{dy}{du} \cdot \frac{du}{dx}  \\
                     &= \frac{d}{du} \left( u^2 \right)  \cdot \frac{d}{dx}\left(  2x^3 + 1 \right)  \\
                     &= 2u (6x^2)  \\
                     &= 2(2x^3+1)(6x^2) .
\end{align*}

Then the slope of the tangent line to the curve $y=(2x^3+1)^2$ at the point $(1,9)$ is 
\[
              \frac{dy}{dx}\Big|_{x=1} = 2(3)(6) = 36 ,
\]
and an equation of the tangent line is
\[
   y - 9 = 36(x-1) .
\]
\end{explanation}
\end{example}

\begin{example}  \label{E"MMDNBRR}
Find an expression for the derivative 
\[
  \frac{d}{d\theta} \left( \theta \cos (5\theta) \right) .
\]

\begin{explanation}

We use the product rule first to get
\begin{align*}
       \frac{d}{d\theta} \left( \theta \cos (5\theta) \right) &= \frac{d}{d\theta} \left(  \theta \right) \cos (5\theta) + \theta \frac{d}{d\theta} \left(  \cos (5\theta) \right)  \\ 
                          &= \cos(5\theta) + \theta  \frac{d}{d\theta} \left(  \cos (5\theta) \right) .
\end{align*}

Now we use the chain rule to compute 
\[
     \frac{d}{d\theta} \left(  \cos (5\theta) \right) .
\]
For this we hide the composition by letting
\[
   y  = \cos (5\theta)
\]  
and 
\[
   u = 5\theta.
\]
Then 
\[
     y = \cos u
\]
and by the chain rule 
\begin{align*}
  \frac{dy}{d\theta}  &= \frac{dy}{du} \cdot \frac{du}{d\theta} \\
                              &= \frac{d}{du}\left( \cos u \right) \cdot \frac{d}{d\theta} \left( 5\theta \right)  \\
                              &= (-\sin u)(5) \\
                              & = -5 \sin (5\theta) . 
\end{align*}


The final result is that 
\begin{align*}
               \frac{d}{d\theta} \left( \theta \cos (5\theta) \right) &=  \cos(5\theta) + \theta  \frac{d}{d\theta} \left( \cos (5\theta) \right) \\
           & = \cos(5\theta) - 5\theta \sin (5\theta) .
\end{align*}


\pskip

Here's a shorter version of the same solution.

We use the product rule first to get
\begin{align*}
       \frac{d}{d\theta} \left( \theta \cos (5\theta) \right) &= \frac{d}{d\theta} \left(  \theta \right) \cos (5\theta) + \theta \frac{d}{d\theta} \left(  \cos (5\theta) \right)  \\ 
                          &= \cos(5\theta) + \theta  \frac{d}{d\theta} \left(  \cos (5\theta) \right) .
\end{align*}

Then we use the chain rule to differentiate $\cos(5\theta)$, giving
\begin{align*}
       \frac{d}{d\theta} \left( \theta \cos (5\theta) \right) &= \frac{d}{d\theta} \left(  \theta \right) \cos (5\theta) + \theta \frac{d}{d\theta} \left(  \cos (5\theta) \right)  \\ 
                          &= \cos(5\theta) + \theta  \frac{d}{d\theta} \left(  \cos (5\theta) \right)  \\
                         &= \cos(5\theta) + \theta (-\sin (5\theta)) \frac{d}{d\theta} \left( 5\theta  \right)  \\
                         &= \cos (5\theta) - 5 \theta \sin (5\theta) .
\end{align*}


\end{explanation}


\end{example}



\section*{Exercises}

\emph{Directions:} Follow the method of Example 1 \emph{exactly} for each of the following problems.


\begin{question}  \label{Q:MMMDNFD}
Find expressions for each of the following derivatives. Simplify your expressions for the derivatives. Do \emph{not} simplify the function being differentiated.

Part 1:

(a)  $\frac{d}{dw} \left( \arctan w + \arctan (1/w) \right)$ 

(b) $\frac{d}{dy} \left(  \arcsin\left(  \sqrt{1-y^2} \right) \right)$

\pskip

Part 2:

Any comments or observations on the derivatives above?


\end{question}


\begin{question}  \label{Q:Dfrg4ttgg}

\begin{onlineOnly}
    \begin{center}
\desmos{zjqsdhvz4p}{900}{600}
\end{center}
\end{onlineOnly}

Desmos activity available at \href{https://www.desmos.com/calculator/zjqsdhvz4p}{151: Building Temperature}

\end{question}

\begin{question}  \label{Q:MXCXV}
The function
\[
   s = f(t) = 
\]
\end{question} 


\end{document}