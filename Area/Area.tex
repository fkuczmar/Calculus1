\documentclass{ximera}
\title{Area}

\newcommand{\pskip}{\vskip 0.1 in}

\begin{document}
\begin{abstract}
Computing the area of regions in the plane bounded by graphs of functions.
\end{abstract}
\maketitle

\emph{Directions:} When computing the area of a region, be sure to do the following.
\begin{itemize}
\item State the direction in which you decide to slice the region.

\item Sketch a differential rectangle and label the coordinates of its endpoints.

\item State the area of the differential rectangle.

\item Explain the role of the definite integral in computing the area of the region. 
\end{itemize}

\section{Parabolas and Power Functions, Part 1}

\begin{question} \label{QWeerFGHG}
Let $a,b>0$ be constants, measured in meters. 

\begin{enumerate}

\item Find the area of the region bounded by the $x$-axis and the parabola
\[
     y  = b - \frac{x^2}{a} ,
\]
where $x$ and $y$ are measured in meters.

Include units and check that your answer is dimensionally correct. Do this twice:

\begin{enumerate}
\item first, by slicing the region perpendicular to the $y$-axis, % and computing the area of a differential rectangle.

\item and again, by slicing the region perpendicular to the $x$-axis % and computing the area of a differential rectangle.
\end{enumerate}
\end{enumerate}
\end{question}

\begin{question} \label{QOLDFKRerbttp}
Express the area of the region shown below bounded by a parabola and a line perpendicular to its axis of symmetry in terms of the height $h$ and the length of the base $b$.

\begin{onlineOnly}
    \begin{center}
\desmos{kkdfpr66rt}{450}{600}  
\end{center}
\end{onlineOnly}

\href{https://www.desmos.com/calculator/kkdfpr66rt}{152: Parabola 1}

\end{question}


\begin{question}  \label{Q9erkmbzxee}
Use ideas of differentilal calculus, \emph{not} the Fundamental theorem to answer Question 2 by comparing the areas of the differential rectangles shown below.

\begin{onlineOnly}
    \begin{center}
\desmos{pvskgbpnew}{450}{600}  
\end{center}
\end{onlineOnly}

\href{https://www.desmos.com/calculator/pvskgbpnew}{152: Parabola 2}

\begin{onlineOnly}
    \begin{center}
\desmos{0w37bixe79}{450}{600}  
\end{center}
\end{onlineOnly}

\href{https://www.desmos.com/calculator/0w37bixe79}{152: Parabola 3}

\end{question}



\begin{question} \label{QOLDefeefKRerbttp}
Let $b,h>0$ be constants with units of meters and let $a=b/2$.

Suppose the curve $y=cx^8$ passes through the point $A$ with coordinates $(b/2,h) = (a,h)$. We wish to express the area of the shaded region below bounded by the curve and the line $y=h$ in terms of its base $b$ and height $h$. Do this as follows.

\begin{enumerate}

\item First express $c$ in terms of $a$ and $h$.

\item Then express the area of the region first in terms of $a$ and $h$, and then in terms of $b$ and $h$.

\item Check that your expression in part (b) has the correct units. 

\item Interpret  your expression for the area geometrically. Does it seem reasonable? Compare it with the area of a triangle and with the area of a rectangle. 

\item Find an equation of the horizontal line that bisects the area of the region. Type your equation in Line 2 above as a check. 

\end{enumerate}

\begin{onlineOnly}
    \begin{center}
\desmos{egdpe7cccl}{450}{600}  
\end{center}
\end{onlineOnly}

\href{https://www.desmos.com/calculator/egdpe7cccl}{152: Power Function 3}

\end{question}




\begin{question} \label{QOidfsfer}
Let $a,b>0$ be constants with units of meters.
\begin{enumerate}
\item Use the Fundamental Theorem to evaluate the definite integral
\[
   \int_0^b \frac{x^2}{a}\, dx . 
\]
Include units.

\item Interpret the result of part (a) geometrically.
\end{enumerate}

\end{question}

\begin{question} \label{QPERer3943}
Let $a,b,c\in \mathbb{R}$ be constants with units of meters.

Find the area of the region bounded by the $x$-axis and the parabola
\[
      y = c+ \frac{(x-b)^2}{a} ,
\]
where the coordinates $(x,y)$ are also measured in meters.
\end{question}


\section{Parabolas, Part 2}

\begin{question} \label{QKdf3rr3}
\begin{enumerate}
\item Find the area of the region bounded by the parabola $y=x^2$ and the line $y=2x+3$. Start by deciding how to slice the region and computing the area of a differential slice.

\item Drag the slider $w$ in Line 1 of the worksheet below from $w=0$ to $w=1$ and describe what you see.

\begin{onlineOnly}
    \begin{center}
\desmos{vci24vc4f1}{450}{600}  
\end{center}
\end{onlineOnly}

\href{https://www.desmos.com/calculator/vci24vc4f1}{152: Area Parabolic Segment}

\item Find an expression for the area of the region bounded by the parabola $y=x^2$ and the line through the points $(a,a^2)$ and $(b,b^2)$.

\end{enumerate}
\end{question}

\begin{question} \label{Q88w88ewe}
Let ${\cal P}$ be the parabola symmetric about the $y$-axis with its vertex at the origin that passes through the point $(a,h)$. Assume $a,h>0$.

\begin{enumerate}

\item Express the area of the region bounded by the parabola and the line $y=h$ in terms of its height $h$ and its base $b=2a$.

\item Check that your expression for the area has the correct units. 

\item Interpret  your expression for the area geometrically. Does it seem reasonable? Compare it with the area of the surrounding rectangle.

\item Find an equation of the horizontal line that bisects the area of the region bounded by ${\cal P}$ and the line $y=h$.

\item Check that your equation is dimensionally correct.

\item Enter your equation for the bisecting line in Line 18 of the worksheet below as a check.

\begin{onlineOnly}
    \begin{center}
\desmos{mbxrsmdho8}{450}{600}  
\end{center}
\end{onlineOnly}

\href{https://www.desmos.com/calculator/mbxrsmdho8}{152: Parabola 23}

\end{enumerate}
\end{question}


\begin{question} \label{Q88w4448ewe}
Let ${\cal P}$ be the parabola symmetric about the $y$-axis with its vertex at the origin that passes through the point $(b,h)$. Assume $b,h>0$.

\begin{enumerate}

\item Find an expression (in terms of $b$ and $h$) for the area of the region bounded by the $x$-axis, the parabola ${\cal P}$, and the tangent line to the parabola at $P$. Check that your expression is dimensionally correct.

\item Compare the area in part (b) with the area of the region bounded by the parabola and the line through the origin and $P$.

\begin{onlineOnly}
    \begin{center}
\desmos{3ulhw9wx4w}{450}{600}  
\end{center}
\end{onlineOnly}

\href{https://www.desmos.com/calculator/3ulhw9wx4w}{152: Parabola and Tangent Line}

\end{enumerate}
\end{question}


\section{Which Area is Greatest? Least?}

\begin{question} \label{QPlerredd}

The vertical segments below are equally-spaced. The horizontal segments cut the shaded region into $20$ smaller regions. Which of these has the greatest area? The least?

\begin{onlineOnly}
    \begin{center}
\desmos{wsgteg55ta}{450}{600}  
\end{center}
\end{onlineOnly}

\href{https://www.desmos.com/calculator/wsgteg55ta}{152: Self-calibrating area}
\end{question}

\begin{question} \label{QPldfefdfsdfd}

The vertical segments below are equally-spaced. The horizontal segments cut the shaded region into $20$ smaller regions. Which of these has the greatest area? The least?

\begin{onlineOnly}
    \begin{center}
\desmos{dir9hpzvse}{450}{600}  
\end{center}
\end{onlineOnly}

\href{https://www.desmos.com/calculator/dir9hpzvse}{152: Self-calibrating area 2}
\end{question}


\section{Trigonometric Functions}

\begin{question} \label{QLkfeREdfd}
Let $a,b,k>0$ be constants with units of cm.

Find a simplified expression for the area of the shaded region below bounded by the $y$-axis and the curves $y=a\cos (x/k)$ and $y=b\sin (x/k)$. Check your expression is dimensionally correct.

\begin{onlineOnly}
    \begin{center}
\desmos{ps1m0hksl0}{450}{600}  
\end{center}
\end{onlineOnly}

\href{https://www.desmos.com/calculator/ps1m0hksl0}{152: Sine Cosine Area}
\end{question}


\section{The Rectangular Elastica}

Let $a>0$ be a constant with units of meters and let
\[
   f(x) = \int_0^a \frac{u^2}{\sqrt{a^4-u^4}} \, du - \int_0^x \frac{u^2}{\sqrt{a^4-u^4}} \, du .
\]

\begin{question} \label{QKDFeefrfr3}

\begin{onlineOnly}
    \begin{center}
\desmos{c3vderglut}{450}{600}  
\end{center}
\end{onlineOnly}

\href{https://www.desmos.com/calculator/c3vderglut}{152: Elastica}

\begin{enumerate}

\item Express the coordinates of the $x$ intercept of the curve $y=f(x)$ in terms of $a$.

\item Express the exact and approximate coordinates of the $y$-intercept in terms of $a$.

\item Find an equation of the tangent line to the curve $u=f(x)$ at the point where the line $x=a/2$ intersects the curve.

\item Find an expression for the area bounded by the curve $y=f(x)$ and the $y$-axis. Activate the \emph{Hint} folder in Line 1 of the worksheet above for a hint. 

\end{enumerate}

\end{question}



\end{document}