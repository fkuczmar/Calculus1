\documentclass{ximera}
\title{Area}

\newcommand{\pskip}{\vskip 0.1 in}

\begin{document}
\begin{abstract}
Computing the area of regions in the plane bounded by graphs of functions.
\end{abstract}
\maketitle

\section{Power Functions}

\begin{question} \label{QWeerFGHG}
Let $a,b>0$ be constants, measured in meters. 

\begin{enumerate}

\item Find the area of the region bounded by the parabola
\[
     y  = b - \frac{x^2}{a} .
\]
Include units and check that your answer is dimensionally correct. Do this twice.

\begin{enumerate}
\item First, by slicing the region perpendicular to the $y$-axis and computing the area of a differential rectangle.

\item Again, by slicing the region perpendicular to the $x$-axis and computing the area of a differential rectangle.
\end{enumerate}
\end{enumerate}
\end{question}

\begin{question} \label{QOLDFKRerbttp}
Express the area of the region shown below bounded by a parabola and a line perpendicular to its axis of symmetry in terms of the height $h$ and the length of the base $b$.

\begin{onlineOnly}
    \begin{center}
\desmos{kkdfpr66rt}{450}{600}  
\end{center}
\end{onlineOnly}

\href{https://www.desmos.com/calculator/kkdfpr66rt}{152: Parabola 1}

\end{question}


We know from the Fundamental Theorem that
\[
   \int_0^b kx^2\, dx = \frac{kb^3}{3}. 
\]

\begin{question} \label{QOidfsfer}
Interpret this result geometrically.
\end{question}



\end{document}