\documentclass{ximera}
\title{Derivatives of Inverse Functions}

\newcommand{\pskip}{\vskip 0.1 in}

\begin{document}
\begin{abstract}
Using the chain rule to compute the derivative of the inverse of a function.
\end{abstract}
\maketitle

We can use the chain rule and our knowledge of the derivative $dy/dx$ of a function $y=f(x)$ to compute the derivative $dx/dy$ of the inverse function $x = f^{-1}(y)$. The Leibniz notation proves very useful here as it suggests that
\begin{align*}
\frac{d}{dy} \left(  f^{-1}(y)  \right)  &= \frac{dx}{dy}  \\
                                                      & = \frac{1}{dy/dx} .
\end{align*}
This turns out to be correct as long as we take care to evaluate the above derivatives at the correct inputs.

Suppose $b = f(a)$, a more precise statement of the relation between the derivative of a function and the derivative of its inverse would be
\begin{align*}
\frac{d}{dy} \left(  f^{-1}(y)  \right)\Big|_{y=b}  &= \frac{dx}{dy} \Big|_{y=b} \\
                                                      & = \frac{1}{\frac{dy}{dx} \Big|_{x=a}} .
\end{align*}

That is, \emph{the derivative of the inverse (of a function) is the reciprocal of the derivative (of that function).}

Here are some examples.




\section*{The Derivative of the Natural Log Function}

\begin{example}   \label{Ex:LDfjbbrt}

(a) Use the graph of the function $y=f(x)$ below to estimate the value of the derivative
\[
    \frac{dy}{dx}\Big|_{x=1.1} =  \frac{dy}{dx}\Big|_{y=3}
\]

(b) Use the result of part (a) to estimate the value of the derivative 
\[
      \frac{dx}{dy}\Big|_{x=1.1} = \frac{dx}{dy}\Big|_{y=3} .
\]

\begin{onlineOnly}
    \begin{center}
\desmos{nnshzdh6jp}{900}{600}
\end{center}
\end{onlineOnly}

(c) Set $u=0.6931$ and $n=90$ in the demonstration above and estimate the value of the derivative
\[
    \frac{dy}{dx}\Big|_{x=0.6931} = \frac{dy}{dx}\Big|_{y=2}
\]

(d) Use the result of part (a) to estimate the value of the derivative 
\[
      \frac{dx}{dy}\Big|_{x=0.6931} = \frac{dx}{dy}\Big|_{y=2} .
\]

\begin{question}  \label{Q:dfrr4f}
Parts (a) and (c) and perhaps a few more derivatives suggest that
\[
     f(x) = \answer{e^x} .
\]
\end{question}

(e) Use the results of parts (b) and (d) to evaluate the derivatives
\[
         \frac{dx}{dy}\Big|_{y=3} = \frac{d}{dy}\left(  \ln y  \right)\Big|_{y=3} ,
\]
\[
         \frac{dx}{dy}\Big|_{y=2} = \frac{d}{dy}\left(  \ln y  \right)\Big|_{y=2}  ,
\]
and more generally
\[
             \frac{dx}{dy}\Big|_{y=a} = \frac{d}{dy}\left(  \ln y  \right)\Big|_{y=a} . 
\]


\begin{question}  \label{Q:Dfdsfgt4gg}
Our conclusion is that
\[
   \frac{d}{dx} \left(  \ln x \right) = \answer{\frac{1}{x}} . 
\]
\end{question}

Geogebra activity available at \href{https://www.desmos.com/calculator/nnshzdh6jp}{151: Magnification Factor 3}

\end{example}


\begin{question}   \label{QDEgfdghb}
Find an equation of the tangent line to the curve $y=\ln x$ at the point $(4,\ln 4)$.
\[
     y - \answer{\ln 4} = \answer{\frac{1}{4}} \left( x - \answer{4}  \right).
\]
\end{question}

\begin{question} \label{Qdfvbtnnn}
(a) Use the chain rule to compute the derivative
\[
    \frac{d}{dx} \left(  \ln (4x) \right) .
\]
(b) Compute the same derivative \emph{without} using the chain rule.

\begin{explanation}
Let 
\[
    y  =\ln (4x)
\]
and
\[
       u = \answer{4x} .
\]
Then 
\[
    y  = \ln \answer{u}
\] 
and
\begin{align*}
   \frac{dy}{dx} & = \frac{dy}{\answer{du}} \cdot \frac{\answer{du}}{\answer{dx}} \\
                       & = \frac{\answer{d}}{\answer{du}} \left( \ln \answer{u}  \right) \cdot \frac{\answer{d}}{\answer{dx}}\left(  \answer{4x} \right)  \\
                        &= \frac{\answer{4}}{u} \\
                        &= \answer{\frac{1}{x}} .
 \end{align*}

(b) We can compute the derivative
\[
    \frac{d}{dx} \left(  \ln (4x) \right) 
\]
wihout the chain rule as follows:
\begin{align*}
    \frac{d}{dx} \left(  \ln (4x) \right)  &=  \frac{d}{dx} \left(  \ln 4 \answer{+} \ln x) \right) \\
                                                       &=  \frac{d}{dx}\left( \ln 4  \right) \answer{+}   \frac{d}{dx}\left( \ln x  \right) \\
                                                      &=\answer{0} + \answer{\frac{1}{x}} .
\end{align*}
\end{explanation}
\end{question}

\begin{question}  \label{Qdfbb44443243}
(a) Compute the derivative 
\[
       \frac{d}{dx} \left( \ln(x^2)  \right)
\]
both with and without the chain rule. Follow the steps \emph{exactly} as in the previous example when using the chain rule.

(b) What are the domains of the functions $f(x)=\ln x$ and $g(x) = \ln (x^2)$ and their derivatives.

(c) Graph the function $g(x) = \ln (x^2)$ and its derivative on the same coordinate system \emph{by hand}.

\end{question}


\end{document}
