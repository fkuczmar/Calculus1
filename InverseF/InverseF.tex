\documentclass{ximera}
\title{Derivatives of Inverse Functions}

\newcommand{\pskip}{\vskip 0.1 in}

\begin{document}
\begin{abstract}
Using the chain rule to compute the derivative of the inverse of a function.
\end{abstract}
\maketitle

We saw at the beginning of the class the relationship between the derivative of a function and the derivative its inverse. Recall the two problems:

\begin{itemize}
\item{You measure the edge length of a cube to be $2$ cm and use this measurement to compute its volume. How is the error in computing the volume related to the error (assumed to be small) in your measurement of the edge length?}

\item{You submerge a cube in water and measure its volume to be $8 \text{ cm}^3$. You then use this measurement to compute the edge length of the cube. How is the error in computing the edge length related to the error (assumed to be small) in your measurement of the volume?}

\end{itemize}

The key to the first problem was to differentiate the function
\[
      V = f(s) = s^3 \, , s>0
\] 
that expresses the volume (in cubic cm) of the cube in terms of its edge length (in cm). We used limits and found that
\[
   \frac{dV}{ds}\Big|_{s=12} = 12 .
\]

\begin{question} \label{Q34rgbds}
(a) What are the units of the above derivative? How do you know?

(b) Interpret the meaning of the above derivative in terms of the geometry of the cube.
\end{question}

From here, we let 
\[
         \Delta s = s- 2
\]
be the error in our measurement (measured in cm) and 
\[
      \Delta V = s^3 -  8
\]
the error in the computed volume (measured in cubic cm). Then for small errors $\Delta s \sim 0$, 
\[
       \frac{\Delta V}{\Delta s} \sim  \frac{dV}{ds}\Big|_{s=2} = \answer{$12$} ,
\]
and so
\[
       \Delta V \sim 12 \Delta s.
\]

For the second problem we used limits to differentiate the function
\[
    s = f^{-1}(V) \, , \, V>0. 
\] 


We can use the chain rule and our knowledge of the derivative $dy/dx$ of a function $y=f(x)$ to compute the derivative $dx/dy$ of the inverse function $x = f^{-1}(y)$. The Leibniz notation proves very useful here as it suggests that
\begin{align*}
\frac{d}{dy} \left(  f^{-1}(y)  \right)  &= \frac{dx}{dy}  \\
                                                      & = \frac{1}{dy/dx} .
\end{align*}
This turns out to be correct as long as we take care to evaluate the above derivatives at the correct inputs.

Suppose $b = f(a)$, a more precise statement of the relation between the derivative of a function and the derivative of its inverse would be
\begin{align*}
\frac{d}{dy} \left(  f^{-1}(y)  \right)\Big|_{y=b}  &= \frac{dx}{dy} \Big|_{y=b} \\
                                                      & = \frac{1}{\frac{dy}{dx} \Big|_{x=a}} .
\end{align*}

That is, \emph{the derivative of the inverse (of a function) is the reciprocal of the derivative (of that function).}

Here are some examples.




\section*{The Derivative of the Natural Log Function}

\begin{example}   \label{Ex:LDfjbbrt}

(a) Use the graph of the function $y=f(x)$ below to estimate the value of the derivative
\[
    \frac{dy}{dx}\Big|_{x=1.1} =  \frac{dy}{dx}\Big|_{y=3}
\]

(b) Use the result of part (a) and the graph below to estimate the value of the derivative 
\[
      \frac{dx}{dy}\Big|_{x=1.1} = \frac{dx}{dy}\Big|_{y=3} .
\]

\begin{onlineOnly}
    \begin{center}
\desmos{nnshzdh6jp}{900}{600}
\end{center}
\end{onlineOnly}

(c) Set $u=0.6931$ and $n=90$ in the demonstration above and estimate the value of the derivative
\[
    \frac{dy}{dx}\Big|_{x=0.6931} = \frac{dy}{dx}\Big|_{y=2}
\]

(d) Use the result of part (c) and the graph above to estimate the value of the derivative 
\[
      \frac{dx}{dy}\Big|_{x=0.6931} = \frac{dx}{dy}\Big|_{y=2} .
\]

\begin{question}  \label{Q:dfrr4f}
Parts (a) and (c) and perhaps a few more derivatives suggest that
\[
     f(x) = \answer{e^x} .
\]
\end{question}

(e) Use the results of parts (b) and (d) to evaluate the derivatives
\[
         \frac{dx}{dy}\Big|_{y=3} = \frac{d}{dy}\left(  \ln y  \right)\Big|_{y=3} ,
\]
\[
         \frac{dx}{dy}\Big|_{y=2} = \frac{d}{dy}\left(  \ln y  \right)\Big|_{y=2}  ,
\]
and more generally
\[
             \frac{dx}{dy}\Big|_{y=a} = \frac{d}{dy}\left(  \ln y  \right)\Big|_{y=a} . 
\]


\begin{question}  \label{Q:Dfdsfgt4gg}
Our conclusion is that
\[
   \frac{d}{dx} \left(  \ln x \right) = \answer{\frac{1}{x}} . 
\]
\end{question}

Geogebra activity available at \href{https://www.desmos.com/calculator/nnshzdh6jp}{151: Magnification Factor 3}

\end{example}

\begin{question}  \label{Qddfs8888}
Here's an equivalent, but more computational way to show that
\[
  \frac{d}{dx} \left(  \ln x  \right) = 1/x.
\]
The key is to recognize that the chain rule tells us that if $u=g(x)$ is a differentiable function of $x$, then
\[
  \frac{d}{dx}\left( e^u  \right) = e^u \cdot \frac{du}{dx} .
\]

Now to compute the derivative above, we know that since the functions $f(x)=\ln x$ and $g(x)=e^x$ are inverses of one another,
\[
    e^{\ln x} = x .
\]
Then differentiate both sides of this equation with respect to $x$ to get
\[
        \frac{d}{dx} \left(  e^{\ln x} \right) = \frac{dx}{dx} .
\]
And by the chain rule we can rewrite this equation as
\[
         \left(  e^{\ln x} \right)   \frac{\answer{d}}{dx} \left(  \answer{\ln x} \right) = \answer{1} . 
\]
And since $e^{\ln x} = \answer{x}$, 
\[
         \frac{d}{dx} \left(  \ln x  \right) = \answer{1/x}.
\]
\end{question}


\begin{question}   \label{QDEgfdghb}
Find an equation of the tangent line to the curve $y=\ln x$ at the point $(4,\ln 4)$.
\[
     y - \answer{\ln 4} = \answer{\frac{1}{4}} \left( x - \answer{4}  \right).
\]
\end{question}

\begin{question} \label{Qdfvbtnnn}
(a) Use the chain rule to compute the derivative
\[
    \frac{d}{dx} \left(  \ln (4x) \right) .
\]
(b) Compute the same derivative \emph{without} using the chain rule.

\begin{explanation}
Let 
\[
    y  =\ln (4x)
\]
and
\[
       u = \answer{4x} .
\]
Then 
\[
    y  = \ln \answer{u}
\] 
and
\begin{align*}
   \frac{dy}{dx} & = \frac{dy}{\answer{du}} \cdot \frac{\answer{du}}{\answer{dx}} \\
                       & = \frac{\answer{d}}{\answer{du}} \left( \ln \answer{u}  \right) \cdot \frac{\answer{d}}{\answer{dx}}\left(  \answer{4x} \right)  \\
                        &= \frac{\answer{4}}{u} \\
                        &= \answer{\frac{1}{x}} .
 \end{align*}

(b) We can compute the derivative
\[
    \frac{d}{dx} \left(  \ln (4x) \right) 
\]
wihout the chain rule as follows:
\begin{align*}
    \frac{d}{dx} \left(  \ln (4x) \right)  &=  \frac{d}{dx} \left(  \ln 4 \answer{+} \ln x \right) \\
                                                       &=  \frac{d}{dx}\left( \ln 4  \right) \answer{+}   \frac{d}{dx}\left( \ln x  \right) \\
                                                      &=\answer{0} + \answer{\frac{1}{x}} .
\end{align*}
\end{explanation}
\end{question}

\begin{question}  \label{Qdfbb44443243}
(a) Compute the derivative 
\[
       \frac{d}{dx} \left( \ln(x^2)  \right)
\]
both with and without the chain rule. Follow the steps \emph{exactly} as in the previous example when using the chain rule.

(b) What are the domains of the functions $f(x)=\ln x$ and $g(x) = \ln (x^2)$ and their derivatives.

(c) Graph the function $g(x) = \ln (x^2)$ and its derivative on the same coordinate system \emph{by hand}.

\end{question}

\begin{question}  \label{Qdefrgbhrdtgr}
(a) Use the chain rule to compute the derivative
\[
   \frac{d}{dx} \left(  |x| \right) .
\]
Do this by noting that
\[
|x| = 
\begin{cases}
    x , \text{ if } x\geq 0 \\
   -x ,  \text{ if } x<0
\end{cases}
\]
and using the chain rule to compute 
\[
  \frac{d}{dx}  \left(  -x \right).
\]

(b) A reflection about the $y$-axis takes the graph of $y=f(x)$ go the graph of $y=g(x)$. Describe the transformation that takes the graph of $y=f^\prime(x)$ to the graph of $y=g^\prime(x)$.

\end{question}

\begin{question} \label{Qmf4566544}
(a) Find a function
\[
      T = f(k) , \, k\geq 1,
\]
that expresses the time (in years) it takes an investment to grow by a factor of $k$. Assume the investment grows at a constant relative instantaneous rate of $i\%$/yr. So, for example if $i=5$, $f(3)$ would be the time it takes an investment to triple at an relative instantaneous growth rate of $5\%$/yr.

(b) Suppose $i=5$ and evaluate the derivative
\[
    \frac{dT}{dk} \Big|_{k=3} .
\]

(c) What are the units of the above derivative? Explain its meaning in terms of small changes.

(d) Use the result of part (b) and the appropriate linear approximation to estimate how much longer it would take an investment growing at an relative instantaneous rate of $5\%$/yr to increase by $210\%$ than it would take to triple.
\end{question}


\section*{The Inverse Sine Function}


\end{document}
