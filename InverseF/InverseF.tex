\documentclass{ximera}
\title{Derivatives of Inverse Functions}

\newcommand{\pskip}{\vskip 0.1 in}

\begin{document}
\begin{abstract}
Using the chain rule to compute the derivative of the inverse of a function.
\end{abstract}
\maketitle


\section*{Introduction}
We saw at the beginning of the class the relationship between the derivative of a function and the derivative its inverse. Recall the two problems:

\begin{itemize}
\item{You measure the edge length of a cube to be $2$ cm and use this measurement to compute its volume. How is the error in computing the volume related to the error (assumed to be small) in your measurement of the edge length?}

\item{You submerge a cube in water and measure its volume to be $8 \text{ cm}^3$. You then use this measurement to compute the edge length of the cube. How is the error in computing the edge length related to the error (assumed to be small) in your measurement of the volume?}

\end{itemize}

The key to the first problem was to differentiate the function
\[
      V = f(s) = s^3 \, , s>0
\] 
that expresses the volume (in cubic cm) of the cube in terms of its edge length (in cm). We used limits and found that
\[
   \frac{dV}{ds}\Big|_{s=2} = \answer{12}.
\]

\begin{question} \label{Q34rgbds}
(a) What are the units of the above derivative? How do you know?

(b) Interpret the meaning of the above derivative in terms of the geometry of the cube.
\end{question}


\begin{question}  \label{QDDFDSFet}
From here, we let 
\[
         \Delta s = s- \answer{2}
\]
be the error in our measurement (measured in cm) and 
\[
      \Delta V = s^3 -  \answer{8}
\]
the error in the computed volume (measured in cubic cm). Then for small errors $\Delta s \sim 0$, 
\[
       \frac{\Delta V}{\Delta s} \sim  \frac{dV}{ds}\Big|_{s=2} = \answer{12} ,
\]
and so
\[
       \Delta V \sim \answer{12} \Delta s.
\]

For the second problem we used limits to differentiate the function
\[
    s = f^{-1}(V) \, , \, V>0 ,
\] 
and found that
\[
    \frac{ds}{dV}\Big|_{V=\answer{8}} = \answer{1/12}. 
\]
\begin{question}  \label{Q:Lldfb9833}
What are the units of the above derivative? How do you know?
\end{question}

Then for small errors  $\Delta \answer{V}$ in our measurement of the volume,
\[
  \Delta s \sim \answer{\frac{1}{12}} \Delta V
\]

But all this work took us back to the same approximate relation between the errors $\Delta V$ and $\Delta s$ that we had already found in solving the first problem.

Expressed another way, having found that
\[
        \frac{dV}{ds}\Big|_{s=2} = \answer{12},
\]
we should have known immediately (without any computations) that
\[
        \frac{ds}{dV}\Big|_{V=\answer{8}} = \answer{1/12}.
\]



For the same reason, this same relationship between the derivative of any function and the derivative of its inverse holds. That is, for any differentiable function $y=f(x)$, the derivative $dx/dy$ of the inverse function $x = f^{-1}(y)$ is
\begin{align*}
\frac{d}{dy} \left(  f^{-1}(y)  \right)  &= \frac{dx}{dy}  \\
                                                      & = \frac{1}{dy/dx} .
\end{align*}
Well almost. We need to take some care in evaluating the above derivatives at the correct inputs.

Supposing that $b = f(a)$, a more precise statement of the relation between the derivative of a function $y=f(x)$ and the derivative $dx/dy$ of its inverse is
\begin{align*}
\frac{d}{dy} \left(  f^{-1}(y)  \right)\Big|_{y=\answer{b}}  &= \frac{dx}{dy} \Big|_{y=\answer{b}} \\
                                                      & = \frac{1}{\frac{dy}{dx} \Big|_{x=\answer{a}}} .
\end{align*}

That is, \emph{the derivative of the inverse (of a function) is the reciprocal of the derivative (of that function).}

Well, not quite. The above relationship is true if
\[
  \frac{dy}{dx} \Big|_{x=\answer{a}} \neq \answer{0}.
\]
This condition also guarantees that the function $y=f(x)$ is one-to-one in a sufficiently small neighborhood near $x=\answer{a}$ and is therefore invertible in that neighborhood.

\end{question}

Here are some examples.




\section*{The Derivative of the Natural Log Function}

\begin{example}   \label{Ex:LDfjbbrt}

(a) Use the graph of the function $y=f(x)$ below to estimate the value of the derivative
\[
    \frac{dy}{dx}\Big|_{x=1.1} =  \frac{dy}{dx}\Big|_{y=3} \sim \answer[tolerance=0.2]{3}
\]

(b) Use the result of part (a) and the graph below to estimate the value of the derivative 
\[
      \frac{dx}{dy}\Big|_{x=1.1} = \frac{dx}{dy}\Big|_{y=3} \sim \answer[tolerance=0.1]{1/3} .
\]

\begin{onlineOnly}
    \begin{center}
\desmos{nnshzdh6jp}{900}{600}
\end{center}
\end{onlineOnly}

(c) Set $u=0.6931$ and $n=90$ in the demonstration above and estimate the value of the derivative
\[
    \frac{dy}{dx}\Big|_{x=0.6931} = \frac{dy}{dx}\Big|_{y=2} \sim \answer[tolerance=0.2]{2}
\]

(d) Use the result of part (c) and the graph above to estimate the value of the derivative 
\[
      \frac{dx}{dy}\Big|_{x=0.6931} = \frac{dx}{dy}\Big|_{y=2} \sim \answer[tolerance=0.1]{0.5}
\]

\begin{question}  \label{Q:dfrr4f}
Parts (a) and (c) and perhaps a few more derivatives suggest that
\[
     f(x) = \answer{e^x} .
\]
\end{question}

(e) Use the results of parts (b) and (d) to evaluate the derivatives
\[
         \frac{dx}{dy}\Big|_{y=3} = \frac{d}{dy}\left(  \ln y  \right)\Big|_{y=3} = \answer{1/3} ,
\]
\[
         \frac{dx}{dy}\Big|_{y=2} = \frac{d}{dy}\left(  \ln y  \right)\Big|_{y=2} = \answer{1/2} ,
\]
and more generally
\[
             \frac{dx}{dy}\Big|_{y=a} = \frac{d}{dy}\left(  \ln y  \right)\Big|_{y=a} = \answer{1/a} . 
\]


\begin{question}  \label{Q:Dfdsfgt4gg}
Our conclusion is that
\[
   \frac{d}{dx} \left(  \ln x \right) = \answer{\frac{1}{x}} . 
\]
\end{question}

Geogebra activity available at \href{https://www.desmos.com/calculator/nnshzdh6jp}{151: Magnification Factor 3}

\end{example}

\begin{question}  \label{Qddfs8888}
Here's an equivalent, but more computational way to show that
\[
  \frac{d}{dx} \left(  \ln x  \right) = 1/x.
\]
The key is to recognize that the chain rule tells us that if $u=g(x)$ is a differentiable function of $x$, then
\[
  \frac{d}{dx}\left( e^u  \right) = e^u \cdot \frac{du}{dx} .
\]

Now to compute the derivative above, we know that since the functions $f(x)=\ln x$ and $g(x)=e^x$ are inverses of one another,
\[
    e^{\ln x} = x .
\]
Then differentiate both sides of this equation with respect to $x$ to get
\[
        \frac{d}{dx} \left(  e^{\ln x} \right) = \frac{dx}{dx} .
\]
And by the chain rule we can rewrite this equation as
\[
         \left(  e^{\ln x} \right)   \frac{\answer{d}}{dx} \left(  \answer{\ln x} \right) = \answer{1} . 
\]
And since $e^{\ln x} = \answer{x}$, 
\[
         \frac{d}{dx} \left(  \ln x  \right) = \answer{1/x}.
\]
\end{question}


\begin{question}   \label{QDEgfdghb}
Find an equation of the tangent line to the curve $y=\ln x$ at the point $(4,\ln 4)$.
\[
     y - \answer{\ln 4} = \answer{\frac{1}{4}} \left( x - \answer{4}  \right).
\]
\end{question}

\begin{question} \label{Qdfvbtnnn}
(a) Use the chain rule to compute the derivative
\[
    \frac{d}{dx} \left(  \ln (4x) \right) .
\]
(b) Compute the same derivative \emph{without} using the chain rule.

\begin{explanation}
Let 
\[
    y  =\ln (4x)
\]
and
\[
       u = \answer{4x} .
\]
Then 
\[
    y  = \ln \answer{u}
\] 
and
\begin{align*}
   \frac{dy}{dx} & = \frac{dy}{\answer{du}} \cdot \frac{\answer{du}}{\answer{dx}} \\
                       & = \frac{\answer{d}}{\answer{du}} \left( \ln \answer{u}  \right) \cdot \frac{\answer{d}}{\answer{dx}}\left(  \answer{4x} \right)  \\
                        &= \frac{\answer{4}}{u} \\
                        &= \answer{\frac{1}{x}} .
 \end{align*}

(b) We can compute the derivative
\[
    \frac{d}{dx} \left(  \ln (4x) \right) 
\]
wihout the chain rule as follows:
\begin{align*}
    \frac{d}{dx} \left(  \ln (4x) \right)  &=  \frac{d}{dx} \left(  \ln 4 \answer{+} \ln x \right) \\
                                                       &=  \frac{d}{dx}\left( \ln 4  \right) \answer{+}   \frac{d}{dx}\left( \ln x  \right) \\
                                                      &=\answer{0} + \answer{\frac{1}{x}} .
\end{align*}
\end{explanation}
\end{question}

\begin{question}  \label{Qdfbb44443243}
(a) Compute the derivative 
\[
       \frac{d}{dx} \left( \ln(x^2)  \right)
\]
both with and without the chain rule. Follow the steps \emph{exactly} as in the previous example when using the chain rule.

(b) What are the domains of the functions $f(x)=\ln x$ and $g(x) = \ln (x^2)$ and their derivatives.

(c) Graph the function $g(x) = \ln (x^2)$ and its derivative on the same coordinate system \emph{by hand}.

\end{question}

\begin{question}  \label{Qdefrgbhrdtgr}
(a) Use the chain rule to compute the derivative
\[
   \frac{d}{dx} \left( \ln |x| \right) .
\]
Do this by noting that
\[
|x| = 
\begin{cases}
    x , \text{ if } x\geq 0 \\
   -x ,  \text{ if } x<0
\end{cases}
\]
and using the chain rule to compute 
\[
  \frac{d}{dx}  \left(  -x \right).
\]

(b) A reflection about the $y$-axis takes the graph of a function $y=f(x)$ to the graph of a function $y=g(x)$. Describe the transformation that takes the graph of $y=f^\prime(x)$ to the graph of $y=g^\prime(x)$. Explain your reasoning.

\end{question}

\begin{question} \label{Qmf4566544}
(a) Find a function
\[
      T = f(k) , \, k\geq 1,
\]
that expresses the time (in years) it takes an investment to grow by a factor of $k$. Assume the investment grows at a constant relative instantaneous rate of $i\%$/yr. So, for example if $i=5$, $f(3)$ would be the time it takes an investment to triple at an relative instantaneous growth rate of $5\%$/yr.

\begin{hint}
Let $B_0$ be the initial investment. Then the value of the investment $T$ years later is 
\[
      kB_0  = B_0 e ^{\frac{i}{100}T} .
\]
Solve this equation for $t$ to find an expression for the function $f$.
\end{hint}

(b) Suppose $i=5$ and evaluate the derivative
\[
    \frac{dT}{dk} \Big|_{k=3} .
\]

(c) What are the units of the above derivative? Explain its meaning in terms of small changes.

(d) Use the result of part (b) and the appropriate linear approximation to estimate how much longer it would take an investment growing at an relative instantaneous rate of $5\%$/yr to increase by $210\%$ than it would take to triple.
\end{question}


\begin{question}  \label{Qd654thmm}
The function 
\[ 
      G = f(v) = 40-0.08\left(\frac{v}{2}-25\right)^{2}\, \ , \, 25\leq v\leq 65 ,
\]
expresses the gas mileage of a car (in miles/gallon) in terms of its speed (in miles/hr).

(a) Explain the meaning of the derivative 
\[
     \frac{d}{dv} \left( \ln (f(v)) \right) .
\]
Include units in your explanation. Also, what are the units of $25$ in the above expression? How do you know?

(b) Evaluate the above derivative at $v=30$ and explain its meaning in terms of small changes.

\end{question}


\begin{question}   \label{Qdfdfnnn}
The function
\[
    v =g(h) \, , \, 50\leq h \leq 200 , 
\]
expresses the speed (in ft/sec) of a hawk in terms of its altitude (in feet) during a portion of its flight.

Suppose that $f(150) = 80$ and
\[
     \frac{dv}{dh}\Big|_{h=150} = -2 .
\]

(a) What are the units of the above derivative? Explain the meaning of the derivative in the context of small changes in this particular scenario. Be specific.

(b) Evaluate the derivative
\[
    \frac{d}{dh} \left( \ln (g(h))\right)\Big|_{h=150}.
\]
Show all work and explain your reasoning.

(c) What are the units of the derivative in part (b)? Explain the meaning of the derivative in the context of small changes in this particular scenario. Be specific.

\end{question}


\section*{The Inverse Sine Function}
Let
\[
  \theta = g(y) = \arcsin y = \sin^{-1}(y).
\]

(a) What is the domain of the function $g$?

(b) What is its range?

(c) Explain the meaning of $\arcsin y$.

(d) True or false: The inverse sine function is the inverse of the sine function.

(e) What function is the inverse of the inverse sine function?

\begin{question}   \label{Q:34fgt44}
The graph of the function
\[
 y  = f(\theta) = \sin \theta \, , \, -\pi/2 \leq \theta \leq \pi/2 ,
\]
is shown below.

\begin{onlineOnly}
    \begin{center}
\desmos{lxwoeir1pt}{900}{600}
\end{center}
\end{onlineOnly}

Worksheet available at \href{https://www.desmos.com/calculator/lxwoeir1pt}{151: Arc Sine}


(a) Use the graph to estimate the derivative 
\[
      \frac{d}{dy} \left(   \arcsin y  \right)\Big|_{y=0.8} .
\]
Explain your reasoning.

(b) Use the graph and the slider $u$ to estimate the derivative
\[
      \frac{d}{dy} \left(   \arcsin y  \right)\Big|_{y=0.8} .
\]

(c) Express the derivative of the function 
\[
 y  = f(\theta) = \sin \theta \, , \, -\pi/2 \leq \theta \leq \pi/2 ,
\]
in terms of $y$.

(d) Use part (c) and the ideas in parts (a),(b) to find an expression for the derivative
\[
   \frac{d}{dy} \left(   \arcsin y  \right) .
\]

(e) Check your answer to part (d) by evaluating the derivative at $y=0.6, 0.8$.

(f) What is the domain of the derivative in part (d)?
\end{question}


\begin{question}  \label{Q:dfbhhyh5g5tr231}

The top of a 25-foot long ladder slides down a vertical wall at the constant rate of $4$ ft/sec. 

(a) Find a function 
\[
       \theta = f(t) \, , \, 0\leq t \leq 6.25 ,
\]
that expresses the angle the ladder makes with the ground (measured in radians) in terms of the number of seconds since the ladder was in the vertical position.

(b) Find the rotation rate of the ladder when the top of the ladder is 

     (i) $15$ feet above the ground. 

     (ii) $24$ feet above the ground.

     (iii) $24.9$ feet above the ground.


(c) Solve this problem again by working directly with the sine function, not the arcsine function.

 


\begin{onlineOnly}
    \begin{center}
\desmos{5c4lssovbi}{900}{600}
\end{center}
\end{onlineOnly}

Worksheet available at \href{https://www.desmos.com/calculator/5c4lssovbi}{151: Ladder and ArcSine}

\end{question}


\begin{question}  \label{Q:dsdfkkkfdvg4}

The bottom end of a 25-foot long ladder slides across a horizontal floor at the constant rate of $4$ ft/sec as the top end slides down a vertical wall. 

(a) Find a function
\[
      \theta = g(u) \, , \, 0\leq u \leq 25 ,
\]
that expresses the angle the ladder makes with the wall (measured in radians) in terms of the distance (in feet) between the wall 
and the bottom end of the ladder.

(b) Use the slider $u$ in the animation below to approximate each of the following derivatives. Include units. Note that the tick marks on the radian protractor are spaced at intervals of $0.1$ radians.

   (i) $\frac{d\theta}{du} \Big|_{u=1}$

   (ii) $\frac{d\theta}{du} \Big|_{u=15}$

    (iii) $\frac{d\theta}{du} \Big|_{u=24.9}$

(c) What do the above derivatives tell you?

(d) Find an expression for the derivative $d\theta/du$. Use your expression to evaluate the three derivatives in part (b) and compare these with your estimates.

(e) Find a function 
\[
       \theta = f(t) \, , \, 0\leq t \leq 6.25 ,
\]
that expresses the angle the ladder makes with the ground (measured in radians) in terms of the number of seconds since the ladder was in the vertical position.

(f) Find an expression for the derivative 
\[
   r=\omega(t)=d\theta/dt. 
\]
Interpret its meaning. Include units.

(g) Find the rotation rate of the ladder when the bottom end of the ladder is 

     (i) $1$ foot from the wall. 

     (ii) $15$ feet from the wall.

     (iii) $24.9$ feet above the wall.

\pskip

(h) Evaluate the limit
\[
   \lim_{t\to 6.25} \omega(t) 
\]
and interpret its meaning.
 


\begin{onlineOnly}
    \begin{center}
\desmos{g6fuhqff02}{900}{600}
\end{center}
\end{onlineOnly}

Worksheet available at \href{https://www.desmos.com/calculator/g6fuhqff02}{151: Ladder and ArcSine 2}

\end{question}


\section*{Exercises}

\begin{question}  \label{Qfdgtnn}
(a) Simplfiy the derivative
\[
     \frac{d}{d\theta}\left(  \arcsin(\sin \theta) \right)
\]

(b) Use the result of part (a) to graph the function
\[
        y = \arcsin (\sin\theta) .
\] 
Explain your reasoning.
\end{question}

\begin{question} \label{QWcsdefv3e354}
Find the measure of the acute angle that the tangent line to the curve
\[
      y = f(\theta) = \ln |\sec\theta|
\]
at the point $(\pi/7, f(\pi/7))$ makes with the $x$-axis Do \emph{not} use a calculator.
\end{question}


\begin{question}   \label{QDfd6mbqs}
The function
\[
       q = f(p) = 0.2 \left( 2p-40 \right)^2 \, , \, 5\leq p \leq 12 ,
\]
expresses the average number of burgers sold per day at Five Guys in Edmonds in terms of the price (in dollars/burger).

(a) Evaluate the derivative
\[
        \frac{d}{dp} \left(  \ln (f(p)) \right)\Big|_{p=7.5}
\]

(b) What are the units of the derivative above? Explain its meaning.
\end{question}


\begin{question}  \label{Q:3e4fg45534}
The bottom end of a $25$-foot ladder lies $24$ feet from the base of a vertical wall. Use the appropriate linear approximation to estimate the angle through which the ladder rotates when the bottom end is pulled an additional $0.1$ feet away from the wall along a horizontal floor. 

Solve this problem twice, first using an inverse trig function and again without an inverse trig function. 

Start by defining your variables, with units.

Use a calculator if need be, but \emph{only} for arithmetic and not to evaluate any trigonometric functions.

Compare your estimate with the actual angle of rotation.

\end{question}

\begin{question}  \label{Qdsfgnbn444}
A tree leans precariously with its trunk making an angle of $\phi = \pi/3$ radians with the ground. One end of a $14$-foot ladder leans against the trunk, the other rests on the horizontal ground. The bottom end of the ladder is pulled away from the trunk at the constant speed of $4$ ft/sec. At what rate is the ladder rotating when the bottom and tops ends are respectively $16$ and $10$ feet from the base of the trunk?

\begin{hint}
Use the law of sines.
\end{hint}

\begin{onlineOnly}
    \begin{center}
\desmos{rpms2jqfpm}{900}{600}
\end{center}
\end{onlineOnly}

Desmos activity available at \href{https://www.desmos.com/calculator/rpms2jqfpm}{151: Tree and Ladder}

\end{question}

\begin{question}  \label{Q:dgynnrzaq}
(a) The animation below shows water draining from a tank. Play the animation and sketch by hand a graph of the function $V=f(t)$ that expresses the depth of the water as a function of time. Explain your reasoning. Label the axes with units and the appropriate variable names.


\pdfOnly{
Access Desmos interactives through the online version of this text at
 
\href{https://www.desmos.com/calculator/pdghky6tie}.
}
 
\begin{onlineOnly}
    \begin{center}
\desmos{pdghky6tie}{900}{600}
\end{center}
\end{onlineOnly}

(b) Torricelli's law says that the rate, say in $\text{cm}^3$/sec, at which water drains out of a small hole in the bottom of a tank is proportional to the square root of the depth of the water. So if $V=f(h)$ is a function that expresses the volume (in $\text{cm}^3$) of water in the tank in terms of the depth (in feet) and $h=g(t)$ is a function that expresses the depth of the water (in feet) in terms of number of seconds past noon, then 
\[
    \frac{d{\answer{V}}}{d\answer{t}} = - k \answer{\sqrt{h}} 
\]
for some positive constant $k$.

For a cylindrical tank of radius $r$ cm, 
\[
     V = f(h) = \answer{\pi r^2h} 
\]
and 
\[
    \frac{dV}{dh} = \answer{\pi r^2}.
\]
So by the chain rule
\begin{align*}
 \frac{dV}{dt} &= \frac{dV}{\answer{dh}} \cdot \frac{\answer{dh}}{\answer{dt}} \\
                     &= \answer{\pi r^2} \frac{\answer{dh}}{\answer{dt}} . \\
\end{align*}
So for the cylindrical tank we can write Torricelli's law as
\[
   \frac{dV}{dt} =  \answer{\pi r^2} \frac{d\answer{h}}{d\answer{t}} = - k \sqrt{h} .
\]
Or equivalently as
\[
      \frac{dh}{dt} = - \frac{k}{\answer{\pi r^2}} \sqrt{h} = -k_2 \sqrt{h} ,
\]
where
\[
       k_2= \frac{k}{\pi r^2}
\]
is a positive constant.

(c) Which of the following functions might express the depth of water in a cylindrical tank (in terms of time) as the water drains out of a small hole in the bottom of the tank? Justify your reasoning.

(i) $h = g(t) = 2 (5-t)^3 , 0\leq t \leq 5$

(ii) $h = g(t) = 2(5-t)^2, 0\leq t \leq 5$

\pskip

\emph{Solution:}

The key idea is to express the derivative $dh/dt$ as a function of the depth $h$ of the water. 

(i) If $h=2(5-t)^3$, then
\begin{align*}
    \frac{dh}{dt} &=\frac{d}{dt}\left( 2(5-t)^3  \right)  \\ 
                        &=     2(3)(5-t)^2  \cdot \frac{d}{dt} \left( 5-t  \right)  \\
                        &= \answer{-6 (5-t)^2} . 
\end{align*}
 
Now to express $dh/dt$ in terms of $h$, solve the equation 
\[
     h = 2(5-t)^3
\]
for $t$ to get
\[
     t = 5 - \left( \frac{h}{2} \right)^{1/3} .
\]
Then, substitute this expression for $t$ into the derivative $dh/dt$:
\begin{align*}
      \frac{dh}{dt} &= -6 (5-t)^2 \\
                          &= -6 \left( \frac{h}{2} \right)^{2/3} \\
                          &= - \left( \frac{6}{2^{2/3}}\right) h^{2/3} .
\end{align*}
This tells us that the rate of change in the depth of the water is \emph{not} proportional to the square root of the water's depth  as Torricelli's law requires. So the function
\[
      h = 2(5-t)^3, 0\leq t \leq 5
\]
is not a possible depth function for water draining from a cylindrical tank.

\begin{freeResponse}
Give a similar analysis for the depth function of part (ii).
\end{freeResponse}

\end{question}

\begin{question}  \label{Q:435gbbrtgt}
This is a continuation of the previous problem.

Now we pour water into a cylindrical tank at a constant rate, while at the same time water leaks out through a small hole in the bottom of the tank. We'll suppose that the tank starts with some initial volume of water.

\begin{freeResponse}

(a) What do you think happens to the water level in the tank initially?

(b) What do you think happens to the water level in the long run?

\end{freeResponse}

To model this situtation, we need to modify Torricelli's law. For this, let's suppose that we pour water into the cylindrical tank (of radius $r$) at the constant rate of $k_3 \text{ cm}^3$/sec. Then with the same notation as before,
\[
      \frac{dV}{dt} = \answer{k_3} - k\sqrt{h} . 
\] 
But since 
\begin{align*}
        \frac{dV}{dt} &= \frac{dV}{dh} \cdot \frac{dh}{dt}   \\
                            & = \answer{\pi r^2} \cdot \frac{dh}{dt} ,
\end{align*}
the above modification of Torricelli's law becomes
\begin{align*}
    \frac{dh}{dt} &=  \frac{k_3}{\answer{\pi r^2}} - \frac{k}{\answer{\pi r^2}}\sqrt{h} \\
                        &= k_1 - k_2 \sqrt{h} ,
\end{align*}
where 
\[
    k_1  = \frac{k_3}{\pi r^2} 
\]
and
\[
        k_2 = \frac{k_1}{\pi r^2} 
\]
are positive constants. 

\begin{freeResponse}
(a) What are the units of $k_1$ and $k_2$? How do you know?

(b) Explain the meaning of $k_1$.
\end{freeResponse}

The equation
\[
   \frac{dh}{dt} =  k_1 - k_2 \sqrt{h} 
\]
expresses the rate of change in the water's depth as a function of the depth. It is called a differential equation and you will learn a little about how to solve equations like this next quarter. 
 
For this particular differential equation, it is not possible to express the depth of the water explicitly as a function of time. But assuming that the depth of the water is $h_0$ cm at time $t=0$, it turns out that the function 
\[
      t = g^{-1}(h) = - \frac{2}{k_2}  \left( \sqrt{h} - \sqrt{h_0} + \frac{k_1}{k_2} \ln \Bigg| \frac{k_1 - k_2 \sqrt{h}}{k_1 - k_2 \sqrt{h_0}}   \Bigg| \right)\, , \, t\geq 0 ,
\]
expresses the time (in seconds) in terms of the depth of the water (in cm). We can check that this is indeed correct as follows:

(a) Show algebraically that the depth of the water at time $t=0$ is $h=h_0$.

(b) Use the above expression for $t=g^{-1}(h)$ to compute and then simplify the derivative $dt/dh$. 

(c) Use the result of part (b) to show that 
\[
   \frac{dh}{dt} =  k_1 - k_2 \sqrt{h} .
\]

\begin{onlineOnly}
    \begin{center}
\desmos{c78kv7wifv}{900}{600}
\end{center}
\end{onlineOnly}

Worksheet available at \href{https://www.desmos.com/calculator/c78kv7wifv}{151: Draining Cylinder 2}

\pskip

(d) Experiment with the sliders above and summarize your observations about how the graph of the function $h=g(t)$ changes depending on the initial depth of the water and the constants $k_1$ and $k_2$. 

(e) Express the equilibrium depth in terms of $k_1$ and $k_2$. Check that your expression has the correct units. The equilibrium depth is the depth at which the water level remains constant. It is also the depth which the water level approaches (independent of the initial depth).

(f) What happens to the equilibrium depth when $k_1$ increases (and $k_2$) is held constant? When $k_2$ changes and $k_1$ is held constant?

\end{question}


\end{document}
