\documentclass{ximera}
\title{152 Review}

\newcommand{\pskip}{\vskip 0.1 in}

\begin{document}
\begin{abstract}
Some problems for review.
\end{abstract}
\maketitle


\begin{question} \label{QOdfe4gbr}
The function 
\[
 r = f(t) \, , \, t\geq 0,
\]
expresses the rate (in gal/hr) at which water flows into a tank in terms of the number of hours past noon, July 23, 2025.

The tank holds 500 gallons at 3pm on July 23, 2025.

\begin{enumerate}
\item Find a function
\[
   V = g(t) \, , \, t\geq 0, 
\]
that expresses the volume of water (measured in gallons) in the tank in terms of terms of the number of hours past noon, July 23, 2025.

\item Suppose now that
\[
     r = f(t) = \frac{200}{(2 + \frac{t}{10})^2} \, , \, t\geq 0.
\]

\begin{enumerate}
\item Find an explicit expression for the function $g$.

\item Assuming the water flows forever, determine if there is a tank that can hold all the water without overflowing? If so, compute the minimum volume of such a tank. If not, explain why.

Use limits to justify your response.
\end{enumerate}

\end{enumerate}
\end{question}

\begin{question} \label{QKdmfe3dv}
The continuous function 
\[
   v = f(s) \, , \, 20\leq s \leq 70,
\]
expresses the speed of a car (in miles/hour) in terms of the trip odometer reading (in miles) during a portion of a 200-mile trip.

The trip odometer reads $50$ miles at 4:00pm.

\begin{enumerate}
\item Find a function
\[
    t = g(s) \, , \, 20\leq s \leq 70
\]
that expresses the time (measured in hours past noon) in terms of the trip odometer reading. \emph{Begin your solution by analyzing the constant case.}

\item Suppose now that the speed is a linear function of the odometer reading during this $50$-mile section of the trip. Suppose also the speed is $60$ miles/hour when the odometer reads $20$ miles and $40$ miles/hour when the odometer reads $70$ miles.

\begin{enumerate}
\item Find an explicit expression for the function $t=g(s)$.

\item Find the exact clock time when the odometer reads $70$ miles. Then use a calculator to approximate this time to the nearest minute.
\end{enumerate}
\end{enumerate}


\begin{explanation}
\begin{enumerate}
\item If the speed is constant at $v$ miles/hour, then the time it takes to travel $\Delta s$ miles is
\[
   \Delta t  =  \frac{\Delta s \text{ miles }}{v \text{ miles/hr}} \text{ hrs}  = \frac{\Delta s}{v} \text{ hrs}.
\]

If the speed is not constant, we divide the stretch of the trip from odometer reading $50$ miles to odometer reading $s$ miles into small intervals of length $ds$ miles over which the speeds are constant. The time to travel travel the distance $ds$ miles at a speed of $v = f(s)$ miles/hr is
\[
     dt = \frac{ds}{v} \text{ hrs}.
\]
Adding these differential times gives the time (in hours) to travel from odometer reading $50$ miles to odometer reading $s$ miles as
\[
     \Delta t = \int_{50}^s \frac{ds}{f(s)} .
\]

Since the odometer reads $50$ miles at time $t=4$ hours past noon, when the odometer reading is $s$ miles, the time is
\begin{align*}
    t  &= g(s)  \\
        &= 4 + \Delta t \\
        &= 4 + \int_{50}^s \frac{1}{f(x)} \, dx .
\end{align*}

\item The function
\begin{align*}
      v  &= f(s)   \\
          &= 60 - \frac{2}{5} \left( s - 20 \right) \\
          & = 68 - \frac{2}{5}s   \, , \, 20\leq s \leq 70, 
\end{align*}
expresses the speed (in miles/hr) in terms of the odometer reading (in miles). So by part (a)
\[
     t = g(s) = 4 + \int_{50}^s \frac{1}{68 - \frac{2}{5}x} \, dx.
\]

Making the substitution  
\[
    u=68 - \frac{2}{5}x 
\]
gives (details left for you)
\begin{align*}
     t  &= g(s) \\
        &= 4 - \int_{50}^s \frac{1}{68 - \frac{2}{5}x} \, dx \\
        &= 4 - \frac{5}{2}\int_{48}^{68-\frac{2}{5}s} \frac{1}{u}  \, du\\
        & = 4 -   \left( \frac{5}{2}\ln |u|  \Big|_{u=48}^{u=68 - \frac{2}{5}s} \right)  \\
        &= 4 - \frac{5}{2}\ln\left( \frac{68 - \frac{2}{5}s}{48} \right) .
\end{align*}

So at the odometer reading $s=70$ miles, the time is
\begin{align*}
  t &= g(70)  \\
     & = 4 - \frac{5}{2}\ln\left( \frac{68 - \frac{2}{5}(70)}{48} \right) \\
     &= 4 -\frac{5}{2}\ln\left( \frac{40}{48}\right) \\
     & \sim  4.45 
\end{align*}
hours past noon. 

\end{enumerate}

\end{explanation}
\end{question}

\begin{question} \label{Q9de3rmre}
Use trigonometric functions to parameterize the circle of radius $a$ cm centered at the origin. Then use the parameterization to compute

\begin{enumerate}
\item the sphere's volume and

\item the sphere's surface area.
\end{enumerate}

\begin{explanation}
 The parameterization in terms of the polar angle $\theta$ of the vector $\overrightarrow{OP}$ from the origin to a point $P(x,y)$ on the circle is 
\[
   (x,y) = (a\cos\theta, a \sin \theta) \, , \, 0\leq \theta \leq 2\pi.
\]
\begin{enumerate}
\item Slice the sphere perpendicular to the $y$-axis, the cross section is a circle with radius $x = a\cos\theta$ and area
\[
 A = \pi a^2\cos^2\theta .
\]
The differential slice has width
\[
  dy = d(a\sin\theta) = a\cos\theta \, d\theta
\]
and volume
\[
  dV = A\, dy = \pi a^3\cos^3\theta \, d\theta .
\]

So the volume of the sphere is 
\begin{align*}
   V &= 2\pi a^3\int_0^{\pi/2} \cos^3\theta \, d\theta \\
       &= 2\pi a^3 \int_0^{\pi/2} \cos^2\theta  \cos \theta\, d\theta \\
       &= 2\i a^3 \int_0^{\pi/2} (1-\sin^2\theta ) \cos \theta\, d\theta 
\end{align*}

Making the substitution $u=\sin\theta$ gives
\begin{align*}
     V &= 2\pi a^3 \int_0^{\pi/2} (1-\sin^2\theta ) \cos \theta\, d\theta \\
        &= 2\pi a^3 \int_0^1 (1-u^2) \, du  \\
        &= \frac{4\pi}{3} a^3 .
\end{align*}

\item Slicing the sphere the same way, the differential surface swept out by the arclength $ds = a\, d\theta$ at distance $x = a\cos\theta$ from the $y$-axis has area
\begin{align*}
   dA  &= (2\pi x) \, ds \\
         & = (2\pi a \cos\theta) (a\, d\theta) \\
         & = 2\pi a^2 \cos\theta \, d\theta.
\end{align*}

So the surface area of the sphere is
\begin{align*}
    S &= 4\pi a^2 \int_0^{\pi/2} \cos\theta \, d\theta  \\
       &= 4\pi a^2 .
\end{align*}


\end{enumerate}
\end{explanation}
\end{question}

\begin{question} \label{QadFefeb}
Use trigonometric functions to parameterize the circle of radius $a$ cm centered at the origin. Then use the parameterization to do the following.
\begin{enumerate}
\item Express the surface area of the spherical cap with height $h$ cm in terms of $a$ and $h$.

\begin{onlineOnly}
    \begin{center}
\desmosThreeD{mlx6kjtjlm}{900}{600}
\end{center}
\end{onlineOnly}

\href{https://www.desmos.com/3d/mlx6kjtjlm}{152: Golden Sphere 2}

\item A sphere is cut into 10-zones by nine equally-spaced planes as shown below. Which zone has the greatest surface area? 

\begin{onlineOnly}
    \begin{center}
\desmosThreeD{0jqijhbw6z}{900}{600}
\end{center}
\end{onlineOnly}

\href{https://www.desmos.com/3d/0jqijhbw6z}{152: Golden Sphere}

\end{enumerate}

\begin{explanation}
\begin{enumerate}
\item From the last question, since $y=a\sin\theta$,
\[
  dy = a\cos\theta\, d\theta.
\]
So
\[
  dA = 2\pi a (a \cos\theta \, d\theta) = 2\pi a \, dy   
\]

and the area of the spherical cap of height $h$ is
\[
   A = 2\pi a \int_{a-h}^h dy = 2\pi a h .
\]

\item Part (a) tells us that the zones all have the same area. Why?
\end{enumerate}
\end{explanation}

\end{question}


\begin{question} \label{QPDFDE3eg45r5r}
This example is about the curve (an astroid)
\[
    (x,y) = (a \cos^3\theta , a\sin^3\theta) \, , \, 0\leq \theta \leq 2\pi,
\]
where $a>0$ is a constant measured in meters.

\begin{onlineOnly}
    \begin{center}
\desmos{up6jvtmspl}{900}{600}
\end{center}
\end{onlineOnly}

\href{https://www.desmos.com/calculator/up6jvtmspl}{152: Astroid}

\begin{enumerate}
\item Find an expression for the function 
\[
   s = f(\phi) \, , \, 0\leq \phi < \pi/2,
\]
that gives the distance from the point $A(a,0)$ to the point $P$  with coordinates $(a\cos^3\phi, a\sin^3\phi)$ on the astroid
\[
     (x,y) = (a\cos^3\theta, a\sin^3\theta) \, , \, 0\leq \theta < 2\pi ,
\]
in terms of $\phi$. Measure the distance from $A$ to $P$ counterclockwise along the curve.

\emph{Start by expressing the differential arclength $ds$ along the astroid in terms of the differential change $d\theta$.}

\item Evaluate $f(\pi/2)$ and interpret its meaning. Compare this distance with the straight-line distance from $(a,0)$ to $(0,a)$. 

\item Express the coordinates of a point $P$ on the curve in terms if its distance $s$ from $A$ measured counterclockwise along the astroid.

\end{enumerate}

\begin{explanation}
\begin{enumerate}
\item Since
\[
  dx = -3a\cos^2\theta \sin\theta \, d\theta
\] 
and
\[
  dy = 3a\sin^2\theta\cos\theta\, d\theta ,
\]
the differential arclength is
\begin{align*}
     ds &= \sqrt{(dx)^2 + (dy)^2}  \\
         & = 3a\sqrt{\cos^2\theta \sin^2\theta (\cos^2\theta + \sin^2\theta)} \, d\theta \\
         &= 3a\sqrt{\cos^2\theta \sin^2\theta} \, d\theta \\
         & = 3a\cos\theta\sin\theta \text{ if } 0\leq \theta \leq \pi/2.
\end{align*}

Then the arclength from $\theta = 0$ to $\theta = \phi$ is
\[
    s = 3a \int_0^\phi  \cos\theta \sin\theta\, d\theta .
\]

Make the substitution $u=\sin\theta$. Then
\begin{align*}
      s &= f(\phi) \\
         &= 3a \int_0^\phi  \cos\theta \sin\theta\, d\theta  \\
         &= 3a\int_0^{\sin\phi} u \, du \\
         & = \frac{3a}{2} \sin^2\phi . 
 \end{align*}

\item The arclength of the astroid from $(a,0)$ to $(0,a)$ in the first quadrant is
\[
   s =f(\pi/2) = 3a/2 = 1.5a.
\]
This is a bit more than the straight-line distance $\sqrt{2}a \sim 1.4a$ between these points.

\item We can skip this one.
\end{enumerate}
\end{explanation}

\end{question}


\begin{question} \label{QK76554s}
The function
\[
    r = f(t) \, , \, 0 \leq t \leq 4,
\]
expresses the rate (in gal/mile) at which a car burns gas in terms of the number of hours past noon during a $240$-mile trip.

The function 
\[
      s = g(t) \, , \, 0\leq t \leq 4 ,
\]
expresses the car's trip odometer reading (in miles) in terms of the number of hours past noon.

The car has $14$ gallons of gas at 1:00pm.

Which of the following functions express the number of gallons in the tank in terms of the number of hours past noon? Select all that apply and explain your reasoning.

\begin{selectAll}
\choice{$w(t) = 14 + \int_1^t f^\prime(u)g(u) \, du$}
\choice{$w(t) = 14 + \int_1^t f(t)g^\prime(t) \, dt$}
\choice[correct]{$w(t) = 14 + \int_1^t f(u)g^\prime(u) \, du$}
\choice{$w(t) = 14 + \int_{g(1)}^{g(t)} f(u)g^\prime(u) \, du$}
\choice[correct]{$w(t) = 14 + \int_{g(1)}^{g(t)} f(g^{-1}(u)) \, du$}
\choice[correct]{$w(t) = 14 + f(t)g(t) - f(1)g(1) - \int_{1}^{t} f^\prime(u)g(u) \, du$}
\end{selectAll}
\end{question}

\begin{explanation}
For the constant case, suppose the car burns gas at a constant rate of $r=f(t) = 0.03$ gal/mile and the car travels at a constant speed of $60$ miles/hr. Then 
\[
      s = g(t) = 60t \, , \, 0\leq t \leq 4
\]
and the car burns gas at the constant rate of
\[
    \left( 0.03 \frac{\text{ gal}}{\text{mile}} \right)\left(60 \frac{\text{ miles}}{\text{hour}}  \right) = f(t)g^\prime(t) \frac{\text{ gal}}{\text{hour}} .
\]

So in the general case, during the differential time $dt$ hours the car burns
\[
   dG = f(t)g^\prime(t)\, dt
\]
gallons of gas. 

And from 1:00pm to time $t$ hours past noon, the car burns
\[
  - \Delta G =    \int_1^t f(v)g^\prime(v) \, dv
\]
gallons. Since the car has $14$ gallons of gas at 1:00pm, 
\[
 G = w(t) = 14 - \int_1^t f(v)g^\prime(v) \, dv
\]  
and (c) is correct.

Integration by parts then tells us that (f) is also correct.

Make the substitution $u=g(v)$ in the expression for $-\Delta G$ above. Then 
\[
   du = g^\prime(t)\, dt
\]
and
\[
      v = g^{-1}(u).
\]
So
\begin{align*}
    - \Delta G &= \int_1^t f(v)g^\prime(v) \, dv \\
                   &= \int_{g(1)}^{g(t)} f(g^{-1}(u)) \, du 
\end{align*}
and (e) is correct.

To understand this, note that $f(g^{-1}(u))$ gives the rate (in gal/hr) at which the car burns gas when the odometer reads $u$ miles. And the product 
\[
  f(g^{-1}(u)) \, du
\]
gives the number of gallons of gas the car burns as it travels $du$ miles. Summing these differential volumes from odometer readings $g(1)$ to $g(t)$ miles gives the number of gallons of gas the car burns from odometer reading $g(1)$ to odometer reading $g(t)$.

\end{explanation}

\begin{question} \label{Q7878dfFGF}
The function
\[
   u = f(t) = u_0 e^{rt} \, , \, 0\leq t \leq 240 ,
\]
expresses the consumption rate of Chromium (measured in $10^8$ tons/yr) in terms of the number of years since 1970. Here $u_0$ is the consumption rate at the start of 1970 and $r>0$ is a constant.

At the start of 1970 the global reserves of Chromium totaled $R_0$ hundreds of millions of tons and would have been depleted in $s$ years had the consumption rate remained constant.

\begin{enumerate}
\item Experss $u_0$ in terms of $R_0$ and $s$.

\item Find a function
\[
     R = g(t) \, , \, 0\leq t \leq 240,
\]
that expresses the remaining Chromium reserves (measured in $10^8$ tons/yr) in terms of the number of years since 1970. Assume the consumption rate described by the function $u=f(t)$.

\item Find an expression in terms of $R_0$, $s$, and $r$ for the number of years since 1970 when the Chromium reserves were projected to be depleted had the consumption rate grew exponentially as described by the function $u=f(t)$.

\item Check that your expression in part (c) has the correct units.
\end{enumerate} 
\end{question}

\begin{question} \label{Q9derJJFe}
Practice integration.
\end{question}

\end{document}