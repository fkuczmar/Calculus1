\documentclass{ximera}
\title{152 Review}

\newcommand{\pskip}{\vskip 0.1 in}

\begin{document}
\begin{abstract}
Review.
\end{abstract}
\maketitle


\begin{question} \label{QOdfe4gbr}
The function 
\[
 r = f(t) \, , \, t\geq 0,
\]
expresses the rate (in gal/hr) at which water flows into a tank in terms of the number of hours past noon, July 23, 2025.

The tank holds 500 gallons at 3pm on July 23, 2025.

\begin{enumerate}
\item Find a function
\[
   V = g(t) \, , \, t\geq 0, 
\]
that expresses the volume of water (measured in gallons) in the tank in terms of terms of the number of hours past noon, July 23, 2025.

\item Suppose now that
\[
     r = f(t) = \frac{200}{(2 + \frac{t}{10})^2} \, , \, t\geq 0.
\]

\begin{enumerate}
\item Find an explicit expression for the function $g$ that does not have a definite integral.

\item Assuming the water flows forever, determine if there is a large enough tank to hold all the water and not overflow? If so, compute the minimum volume of such a tank. If not, explain why.

Use limits to justify your response.
\end{enumerate}

\end{enumerate}
\end{question}

\begin{question} \label{QKdmfe3dv}
The function 
\[
   v = f(s) \, , \, 20\leq s \leq 70,
\]
expresses the speed of a car (in miles/hour) in terms of the trip odometer reading (in miles) during a portion of a 200-mile trip.

The trip odometer reads $50$ miles at 4:00pm.

\begin{enumerate}
\item Find a function
\[
    t = g(s) \, , \, 20\leq s \leq 70
\]
that expresses the time (measured in hours past noon) in terms of the trip odometer reading. \emph{Begin your solution by analyzing the constant case.}

\item Suppose now that the speed is a linear function of the odometer reading during this $50$-mile section of the trip. Suppose also the speed is $60$ miles/hour when the odometer reads $20$ miles and $40$ miles/hour when the odometer reads $70$ miles.

\begin{enumerate}
\item Find an explicit expression for the function $t=g(s)$ that does not have a definite integral.

\item Find the exact clock time when the odometer reads $70$ miles. Then use a calculator to approximate this time to the nearest minute.
\end{enumerate}
\end{enumerate}
\end{question}

\begin{question} \label{Q9de3rmre}
Use trigonometric functions to parameterize the circle of radius $a$ cm mcentered at the origin. Then use the parameterization to compute

\begin{enumerate}
\item the sphere's volume and

\item the sphere's surface area.
\end{enumerate}
\end{question}

\begin{question} \label{QadFefeb}
Use trigonometric functions to parameterize the circle of radius $a$ cm centered at the origin. Then use the parameterization to do the following.
\begin{enumerate}
\item Express the surface area of the spherical cap with height $h$ cm in terms of $a$ and $h$.

\begin{onlineOnly}
    \begin{center}
\desmosThreeD{mlx6kjtjlm}{900}{600}
\end{center}
\end{onlineOnly}

\href{https://www.desmos.com/3d/mlx6kjtjlm}{152: Golden Sphere 2}

\item A sphere is cut into 10-zones by nine equally-spaced planes as shown below. Which zone has the greatest surface area? 

\begin{onlineOnly}
    \begin{center}
\desmosThreeD{0jqijhbw6z}{900}{600}
\end{center}
\end{onlineOnly}

\href{https://www.desmos.com/3d/0jqijhbw6z}{152: Golden Sphere}

\end{enumerate}
\end{question}



\end{document}