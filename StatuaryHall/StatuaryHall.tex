\documentclass{ximera}
\title{Statuary Hall and Implicit Differentiation}

\newcommand{\pskip}{\vskip 0.1 in}

\begin{document}
\begin{abstract}
Statuary Hall and John Quincy Adams
\end{abstract}
\maketitle


\begin{question}  \label{Q656g5y4546}

\begin{center}
\youtube{watch?v=FX6rUU_74kk}
\end{center}

\href{https://www.youtube.com/watch?v=FX6rUU_74kk}{Statuary Hall}


Up until the middle of the 19th centuary, Statuary Hall in the U.S. Capitol was the meeting place of the House of Representatives. The room is in the shape of an ellipse and John Quincy Adams had his desk at one of the foci. Legend has it that he was able to eavesdrop on the whispered conversations of his political opponents when they were standing at the other focus.

The aim of this problem is to explain this and prove the \emph{reflective property of ellipses.}

\pskip

\emph{A light wave (or a sound wave) emitted from one focus of an ellipse passes through the other focus after reflecting off the ellipse.}

\pskip

We need to know the definition of an ellipse:

\emph{An ellipse is the set of points, the sum of whose distances from two fixed points (the foci) is constant.} 

\begin{onlineOnly}
    \begin{center}
\desmos{6kxtojk72r}{900}{600}
\end{center}
\end{onlineOnly}

Access Desmos interactives through the online version of this text at
 
\href{https://https://www.desmos.com/calculator/6kxtojk72r}{151: Statuary Hall}.

In Calculus 3 you'll learn about the gradient vector and a short way to prove the reflective property of ellipses in general. But we'll work with a specific example, and take our ellipse to have focal points $F_1(0,0)$ and $F_2(4,0)$. We'll also suppose the ellipse passes through $P(0,3)$. Our problem is to first find an equation of the normal line to the ellipse at $P$. Then we'll show that this line bisects angle $F_1PF_2$. 

\begin{enumerate}
\item Start by using the definition of the ellipse to write an equation of the ellipse.

\item Use implicit differentiation to find the slope of the tangent line to the ellipse at $P(0,3)$.

\item Find an equation of the normal line at $P$. Enter this equation on Line 16 in the worksheet above. 

\item For a shorter way to describe the normal line, do the following.

\begin{enumerate}
\item Use vector arithmetic to find a vector parallel to the normal line at $P$.

\item Use your vector from part (i) to parameterize the normal line at $P$.

\end{enumerate}
\end{enumerate}
\end{question}


\begin{question}  \label{Q5445rggfbhyhyrdt}

Another way to prove the reflective property of ellipse is related to an optimization problem.

The problem is this. Given points $A$ and $B$, to find the point on a give line $L$ that minimizes the sum 
\[
   \text{dist}(P,A) + \text{dist}(P,B)
\]
of the distances from $P$ to $A$ and $B$.

\begin{onlineOnly}
    \begin{center}
\desmos{vxqdjba3hm}{900}{600}
\end{center}
\end{onlineOnly}

Access Desmos interactives through the online version of this text at
 
\href{https://https://www.desmos.com/calculator/vxqdjba3hm}{151: Minimization Property of Ellipses}.

\begin{enumerate}
\item Use calculus to show that the marked angles at $P$ above are congruent for the point $P$ on the line that minimizes the above sum.

\item Explain how part (a) proves the reflective property of ellipses.
\end{enumerate}




\end{question}


\end{document}