\documentclass{ximera}
\title{Gas Consumption}

\newcommand{\pskip}{\vskip 0.1 in}

\begin{document}
\begin{abstract}
Gas consumption and derivatives.
\end{abstract}
\maketitle


\begin{question} \label{Q5hhhhgeyghhg}
The function
\[
    G = f(v) = \frac{v^2}{20} + 5v -90 \, , \, 35\leq v \leq 70 ,
\]
graphed below expresses the gas mileage (in miles/gal) of a car in terms of its speed (in miles/hour).

\begin{onlineOnly}
    \begin{center}
\desmos{ttmyizszfb}{450}{600}  
\end{center}
\end{onlineOnly}

\href{https://www.desmos.com/calculator/ttmyizszfb}{151: Burning Gas}

\begin{enumerate}
\item Use the graph to determine the rate (in gal/hr) at which the car burns gas at a speed of $50$ miles/hour.

\item Drag the slider $v$ in Line 1 to approximate the speeds between $35$ miles/hour and $70$ miles/hour at which the car burns gas at the maximum and minimum rates. Explain your reasoning.

\item Use calculus to determine the exact speeds in part (b).

\end{enumerate}


\end{question}



\end{document}