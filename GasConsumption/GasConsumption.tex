\documentclass{ximera}
\title{Gas Consumption}

\newcommand{\pskip}{\vskip 0.1 in}

\begin{document}
\begin{abstract}
Gas consumption and derivatives.
\end{abstract}
\maketitle


\begin{question} \label{Q5hhhhgeyghhg}
The function
\begin{align*}
    G  &= f(v)    \\
         &=  -\frac{v^2}{20} + 5v -90 \, , \, 35\leq v \leq 70 ,
\end{align*}
graphed below expresses the gas mileage (in miles/gal) of a car in terms of its speed (in miles/hour).

\begin{onlineOnly}
    \begin{center}
\desmos{fapdhcqptl}{450}{600}  
\end{center}
\end{onlineOnly}

\href{https://www.desmos.com/calculator/fapdhcqptl}{151: Burning Gas}

\begin{enumerate}
\item Use the graph to determine the rate (in gal/hr) at which the car burns gas at a speed of $50$ miles/hour.

\item Drag the slider $v$ in Line 1 to approximate the speeds between $35$ miles/hour and $70$ miles/hour at which the car burns gas at the maximum and minimum rates (measured in gal/hr). Explain your reasoning.

\item Use calculus to determine the exact speeds in part (b). Find a way that avoids using the quotient rule.

\end{enumerate}


\end{question}

\begin{question} \label{Qjjjkmmmadgt4e}
The function 
\[
   G = f(s) = \frac{11}{5} +\frac{1}{5000}\left( s^3-50s^2+300s \right) , \, 3\leq s \leq 28 ,
\]
expresses the number of gallons of gas in your car in terms of your distance from home. The distance is measured in miles along your route. 

\begin{onlineOnly}
    \begin{center}
\desmos{pb8v4t3cxg}{900}{600}
\end{center}
\end{onlineOnly}

Desmos activity available at
\href{https://www.desmos.com/calculator/pb8v4t3cxg}{151: Gas as a Function of Distance 33}

\begin{enumerate}
\item Use the graph above to determine if you are driving toward or away from home. Explain your reasoning.

\item Sensors on your car measure both the (instantaneous) gas mileage and the number of gallons of gas in your tank at each instant. A computer then uses these measurements to estimate the number of additional miles you can drive before running out of gas.

\begin{enumerate}

\item Use the graph and the slider $s_0$ above to approximate the reading for the number of additional miles you can drive when you are $10$ miles from home and $20$ miles from home. Explain your reasoning.

\item Find a function 
\[
  m =g(s) \, , \, 3\leq s \leq 28 ,
\]
that expresses the number of miles you can drive before running out of gas (assuming your gas mileage remains constant for the remainder of your trip) in terms of your distance from home. Explain your reasoning. 

\item Find a function 
\[
     r = h(s) \, , \, 3\leq s \leq 28 ,
\]
that expresses the \emph{relative} rate of change in the function $f$ with respect to and in terms of the number of miles from home. 

\begin{enumerate}
\item Evaluate $h(10)$ and interpet its meaning. Include units.

\item Compare the functions $r=h(s)$ and $m=g(s)$. How are they related? Explain the logic behind this relation.
\end{enumerate}

\end{enumerate}
\end{enumerate}

\end{question}


\end{document}