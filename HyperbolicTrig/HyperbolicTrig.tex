\documentclass{ximera}
\title{Hyperbolic Trigonometry}

\newcommand{\pskip}{\vskip 0.1 in}

\begin{document}
\begin{abstract}
An introduction to the hyperbolic trigonometric functions.
\end{abstract}
\maketitle


\section{The Hyperbolic Cosine and Sine Functions}

The function \emph{hyperbolic cosine} is defined as
\[
    \cosh(x) = 0.5(e^x + e^{-x}  ).
\]

Its graph is shown below.

\begin{onlineOnly}
    \begin{center}
\desmos{nubdwzgsa9}{450}{600}  
\end{center}
\end{onlineOnly}

\href{https://www.desmos.com/calculator/nubdwzgsa9}{152: Hyp Cosine}

\begin{enumerate}
\item Drag the slider $u$ in Line 2 of the worksheet above. Explain how to get the graph of the function $y=\cosh x$ from the graphs of the functions $y=e^x$ and $y=e^{-x}$.

\begin{freeResponse}
\end{freeResponse}

\item What is the range of the function $f(x)=\cosh x$?

\item Activate the folder \emph{Hyperbolic Sine Function} in Line 15 above to see the graph of the hyperbolic sine function
\[
   \sinh(x) = 0.5(e^x - e^{-x}  ).
\]
\item Drag the slider $u$ in Line 2 of the worksheet above. Explain how to get the graph of the function $y=\cosh x$ from the graphs of the functions $y=e^x$ and $y=e^{-x}$.
\begin{freeResponse}
\end{freeResponse}

\item What is the range of the function $g(x)=\sinh x$?

\item Prove the identity 
\[
      \cosh^2x - \sinh^2x = 1.
\]

\emph{Hint: For a quick way, factor the above expression.}

\item Show that 
\[
   \frac{d}{dx} \left( \cosh x  \right) = \sinh x
\]
and
\[
   \frac{d}{dx} \left( \sinh x  \right) = \cosh x
\]

\end{enumerate}


\section{Hanging Chains}
The graph of the hyperbolic cosine function models a chain with uniform density hanging in a uniform graviatational field. We'll learn why later in the class.

Drag the slider $a$ in Line 2 to adjust the graph of the function
\[
     f(x) = -a + a\cosh(x/a)
\]
to match the highlighted (orange) chain below.

\begin{onlineOnly}
    \begin{center}
\desmos{tj3dz2cnf0}{450}{600}  
\end{center}
\end{onlineOnly}

\href{https://www.desmos.com/calculator/tj3dz2cnf0}{152: Hanging Chain 1}


\end{document}
