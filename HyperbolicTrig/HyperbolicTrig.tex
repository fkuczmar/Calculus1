\documentclass{ximera}
\title{Hyperbolic Trigonometry}

\newcommand{\pskip}{\vskip 0.1 in}

\begin{document}
\begin{abstract}
An introduction to the hyperbolic trigonometric functions.
\end{abstract}
\maketitle


\section{The Hyperbolic Cosine and Sine Functions}

The function \emph{hyperbolic cosine} is defined as
\[
    \cosh(x) = 0.5(e^x + e^{-x}  ).
\]

Its graph is shown below.

\begin{onlineOnly}
    \begin{center}
\desmos{nubdwzgsa9}{450}{600}  
\end{center}
\end{onlineOnly}

\href{https://www.desmos.com/calculator/nubdwzgsa9}{152: Hyp Cosine}

\begin{enumerate}
\item Drag the slider $u$ in Line 2 of the worksheet above. Explain how to get the graph of the function $y=\cosh x$ from the graphs of the functions $y=e^x$ and $y=e^{-x}$.

\begin{freeResponse}
\end{freeResponse}

\item What is the range of the function $f(x)=\cosh x$?

\item Activate the folder \emph{Hyperbolic Sine Function} in Line 15 above to see the graph of the hyperbolic sine function
\[
   \sinh(x) = 0.5(e^x - e^{-x}  ).
\]
\item Drag the slider $u$ in Line 2 of the worksheet above. Explain how to get the graph of the function $y=\cosh x$ from the graphs of the functions $y=e^x$ and $y=e^{-x}$.
\begin{freeResponse}
\end{freeResponse}

\item What is the range of the function $g(x)=\sinh x$?

\item Prove that 
\[
      \cosh^2x - \sinh^2x = 1.
\]

\emph{Hint: For a quick way, first factor the above expression.}

\item Show that 
\[
   \frac{d}{dx} \left( \cosh x  \right) = \sinh x
\]
and
\[
   \frac{d}{dx} \left( \sinh x  \right) = \cosh x
\]

\end{enumerate}


\section{Hanging Chains}
Hold the ends of a chain and let the chain sag under its own weight. The resulting curve looks like a parabola, and so it was thought. But it turns out that this is not correct. 


\begin{onlineOnly}
    \begin{center}
\desmos{tj3dz2cnf0}{450}{600}  
\end{center}
\end{onlineOnly}

\href{https://www.desmos.com/calculator/tj3dz2cnf0}{152: Hanging Chain 1}


\begin{enumerate} 

\item To see that the dashed (orange) chain in the worksheet above is \emph{not a parabola} drag the slider $c$ in Line 2.

\item It turns out that a chain with uniform density hanging in a uniform graviatational field assumes the shape of the hyperbolic cosine function $y=\cosh x$ (we'll see why later in the course). All that's needed to describe a particular chain is to scale the size of the graph by some factor. Test this by dragging the slider $a$ in Line 6 to make the graph of the function
\[
     f(x) = -a + a\cosh(x/a)
\]
match the highlighted (orange) chain below.
\end{enumerate}

\section{The Hyperbolic Tangent Function}
The hyperbolic tangent function is defined as
\begin{align*}
 \tanh x &= \frac{\sinh x}{\cosh x}  \\
             &= \frac{e^x - e^{-x}}{e^x + e^{-x}}.
\end{align*}

\begin{enumerate}
\item Find the domain and range of the function $y=\tanh x$.

\item Sketch by hand a graph of the function $y=\tanh x$.

\item Use your graph from part (b) to sketch a graph of the derivative
\[
    y = \frac{d}{dx} \left( \tanh x \right) .
\]

\item Find an expression for the derivative in part (c) in terms of $\cosh x$ and/or  $\sinh x$. 

\end{enumerate}

\section{Inverse Functions}
The key idea here is that \emph{the derivative of the inverse of a function is the reciprocal of the function's derivative}. This should be intuitive.

So to find the derivative of the inverse of a one-to-one function $y=f(x)$, we should first express $dy/dx$ in terms of $y$. Then take the reciprocal.


\begin{enumerate}
\item Express the derivaive of each function below in terms of its output.
\begin{enumerate}
\item
\[
   y = \sinh x 
\]

\item
\[
   y = \cosh x \, , \, x\geq 0
\]

\item 
\[
    y = \tanh x 
\]
\end{enumerate}

\item Use the results of part (a) to find expressions for the derivatives
\begin{enumerate}
\item 
\[
    \frac{d}{dy} (\sinh^{-1} y)
\]

\item 
\[
    \frac{d}{dy} (\cosh^{-1} y)
\]

\item 
\[
    \frac{d}{dy} (\tanh^{-1} y)
\]

\end{enumerate}

\end{enumerate}



\end{document}
