\documentclass{ximera}
\title{Derivatives of Exponential Functions}

\newcommand{\pskip}{\vskip 0.1 in}

\begin{document}
\begin{abstract}
Working with exponential functions and their derivatives.
\end{abstract}
\maketitle


\section*{Relative Changes and Relative Rates of Change}
Relative changes and relative errors are often more meaningful than absolute changes and errors. For example, I might measure the distance from Shoreline's Central Market to the Richmond beach library to be $5$ km with an  error of at most $0.2$ km, while NASA might measure the between the earth and the moon on the first day of spring to be $384,400$ km with an error of at most $100$ km. The relative error in my measurement is at most
\[
  \frac{0.2 \text{ km} }{5 \text{ km}} = 0.04 = 4\% ,
\] 
while the relative error in NASA's measurement is at most
\[ 
   \frac{100 \text{ km} }{384,400 \text{ km}} \sim 0.00026 = 0.026\% .
\]
Relatively speaking, NASA's measurement was about 150 times more accutate than mine.

\begin{question}  \label{Qfnljhn}
At 10:00am the preices of Stock A and Stock B are both increasing at the rate of $(\$2/\text{share})/\text{hour}$. At 10:00am Stock A sells for $\$50$/share and Stock B for $\$10$/share. Compare the relative rates at which the share prices are changing at 10:00am.
\end{question}


\begin{question} \label{Q43tbbtt}
The function 
\[
      P = f(t) = 5 -3t + t^2 \, , \, 0\leq t \leq 4 , 
\]
expresses the price in $\$$/share of a stock in terms of the number of hours past 9am.

(a) At what relative rate is the price of the stock changing at 10am? 

(b) When is the share price increasing at a rate of $60\%$/hr?

(c) During what time interval is the price of the stock increasing?

(d) During what time interval is the relative rate of change in the price of the stock increasing? 

\begin{onlineOnly}
    \begin{center}
\desmos{hhkveu6lxp}{900}{600}
\end{center}
\end{onlineOnly}

Desmos activity available at \href{https://www.desmos.com/calculator/hhkveu6lxp}{151: Stock Price}
\end{question}


\begin{question}  \label{Qdgbchhtt}

The function 
\[
      P = f(t) \, , \, 1\leq t \leq 3.6 , 
\]
expresses the price in $\$$/share of a stock in terms of the number of hours past 9am. Use the graph below to approximate the answers to the following questions \emph{without} putting a scale on the $P$ axis.

(a) At what relative rate is the price changing at 11am? At 12:30pm?

(b) When is the stock price increasing at its maximum relative rate? At its minimum relative rate? Approximate these rates.




\begin{onlineOnly}
    \begin{center}
\desmos{jyebaj5jif}{900}{600}
\end{center}
\end{onlineOnly}


Desmos activity available at \href{https://www.desmos.com/calculator/jyebaj5jif}{151: Stock Price 2}



\end{question}


\begin{question}  \label{Q:354rgbytt}
The function
\[
    W = f(t) = 200+4t +2t^2 , 0\leq t \leq 12 ,
\]
expresses the weight (in pounds) of a baby elephant in terms of its age (in months).

(a) Find the average rate at which the elephant gained weight between ages 4 and 10 months.

(b) Find the relative average rate at which the elephant gained weight between ages 4 and 10 months.

(c) Find the relative instantaneous rate at which the elephant is gaining weight at age 4 month.

(d) Find the relative instantaneous rate at which the elephant is gaining weight at age 10 months.

(e) Use the graph below to interpret your answers to parts (b)-(d) geometrically.


\begin{onlineOnly}
    \begin{center}
\desmos{2xj6xy7ggo}{900}{600}
\end{center}
\end{onlineOnly}


Desmos activity available at \href{https://www.desmos.com/calculator/2xj6xy7ggo}{151: Elephant}



\end{question}



\section*{Exponential Growth}
\begin{question}  \label{Qr435rhgnbh}
(a) What does it mean for a population to grow exponentially?

(b) Is it possible for a population to increase by $20\%$ every year and not grow exponentially?
\end{question}


\begin{question} \label{Qewrdfst5t}
Suppose between noon and 10pm a colony of bacteria grows exponentially. The population is 200,000 at noon and 242,200 at 1pm.

(a) Describe how the population grows. Keeping Question 2(b) in mind, is  your description sufficient?

(b) How might we find a complete description of the exponential growth?

(c) Determine the relative average growth rate between noon and 12:30pm. Between 1pm and 1:30pm. Over any half-hour time period. Use the slider $u$ in the graph below to intrepret these rates geometrically.

(d) Approximate the instantaneous relative growth rates in the population at noon, at 1pm, and at 2pm. Modify the definition of $v=u+1/2$ in the demonstration below and interpret these rates geometrically.

(e) Use limits to write an expression that gives the instantaneous growth rate at time $t$ hours past noon. What can you conclude? 

(f) Try to answer part (b) again.


\begin{onlineOnly}
    \begin{center}
\desmos{wvpsotdhby}{900}{600}
\end{center}
\end{onlineOnly}


Desmos activity available at \href{https://www.desmos.com/calculator/wvpsotdhby}{151: Exp Growth 1}

\end{question}



\begin{question}  \label{Qdrg5h888}
The function 
\[
        P = f(t) = 3(2)^t , -2\leq t \leq 4 ,
\] 
expresses the population (in millions) of a colony of bacteria in terms of the number of hours past noon.

(a) Describe how the population grows. Is your description sufficient?

\pskip

The population $\answer{doubles}$ every $\answer{hour}$.

\pskip

(b) Find an expression for the relative growth \emph{rate} between time $t=u$ hours past noon and time $t=u+h$ hours past noon. Measure the rate relative to the population at time $t=u$. Is this question asking about an average or an instantaneous relative growth rate?

\pskip

The relative growth rate is
\begin{align*}
    \frac{1}{P} \left(  \frac{\Delta \answer{P}}{\Delta \answer{t}} \right)   &= \frac{1}{\answer{f(u)}} \left( \frac{\answer{f(u+h)} - f(u)}{\answer{h}}     \right)   \\ \\
                &= \frac{1}{3(2)^{\answer{u}}}   \left(   \frac{\answer{3(2)^{u+h}-3(2)^u}}{\answer{h}}     \right)  \\ \\
                &= \frac{1}{3(2)^u}   \left(   \frac{\answer{3(2)^u}(2^{\answer{h}}  -1}{h}  \right)  \\ \\
                &= \frac{\answer{2^h} - 1}{h}
\end{align*}

(c) What are the units of the relative average growth rate in part (b)?

(d) Input your function from part (b) on Line 5 in the worksheet below. 

(e) What do you notice about the distance between points $R$ and $S$ as you drag the slider $u$ below. How is this distance related to the relative average growth rate in part (b)?

(f) Use limits to write an expression for the relative growth rate at time $t=u$ hours past noon. Simplify this expression as much as possible.  What can you conclude about the relative growth rate at any instant?

\pskip

The relative growth rate is 
\[
    \lim_{h \to \answer{0}} \frac{\answer{2^h} - 1}{h} .
\]


(g) Interpret your expression from part (f) as the derivative of a specific function evaluated at a specific input. What does this tell you about the relative growth rate of this particular population?

(h) Use part (f) to numerically approximate the relative (instantaneous) growth rate of the population. Show a table that suggests a progression toward a limit.

(i) Use a similar method to approximate the relative instantaneous growth rate of the population
\[
   P = f(t) = 5 (3)^t .
\]



\begin{onlineOnly}
    \begin{center}
\desmos{omjbec2hpu}{900}{600}
\end{center}
\end{onlineOnly}


Desmos activity available at \href{https://www.desmos.com/calculator/omjbec2hpu}{151: Exponential Growth 1}
\end{question}


\begin{question}  \label{Qcgbt4tt}
Parts (h) and (i) of the previous question suggest that there is a number $e$ between $2$ and $3$ that makes the 
relative growth rate of the function 
\[
         P = f(t) = P_0 e^t , -3\leq t \leq 5
\]
equal to $100\%$/hr, where we assume here that $t$ is measured in hours. 

(a) What is the one-hour growth factor for this population?

(b) Describe what happens to the population every hour.

(c) At what relative rate is the population increasing at 1:00pm?

(d) Suppose at 1:00pm the population is $500,000$. Approximate the population at 1:03pm and compare your approximation to the actual population at that time.
\end{question}


\section*{Exponential Functions with Bases other than $e$}


\begin{question}  \label{Qcgt4ghggyt4r}
The function
\[
     P = g(t) = P_0 e^{t/2}, -6 \leq t \leq 10
\]
expresses the population (Colony B) of bacteria in terms of the number of hours past noon.

(a) Describe a transformation that takes the graph of the population function 
\[
           P = f(t) = P_0 e^t , -3\leq t \leq 5 
\]
for Colony A (where $t$ is also the number of hours past noon) to the graph of $P=g(t)$.

(b) Suppose that the population of Colony A is $400,000$ at 4:00pm.

(i) When is the population of Colony B equal to $400,000$?

(ii) What are the growth rates of the two populations when they each have respective populations of 400,000 bacteria?

(iii) What are the relative growth rates of the two populations when they each have respective populations of 400,000 bacteria?

\end{question}

\begin{question} \label{Q34324gbe43}
Here's another way to think about differentiating the function
\[
      P = g(t) = P_0 e^{t/2} , -6 \leq t \leq 10
\]
that expresses the population of a colony of bacteria in terms of the number of hours past noon.

We'll let $u=t/2$ be the number of two-hour periods since noon. 

(a) Express the population in terms of $u$.

(b) Use what you know about the exponential function base $e$ to express the growth rate of the population in terms of $u$.

(c) Use part (b) to find the growth rate of the population at 6pm. Pay careful attention to units.

(d) Use the idea of part (c) to express the growth rate
\[
    \frac{dP}{dt} = g^\prime(t)
\]
in terms of $t$.

(e) Suppose instead that the population grows exponentially and doubles every hour. Find the relative instantaneous growth rate of the population.

\end{question}


\begin{question}  \label{Qe545tgbvb}
Between 11am and 8pm, a population of bacteria grows exponentially. The population is 4 million at noon and 5 million at 1pm.

(a) What is the one year growth factor?  $\answer{1.25}$

(b) Describe how the population grows. 

\pskip

The population increases by $\answer{25}\%$ every hour.

(c) Use your description from part (b) to find a function that expresses the population (in millions of bacteria) in terms of the number of hours past noon. Do \emph{not} use $e$ in your function. Define meaningful variables and include a domain.

(d) Use the fact that $k = e^{\ln k}$ for $k>0$ to express your function from part (c) using an exponential function with base $e$.

(e) Use the fact that 
\[
     \frac{d}{dt}(e^{kt}) = k e^{kt}
\]
to find the relative instantaneous growth rate of the population.

(f) At what rate is the population growing when there are $10$ million bacteria?

(g) Find the population when it is increasing at the rate of $3$ million bacteria/hour.

\end{question}


\begin{question}  \label{Q234gt44}
Between 11am and 8pm, a population of bacteria decreases exponentially. The population is 5 million at noon and 4 million at 4pm.

(a) What is the one year growth factor?

(b) Describe how the population declines.

(c) Find a function that expresses the population (in millions of bacteria) in terms of the number of hours past noon. Define meaningful variables and include a domain.

(d) What is the relative instantaneous growth rate of the population?

(e) At what rate is the population decreasing when there are $2$ million bacteria?

(f) Find the population when it is decreasing at the rate of $300,000$ bacteria/hour.

(g) Find a three-hour time interval over which the population decreases at an average rate of $600,000$ bac/hr. Start by defining an unknown.

\end{question}


\section*{Relative Rates Again}

\begin{question}   \label{Q45fggfbhyhy}
The function
\[
     P = f(t) , 0\leq t \leq 2,
\]
expresses the balance (in dollars) in an account in terms of the number of years since the start of 2022. Suppose 
\[
      \frac{1}{P} \frac{dP}{dt} \Big|_{P=5000} = 0.08 .
\]

(a) What are the units of the above derivative? How do you know?

(b) Interpret the meaning of the above derivative.

(c) Approximate the balance in the account four days after the account has $\$5,000$. Explain your reasoning.

\end{question}


\begin{question}  \label{Qe5r54tggrgtre}
The function 
\[
    q = f(p) = 0.5(p-16)^2 \, , \, 6\leq p \leq 15 , 
\]
expresses the average number of burgers/day sold at Five Guys of Edmonds in terms of the price (in $\$$/burger).

(a) At what relative rate does the quantity sold ($q$) change with respect to the price ($p$) at a price of $\$10$/burger?

(b) What are the units of the above relative rate of change?

(c) Explain the meaning of the relative rate of change in part (a).

(d) Use the graph of the function $q=f(p)$ and the slider $u$ in the desmos activity below to interpret the relative rate of change in part (a) geometrically. Explain your reasoning.

(e) Use the result of part (a) to approximate the relative change in the average number of burgers sold per day if the Five Guys increases the price from $\$10$/burger to $\$10.25$/burger. Explain your reasoning.

(f) Use the result of part (a) to approximate the relative change in the average number of burgers sold per day in terms of a small relative change in the price from $\$10$/burger. Explain your reasoning.


\begin{onlineOnly}
    \begin{center}
\desmos{ylgk03oaza}{900}{600}
\end{center}
\end{onlineOnly}


Desmos activity available at \href{https://www.desmos.com/calculator/ylgk03oaza}{151: Burgers 1}

\end{question}

\begin{question}  \label{Qdsfsadfghbhhhhyy}
The function
\[
        P = f(t) , -2 \leq t \leq 5,
\]
expresses the population of a colony (call it Colony A) of bacteria in terms of the number of hours past noon.

The function 
\[
          P =g(t) = f(t/2), -4 \leq t \leq 10,
\]
expresses the population of Colony B in terms of the number of hours past noon.

The populations do \emph{not} necessarily grow exponentially.


(a) Compare the populations at noon.


(b) Suppose Colony A as 50,000 bacteria at 3:00pm. When does Colony B have 50,000 bacteria? Explain.

(c) Suppose the population of Colony A takes three hours to grow from 20,000 to 50,000. How long does it take the population of Colony B to grow from 20,000 to 50,000?

(d) Suppose the population of Colony A is increasing at the rate of 10,000 bac/hr at 3pm. What is the growth rate of Colony B at 6pm? Explain. 

(e) What is the relative growth rate of Colony A at 3pm? What is the relative growth rate of Colony B at 6pm? Explain. 

\end{question}

\begin{question} \label{Q:432g4gh}
The function 
\[
       P = f(u) = 10e^u , -2\leq u \leq 5
\]
expresses the population (in millions of bacteria) of a colony of bacteria in terms of the number of hours since noon.

(a) What are the units of the input to the exponential function in the above expression for $P$?

(b) At what rate is the population growing when there are 30 million bacteria?

(c) At what rate is the population growing when there are $P$ million bacteria? Do the units of your answer make sense?

(d) What is the relative instantaneous growth rate of the population?

(e) Find a function
\[
         P = g(t) 
\]   
that expresses the population  (in millions of bacteria) of the colony of bacteria in terms of the number of \emph{minutes} since noon. Include a domain.

(f) Use common sense to evaluate the derivative
\[
      \frac{dP}{dt}\Big|_{P=30} .
\]
Explain your reasoning.

(g) Use the idea of part (f) to find an expression for the derivative
\[
    \frac{dP}{dt} = g^\prime(t)
\]
at time $t$ minutes past noon.

\end{question}


\begin{question}  \label{Q:3dgtnzz}
The function 
\[
      P = P_0 e^{kt}, -4\leq t \leq 5 ,
\]
expresses the population of a colony of bacteria in terms of the number of hours past noon.


(a) What are the units of the constant $P_0$?

(b) What are the units of the constant $k$?

(c) Use the ideas of the previous question to find an expression for the growth rate of the population at time $t$ hours past noon. Include units in your answer.

(d) Find an expression for the relative growth rate at time $t$ hours past noon. Include units in your answer.

\end{question}



\begin{question} \label{Qdcvbrtt}
One of the two functions graphed below is an exponential function. Which one? How do you know?
\end{question}



\section*{The Derivative as a Magnification Factor}
\begin{exploration}   \label{Ex:325gyt}

It sometimes helps to think of the derivative as a magnification factor that maps a small interval around an input to a function to a corresponding interval around the output.

(a) Use this idea for the function $y=f(x)$ graphed below to approximate the derivatives 
\[
    \frac{d}{dx} (f(x))\Big|_{x=4} \text{  and  } \,\, \frac{dy}{dx}\Big|_{x=5} .
\]


\begin{onlineOnly}
    \begin{center}
\desmos{la4f5ots3r}{900}{600}
\end{center}
\end{onlineOnly}


Desmos activity available at \href{https://www.desmos.com/calculator/la4f5ots3r}{151: Magnification Factor 1}

\end{exploration}


\section*{Differentiating the Exponential Function $e^x$}

\begin{exploration}

(a) Use the graph of the function $y=f(x)$ below to approximate the derivatives 
\[
    \frac{dy}{dx}\Big|_{y=k} \text{ for } k=1, 2, \ldots , 6.
\]
Note the above derivatives are evaluated at the \emph{outputs} of the function $f$. 

(b) What do  you notice?

\begin{onlineOnly}
    \begin{center}
\desmos{k08dphtuca}{900}{600}
\end{center}
\end{onlineOnly}


Desmos activity available at \href{https://www.desmos.com/calculator/k08dphtuca}{151: Magnification Factor 2}


\end{exploration}

\begin{question}  \label{Qdefrt4trt}
The function 
\[
    P = 400 e^t , 0\leq t \leq 2
\]
expresses the population (in thousands) of a colony of bacteria in terms of the number of hours past noon.

(a) What are the units of the factor $400$?

(b) What are the units of the exponent in the factor $e^t$? Be careful.

(c) Find the (instantaneous) growth rate of the population when there are 1,200,000 bacteria.

(d) Find the relative (instantaneous) growth rate when there are 1,200,000 bacteria.

(e) Find the  relative (instantaneous) growth rate at any time.

(f) Approximate the population $30$ seconds after there are 1,200,000 bacteria.
\end{question}


\begin{question}  \label{Qdfdsf4thn}
Let $k>0$ be a constant and let
\[
     f(x) = k e^x .
\]
For $a\in \mathbb{R}$ let point $P$ with coordinates $(a, f(a))$ be a curve on the curve $y=f(x)$. Let $Q$ be the point where the tangent line to the curve intersects the $x$-axis and let $R$ be the point with coordinates $(a,0)$

(a) Find the length of segment $\overline{QR}$.

(b) How is part (a) related to part (e) of the previous question?

\end{question}


\end{document}