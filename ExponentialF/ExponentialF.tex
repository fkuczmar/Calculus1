\documentclass{ximera}
\title{Derivatives of Exponential Functions}

\newcommand{\pskip}{\vskip 0.1 in}

\begin{document}
\begin{abstract}
Working with exponential functions and their derivatives.
\end{abstract}
\maketitle



\section*{The Derivative as a Magnification Factor}
\begin{exploration}   \label{Ex:325gyt}

It sometimes helps to think of the derivative as a magnification factor that maps a small interval around an input to a function to a corresponding interval around the output.

(a) Use this idea for the function $y=f(x)$ graphed below to approximate the derivatives 
\[
    \frac{d}{dx} (f(x))\Big|_{x=4} \text{  and  } \,\, \frac{dy}{dx}\Big|_{x=5} .
\]


\begin{onlineOnly}
    \begin{center}
\desmos{la4f5ots3r}{900}{600}
\end{center}
\end{onlineOnly}


Desmos activity available at \href{https://www.desmos.com/calculator/la4f5ots3r}{151: Magnification Factor 1}

\end{exploration}


\section*{Differentiating Exponential Functions}

\begin{exploration}

(a) Use the graph of the function $y=f(x)$ below to approximate the derivatives 
\[
    \frac{dy}{dx}\Big|_{y=k} \text{ for } k=1, 2, \ldots , 6.
\]
Note the above derivatives are evaluated at the \emph{outputs} of the function $f$. 

(b) What do  you notice?

\begin{onlineOnly}
    \begin{center}
\desmos{k08dphtuca}{900}{600}
\end{center}
\end{onlineOnly}


Desmos activity available at \href{https://www.desmos.com/calculator/k08dphtuca}{151: Magnification Factor 2}


\end{exploration}


\end{document}