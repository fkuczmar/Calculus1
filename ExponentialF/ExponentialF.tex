\documentclass{ximera}
\title{Derivatives of Exponential Functions}

\newcommand{\pskip}{\vskip 0.1 in}

\begin{document}
\begin{abstract}
Working with exponential functions and their derivatives.
\end{abstract}
\maketitle


\section*{Relative Changes and Relative Rate of Change}
Relative changes and relative errors are often more meaningful than absolute changes and errors. For example, I might measure the distance from Shoreline's Central Market to the Richmond beach library to be $5$ km with an  error of at most $0.2$ km, while NASA might measure the between the earth and the moon on the first day of spring to be $384,400$ km with an error of at most $100$ km. The relative error in my measurement is at most
\[
  \frac{0.2 \text{ km} }{5 \text{ km}} = 0.04 = 4\% ,
\] 
while the relative error in NASA's measurement is at most
\[ 
   \frac{100 \text{ km} }{384,400 \text{ km}} \sim 0.00026 = 0.026\% .
\]


\begin{question}  \label{Q:354rgbytt}
The function
\[
    W = f(t) = 200+4t +2t^2 , 0\leq t \leq 12 ,
\]
expresses the weight (in pounds) of a baby elephant in terms of its age (in months).

(a) Find the average rate at which the elephant gained weight between ages 4 and 10 months.

(b) Find the relative average rate at which the elephant gained weight between ages 4 and 10 months.

(c) Find the relative instantaneous rate at which the elephant is gaining weight at age 4 month.

(d) Find the relative instantaneous rate at which the elephant is gaining weight at age 10 months.

(e) Use the graph below to interpret your answers to parts (b)-(d) geometrically.


\begin{onlineOnly}
    \begin{center}
\desmos{2xj6xy7ggo}{900}{600}
\end{center}
\end{onlineOnly}


Desmos activity available at \href{https://www.desmos.com/calculator/2xj6xy7ggo}{151: Elephant}



\end{question}



\section*{Exponential Growth}
\begin{question}  \label{Qr435rhgnbh}
(a) What does it mean for a population to grow exponentially?

(b) Is it possible for a population to increase by $20\%$ every year and not grow exponentially?
\end{question}


\begin{question} \label{Qewrdfst5t}
Suppose between noon and 10pm a colony of bacteria grows exponentially. The population is 200,000 at noon and 242,200 at 1pm.

(a) Describe how the population grows. Keeping Question 2(b) in mind, is  your description sufficient?

(b) How might we find a complete description of the exponential growth?

(c) Determine the relative average growth rate between noon and 12:30pm. Between 1pm and 1:30pm. Over any half-hour time period. Use the slider $u$ in the graph below to intrepret these rates geometrically.

(d) Approximate the instantaneous relative growth rates at noon and at 1pm. Modify the definition of $v=u+1/2$ in the demonstration below and interpret these rates geometrically.

(e) Use limits to write an expression that gives the instantaneous growth rate at time $t$ hours past noon. What can you conclude? 

(f) Try to answer part (b) again.


\begin{onlineOnly}
    \begin{center}
\desmos{wvpsotdhby}{900}{600}
\end{center}
\end{onlineOnly}


Desmos activity available at \href{https://www.desmos.com/calculator/wvpsotdhby}{151: Exp Growth 1}





\end{question}






\section*{The Derivative as a Magnification Factor}
\begin{exploration}   \label{Ex:325gyt}

It sometimes helps to think of the derivative as a magnification factor that maps a small interval around an input to a function to a corresponding interval around the output.

(a) Use this idea for the function $y=f(x)$ graphed below to approximate the derivatives 
\[
    \frac{d}{dx} (f(x))\Big|_{x=4} \text{  and  } \,\, \frac{dy}{dx}\Big|_{x=5} .
\]


\begin{onlineOnly}
    \begin{center}
\desmos{la4f5ots3r}{900}{600}
\end{center}
\end{onlineOnly}


Desmos activity available at \href{https://www.desmos.com/calculator/la4f5ots3r}{151: Magnification Factor 1}

\end{exploration}


\section*{Differentiating the Exponential Function $e^x$}

\begin{exploration}

(a) Use the graph of the function $y=f(x)$ below to approximate the derivatives 
\[
    \frac{dy}{dx}\Big|_{y=k} \text{ for } k=1, 2, \ldots , 6.
\]
Note the above derivatives are evaluated at the \emph{outputs} of the function $f$. 

(b) What do  you notice?

\begin{onlineOnly}
    \begin{center}
\desmos{k08dphtuca}{900}{600}
\end{center}
\end{onlineOnly}


Desmos activity available at \href{https://www.desmos.com/calculator/k08dphtuca}{151: Magnification Factor 2}


\end{exploration}

\begin{question}  \label{Qdefrt4trt}
The function 
\[
    P = 400 e^t , 0\leq t \leq 2
\]
expresses the population (in thousands) of a colony of bacteria in terms of the number of hours past noon.

(a) What are the units of the factor $400$?

(b) What are the units of the exponent in the factor $e^t$? Be careful.

(c) Find the (instantaneous) growth rate of the population when there are 1,200,000 bacteria.

(d) Find the relative (instantaneous) growth rate when there are 1,200,000 bacteria.

(e) Find the  relative (instantaneous) growth rate at any time.

(f) Approximate the population $30$ seconds after there are 1,200,000 bacteria.
\end{question}


\begin{question}  \label{Qdfdsf4thn}
Let $k>0$ be a constant and let
\[
     f(x) = k e^x .
\]
For $a\in \mathbb{R}$ let point $P$ with coordinates $(a, f(a))$ be a curve on the curve$y=f(x)$. Let $Q$ be the point where the tangent line to the curve intersects the $x$-axis and let $R$ be the point with coordinates $(a,0)$

(a) Find the length of segment $\overline{QR}$.

(b) How is part (a) related to part (e) of the previous question?

\end{question}


\end{document}