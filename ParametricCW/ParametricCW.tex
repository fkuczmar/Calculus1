\documentclass{ximera}
\title{Parametrically-defined Curves, CW}

\newcommand{\pskip}{\vskip 0.1 in}

\begin{document}
\begin{abstract}
Curves defined parametrically and signed area.
\end{abstract}
\maketitle

\section{The Basics}

Not all curves are the graphs of functions.

The green curve below in the first quadrant, for example, is neither the graph of $y$ as a function of $x$ (ie. $y=f(x)$) nor of $x$ as a function of $y$ (ie. $x=f(y)$).

Why not?
\begin{freeResponse}
\end{freeResponse}

\begin{onlineOnly}
    \begin{center}
\desmos{vio0b8ztnl}{450}{600}  
\end{center}
\end{onlineOnly}

\href{https://www.desmos.com/calculator/vio0b8ztnl}{152: Parametric 2}


But we can describe the curve \emph{parametrically}. This means to give functions
\[
      x = f(t) \text{ and } y=g(t)
\]
that express the coordinates $(x,y)$ of a point on the curve in terms of some parameter ($t$ here) which is often helpful to think of as being time. We can then think dynamically of the \emph{motion} of a point along its path instead of just a static path.

These coordinate functions for the green curve above are graphed in the fourth ($x=f(t)$) and second ($y=g(t)$) quadrants and share the domain $0\leq t\leq 4$ (you should ignore the negative signs on the vertical and horizontal axes). To see the motion of the point along its path, play the slider $u=t$ in Line 2. 

The question we would like to address here is to compute the unsigned (ie. positive) area of the region bounded by the curve 
\[
  (x,y) = (f(t), g(t)) \, , \, 0\leq t \leq 4 .
\]

As usual, our first choice is to decide which way to slice the region and we'll start by slicing perpendicular to the $x$-axis. But this is not really an accurate description because the differential slices run from the $x$-axis to the curve and do not always lie within the region. 

To see this, move the slider the slider $u$ in Line 2 to $u=0.50$ and the slider $v = \Delta t$ in Line 4 to $v=0.3$. You'll see a shaded rectangle that does not lie within the region. Nevertheless, we can use these differential rectangles to compute the area of the region.

The key idea is to work with their \emph{signed} areas. Cutting perpendicular to the $x$-axis, the differential rectangles have signed area
\begin{align*}
 dA & = y \, dx \\
      & = g(t) d(f(t)) \\
      & = g(t) (f^\prime(t) \, dt) \\
      & = g(t) f^\prime (t) \, dt .
\end{align*}

And so the region bounded by the green curve and the $x$-axis has signed area
\[
   A = \int_0^4 g(t) f^\prime(t) \, dt.
\]

Let's stop and think a minute about what we mean by signed area. The sign of the differential area $dA$ depends upon \emph{both} the signs of $y=g(t)$ and $dx = f^\prime (t) \, dt$. For our example $y = g(t) \geq 0$, and so $dA$ has the same sign as $dx$. 

\begin{enumerate}

\item Drag the slider $u$ in Line 2 from $u=0$ to $u=b=4$. For what values of $u$ is $dA$ positive? Negative?

\item Sketch by hand a rough graph of the cumulative signed area function
\[
      A_1(t) = \int_0^t g(u) f^\prime(u) \, du.
\] 
Then drag the slider $u$ in Line 2 to $u=0$ and activate the folder \emph{Graph of signed area function, slices perp to $x$-axis} in Line 16. Drag $u$ to $u=b=4$ and see the graph of the signed area function. Compare it with your graph.

\item How is the integral 
\[
  A_1(4) = \int_0^4 g(t) f^\prime(t) \, dt
\]
related to the area of the region bounded by the $x$A-axisa and the curve $(x,y) = (f(t), g(t))$? Explain your reasoning.
\begin{freeResponse}
\end{freeResponse}

We can also take our differential rectangles perpendicular to the $y$=axis. Then
\[
   dA = x \, dy = \answer{f(t) g^\prime(t)}\, dt .
\]

\item Repeat parts (a)-(c) above for these differential rectangles and the signed area function
\[
     A_2(t) = \int_0^t f(u) g^\prime(u) \, du .
\]

\item Now use the actual coordinate functions
\[
     x = f(t) = t^2 - 3t + 3 \, , \, 0\leq t \leq 4
\]
and
\[
   y = g(t) = 4t - t^2 \, , \, 0\leq t \leq 4
\]
to compute the area of the region bounded by the $x$-axis and the curve
\[
   (x,y) = (f(t), g(t)) \, , \, 0\leq t \leq 4.
\]
Do this twice. Once with slices perpendicular to the $x$-axis and again with slices perpendicular to the $y$-axis.

\end{enumerate}

\section{Exercises}

\begin{exercise}  \label{EOfidf3r3r2}

This question is about the circle
\[
     (x,y) = (f(\theta), g(\theta)) = (a \cos \theta , b+a\sin\theta) \, , \, 0\leq \theta \leq 2\pi .
\]
Here $a,b$ are positive constants measured in meters.

\begin{onlineOnly}
    \begin{center}
\desmos{weihqr1fxu}{450}{600}  
\end{center}
\end{onlineOnly}

\href{https://www.desmos.com/calculator/weihqr1fxu}{152: Parametric 3}

\begin{enumerate}

\item What 
\item Use the worksheet above to sketch by hand the signed area functions
\[
     A_1(\theta) = \int_0^{t} g(t) f^\prime (t) \, dt
\]
and
\[
  A_2 (\theta) = \int_0^{t} f(t) g^\prime (t) \, dt .
\]

\item Evaluate the integrals
\[
       A_1(2\pi) = \int_0^{2\pi} g(\theta) f^\prime (\theta) \, d\theta
\]
and 
\[
   A_2(2\pi) = \int_0^{2\pi} f(\theta)g^\prime (\theta) \, d\theta 
\]
\emph{geometrically}.

\emph{Hint:} Use the graph of the function
\[
     A = a^2 \sin^2\theta \, , \, 0\leq \theta \leq \pi
\]
shown below. Work in general, not with the specific value of $a$ in the worksheet.


\begin{onlineOnly}
    \begin{center}
\desmos{dsoqsyyfba}{450}{600}  
\end{center}
\end{onlineOnly}

\href{https://www.desmos.com/calculator/dsoqsyyfba}{152: Sine Squared}
\end{enumerate}
\end{exercise}

\begin{exercise} \label{EKdfe34dfds}
Use the parameteriziation
\[
    (x,y) = (a\cos\theta, a \sin\theta) \, , \, 0\leq \theta \leq 2\pi,
\]
of the circle with radius $a$ meters centered at the origin to find an expression for the volume of a sphere with the same radius.

\end{exercise}

\begin{exercise} \label{Epdfsderenn}
Find the area bounded by the astroid
\[
 (x,y) = (a\cos^3\theta, a \sin^3\theta) \, , \, 0\leq \theta \leq 2\pi ,
\]
where $a>0$ is a constant measured in meters.
\end{exercise}


\begin{exercise}  \label{EKDferf3vbbb}
Find the area bounded by the ellipse
\[
   \frac{x^2}{a^2} + \frac{y^2}{b^2} = 1, 
\]
where $a,b>0$ are constants measured in meters. 

Start by parameterizing the ellipse.
\end{exercise}

\end{document}