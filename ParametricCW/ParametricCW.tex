\documentclass{ximera}
\title{Parametrically-defined Curves, CW}

\newcommand{\pskip}{\vskip 0.1 in}

\begin{document}
\begin{abstract}
Curves defined parametrically and signed area.
\end{abstract}
\maketitle

\section{The Basics}

Not all curves are the graphs of functions.

The green curve below in the first quadrant, for example, is neither the graph of $y$ as a function of $x$ (ie. $y=f(x)$) nor of $x$ as a function of $y$ (ie. $x=f(y)$).

Why not?
\begin{freeResponse}
\end{freeResponse}

\begin{onlineOnly}
    \begin{center}
\desmos{vio0b8ztnl}{450}{600}  
\end{center}
\end{onlineOnly}

\href{https://www.desmos.com/calculator/vio0b8ztnl}{152: Parametric 2}


But we can describe the curve \emph{parametrically}. This means to give functions
\[
      x = f(t) \text{ and } y=g(t)
\]
that express the coordinates $(x,y)$ of a point on the curve in terms of some parameter ($t$ here) which is often helpful to think of as being time. We can then think dynamically of the \emph{motion} of a point along its path instead of just a static path.




\end{document}