\documentclass{ximera}
\title{The Chain Rule}

\newcommand{\pskip}{\vskip 0.1 in}

\begin{document}
\begin{abstract}
An introduction to the chain rule.
\end{abstract}
\maketitle

You ride a ferris wheel for one revolution and get off. The function 
\[
    h = f(\theta) = \, , \, 0\leq \theta \leq 2\pi    %40 - 30\cos \theta \, , \, 0\leq \theta \leq 2\pi ,
\]
expresses your height (in feet) in terms of the wheel's angle of rotation (measured in radians from the moment you boarded).

The function 
\[
    \theta = g(t) \, , \, 0\leq t \leq 44, 
\]
expresses the rotation angle of the wheel in terms of the number of seconds since  you boarded.

Use the graphs of the function $f$ and $g$ below (take the times $t$ to be postive, not negative as shown) to approximate each of the following:

(a) The value of the derivative 
\[
   \frac{dh}{d\theta}\Big|_{t = 16} .
\]

(b) The rotation rate of the wheel 16 seconds after you boarded.

(c) The value of the derivative 
\[
    \frac{d\theta}{dt}\Big|_{t=16} .
\]

(d) The rate at which you are ascending at time $t=16$ seconds after you boarded.

(e) The value of the derivative
\[
   \frac{dh}{dt}\Big|_{t=16}  =  h^\prime(16),
\]
where $h(t) = f(g(t))$.

\begin{onlineOnly}
    \begin{center}
\desmos{bqlbmfzzxf}{900}{600}
\end{center}
\end{onlineOnly}


Desmos activity available at \href{https://www.desmos.com/calculator/bqlbmfzzxf}{151: Ferris Wheel 2}



\end{document}
