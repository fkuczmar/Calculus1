\documentclass{ximera}
\title{The Chain Rule}

\newcommand{\pskip}{\vskip 0.1 in}

\begin{document}
\begin{abstract}
An introduction to the chain rule.
\end{abstract}
\maketitle


The chain rule tells us how to differentiate the composition of two functions. It say that the derivative of the composition is the product of the derivatives of the two functions.

Here is a more formal statement.

\begin{theorem}
(The Chain Rule) If $y=f(u)$ is a differentiable function of $u$ and $u=g(t)$ is a differentiable function of $t$, then the composition
\[
      y = f(g(t))
\]
is a differentiable function of $t$ and
\[
   \frac{dy}{dt} = \frac{dy}{du} \cdot \frac{du}{dt} .
\]
Or paying closer attention to the inputs,
\[
      \frac{dy}{dt}\Big|_{t=t_0} = \frac{dy}{du}\Big|_{u=g(t_0)} \cdot \frac{du}{dt}\Big|_{t=t_0} . 
\]

\end{theorem}


We'll go through some examples to get an understanding of how to use the chain rule and also why it works.


\begin{example}  \label{Ex:GDFGDFGff}
Find an equation of the tangent line to the curve
\[
  y = f(x) = \left( 2x^3 +1  \right)^2
\]
at the point $(1,9)$.

\begin{explanation}
Let 
\[
       y = \left( 2x^3 +1  \right)^2
\]
and 
\[
      u = 2x^3 + 1 .
\]
Then
\[
     y = u^2
\]
and
\begin{align*}
\frac{dy}{dx} &= \frac{dy}{du} \cdot \frac{du}{dx}  \\
                     &= \frac{d}{du} \left( u^2 \right)  \cdot \frac{d}{dx}\left(  2x^3 + 1 \right)  \\
                     &= 2u (6x^2)  \\
                     &= 2(2x^3+1)(6x^2) .
\end{align*}

Then the slope of the tangent line to the curve $y=(2x^3+1)^2$ at the point $(1,9)$ is 
\[
              \frac{dy}{dx}\Big|_{x=1} = 2(3)(6) = 36 ,
\]
and an equation of the tangent line is
\[
   y - 9 = 36(x-1) .
\]

\pskip 

We can get the same result without the chain rule, by rewriting the original function as
\[
     y = f(x) = (2x^3+1)^2 = 4x^6 + 4x^3 + 1.
\] 
Then
\begin{align*}
      \frac{dy}{dx} &= \frac{d}{dx} \left(  4x^6 + 4x^3 + 1  \right) \\
                         &= 4\frac{d}{dx}\left( x^6 \right) + 4 \frac{d}{dx}\left( x^3 \right) + \frac{d}{dx}\left(1 \right) \\
                        &=   24x^5 + 12x^2 .
\end{align*}

Then the slope of the tangent line to the curve $y=(2x^3+1)^2$ at the point $(1,9)$ is 
\[
              \frac{dy}{dx}\Big|_{x=1} =14 + 12 = 36 ,
\]
as before.

\end{explanation}
\end{example}


\begin{example}  \label{Ex:DFdg4ergg}
Find an equation of the tangent line to the curve
\[
     x^2 + y^2 = 25
\]
at the point $P(4,-3)$.

\begin{explanation}
Solve the above equation for $y$ in terms of $x$ to get
\[
      y = \pm \sqrt{25-x^2} .
\] 
While this equation does not define $y$ as a function of $x$, near the point $(4,-3)$, the equation
\[
         y = - \sqrt{25-x^2} 
\]
does define $y$ as a function of $x$.

Now to find the derivative of the function
\[
          y = - \sqrt{25-x^2} ,
\]
let 
\[
       u = 25 - x^2 .
\]
Then
\[
    y = \sqrt{u} = - u^{1/2}
\]
and 
\begin{align*}
\frac{dy}{dx} &= \frac{dy}{du} \cdot \frac{du}{dx}  \\
                     &= \frac{d}{du} \left(- u^{1/2} \right)  \cdot \frac{d}{dx}\left(  25-x^2 \right)  \\
                     &= -\frac{1}{2\sqrt{u}} (-2x)  \\
                     &= \frac{x}{\sqrt{25-x^2}}.
\end{align*}

So the slope of the tangent line to the curve
\[
       x^2 + y^2 =25
\]
at the point $(P(4,-3)$ is
\[
             \frac{dy}{dx}\Big|_{x=4} = \left(  \frac{x}{\sqrt{25-x^2}} \right) \Big|_{x=4} = \frac{4}{3}
\]
and an equation of the tangent line is
\[
     y +3 = \frac{4}{3} (x-4).
\]

To find the slope of the tangent line without appealing to the chain rule, note that the curve
\[
              x^2 + y^2  = 25
\]
is a circle centered at the origin. So the tangent line at $P(4,-3)$ is perpendicular to the radius $\overline{OP}$ from the origin to $P$. Since $\overline{OP}$ has slope $m_1 = -3/4$, the tangent line at $P(4,-3)$ has slope
\[
     m_2 = - 1 / m_1  = 4/3 .
\]
\end{explanation}
\end{example}


\begin{question}  \label{Ex:LKKkdftr}
(a) Describe a transformation that takes the circle
\[
     x^2 + y^2 = 25
\]
to the ellipse
\begin{equation}
     4x^2 + y^2 = 25.     \label{Eq:Ellipse}
\end{equation}

(b) Find the image (call it $Q$) of the point $P(4,-3)$ under the transformation in part (a).

(c) Use the result of Example 2 to find an equation of the tangent line to the ellipse (\ref{Eq:Ellipse}) at $Q$.

(d) Use the chain rule to first find the slope of the tangent line to the ellipse (\ref{Eq:Ellipse}) at $Q$. Then find an equation of the tangent line.

\end{question}


\begin{question}  \label{Ex:pdfsd0tr}
The function 
\[
       P  = f(t) = 22 + \frac{t}{2} - \frac{3t^2}{4} , 0\leq t \leq 3 , 
\]
expresses the price of a stock (in dollars/share) in terms of the number of hours past 9am.

(a) Is the price of the stock increasing or decreasing at 11am? At what rate? At what relative rate?

(b) Is the number of shares you can buy with $\$1000$ increasing or decreasing at 11am? At what rate? At what relative rate? 
Start your solution to this problem by defining a new function with a new function name and a new depedent variable.

\end{question}


\begin{question} \label{Ex:sd0fset}
At 10am the price of a stock is increasing at the relative rate of $p\%$/hr. Is the number of shares you can buy with $\$1000$ increasing or decreasing at 10am? At what relative rate?

Answer this question in two ways:

(i) Going back to limits and using the definition of the derivative.

(ii) Using the chain rule.

Either way, start your solution with defintions as in the previous problem.

\end{question}


\begin{question}  \label{Q:df0g43rb4}
A tree leans precariously with its trunk making an angle of $\phi = \pi/6$ radians with the ground. One end of a ten-foot ladder leans against the trunk, the other rests on the horizontal ground.

Let $t$ be the distance between the top of the ladder and the base of the trunk (measured in feet) and $s$ the distance between the bottom of the ladder and the base of the trunk (also measured in feet).

(a) Use the law of cosines to write an equation relating $t$ and $s$.

(b) Use the result of part (a) to find the two possible values of $t$ when $s=16$. Give exact values (ie. not decimal approximations from a calculator).

(c) We'll focus now on the ladder in its various positions when $s \sim 16$ and $t\sim 8$. For these positions, complete the square to find a function 
\[
    t = f(s)
\]
that expresses the distance (in feet) from the top of the ladder to the base of the trunk in terms of the distance (in feet) from the bottom of the ladder to the trunk's base.

(d) Use your function from part (c) to find an expression for the derivative $dt/ds$.

(e) Evaluate the derivative 
\[
       \frac{dt}{ds}\Big|_{s=16} .
\]

(f) What are the units of the derivative in part (e)? Explain the meaning of the derivative, \emph{not} by giving a standard and meaningless response about the rate of change of some quantity with respect to another, but by relating small changes.

(g) Use your response to part (f) to write an approximation for the change
\[
         \Delta t = f(s) - f(16)
\]
in terms of the change 
\[
     \Delta s = s - 16
\]
for values of $s$ near $s=16$. Use this approximation to estimate the distance between the top of the ladder and the base of the trunk when the bottom of the ladder is 15.6  feet from the trunk’s base. Compare your approximation with the exact distance.
\end{question}


\begin{example}  \label{Ex:CHr34rrer}
(a) Find the slope of the tangent line to the curve
\[
   y = f(\theta) = 6 \sin \theta
\]
at the point with coordinates $P(2\pi/3, 3\sqrt{3})$. We do not need the chain rule for this.

(b) Describe a transformation that takes the graph of $y=f(\theta)$ to the graph of the function
\[
        y = g(\theta) = 6\sin (\theta/2) .
\]

(c) Find the image of the point $P(2\pi/3,3\sqrt{3})$ under the above transformation.

(d) Use the results of parts (a) - (c) and the demonstration below to guess the slope of the tangent line to the curve $y=g(\theta)$ at the point $Q(4\pi,3\sqrt{3})$.

\begin{onlineOnly}
    \begin{center}
\desmos{mqjxpsqyo5}{900}{600}
\end{center}
\end{onlineOnly}

Desmos activity available at \href{https://www.desmos.com/calculator/mqjxpsqyo5}{151: Chain Rule 1}

\pskip

(e) Use the chain rule to confirm your guess from part (d).

\end{example}


\begin{example}  \label{Ex:Chdfr5llk}
The function 
\[
       P = f(t) = 10 e^{\frac{1}{4}t} \, , \, -4\leq t \leq 12 ,
\] 
expresses the population (in millions) of a colony of bacteria in terms of the number of hours past noon.

(a) What are the units of $1/4$ in the function above? How do you know?

(b)  Use the chain rule to find an expression for the growth rate of the population at time $t$ hours past noon. What are the units of the growth rate?

(c) Express the growth rate from part (b) in terms of the population $P=f(t)$ at time $t$ hours past noon.

(d) What is the relative instantaneous growth rate of the population? Include units.

(e) Find the relative average growth rate of the population over a one-hour period. 

(f) Describe what happens to the population every hour.

\end{example}

\begin{example}  \label{Ex:CHHODDrer}
 (a) Use the chain rule to find the relative instantaneous growth rate of the population function
\[
      P = 10 (2^t)  \, , \, -2\leq t \leq 5 ,
\]
where $P$ is measured in millions of bacteria and $t$ is the number of hours past noon.

(b) Describe what happens to the population every hour.

(c) Estimate the population 2 minutes after there are 30 million bacteria.
\end{example}




\begin{example}  \label{Ex:CHdfthghhyhpp}

You ride a ferris wheel for one revolution and get off. The function 
\[
    h = f(\theta) = \, , \, 0\leq \theta \leq 2\pi    %40 - 30\cos \theta \, , \, 0\leq \theta \leq 2\pi ,
\]
expresses your height (in feet) in terms of the wheel's angle of rotation (measured in radians from the moment you boarded).

The function 
\[
    \theta = g(t) \, , \, 0\leq t \leq 44, 
\]
expresses the rotation angle of the wheel in terms of the number of seconds since  you boarded.

Use the graphs of the function $f$ and $g$ below (take the times $t$ to be postive, not negative as shown) to approximate each of the following:

(a) The value of the derivative 
\[
   \frac{dh}{d\theta}\Big|_{t = 16} .
\]

(b) The rotation rate of the wheel 16 seconds after you boarded.

(c) The value of the derivative 
\[
    \frac{d\theta}{dt}\Big|_{t=16} .
\]

(d) The rate at which you are ascending at time $t=16$ seconds after you boarded.

(e) The value of the derivative
\[
   \frac{dh}{dt}\Big|_{t=16}  =  h^\prime(16),
\]
where $h(t) = f(g(t))$.

\begin{onlineOnly}
    \begin{center}
\desmos{bqlbmfzzxf}{900}{600}
\end{center}
\end{onlineOnly}

Desmos activity available at \href{https://www.desmos.com/calculator/bqlbmfzzxf}{151: Ferris Wheel 2}

\end{example}

\end{document}
