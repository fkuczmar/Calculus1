\documentclass{ximera}
\title{Substitution and Parametrically-defined Curves}

\newcommand{\pskip}{\vskip 0.1 in}

\begin{document}
\begin{abstract}
Substitution and parameterized curves
\end{abstract}
\maketitle

\section{Theh Basics}

\begin{question}\label{Q7yyweff}
Identify for which of the following definite integrals a $u$-substitution would work. Make the substitution for these and evaluate the integral. Do this by changing the lower and upper bounds and not substituting back.

\begin{enumerate}

\item $\int_0^{\sqrt{\pi}} \theta \sin(\theta^2)\, d\theta$

\item $\int_0^{\pi} \theta^2 \sin(\theta)\, d\theta$

\item $\int_0^2 x^2 (1+x^2)^2 \, dx$

\item $\int_0^2 x^2 (1+x^3)^2 \, dx$

\item $\int_1^2 \frac{\cos \left( \sqrt{x} \right)}{\sqrt{x}}\, dx$

\item $\int_1^2 \sqrt{x} \cos \left( \sqrt{x} \right)\, dx$

\item $\int_0^1 \frac{e^x}{(1+e^x)^2} \ , dx$

\item $\int_0^1 \frac{e^x}{1+e^x} \ , dx$

\item $\int_0^1 \frac{e^x}{1+e^{2x}} \ , dx$


\end{enumerate}

\end{question}


\section{A Geometric Interpretation}
To understand what $u$-substitution does geometrically, it helps to think about making a \emph{reverse} $u$-substitution. This means starting with a simple integral and making it more complicated with a substitution. Here's an example.

\begin{example} \label{EKEMrerMMEF}
We'll start with the integral
\[
      I = \int_0^\pi \sin u \, du
\]
and make the substitution 
\[
     u = t^2 .
\]
This substitution really reparameterizes the curve
\[
   y = f(u) = \sin u \, , \, 0\leq u \leq \pi
\] 
from the standard parameterization
\[
   (x,y) = (u,\cos u) \, , \, 0\leq u \leq \pi
\]
to the new parameterization
\[
   (x,y) = (t^2,\cos (t^2) \, , \, 0\leq t \leq \sqrt{\pi}.
\]

The substitution maps the rectangle with height $y=\sin u$ and width $du$ to the rectangle with the same height $y=\sin(t^2)$ and width
\[
   du = 2t \,dt .
\]
The effect is to transform a partition of the interval of integration, originally $u\in [0,\pi]$ with equal subintervals into a partition of the new interval of integration $t\in [0,\sqrt{2\pi}]$ with \emph{unequal} subintervals $du=2t\, dt$. You can see this transformation by dragging the slider $w$ in Line 2 of the worksheeet below from $w=0$ to $w=1$.



\begin{onlineOnly}
    \begin{center}
\desmos{toiwzzqkva}{450}{600}  
\end{center}
\end{onlineOnly}

\href{https://www.desmos.com/calculator/toiwzzqkva}{152: Sub 2}

\end{example}

The next example puts this idea in more a more meaningul context.

\end{document}

