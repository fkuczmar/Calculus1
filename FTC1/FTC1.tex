\documentclass{ximera}
\title{The Fundamental Theorem of Calculus, Part 2}

\newcommand{\pskip}{\vskip 0.1 in}

\begin{document}
\begin{abstract}
FTC
\end{abstract}
\maketitle


\begin{question} \label{QPfeEFVVD}

The function 
\[
 v=f(t) \, , \, , 0 \leq t \leq 8, 
\]
expresses the speed (in feet/min) of a beetle in terms of the number of minutes past noon.

\begin{enumerate}

\item Write an expression for the distance the beetle crawls from 12:03pm to 12:07pm. Use the graph of the function $v=f(t)$ below to approximate this distance.

\item Write an equation you would solve to determine when the beetle is halfway to its destination. Start by defining an unknown (in a complete sentence, with units). Do not use $t$, it is already taken.

\item Use the graph of the function $v=f(t)$ below to approximate the time in part (b). But first determine whether this time is before or after 12:04pm.

\begin{onlineOnly}
    \begin{center}
\desmos{gd54bqgk6l}{450}{600}  
\end{center}
\end{onlineOnly}

\href{https://www.desmos.com/calculator/gd54bqgk6l}{152:Beetle 1}


\item Suppose 
\[
      v(t) = \frac{20}{(t+2)^2}. 
\] 
and determine the exact time when the beetle is halfway to its destination.



\end{enumerate}

\end{question}


\begin{question} \label{Q3edFGee}

The function
\[
 r =f(t) \, , \,  2 \leq t  \leq 36,
\]
 gives the net rate (in gal/min) at which water flows into a tank at time $t$ minutes past noon.

\begin{enumerate}
\item Write an expression that gives the change in volume from 12:15pm to 12:32pm.

\item Write an equation you would solve to find when the tank holds $22$ fewer gallons than it does at 12:09pm. Start by defining an unknown (in a complete sentence, with units). Do not use $t$, it is already taken.

\item Use the graph of the function
 \[
 r =f(t) \, , \,  2 \leq t  \leq 36,
\]
 shown below to approximate the time(s) when the tank holds 22 fewer gallons than it does at 12:09. 
\end{enumerate}

\begin{onlineOnly}
    \begin{center}
\desmos{rcsfzdhsax}{450}{600}  
\end{center}
\end{onlineOnly}

\href{https://www.desmos.com/calculator/rcsfzdhsax}{152:Tank 1}

\item Suppose 
\[
     f(t) = 12 - 3\sqrt{t} , 2\leq t \leq 36 .
\] 
Use calculus and algebra to find the exact time(s) in part (c). Then use a calculator to approximate these times to the nearest second.
\end{question}


\begin{question} \label{QKDfer3fr}

%Explain your understanding of what our OER text calls Part II of the Fundamental Theorem of Calculus (page 396) and
% how it relates to the Net Change Theorem. State the theorem, explain what
 %it says, and give some computational examples.
 %Then explain why the theorem is true, giving intuitive examples illustrat
%ing a variety (at least 3) applications. In the course of your explanation,
% address and answer the following question.
 
The function
\[
 m=f(s) \, , \, 30\leq s \leq 300
\]
expresses your gas mileage (in miles/gallon) in terms of your distance from home. The distance is measured along your route in miles. Its graph is shown below.

\begin{onlineOnly}
    \begin{center}
\desmos{v34aycmz93}{450}{600}  
\end{center}
\end{onlineOnly}

\href{https://www.desmos.com/calculator/v34aycmz93}{152:Mileage 89}



Your tank has $12$ gallons of gas when you are $50$ miles from home.

\begin{enumerate}
\item Suppose you are driving away from home. 

\begin{enumerate}

\item Use the graph above to approximate the number of gallons of gas in your tank when you are $150$ miles from home.

\item Find a function $G_1 = g_1(s)$ that expresses the number of gallons in your tank in terms of your distance (measured along your route in miles) from home.

\end{enumerate}

\item Suppose you are driving home. 

\begin{enumerate}

\item Use the graph above to approximate the number of gallons of gas in your tank when you are $150$ miles from home.

\item Find a function $G_2 = g_2(s)$ that expresses the number of gallons in your tank in terms of your distance (measured along your route in miles) from home.


\end{enumerate}

\end{enumerate}

% Finally, point out any aspects of the theorem that you do not fully under
%stand or ask about something you would like to know more about.

\end{question}

\begin{question} \label{QLkdR343RE}
The function 
\[
r = f(t) = 100/t \, , \, a \leq  t \leq b
\]
expreses the rate (in gal/min) at which water flows into a tank in terms of the number of minutes past noon. The tank
 is empty at time $t = a$ minutes past noon and full at time $t = b$ minutes past noon.

\begin{enumerate}
\item Sketch a graph of the function $r = f(t)$. Label the axes with the appropriate units and variable names.

\item Would you expect the tank to be half-full before or after time $t = (a+b)/2$? Explain.

\item When is the tank half-full?

\item  When does water flow into the tank at a rate equal the average rate at which it ows into the tank between $t = a$ and $t = b$?

\end{enumerate}
\end{question}


\begin{question} \label{QPLDFRr3}
The function 
\[
     r = f(t) \, , \, 0\leq t \leq 12,
\]
expresses a balloon's rate of ascent (in ft/min) in terms of the number of minutes past noon. Its graph is shown below.

\begin{onlineOnly}
    \begin{center}
\desmos{yp3awkyck8}{450}{600}  
\end{center}
\end{onlineOnly}

\href{https://www.desmos.com/calculator/yp3awkyck8}{152:Balloon 11}

\begin{enumerate}
\item When is the balloon the same height as it is at 12:02pm? Explain.

\item Use the graph to approximate the time(s) when the balloon is $20$ feet higher than it was at 12:02pm.

\item Write an equation whose solution(s) give the time(s) when the balloon is $20$ feet higher than it was at 12:02pm.

\item Suppose
\[
      f(t)=80\sin\left( \frac{\pi}{6}t \right) .
\]
Use algebra to find exact solution(s) to your equation from part (c). Then use a calculator to approximate the clock time(s) to the nearest second.
\end{enumerate}

\end{question}



\end{document}