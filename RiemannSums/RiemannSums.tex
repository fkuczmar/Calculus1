\documentclass{ximera}
\title{Riemann Sums}

\newcommand{\pskip}{\vskip 0.1 in}

\begin{document}
\begin{abstract}
Riemann sums and summation notation.
\end{abstract}
\maketitle


\begin{question} \label{QPdfer34334}
The function
\[
   v = g(s) = 40 + \frac{s}{5} \, , \, 20\leq s \leq 100 ,
\]
expresses the speed (in miles/hr) of a car in terms of the trip odometer reading (measured in miles) during part of a 200-mile trip.

\begin{enumerate}
\item Use summation notation for a Riemann sum with $10$ equal intervals of distance to find an upper bound for the time it takes to travel the $80$ miles in this portion of the trip. Enter this in Line 1 of the worksheet below to get an  aproximation for the time.

\item Use summation notation for a Riemann sum with $10$ equal intervals of distance to find a lower bound for the time it takes to travel the $80$ miles in this portion of the trip. Enter this in Line 1 of the worksheet below to get an  aproximation for the time.

\item Use summation notation for a Riemann sum with $n$ equal intervals of distance to find an upper bound for the time it takes to travel the $80$ miles in this portion of the trip. Enter this in Line 1 below. Then drag the slider $n$ in Line 2 to get a more accurate upper bound.

\item Use summation notation for a Riemann sum with $n$ equal intervals of distance to find a lower bound for the time it takes to travel the $80$ miles in this portion of the trip. Enter this in Line 1 below. Then drag the slider $n$ in Line 2 to get a more accurate upper bound.


\end{enumerate}

\begin{onlineOnly}
    \begin{center}
\desmos{k10phbukon}{900}{600}
\end{center}
\end{onlineOnly}
 
\href{https://www.desmos.com/calculator/k10phbukon}{152: Distance Speed 2}


\end{question}



 
\begin{onlineOnly}
    \begin{center}
\desmos{vt6utnkfve}{900}{600}
\end{center}
\end{onlineOnly}



\end{document}
