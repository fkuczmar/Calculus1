\documentclass{ximera}
\title{Riemann Sums}

\newcommand{\pskip}{\vskip 0.1 in}

\begin{document}
\begin{abstract}
Riemann sums and summation notation.
\end{abstract}
\maketitle



\begin{question} \label{QPdfer34334}
The function
\[
   v = g(s) = 40 + \frac{s}{5} \, , \, 20\leq s \leq 100 ,
\]
expresses the speed (in miles/hr) of a car in terms of the trip odometer reading (measured in miles) during part of a 200-mile trip.

\begin{enumerate}
\item Use summation notation for a Riemann sum with $10$ equal intervals of distance to find an upper bound for the time it takes to travel the $80$ miles in this portion of the trip. Enter this in Line 1 of the worksheet below to get an approximation for the time. Activate the appropriate folder in Line 12 (Left endpoints) or Line 19 (Right Endpoints) to interpret the sum geometrically.

\item Use summation notation for a Riemann sum with $10$ equal intervals of distance to find a lower bound for the time it takes to travel the $80$ miles in this portion of the trip. Enter this in Line 1 of the worksheet below to get an  approximation for the time.
Activate the appropriate folder in Line 12 (Left endpoints) or Line 19 (Right Endpoints) to interpret the sum geometrically.

\item Use summation notation for a Riemann sum with $n$ equal intervals of distance to find an upper bound for the time it takes to travel the $80$ miles in this portion of the trip. Enter this in Line 1 below. Then drag the slider $n$ in Line 2 to get a more accurate upper bound.

\item Use summation notation for a Riemann sum with $n$ equal intervals of distance to find a lower bound for the time it takes to travel the $80$ miles in this portion of the trip. Enter this in Line 1 below. Then drag the slider $n$ in Line 2 to get a more accurate upper bound.


\end{enumerate}

\begin{onlineOnly}
    \begin{center}
\desmos{k10phbukon}{900}{600}
\end{center}
\end{onlineOnly}
 
\href{https://www.desmos.com/calculator/k10phbukon}{152: Distance Speed 2}
\end{question}

\begin{question} \label{QOErermern333}
The function
\[
 r = f(t) = 70 \cos^3 \sqrt{t+40} \, , \, 0\leq t \leq 48,
\]
expresses a balloon’s rate of ascent (in ft/min) in terms of the number of minutes past noon. The balloon was 300 feet high at 12:04pm.

\begin{enumerate}
\item Use summation notation for an expression with $10$ equal time intervals to find a lower bound for the balloon's height at 12:44pm. Enter this expression in Line 1 of the worksheet below to get an  approximation for the height. Activate the appropriate folder in Line 12 (Left endpoints) or Line 19 (Right Endpoints) to interpret the sum geometrically.

\item Use summation notation for an expression with $10$ equal time intervals to find an upper bound for the balloon's height at 12:44pm. Enter this expression in Line 1 of the worksheet below to get an  approximation for the  height. Activate the appropriate folder in Line 12 (Left endpoints) or Line 19 (Right Endpoints) to interpret the sum geometrically.


\item Use summation notation for an expression with $n$ equal time intervals to find a lower bound for the balloon's height at 12:44pm. Enter this expression in Line 1 of the worksheet below to get an  approximation for the height. Enter this in Line 1 below. Then drag the slider $n$ in Line 2 to get a more accurate upper bound.

\item Use summation notation for an expression with $n$ equal time intervals to find an upper bound for the balloon's height at 12:44pm. Enter this expression in Line 1 of the worksheet below to get an  approximation for the height. Enter this in Line 1 below. Then drag the slider $n$ in Line 2 to get a more accurate upper bound.

\begin{onlineOnly}
    \begin{center}
\desmos{qwuqprd9vf}{900}{600}
\end{center}
\end{onlineOnly}
 
\href{https://www.desmos.com/calculator/qwuqprd9vf}{152: Balloon 56}
 
\end{enumerate}
\end{question}


\begin{question}  \label{QIIIUER3355}
The function
\[
      \theta = g(s) = \frac{s}{2} - \frac{s^3}{12} \, , \, 0\leq s \leq 3 , 
\]
expresses the inclination angle (in radians) of the trail to Nada Lake in terms of the distance from the trailhead (measured along the trail in miles). The inclination angle is the angle the trail makes with the horizontal. It is positive when the trail slopes upward in the direction away from the trailhead.

The trail is at an elevation of $3200$ feet $0.5$ miles from the trailhead.

\begin{enumerate}
\item Use summation notation for an expression with $10$ equal intervals of distance to find a lower bound for the trail's elevation (in feet) at Nada Lake (3 miles from the trailhead). Enter this in Line 1 of the worksheet below to get an approximation for the elevation. Activate the appropriate folder in Line 12 (Left endpoints) or Line 19 (Right Endpoints) to interpret the sum geometrically.

\item Use summation notation for an expression with $10$ equal intervals of distance to find an upper bound for the trail's elevation (in feet) at Nada Lake. Enter this in Line 1 of the worksheet below to get an approximation for the elevation. Activate the appropriate folder in Line 12 (Left endpoints) or Line 19 (Right Endpoints) to interpret the sum geometrically.


\item  Use summation notation for an expression with $n$ equal intervals of distance to find an upper bound for the trail's elevation (in feet) at Nada Lake. Enter this in Line 1 below. Then drag the slider $n$ in Line 2 to get a more accurate upper bound.

\item Use summation notation for an expression with $n$ equal intervals of distance to find a lower bound for the trail's elevation (in feet) at Nada Lake. Enter this in Line 1 below. Then drag the slider $n$ in Line 2 to get a more accurate upper bound.

\begin{onlineOnly}
    \begin{center}
\desmos{qhyco8cuou}{900}{600}
\end{center}
\end{onlineOnly}
 
\href{https://www.desmos.com/calculator/qhyco8cuou}{152: Inclination Angle 34}

\end{enumerate}

\end{question}



\end{document}
