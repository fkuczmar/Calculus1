\documentclass{ximera}
\title{Linear Approximation, Part 2}

\newcommand{\pskip}{\vskip 0.1 in}

\begin{document}
\begin{abstract}
Linear approximation and relative changes.
\end{abstract}
\maketitle


\section{Review of the Natural Log Fuction}








\section{Distance to the Horizon}
\begin{example} \label{ExKdfdKREGER}
An astronaut above the surface of a planet sees only a fraction of the surface as suggested by the figure below.

\begin{onlineOnly}
    \begin{center}
\desmos{8shf1msp4m}{450}{600}  
\end{center}
\end{onlineOnly}

\href{https://www.desmos.com/calculator/8shf1msp4m}{151: Distance to Horizon 44}

The visible part of the surface is a spherical disk with spherical radius $BC$ above. We can think of this distance (an arc of a circle) as the distance to the horizon.

\begin{enumerate}
\item Find a function 
\[
   s = f(h) \, , \, h\geq 0,
\]
that expresses the distance to the horizon on a planet of radius $R$ kilometers in terms of the altitude of the astronaut (in km) above the surfarce.

\item Evaluate the derivative
\[
      \frac{ds}{dh} \Big|_{h=2R/3} .
\]
Include units.

\item Explain the meaning of the derivative in terms of a scaling factor and small changes.

\item Approximate the change in the distance to the horizon when the altitude above the surface increases from $h=(2/3)R$ to $h=(0.01 + 2/3)R$.
\end{enumerate}
\end{example}




\end{document}
