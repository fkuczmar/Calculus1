\documentclass{ximera}
\title{Linear Approximation, Part 2}

\newcommand{\pskip}{\vskip 0.1 in}

\begin{document}
\begin{abstract}
Linear approximation and relative changes.
\end{abstract}
\maketitle


\section{Review of the Natural Log Fuction}

\begin{question} \label{QPodREERve3}
Find expressions for each of the following derivatives.

\begin{enumerate}
\item $\frac{d}{dx} \left( \ln x \right)$  

\item $\frac{d}{dx} \left( \ln (4x) \right)$ 

\item $\frac{d}{dx} \left( \ln (-x) \right)$ 

\item $\frac{d}{dx} \left( \ln |x| \right)$ 

\item $\frac{d}{dx} \left( \ln (x^5) \right)$ 

\item $\frac{d}{dx} \left( \ln (1+x^2) \right)$ 

\end{enumerate}

\end{question}


\section{First Derivatives}

\begin{question}  \label{Q5dfdgndfdfdfnhn}
The function $P=f(t)$, $0\leq t \leq 12$, expresses the population of a colony of bacteria (measured in the number of bacteria) in terms of the number of hours past noon.

Suppose that
\[
     \frac{dP}{dt}\Big|_{t=4} = 400,000.
\]

\begin{enumerate}
\item What are the units of this derivative?

\item Explain the meaning of the derivative without using the language of small changes.

\item Explain the meaning of the derivative using the language of small changes by completing the following sentence.

Between 4:00pm and 4:01pm the population ...
\end{enumerate}

\end{question}


\begin{question}  \label{Q5ghldfdfgng}
The function $P=f(t)$, $0\leq t \leq 12$, expresses the population of a colony of bacteria (measured in the number of bacteria) in terms of the number of hours past noon.

Suppose that
\[
   \frac{d}{dt} \left(  \ln P \right)\Big|_{t=4} = \frac{d}{dt} \left(  \ln (f(t)) \right)\Big|_{t=4}  = \frac{3}{10} .
\]

\begin{enumerate}
\item What are the units of this derivative?

\item Explain the meaning of the derivative without using the language of small changes.

\item Explain the meaning of the derivative using the language of small changes by completing the following sentence.

Between 4:00pm and 4:01pm the population ...
\end{enumerate}
\end{question}



\begin{question} \label{QPErerDR}
The function
\[
     G =f(s) \, , \, 0\leq s \leq 130 ,
\]
graphed below expresses the number of gallons of gas in a car in terms of the trip odometer reading (measured in miles).


\begin{onlineOnly}
    \begin{center}
\desmos{gdeybih5t2}{450}{600}  
\end{center}
\end{onlineOnly}

\href{https://www.desmos.com/calculator/gdeybih5t2}{151: Relative Rates Gas}

\begin{enumerate}

\item A reading on the car's dashboard shows the number of miles left to drive before running out of gas assuming the car continues to burn gas at the current rate. Drag the slide $u$ (another name for $s$) in Line 1 to approximate this reading when the trip odometer reads $80$ miles.

\item Drag the slider $u$ in Line 1 to approximate the derivative 
\[
     \frac{d}{ds} \left(   \ln \left(\frac{G}{1}\right)  \right)\Big|_{s=80}.
\] 

\item What are the units of the derivative above?

\item Explain the meaning of the derivative.

\item Approximate the derivative $d/ds (\ln G)$ for some other values of $s$. What do you notice? What does this suggest about the function $f$?

\item In light of part (e), how would you describe what it means for a population of bacteria to decrease at the constant relative rate of $5\%$/hour? The rate is instantaneous.
\end{enumerate}

\end{question}




\section{Second Derivatives}

\begin{question} \label{Q59dfgnmnmcxcv}
The function $P=f(t)$, $0\leq t \leq 12$, expresses the population of a colony of bacteria (measured in the number of bacteria) in terms of the number of hours past noon.

Suppose that
\begin{equation}
   \frac{d^2 P}{dt^2} \Big|_{t=4} =  -6,000 .  %\label{Eq:SecondD2}
\end{equation}


\begin{enumerate}
\item What are the units of this derivative?

\item Explain the meaning of the derivative without using the language of small changes.

\item Explain the meaning of the derivative using the language of small changes by completing the following sentence.

Between 4:00pm and 4:01pm  ...

\item Suppose also that
\[
   \frac{dP}{dt}\Big|_{t=4} = 400,000 .
\]
Explain the meaning of the second derivative above \emph{without} using the language of small changes by completing the following sentence.

At 4:01pm  ...

\end{enumerate}
\end{question}






\end{document}
