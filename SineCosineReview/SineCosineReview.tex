\documentclass{ximera}
\title{Review of Sine and Cosine}

\newcommand{\pskip}{\vskip 0.1 in}

\begin{document}
\begin{abstract}
Some problems.
\end{abstract}
\maketitle


\section{Transformations}

\begin{question}  \label{QPODFerRE}
\begin{enumerate}

\item The point with coordinates $(a,b)$ lies on the graph of the function $y=f(x)$. What are the coordinates of the corresponding point on the graph of the function 
\[
  y = g(x) = f(x/2)?
\]

\item Express the derivative
\[
    \frac{d}{dx}\left( g(x) \right)\Big|_{x=2a} =  \frac{d}{dx}\left( f(x/2) \right)\Big|_{x=2a}
\]
in terms of the derivative
\[
       \frac{d}{dx}\left(  f(x) \right)\Big|_{x=a} .
\]

\item Compute the slopes of the tangent lines to the curves $y=\sin\theta$ and $y=\sin(\theta/2)$ at the origin. Drag the slider $k$ in the worksheet below to visualize the transformation.

\href{https://www.desmos.com/calculator/2ws2wqg5hv}{151: Tranform Sine}.

 
\begin{onlineOnly}
    \begin{center}
\desmos{2ws2wqg5hv}{900}{600}
\end{center}
\end{onlineOnly}


\item Which of the following curves share the same tangent line at the origin with the curve $y=\sin\theta$?

\begin{selectAll}
\choice{$y=2\sin\theta$}
\choice{$y=2\sin (2\theta)$}
\choice[correct]{$y=2\sin(\theta/2)$}
\choice[correct]{$y=\frac{1}{3}\sin(3\theta)$}
\choice{$y=\frac{1}{3}\sin(\theta/3)$}
\end{selectAll}


\end{enumerate}
\end{question}


\section{Simple Harmonic Motion}

\begin{question} \label{QODFIER33r3r}
A relaxed spring has a length of $3$ meters. A mass is attached to the right end of the relaxed spring and then pulled an additional $2$ meters to the right. The mass is then released from rest and oscillates without friction on a horizontal surface about its equilibirium postion ($x=0$ meters) with a period of $5$ seconds as shown below. 

A consequence of Hooke's Law (that the force of the spring acting on the mass points toward equilibrium and has magnitude proportional to the distance of the mass from equilibrium) is that the mass oscillates in simple harmonic motion about its equilibrium position.  This means that the motion of the mass is driven by uniform circular motion around the circle of radius $5$ meters below.

\href{https://www.desmos.com/calculator/noxooak1au}{151: Simple Harmonic motion 23}.

 
\begin{onlineOnly}
    \begin{center}
\desmos{noxooak1au}{900}{600}
\end{center}
\end{onlineOnly}


\begin{enumerate}

\item Find a function
\[
     s = f(t) \, , \, t\geq 0, 
\]
that expresses the displacement of the mass from equilibrium in terms of the number of seconds since the mass was released.

\item Find a function 
\[
 v = g(t) \, , \, t\geq 0, 
\]
that gives the horizontal component of the velocity of the mass (in meters/sec) in terms of the number of seconds since release. Take this component to be negative when the mass is moving to the left, positive when moving to the right.

\item Use your function from part (b) to find the maximum speed of the mass.

\item Find an equation relating $s$ and $v$. Use it to answer the following questions.

\begin{enumerate}
\item Where is the mass when it is moving at $2/3$ of its maximum speed? 

\item At what fraction of its maximum speed is the mass traveling when it is $3/4$ of the way from its equilibrium position to one if its extreme positions?
\end{enumerate}

\end{enumerate}

\end{question}

\section{Ferris Wheels}
\begin{question}  \label{QOERERERERQ}
A ferris wheel rotates at a constant rate, making one revolution every $90$ seconds. The wheel has a radius of $50$ feet and its center is $60$ feet above the ground.

You ride the wheel for one revolution and jump off.

\begin{enumerate}
\item What is your speed as you ride the wheel?

\item Find an equation relating your height $h$ above the ground (in feet) and your rate of ascent $r$ (in ft/sec) at any instant. Start by finding an expression for the height as a function of time. 

\item Use your equation from part (b) to determine your rate of ascent when you are $20$ feet above the ground for the second time.
\end{enumerate}
\end{question}


\begin{question} \label{QsdfPERdfere}
The graph below shows your height (in feet) as a function of the number of seconds since you boarded a ferris wheel. The wheel does not change its sense of rotation, but its rate of rotation might vary during the time you make your first revolution.

\href{https://www.desmos.com/calculator/eeiburo7bd}{151: Ferris Wheel 99}.

 
\begin{onlineOnly}
    \begin{center}
\desmos{eeiburo7bd}{900}{600}
\end{center}
\end{onlineOnly}

\begin{enumerate}
\item Use the graph above to approximate your rate of ascent when you are $150$ feet above the ground for the first time.

\item Use part (a) to approximate the wheel's rotation rate when you are $150$ feet above the ground for the first time. 

\emph{Hint:} Start by expressing your height in terms of the wheel's angle of rotation since you boarded. 
\end{enumerate}



\end{question}


\section{Flagpole and Track}


\begin{question} \label{QPEErmE}

You run around a circular track of radius $50$ feet at a constant speed of $v$ ft/sec, passing point $A$ at time $t=0$ seconds psat noon (see the animation below). A flagpole is located at point $F$, $80$ feet from the track's center and $30$ feet from $A$ as shown.


\href{https://www.desmos.com/calculator/xymelsnabl}{151: Track and Flagpole}.

 
\begin{onlineOnly}
    \begin{center}
\desmos{xymelsnabl}{900}{600}
\end{center}
\end{onlineOnly}

\begin{enumerate}
\item Find a function
\[
  s =g(t) \, , \, t\geq 0 ,
\]
that expresses your distance to the flagpole (in feet) in terms of the number of seconds past noon.

\item Find an expression for the rate of change (with respect to time) in your distance to the flagpole at time $t$ seconds past noon.

\item Express this rate of change in terms of your speed $v$ and the angle $\phi = \angle OPF$ above as you run around the ``top'' half of the track.


\end{enumerate}

\end{question}

\end{document}
