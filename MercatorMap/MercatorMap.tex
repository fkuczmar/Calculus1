\documentclass{ximera}
\title{The Mercator Map}

\newcommand{\pskip}{\vskip 0.1 in}

\begin{document}
\begin{abstract}
The Mercator map and its relation to Lambert's equal-area map.
\end{abstract}
\maketitle

\href{https://www.nytimes.com/2025/08/19/world/africa/africa-map-mercator.html}{Equal Earth Projection}


\section{Loxodromes}

\href{https://mathcurve.com/courbes3d.gb/loxodromie/sphereloxodromie.shtml}{Rhumb Lines on the Sphere}

For a curve on a sphere of radius $a$ that cuts the meridians at a fixed angle $\alpha$,
\[
  d\theta = \frac{\tan \alpha \, d\phi}{\cos\phi} ,
\]
where $\phi$ is the latitude and $\theta the longitude. Taking \theta = 0$ at the equator where $\phi=0$,
\begin{align*}
 \theta     &= \tan \alpha \int_0^{\phi^*} \sec \phi^* \, d\phi^*\\
               &= \tan \alpha \operatorname{arctanh}(\sin\phi) .
\end{align*}

So
\[
   \sin\phi = \operatorname{tanh}(k \theta),
\]
where $k=\cot\alpha$.

Because $-\pi/2 < \phi < \pi/2$,
\[
  \cos\phi =  \frac{1}{\cosh (k \theta)},
\]
a parameterization of the rhumb line by longitude is
\[
  (x,y,z) = \left( \frac{a\cos\theta}{\cosh(k\theta)} , \frac{a\sin\theta}{\cosh(k\theta)}  , a \tanh(k\theta)  \right) .
\]


And a parameterization by latitude,
\[
   (x,y,z) = \left(  a    \cos\phi  ,   a    \cos\phi    ,   a \sin \phi   \right)
\]

\begin{onlineOnly}
    \begin{center}
\desmosThreeD{6cqnjsuew7}{800}{600}  
\end{center}
\end{onlineOnly}

\href{https://www.desmos.com/3d/6cqnjsuew7}{152: Mercator 1}


\begin{onlineOnly}
    \begin{center}
\desmos{rbcbxdrvcp}{450}{600}  
\end{center}
\end{onlineOnly}

\begin{onlineOnly}
    \begin{center}
\geogebra{mvp9zvge}{450}{600}  
\end{center}
\end{onlineOnly}

\href{https://www.geogebra.org/classic/mvp9zvge}{152: Mercator}

\end{document}