\documentclass{ximera}
\title{The Mercator Map}

\newcommand{\pskip}{\vskip 0.1 in}

\begin{document}
\begin{abstract}
The Mercator map and its relation to Lambert's equal-area map.
\end{abstract}
\maketitle

\section{Scaling Factors and Elastic Bands}
The interpretation of the derivative as a rate of change is limited. A more universal description of the derivative, and one that generalizes to higher dimensions, is as a scaling factor. Multiplying the derivative by a sufficiently small change in the input to a function gives a good approximation to the change in the output. Of all possible scaling factors the derivative is the best in the sense that the error in the approximation approaches zero faster than the change in the output.

We can visualize the derivative as a scaling factor by regarding a function's domain as an elastic band and the function as acting on that band by stretching or compressing it.

For example, imagine a thin elastic band of length $4$ meters running along the horizontal $L$-axis from $L=0$ meters to $L=4$ meters.  Now hold the left end fixed at $L=0$ and stretch the band by moving right end an additional four meters to the right . Then the function
\[
      H = g(L) = 2L \, , \, 0\leq L \leq 4 ,
\]
describes this stretching action. It takes as an input the distance (in meters) of a point on the band from the origin ($L=0$) and returns as an output the distance between the origin and the corresponding point on the stretched band. The exploration below shows this stretching action.

\begin{exploration} \label{Ex:98f3rgafgbb}
Drag the slider $k$ in Line 2 below from $k=1$ to $k=2$ to illustrate the stretching action for the function $g$.

\begin{onlineOnly}
    \begin{center}
\desmos{qejivz36ui}{450}{600}  %qvk0mzy26u
\end{center}
\end{onlineOnly}

\href{https://www.desmos.com/calculator/qejivz36ui}{151: Rubber Band 1}

\end{exploration}

The \emph{global stretching factor} for a linear function like the one above, %The average rate of change of the function,
\[
      H = g(L) = 2L \, , \, 0\leq L \leq 4 ,
\]
is the slope of its graph. We can calculate this factor as an average stretching factor (ie. an average rate of change) between the points $L=a$  meters and $L=b$ meters from the origin. For the function $H=f(L) = 2L$, the stretching factor is
\begin{align*}
  \frac{\Delta H}{\Delta L} &= \frac{f(b) - f(a)}{b-a} \\
                                       &= \frac{\answer{2(b-a)}}{b-a} \\
                                       &= \answer{2},
\end{align*}
This tells us that any two points on the stretched band are twice as far apart as they were on the unstretched band.

\begin{question}  \label{Q:LDJJNMDesd}
\begin{enumerate}
\item What are the units of the stretching factor?
\begin{freeResponse}
\end{freeResponse}

\item Find an expression for the inverse function
\[
      L = f^{-1}(H).
\]
Include a domain. 

\item Interpret the inverse function as a deformation of a thin elastic band. What is the global stretching factor for this function?
\begin{freeResponse}
\end{freeResponse}
\end{enumerate}
\end{question}



\begin{example} \label{Ex:JDJFHDtet434t}
Here's an example
\[
      H = f(L) = 10 - \sqrt{100-L^2} \, , \, 0\leq L \leq 10,
\] 
of a non-linear stretching function (where $L$ and $H$ are measured in meters as before). Like most functions in this class, it acts like a linear function near almost all points in its domain. 

To stretch the elastic band in the demonstration below, drag the slider $u$ in Line 2 from $u=0$ to $u=1$. Then zoom in close enough to the point $H=f(9.5)$ in the stretched band (highlighted in black) to make the stretching function look linear. %the band looks like it's stretched by a constant factor. 


\begin{onlineOnly}
    \begin{center}
\desmos{xk8dvcfgwi}{450}{600}  %qvk0mzy26u
\end{center}
\end{onlineOnly}

\href{https://www.desmos.com/calculator/xk8dvcfgwi}{152: Rubber Band Ladder 5B}
%\end{exploration}


\begin{enumerate}

\item Use the close-up view of the stretched band to approximate the local stretching factor at the input $L=9.5$.

%\item Drag the slider $m$ in Line 4 to $m=95$ and repeat part (a) to approximate the stretching factor at the input $L=9.5$ meters.

\item Parts of the elastic band get stretched, others compressed. Identify these.
\end{enumerate}

%We can also approximate the stretching factors at different inputs by zooming in near enough to the graph of the function $H=f(L)$ (shown below) to make the graph look like a line. %To see the graph of the function, zoom back out and then activate the folder in Line 22 of the worksheet above.

%\begin{onlineOnly}
%    \begin{center}
%\desmos{jfwvbbrts1}{450}{600}  %qvk0mzy26u
%\end{center}
%\end{onlineOnly}

%\href{https://www.desmos.com/calculator/jfwvbbrts1}{151: Rubber Band Ladder 6}



\end{example}



\section{Lambert's Equal-Area Map}


You probably saw the idea of a stretching (scaling) factor long before taking calculus. Look at a map of Seattle and you'll see a scale, probably near the bottom. It might be 2 miles/inch, or something like that. But you will not see such a scale on a map of the earth for the simple reason that one does not exist. The scale varies from place to place and is usually a function of latitude. Even at a specific location, there is often not one but infinitely many scaling factors that depend on the direction in which you head. 

The worksheet below shows a way to create a map of the earth by wrapping a cylinder around its equator. The map sends each point of the earth to its nearest point on the cylinder (ie. straight out, directly away from the earth's axis of rotation). Unwrapping the cylinder gives Lambert's equal-area map, from the earth to the points in a rectangle.

\begin{exploration}  \label{EXPDkjkeFERerRd8}
Drag the slider in Line 6  from $r_1=0$ to $r_1=1$ to uwrap the cylinder. Drag sliders $theta_1$, $\phi_1$ in LInes 2 and 4 to vary the longitude and latitude of the point on the sphere.
\begin{onlineOnly}
    \begin{center}
\desmosThreeD{ehyvvttdeo}{800}{600}  
\end{center}
\end{onlineOnly}

\href{https://www.desmos.com/3d/ehyvvttdeo}{152: Mercator 0}

\end{exploration}

Suppose the earth has radius one. Lambert's map sends the circle of latitude $\phi$ (the angle between the plane of the equator and the segments from the sphere's center to points on the circle) to a circle on the cylinder with unit radius. So the map stretches the circle of latitude $\phi$, with radius $\cos\phi$, by the factor
\[
 \lambda_p = \frac{1}{\cos\phi}.
\]
This is the scaling factor along the circle in the east-west direction at latitude $\phi$.

In the north-south direction the scaling factor is also a function of latitude. The key point is that the map sends a point on the circle of latitude $\phi$ to signed height
\[
    h = f(\phi) = \sin\phi 
\]
above the plane of the equator. So the scaling factor at latitude $\phi$ in the north-south direction is
\[
 \lambda_m = \frac{dh}{d\phi} = \cos \phi.
 \]
Drag the slider $m$ in Line 2 of the worksheet below to visualize the scaling factor.

\begin{onlineOnly}
    \begin{center}
\desmos{kbryuogdaj}{450}{600}  %qvk0mzy26u
\end{center}
\end{onlineOnly}

\href{https://www.desmos.com/calculator/kbryuogdaj}{151: Lambert Map Cosine Function}



\section{The Mercator Map}

Watch this video.

\href{https://www.nytimes.com/2025/08/19/world/africa/africa-map-mercator.html}{Equal Earth Projection}


\section{Loxodromes}

\href{https://mathcurve.com/courbes3d.gb/loxodromie/sphereloxodromie.shtml}{Rhumb Lines on the Sphere}

For a curve on a sphere of radius $a$ that cuts the meridians at a fixed angle $\alpha$,
\[
  d\theta = \frac{\tan \alpha \, d\phi}{\cos\phi} ,
\]
where $\phi$ is the latitude and $\theta the longitude. Taking \theta = 0$ at the equator where $\phi=0$,
\begin{align*}
 \theta     &= \tan \alpha \int_0^{\phi^*} \sec \phi^* \, d\phi^*\\
               &= \tan \alpha \operatorname{arctanh}(\sin\phi) .
\end{align*}

So
\[
   \sin\phi = \operatorname{tanh}(k \theta),
\]
where $k=\cot\alpha$.

Because $-\pi/2 < \phi < \pi/2$,
\[
  \cos\phi =  \frac{1}{\cosh (k \theta)},
\]
a parameterization of the rhumb line by longitude is
\[
  (x,y,z) = \left( \frac{a\cos\theta}{\cosh(k\theta)} , \frac{a\sin\theta}{\cosh(k\theta)}  , a \tanh(k\theta)  \right) .
\]


And a parameterization by latitude,
\[
   (x,y,z) = \left(  a    \cos\phi  ,   a    \cos\phi    ,   a \sin \phi   \right)
\]

\begin{onlineOnly}
    \begin{center}
\desmosThreeD{6cqnjsuew7}{800}{600}  
\end{center}
\end{onlineOnly}

\href{https://www.desmos.com/3d/6cqnjsuew7}{152: Mercator 1}


\begin{onlineOnly}
    \begin{center}
\desmos{rbcbxdrvcp}{450}{600}  
\end{center}
\end{onlineOnly}

\begin{onlineOnly}
    \begin{center}
\geogebra{mvp9zvge}{450}{600}  
\end{center}
\end{onlineOnly}

\href{https://www.geogebra.org/classic/mvp9zvge}{152: Mercator}

\end{document}