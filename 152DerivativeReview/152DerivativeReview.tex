\documentclass{ximera}
\title{Review of the Chain Rule}

\newcommand{\pskip}{\vskip 0.1 in}

\begin{document}
\begin{abstract}
Derivative review.
\end{abstract}
\maketitle


\section{Example}

Here's an example of how to use $u$-substitution with the chain rule. Just as importantly, it's an example of how to show your work clearly and precisely.

\begin{example}  \label{Ex:GdfeerDFGff}
\begin{enumerate}
\item Ues the chain rule to find an expression for the derivative
\[
  \frac{d}{dx} \left(   (2x^3+1)^2 \right).
\]

\item Use part (a) to find an equation of the tangent line to the curve
\[
        y = (2x^3+1)^2
\]
at the point $(1,9)$.

\item Answer part (a) without the chain rule.

\end{enumerate}

\begin{explanation}
\begin{enumerate}
\item Let 
\[
       y = \left( 2x^3 +1  \right)^2
\]
and 
\[
      u = 2x^3 + 1 .
\]
Then
\[
     y = u^2
\]
and from the chain rule,
\begin{align*}
\frac{dy}{dx} &= \frac{dy}{du} \cdot \frac{du}{dx}  \\
                     &= \frac{d}{du} \left( u^2 \right)  \cdot \frac{d}{dx}\left(  2x^3 + 1 \right)  \\
                     &= 2u (6x^2)  \\
                     &= 2(2x^3+1)(6x^2) .
\end{align*}

\item The slope of the tangent line to the curve $y=(2x^3+1)^2$ at the point $(1,9)$ is 
\[
              \frac{dy}{dx}\Big|_{x=1} = 2(3)(6) = 36 ,
\]
and an equation of the tangent line is
\[
   y - 9 = 36(x-1) .
\]

\pskip 

\item We can get the same result without the chain rule, by rewriting the original function as
\[
     y = f(x) = (2x^3+1)^2 = 4x^6 + 4x^3 + 1.
\] 
Then
\begin{align*}
      \frac{dy}{dx} &= \frac{d}{dx} \left(  4x^6 + 4x^3 + 1  \right) \\
                         &= 4\frac{d}{dx}\left( x^6 \right) + 4 \frac{d}{dx}\left( x^3 \right) + \frac{d}{dx}\left(1 \right) \\
                        &=   24x^5 + 12x^2 .
\end{align*}

Then the slope of the tangent line to the curve $y=(2x^3+1)^2$ at the point $(1,9)$ is 
\[
              \frac{dy}{dx}\Big|_{x=1} =14 + 12 = 36 ,
\]
as before.

\end{enumerate}

\end{explanation}
\end{example}


\section{Exercises}

\emph{Directions:} Follow all the steps of \emph{Example 1}(a) exactly for each of the following. Show all work as above. Do not skip steps.

\begin{exercise}  \label{Q9edRersfrre}
Find simplified expressions for each of the following derivatives.

\begin{enumerate}

\item
\[
\frac{d}{d\theta}\left( \ln \left| \cot (\theta/2)\right| \right)
\]

\item 
\[
\frac{d}{d\theta}\left( \ln \left| \sec\theta + \tan\theta \right| \right)
\]

\item 
\[
\frac{d}{d\theta}\left( \operatorname{arctanh}(\sin(\theta)) \right)
\]

\item 
\[
\frac{d}{d\theta}\left( \operatorname{arcsinh}(\tan(\theta)) \right)
\]

\item 
\[
\frac{d}{dt} \left(   \tan \left( k \arctan(t/2)  \right)       \right),
\]
where $k$ is a constant.
\end{enumerate}

\end{exercise}

\begin{exercise}  \label{EX:JDHFeMERE34}
Find the acute angle the tangent line to the curve
\[
      y = \ln \left|  \sec\theta \right|
\]
at the point $P(\pi/7, \ln (\sec(\pi/7))$ makes with the $x$-axis. 
\end{exercise}

\begin{exercise} \label{EX:Her4r3erf}
Find simplified expressions for each of the following derivatives. Follow all steps of \emph{Example 1(a)}.

\begin{enumerate}
\item 
\[
      \frac{d}{dx}\left(  x \arctan x - \ln \sqrt{1+x^2} \right)
\]

\item 
\[
  \frac{d}{dt}\left(  t \ln |t| - t \right)
\]

\item 
\[
\frac{d}{dx} \left(    \ln (x + \sqrt{x^2-1})   \right)
\]

\item 
\[
\frac{d}{dx} \left(    \ln (x + \sqrt{x^2+1})   \right)
\]

\item 
\[
\frac{d}{dx} \left(    \ln (x + \sqrt{x^2-1})   \right)
\]

\item 
\[
\frac{d}{dt} \left(    \ln \sqrt{\frac{1+t}{1-t}}   \right)
\]


\end{enumerate}

\end{exercise}


\end{document}