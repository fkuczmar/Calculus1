\documentclass{ximera}
\title{Introduction}

\newcommand{\pskip}{\vskip 0.1 in}

\begin{document}
\begin{abstract}
Introduction to Differential Calculus
\end{abstract}
\maketitle


This course is listed as Calculus I in the catalogue, but it really should be called Differential Calculus. \emph{Differential} has the same root as \emph{difference} and \emph{calculus} the same root as \emph{calculate}. So this class is really about calculating differences, or more simply put it's about subtracting; but subtracting in the context of functions. We start by approximating the change in the output of a function given a small change in its input. Let's look at a few examples to get an idea what this means.

\section{Odometer Readings}

\begin{example} \label{Ex:ldfdtgjh5}
Suppose the function
\[
    s = f(t) , 0\leq t \leq 2 ,
\]
expresses the trip odometer reading (measured in miles) on your car in terms of the number of hours past noon during a two-hour trip. The graph of $f$  is shown below.

\pdfOnly{
Access Desmos interactives through the online version of this text at
 
\href{https://www.desmos.com/calculator/tumgpu4n0w}.
}
 
\begin{onlineOnly}
    \begin{center}
\desmos{tumgpu4n0w}{900}{600}
\end{center}
\end{onlineOnly}

Desmos activity available at

\href{https://www.desmos.com/calculator/tumgpu4n0w}{151: Odometer}

\pskip \pskip

We first approximate the car's speed at 12:30pm.

\pskip

(a) Use the graph above to describe how the speed of the car varies over the two-hour period. Over what time intervals is the speed increasing? Decreasing? Explain your reasoning. Then play the Slider $u$ in Line 2 and use the animation of the motion to check if your description was accurate. Explain.

(b) To approximate the car's speed at 12:30pm, set the Slider $u=0$ and zoom in close enough to point $P$ in the graph above to make the graph of $s=f(t)$ look like a line. Use the coordinates of point $P$ and a second point in the window far away from $P$ to estimate the slope of this line. How is the slope related to the car's speed at 12:30pm? Explain your reasoning.

(c) Use your approximation to write an equation of the line through $P$ with your slope from part (b). Use point-slope:
\begin{question}  \label{Q:3422d3}
\[
     s =  \answer{10} + \answer{36}(t - \answer{0.5}) , 0\leq t \leq 2 .
\]

(d) Enter your equation from part (c) on Line 30. Change the color to something other than red. What can you say about the line and the graph of $f$ near $P$? Now zoom back out. How is the line related to the graph of the function $f$?

(e) We can use the equation of the line to approximate the odometer reading at times near 12:30pm, or equivalently the change
\[
   \Delta s = f(0.5) - 10
\]
in the odometer reading in terms of the change
\[
   \Delta t = t -0.5
\]   
in time. To do this write your equation from part (c) in the form
\[
      s - \answer{10}= \answer{36}(t - \answer{0.5}) , 0\leq t \leq 2 .
\]
This tells us that for $t\sim 0.5$,
\[
       f(0.5) - \answer{10} \sim  \answer{36}(t - \answer{0.5}) 
\]
or that
\[
    \Delta s \sim  \answer{36} \Delta \answer{t} .
\]    


(f) Use the result of part (e) to approximate when the odometer reads $10.1$ miles by substituting 
\[
    \Delta s = 10.1 - 10  = 0.1
\]
and solving for $\Delta t$ to get
\[
     \Delta t \sim \answer{1/360} .
\]
Check your work by entering $f(0.5+\answer{1/360})$ in Line 31 above.

(g) Was your estimate from part (f) greater or less than the actual time? Explain why.


\end{question}
\end{example}

\section{Exponential Growth}

\begin{example} \label{Ex9d8gsd8}
Suppose between noon and 10pm, the population of a colony of bacteria grows exponentially. Suppose also that the population was $400,000$ at noon and $496,000$ at 3pm.

\pskip

(a) Find a function
\[
       P = f(t) , 0\leq t \leq 10,
\]
that expresses the population (measured in hundreds of thousands of bacteria) in terms of the number of hours past noon. Explain your reasoning. Do this by first describing the exponential growth and then by writing about about growth factors. Do \emph{not} use the number $e$.
 
(b) Enter your function from part (a) below and in Line 1 of the desmos worksheet below.

\begin{question} \label{Qer45gh54}
\[
    P = f(t) = \answer{4}(\answer{1.24})^{\answer{t/3}} , 0\leq t \leq 10 .
\]


\pdfOnly{
Access Desmos interactives through the online version of this text at
 
\href{https://www.desmos.com/calculator/lmvoqovxjz}.
}
 
\begin{onlineOnly}
    \begin{center}
\desmos{lmvoqovxjz}{900}{600}
\end{center}
\end{onlineOnly}

Desmos activity available at

\href{https://www.desmos.com/calculator/lmvoqovxjz}{151: Exponential Growth 1}

\pskip \pskip

(c) Follow steps (b)-(g) of Example 1, adjusted accordingly, in regard to the (instantaneous) growth rate of the population at 3:00pm.

\end{question}
\end{example}



\section{Riding a Ferris Wheel}

\begin{example} \label{Exdfgt4ttth4}


\end{example}



\end{document}