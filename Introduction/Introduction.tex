\documentclass{ximera}
\title{Introduction}

\newcommand{\pskip}{\vskip 0.1 in}

\begin{document}
\begin{abstract}
Introduction to Differential Calculus
\end{abstract}
\maketitle


This course is listed as Calculus I in the catalogue, but it really should be called Differential Calculus. \emph{Differential} has the same root as \emph{difference} and \emph{calculus} the same root as \emph{calculate}. So this class is really about calculating differences, or more simply put it's about subtracting. But it's about subtracting in the context of functions. 

%We start by approximating the change in the output of a function given a small change in its input. Let's look at a few examples to get an idea what this means.

Pick a specific input to specific function and you'll likely find that the changes in the function's output are approximately proportional to \emph{small} changes in the input. If so, we say that the function is \emph{differentiable} at that input and we call the  proportionality constant the \emph{derivative}.

For example, let's look at the behavior of the function
\[
   A = f(s) = s^2 , s\geq 0,
\]
near the input $s=5$. To emphasize the importance of units, let's define the input $s$ to be the side length of a square measured in feet and the output $f(s)$ to be the area of that square, measured in square feet. The problem before us is to describe a simple relationship between a small change in the side length 
\[
     \Delta s = s -5
\]
of the square and the change 
\[
   \Delta A = f(s) - f(5)
\]
in its area. 

\begin{question}  \label{Qfrdsft4t4tt}
(a) We'll first take a numerical approach and compute some small changes and their ratios. Fill in the missing entries in the table below.

\begin{center}
  \begin{tabular}{ | c| c | c | c | c |}
    \hline
    $s$ (ft) & $A = s^2$ ($\text{ft}^2$) & $\Delta s = s-5$ (ft)  & $\Delta A = s^2 - 25$ ($\text{ft}^2)$ & $\Delta A/\Delta s \, (\text{ft}^2/\text{ft})$ \\ \hline
    $4.9$ & $\answer{24.01}$ & $\answer{-0.1}$ &  $\answer{-0.99}$ & $\answer{9.9}$ \\ \hline
    $4.99$ & $\answer{24.9001}$ & $\answer{-0.01}$ &  $\answer{-0.0999}$ & $\answer{9.99}$ \\ \hline
    $5$  &  25 & $0$ & $0$ & $-$  \\ \hline
    $5.01$ & $\answer{25.1001}$ & $\answer{0.01}$ &  $\answer{0.1001}$ & $\answer{10.01}$ \\ \hline
    $5.1$ & $\answer{26.01}$ & $\answer{0.1}$ &  $\answer{1.01}$ & $\answer{10.1}$ \\ \hline
    \hline
  \end{tabular}
\end{center}

\pskip

(b) The data in the table above suggests an approximate proportional relationship between $\Delta A$ and $\Delta s$. We can guess the constant of proportionality from the fifth column. As $s\to 5$ (as $s$ approaches $5$), it looks like the ratio $\Delta A/\Delta s$ approaches some number, the constant of proportionality.

i) What is that number? $\answer{10}$

ii) What are its units?

\pskip

So for $\Delta s \sim 0$, we suspect that
\[
   \Delta A \sim \answer{10} \Delta s .
\]

(c) The constant of proportionality is called the derivative, in this case of the function $A=s^2$, at the input $s=5$. We write this as
\[
   \frac{dA}{ds}\Big|_{s=5} = \answer{10} .
\]

We could take an algebraic approach to determe this constant of proportionality instead. The idea is to first simplfiy the quotient $\Delta A/\Delta s$ as
\begin{align*}
            \frac{\Delta A}{\Delta s} &= \frac{s^2-25}{s-5}  \\
                                                &= \frac{(s + \answer{5})((s - \answer{5})}{s-5} \\
                                               &= \answer{s+5} \text{ if } s\neq {\answer{5}} .
\end{align*}

So, for example, if $s=4.99$, then
\[
\frac{\Delta A}{\Delta s} = \answer{4.99} + 5 = \answer{9.99} 
\]
as shown in the second row of the above table.

The advantage of this algebraic approach is that we can now compute the proportionality constant as a limit:
\begin{align*}
                    \frac{dA}{ds}\Big|_{s=5} &= \lim_{s\to 5} \frac{\Delta A}{\Delta s} \\
                                                         &=  \lim_{s\to 5} (s+\answer{5}) \\
                                                         &= \answer{5} + \answer{5} \\
                                                           &= \answer{10} .
\end{align*}

(d) We can also take a graphical approach by interpreting the ratios $\Delta A/\Delta s$ as slopes

\end{question}

\section{Odometer Readings}

\begin{example} \label{Ex:ldfdtgjh5}
Suppose the function
\[
    s = f(t) , 0\leq t \leq 2 ,
\]
expresses the trip odometer reading (measured in miles) on your car in terms of the number of hours past noon during a two-hour trip. The graph of $f$  is shown below.

\pdfOnly{
Access Desmos interactives through the online version of this text at
 
\href{https://www.desmos.com/calculator/tumgpu4n0w}.
}
 
\begin{onlineOnly}
    \begin{center}
\desmos{tumgpu4n0w}{900}{600}
\end{center}
\end{onlineOnly}

Desmos activity available at

\href{https://www.desmos.com/calculator/tumgpu4n0w}{151: Odometer}

\pskip \pskip

We first approximate the car's speed at 12:30pm.

\pskip

(a) Use the graph above to describe how the speed of the car varies over the two-hour period. Over what time intervals is the speed increasing? Decreasing? Explain your reasoning. Then play the Slider $u$ in Line 2 and use the animation of the motion to check if your description was accurate. Explain.

(b) To approximate the car's speed at 12:30pm, set the Slider $u=0$ and zoom in close enough to point $P$ in the graph above to make the graph of $s=f(t)$ look like a line. Use the coordinates of point $P$ and a second point in the window far away from $P$ to estimate the slope of this line. How is the slope related to the car's speed at 12:30pm? Explain your reasoning.

(c) Use your approximation to write an equation of the line through $P$ with your slope from part (b). Use point-slope:
\begin{question}  \label{Q:3422d3}
\[
     s =  \answer{10} + \answer{36}(t - \answer{0.5}) , 0\leq t \leq 2 .
\]

(d) Enter your equation from part (c) on Line 30. Change the color to something other than red. What can you say about the line and the graph of $f$ near $P$? Now zoom back out. How is the line related to the graph of the function $f$?

(e) We can use the equation of the line to approximate the odometer reading at times near 12:30pm, or equivalently the change
\[
   \Delta s = f(0.5) - 10
\]
in the odometer reading in terms of the change
\[
   \Delta t = t -0.5
\]   
in time. To do this write your equation from part (c) in the form
\[
      s - \answer{10}= \answer{36}(t - \answer{0.5}) , 0\leq t \leq 2 .
\]
This tells us that for $t\sim 0.5$,
\[
       f(0.5) - \answer{10} \sim  \answer{36}(t - \answer{0.5}) 
\]
or that
\[
    \Delta s \sim  \answer{36} \Delta \answer{t} .
\]    


(f) Use the result of part (e) to approximate when the odometer reads $10.1$ miles by substituting 
\[
    \Delta s = 10.1 - 10  = 0.1
\]
and solving for $\Delta t$ to get
\[
     \Delta t \sim \answer{1/360} .
\]
Check your work by entering $f(0.5+\answer{1/360})$ in Line 31 above.

(g) Was your estimate from part (f) greater or less than the actual time? Explain why.


\end{question}
\end{example}

\section{Exponential Growth}

\begin{example} \label{Ex9d8gsd8}
Suppose between noon and 10pm, the population of a colony of bacteria grows exponentially. Suppose also that the population was $400,000$ at noon and $496,000$ at 3pm.

\pskip

(a) Find a function
\[
       P = f(t) , 0\leq t \leq 10,
\]
that expresses the population (measured in hundreds of thousands of bacteria) in terms of the number of hours past noon. Explain your reasoning. Do this by first describing the exponential growth and then by writing about about growth factors. Do \emph{not} use the number $e$.
 
\begin{question} \label{Qer45gh54}
(b) Enter your function from part (a) below and in Line 1 of the desmos worksheet below.
\[
    P = f(t) = \answer{4}(\answer{1.24})^{\answer{t/3}} , 0\leq t \leq 10 .
\]


\pdfOnly{
Access Desmos interactives through the online version of this text at
 
\href{https://www.desmos.com/calculator/ntguz3lsmd}.
}
 
\begin{onlineOnly}
    \begin{center}
\desmos{ntguz3lsmd}{900}{600}
\end{center}
\end{onlineOnly}

Desmos activity available at

\href{https://www.desmos.com/calculator/ntguz3lsmd}{151: Exponential Growth 1}

\pskip \pskip

(c) Follow steps (b)-(d) of Example 1, adjusted accordingly, and estimate (instantaneous) growth rate of the population at 3:00pm.
Use this to write an equation of the line that best approximates the graph near the point $(3,496)$.

(d) Follow step of Example 1, adjusted accordingly, to approximate the change 
\[
\Delta P = f(t) - f(3)
\]
in terms of the change 
\[
  \Delta t = t-3
\]
for $t\sim 3$. 

(e) Make up and solve your own questions like parts (f) and (g) of Example 1.

\end{question}
\end{example}



\section{Riding a Ferris Wheel, Part 1}

\begin{example} \label{Exdfgt4ttth4}
A ferris wheel has radius $30$ meters and the center of the wheel is $40$ meters above the ground. You ride the wheel for one revolution and get off.

We'll let $\theta$, $0\leq \theta \leq 2\pi$, be the rotation angle of the wheel, measured in radians from the time you got on and 
\[
   h = f(\theta) , 0\leq \theta \leq 2\pi,
\]
be the function that expresses your height above the ground (in meters) in terms of the rotation angle.

Suppose the wheel gets stuck after turning through the angle $\theta = 5\pi/24$ radians. It then gets unstuck and turns through a small angle $\Delta \theta$ before getting stuck again. We wish to approximate the change $\Delta h$ in your height after the wheel turns through this small angle $\Delta \theta$.

\pdfOnly{
Access Desmos interactives through the online version of this text at
 
\href{https://www.desmos.com/calculator/cxjdzyfa8s}.
}
 
\begin{onlineOnly}
    \begin{center}
\desmos{cxjdzyfa8s}{900}{600}
\end{center}
\end{onlineOnly}

Desmos worksheet available at

\href{https://www.desmos.com/calculator/cxjdzyfa8s}{151: Ferris Wheel 1}


\pskip

\begin{center}
  \begin{tabular}{ | l | c | r |}
    \hline
    $\Delta \theta$ (radians) & $\Delta h$ (meters) & $\Delta h/\Delta \theta$ (m/rad) \\ \hline
    $\pi/24$ &  &  \\ \hline
    $\pi/48$ &  &  \\  \hline
    $pi/72$  &   &  \\
    \hline
  \end{tabular}
\end{center}

\end{example}

\end{document}





Let 
\[
    h = f(\theta) \, , 0\leq \theta \leq 2\pi ,
\]
be the function that expresses your height above the ground in terms of the wheel's angle of rotation, measured from the time you  boarded.

(a)  Assume $0 < \theta < \pi/2$ and draw a picture that captures the scenario. Include the ground, the ferris wheel, a point $P$ on the wheel representing your position after the wheel has turned through $\theta$ radians, and other lines as necessary. Label the angle $\theta$.

(b) Use your picture from part (a) to find an expression for $f(\theta)$.

(c) Suppose the wheel stops when you are at the angular position $\theta$ and that it then turns through a small angle $\Delta \theta$ and stops again.  Use the exploration below to describe qualitatively how the small change
\[
    \Delta h = f(\theta + \Delta \theta) - f(\theta)
\]
in your height depends on the angle $\theta$. 

(d) Fix $\Delta \theta \sim 0$ and use the exploration to sketch by hand a graph of the function 
\[
     \Delta h = g(\theta) \, , 0\leq \theta \leq 2\pi ,
\]
that takes as an input the wheel's angle of rotation $\theta$ and returns as an output the change $\Delta h$
in your height as the wheel turns through the small angle $\Delta \theta$.

\begin{exploration}

\pdfOnly{
Access Desmos interactives through the online version of this text at
 
\href{https://www.desmos.com/calculator/8swp20zond}.
}
 
\begin{onlineOnly}
    \begin{center}
\desmos{8swp20zond}{900}{600}
\end{center}
\end{onlineOnly}
\end{exploration} 
\end{question}



\end{example}



\end{document}