\documentclass{ximera}
\title{Introduction}

\newcommand{\pskip}{\vskip 0.1 in}

\begin{document}
\begin{abstract}
Introduction to Differential Calculus
\end{abstract}
\maketitle


This course is listed as Calculus I in the catalogue, but it really should be called Differential Calculus. \emph{Differential} has the same root as \emph{difference} and \emph{calculus} the same root as \emph{calculate}. So this class is really about calculating differences, or more simply put it's about subtracting. But it's about subtracting in the context of functions. 

%We start by approximating the change in the output of a function given a small change in its input. Let's look at a few examples to get an idea what this means.

Pick a specific input to specific function and you'll likely find that the changes in the function's output are approximately proportional to \emph{small} changes in the input. If so, we say that the function is \emph{differentiable} at that input and we call the  proportionality constant the \emph{derivative}.

For example, let's look at the behavior of the function
\[
   A = f(s) = s^2 , s\geq 0,
\]
near the input $s=5$. To emphasize the importance of units, let's define the input $s$ to be the side length of a square measured in feet and the output $f(s)$ to be the area of that square, measured in square feet. The problem before us is to describe a simple relationship between a small change in the side length 
\[
     \Delta s = s -5
\]
of the square and the change 
\[
   \Delta A = f(s) - f(5)
\]
in its area. 

\begin{question}  \label{Qfrdsft4t4tt}
(a) We'll first take a numerical approach and compute some small changes and their ratios. Fill in the missing entries in the table below.

\begin{center}
  \begin{tabular}{ | c| c | c | c | c |}
    \hline
    $s$ (ft) & $A = s^2$ ($\text{ft}^2$) & $\Delta s = s-5$ (ft)  & $\Delta A = s^2 - 25$ ($\text{ft}^2)$ & $\Delta A/\Delta s \, (\text{ft}^2/\text{ft})$ \\ \hline
    $4.9$ & $\answer{24.01}$ & $\answer{-0.1}$ &  $\answer{-0.99}$ & $\answer{9.9}$ \\ \hline
    $4.99$ & $\answer{24.9001}$ & $\answer{-0.01}$ &  $\answer{-0.0999}$ & $\answer{9.99}$ \\ \hline
    $5$  &  25 & $0$ & $0$ & $-$  \\ \hline
    $5.01$ & $\answer{25.1001}$ & $\answer{0.01}$ &  $\answer{0.1001}$ & $\answer{10.01}$ \\ \hline
    $5.1$ & $\answer{26.01}$ & $\answer{0.1}$ &  $\answer{1.01}$ & $\answer{10.1}$ \\ \hline
    \hline
  \end{tabular}
\end{center}

\pskip

(b) The data in the table above suggests an approximate proportional relationship between $\Delta A$ and $\Delta s$. We can guess the constant of proportionality from the fifth column. As $s\to 5$ (as $s$ approaches $5$), it looks like the ratio $\Delta A/\Delta s$ approaches some number, the constant of proportionality.

i) What is that number? $\answer{10}$

ii) What are its units?
\begin{freeResponse}
\end{freeResponse}

\pskip

So for $\Delta s \sim 0$, we suspect that
\[
   \Delta A \sim \answer{10} \Delta s .
\]

(c) The constant of proportionality is called the derivative, in this case of the function $A=s^2$, at the input $s=5$. We write this as
\[
   \frac{dA}{ds}\Big|_{s=5} = \answer{10} .
\]

(d) We could have taken an algebraic approach to determe this constant of proportionality instead. The idea is to first simplfiy the quotient $\Delta A/\Delta s$ as
\begin{align*}
            \frac{\Delta A}{\Delta s} &= \frac{s^2-25}{s-5}  \\
                                                &= \frac{(s + \answer{5})((s - \answer{5})}{s-5} \\
                                               &= \answer{s+5} \text{ if } s\neq {\answer{5}} .
\end{align*}

So, for example, if $s=4.99$, then
\[
\frac{\Delta A}{\Delta s} = \answer{4.99} + 5 = \answer{9.99} 
\]
as shown in the last column of the second row of the above table.

The advantage of this algebraic approach is that we can now compute the proportionality constant as a limit:
\begin{align*}
                    \frac{dA}{ds}\Big|_{s=5} &= \lim_{s\to 5} \frac{\Delta A}{\Delta s} \\
                                                         &=  \lim_{s\to 5} (s+\answer{5}) \\
                                                         &= \answer{5} + \answer{5} \\
                                                           &= \answer{10} .
\end{align*}

(d) We can also use the graph of the function $A=f(s)=s^2$ to intrepret the ratios 
\[
  \frac{\Delta A}{\Delta s} = \frac{f(s) - f(5)}{s-5} 
\]
geometrically. Move the slider $s$ in the demonstration below and describe

(i) how the line through the points $P$ and $Q$ is related to the ratio $\Delta A/\Delta s$ show on Line 2, 

(ii) what happens to the line $PQ$ as $s\to 5$, and

(iii) what happens to the line $PQ$ when $s=5$.

\begin{freeResponse}
\end{freeResponse}

\pdfOnly{
Access Desmos interactives through the online version of this text at
 
\href{https://www.desmos.com/calculator/vz9ud5txva}.
}
 
\begin{onlineOnly}
    \begin{center}
\desmos{vz9ud5txva}{900}{600}
\end{center}
\end{onlineOnly}

\pskip

Continuing with the above demonstration, 

(i) Open the Code folder in Line 3 and turn off the line $PQ$ in Line 7.

(ii) Write an equation for the line through the point $P$ with slope equal to the proportionality constant in the line below and on Line 8 in the desmos worksheet:
\[
    A = L(s) = \answer{25}+ \answer{10}(s- \answer{5}) .
\]

(iii) Zoom in close enough to the point $P$ to make the graph of the function $A=f(s)$ look like a line. How do the graph of the function and the graph of the line $A=L(s)$ compare in this close-up view?
\begin{freeResponse}
\end{freeResponse}

(e) {\bf Summary:}
\begin{itemize}
\item{If we change the side of a square from a length of $5$ feet to a length of $s\sim 5$ feet, then the area of the square changes by approximately
\[
 \Delta A = s^2-25  \sim 10 \Delta s = 10 (s-5)
\]
square feet. The proportionality constant $10$ has units $ft^2/ft = ft$.
}

\item{Zoom in close enough to the graph of the function $A=f(s)=s^2$ near the point $P(5,25)$ and the graph looks like a line with slope equal to the proportionality constant.}

\item{We can compute the proportionality constant as the limit
\[
    \frac{dA}{ds}\Big|_{s=5} = \lim_{s\to 5} \frac{f(s)-f(5)}{s-5} . 
\]
}

\end{itemize}



\end{question}


\begin{question}  \label{Q435rdfgbyt}
On a clear day with an unobstructed view (like you might have at the beach or in a hot air balloon), the distance to the horizon is limited by the curvature of the earth as illustrated in the demonstration below.

In fact, as long as you are not too high above the surface of the earth, the function
\[
      s = f(h) = 1.22\sqrt{h}, 0\leq h \leq 20,000,
\]
gives a good approximation to the distance to the horizon (the length of the red arc $AT$ below, measured in miles) in terms of your height above the ground (the distance $AP$ below, measured in feet).


\pdfOnly{
Access Desmos interactives through the online version of this text at
 
\href{https://www.desmos.com/calculator/ewowig5sgk}.
}
 
\begin{onlineOnly}
    \begin{center}
\desmos{ewowig5sgk}{900}{600}
\end{center}
\end{onlineOnly}

Desmos activity available at

\href{https://www.desmos.com/calculator/ewowig5sgk}{151:Distance to Horizon 1}

\pskip \pskip

Our aim is to approximate the change in the distance to the horizon (in miles) in terms of a small change in height (in feet) from a height of $25$ feet.

(i) To start, what are the units of the constant $1.22$ above? Explain how  you know.

\begin{freeResponse}
\end{freeResponse}

(ii) Go through a similar analysis as in parts (a)-(e) of Example 1, to approximate the change $\Delta s = s - f(25)$ in the distance to the horizon in terms of the change $\Delta h = h-25$ in your height above the ground. Start by completing the column headings (with units) and the missing entries in the table below.

\begin{center}
  \begin{tabular}{ | c| c | c | c | c |}
    \hline
    $h$ (ft) & $s = 1.22\sqrt{h}$ (miles) & $\Delta h = h-25$ (ft)  & $\Delta s =f(h) - f(25)$ (miles) & $\Delta s/\Delta h \, (units?)$ \\ \hline
    $4.9^2$ &  &  &   & \\ \hline
    $4.99^2$ &  &  &   &  \\ \hline
    $25$  &  & $0$ & $0$ & $-$  \\ \hline
    $5.01^2$ &  &  &   &  \\ \hline
    $5.1^2$ &  &  &   &  \\ \hline
    \hline
  \end{tabular}
\end{center}

\end{question}


\begin{question}  \label{Q34eefg4t3}
This question is similar to the last, but suppose instead we are looking down on the earth from the space station or a rocket. Then the approximation to the distance to the horizon from the previous problem will not work.

So our first step is to find a function
\[
   s = f(h), h\geq 0 ,
\]
that expresses the distance to the horizon (still measured in miles) in terms of our height above the earth's surface, \emph{now measured in miles instead of feet}. We'll suppose the earth to be a perfect sphere of radius $3960$ miles. The distance to the horizon is the arclength $AT$ below, meaured along the surface of the earth (you can think of this distance as the radius of the spherical disk visible to us). Our height is the distance $AP$. 

 
\begin{onlineOnly}
    \begin{center}
\desmos{ewowig5sgk}{900}{600}
\end{center}
\end{onlineOnly}

(a) Find an expression for the above function. 
\begin{hint}
Use right triangle $\Delta OTP$ to find an expression for the radian measure of angle $\angle POT$. Then use this angle to find an expression for the arclength $AT$.
\end{hint}

(b) Now suppose we are $165$ miles above the surface of the earth and we wish to approximate how a small change in our altitude changes the distance to the horizon.

To do this, fill in the missing entries in the table below.

\begin{center}
  \begin{tabular}{ | c| c | c | c | c |}
    \hline
    $h$ (miles) & $s = f(h)$ (miles) & $\Delta h = h-165$ (miles)  & $\Delta s =f(h) - f(165)$ (miles) & $\Delta s/\Delta h \, (units?)$ \\ \hline
     $162$ &  &  &   & \\ \hline
    $163$ &  &  &   & \\ \hline
    $164$ &  &  &   &  \\ \hline
    $165$  &  &  &  &   \\ \hline
    $166$ &  &  &   &  \\ \hline
    $167$ &  &  &   &  \\ \hline
    $168$ &  &  &   &  \\ \hline
    \hline
  \end{tabular}
\end{center}

(c) Do the data above suggest that the quotients $\Delta s / \Delta h$ approach some number as $h$ approaches 165? If so, use the data to approximate that number. If not, explain why not.

(d) Make your own table similar to the one above to get a better approximation, correct to the nearest thousandth, to
\[
   \lim_{h\to 165} \frac{f(h) - f(165)}{h-165} .
\]


(e) Use your result from part (d), rounded to the nearest thousandth, to approximate $\Delta s$ in terms of $\Delta h$ and enter your result below.
\[
  \Delta s \sim \answer{3.291} \Delta h , \text{ for } \Delta h \sim 0.
\] 

(f) Explain the meaning of the proportionality constant in parts (d) and (e). Be sure to include units in your explanation.


\end{question}






\end{document}

