\documentclass{ximera}
\title{More Related Rates Problems}

\newcommand{\pskip}{\vskip 0.1 in}

\begin{document}
\begin{abstract}
Related rates.
\end{abstract}
\maketitle


\emph{Directions:}
\begin{itemize}
\item Start by defining all variables, each in a complete sentence with units. Be sure to precisely define the variable for time as well.

\item Answer each problem with a concluding sentence.
\end{itemize}


\section{Kite Flying}
\begin{question} \label{Q9oigohonn}
A kite drifts due east as it maintains a constant altitude. Play the slider $u$ in Line 1 of the worksheet below to watch the motion.

\pdfOnly{
Access Desmos interactives through the online version of this text at
 
\href{https://www.desmos.com/calculator/qdrzdt4erc}.
}
 
\begin{onlineOnly}
   \begin{center}
\desmos{qdrzdt4erc}{900}{600}
\end{center}
\end{onlineOnly}

\href{https://www.desmos.com/calculator/qdrzdt4erc}{151: Kite}

\begin{enumerate}
\item Which do you think is greater, the rate at which the string is being let out or the speed of the kite? No computations, just go with your intuition.

\item Express the speed of the kite (not assumed constant) in terms of the rate at which the string is being let out (\emph{not} assumed constant) and the angle $\theta$ the string makes with the horizontal.

\item What can you say if $\theta =  \pi/3$?

\item Was your intuition correct?
\end{enumerate}
\end{question}


\section{A Crawling Beetle}
\begin{question}  \label{Qbnmkgytyuu}
As a beetle crawling in the $xy$-plane passes the point $(-4,-3)$ (coordinates in centimeters) its distance from the $x$-axis is increasing at the rate of $2$ cm/sec and its distance from the origin is increasing at the rate of $3$ cm/sec.

Is the beetle rotating clockwise or counterclockwise at this instant? At what rate?

\end{question}

\section{A Mechanical Motion}
\begin{question}  \label{Q34fdffffhggg}
As a rod $OP$ of length $R$ meters rotates about its endpoint $O$, the other endpoint $P$ drags along a second rod $PA$ of length $L$ meters as illustrated below.

\pdfOnly{
Access Desmos interactives through the online version of this text at
 
\href{https://www.desmos.com/calculator/l03dlyifb0}.
}
 
\begin{onlineOnly}
   \begin{center}
\desmos{r20xwnok5r}{900}{600}
\end{center}
\end{onlineOnly}

\href{https://www.desmos.com/calculator/l03dlyifb0}{151: Engine}

\begin{enumerate}
\item Express the speed of $A$ in terms of $R$, $L$, the angle $\theta$ shown above, and the rotation rate $\omega$ (measured in radians/sec) of rod $OP$. Do \emph{not} assume $\omega$ is constant. Work in general, not with the specific values of $L$ and $R$ in the worksheet.

\item Check that your expression in  part (a) has the correct units.

\item Express the signed rotation rate of $PA$ in terms of the same parameters. Measure the rate to be postive when $PA$ rotates counterclockwise. Do not assume $\omega$ is constant.

\item Check that your expression in  part (c) has the correct units.
\end{enumerate}

\end{question}

\section{Tracking a Helicopter}
Sensors on the ground four hundred feet apart track a helicopter. At the instant shown below, the sensors at $A$ and $B$ are rotating counterclocwise at the respective rates of $2$ rad/min and $3$ rad/min. And the marked angles at $A$ and $B$ have respective measures $\angle A = \pi/6$ and $\angle B = \pi/3$.

Is the helicopter ascending or descending at this instant? At what rate?

 
\begin{onlineOnly}
   \begin{center}
\desmos{xl8t3toppg}{900}{600}
\end{center}
\end{onlineOnly}

\href{https://www.desmos.com/calculator/xl8t3toppg}{151: Tracking a Helicopter}



\section{Distance to the Horizon}

\begin{question}  \label{Qthhvhrdfgbyt}
The distance to the horizon is limited by the curvature of the earth as illustrated in the demonstration below. This distance is the length of the (red) arc $AT$ on the earth's surface. 


\pdfOnly{
Access Desmos interactives through the online version of this text at
 
\href{https://www.desmos.com/calculator/ewowig5sgk}.
}
 
\begin{onlineOnly}
    \begin{center}
\desmos{8qwt6mfirt}{900}{600}
\end{center}
\end{onlineOnly}

Desmos activity available at

\href{https://www.desmos.com/calculator/8qwt6mfirt}{151:Distance to Horizon 11}


Let 
\[
    s = f(h)
\]
be the function that expresses the distance to the horizon (in thousands of miles) in terms of your altitude above the surface (in thousands of miles).

\begin{enumerate}

\item Sketch by hand a graph of the function $f$. Drag the slider $h$ in the worksheet above to help with your sketch. Then activate the Graph folder in Line 17 to see how you did.

\item Find an expression for the function $f$.

\item Use the graph of the function $f$ to approximate the derivative
\[
       \frac{ds}{dh}\Big|_{h=1} .
\]
Include units.

\item Use your expression for the function $s = f(h)$ to find the exact value of the derivative above.

\item Interpret the meaning of the derivative in terms of small changes.

\item Suppose at some instant you are in a rocket $1000$ miles above the earth and descending at the rate of $v$ miles/hour.
Find an expression (in terms of $v$) for the rate (in miles/hour) at which your distance to the horizon (the length of the red arc $AT$ above) is changing at this instant.  Do \emph{not} assume $v$ is constant.

\end{enumerate}

\end{question}



\section{Jar Lid and Rubber Band}
\begin{question}  \label{Qbnmdfgh}
You wrap a rubber band around a circluar jar lid and pull the band tight as illustrated below.

 
\begin{onlineOnly}
   \begin{center}
\desmos{jbku3wrtdq}{900}{600}
\end{center}
\end{onlineOnly}

\href{https://www.desmos.com/calculator/jbku3wrtdq}{151: String and Jar Lid}

Suppose at some instant you are pulling the point $P$ of the rubber band directly away from the center of the lid at a speed of $v$ inches/sec. Express the rate (in inches/sec) at which the length of the rubber band is changing at this instant in terms of $v$ and the marked angle $\theta$ between the straight segments of the band at $P$.

\emph{Hint:} Express the length of the rubber brand in terms of $\theta$ and the distance $h$ (mesaured in inches) between $P$ and the center of the lid. Keep in mind the band has a curved section in addition to the straight parts.
\end{question}

\section{Disco Dancing}
\begin{question}  \label{QLKDKGDCCCC}
A spotlight in a circular dance hall of radius $r$ meters is located $b$ meters from the center of the hall. 

At some instant the light is rotating at the rate of $\omega$ rad/sec. 

\begin{onlineOnly}
   \begin{center}
\desmos{m2o267u9ur}{900}{600}
\end{center}
\end{onlineOnly}

\href{https://www.desmos.com/calculator/m2o267u9ur}{151: Disco Dancing}


\begin{enumerate}
\item Find an expression for the speed of the light beam as it moves along the wall at this instant. Your expression should be in terms of $\omega$, $r$, $b$, and the angle $\theta$ marked below.

\item Check that your expression in  part (a) has the correct units.

\end{enumerate}

\end{question}


\end{document}