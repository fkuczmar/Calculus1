\documentclass{ximera}
\title{Sinusoidal Modeling, Math 142}

\newcommand{\pskip}{\vskip 0.1 in}

\begin{document}
\begin{abstract}
Implicit differentiation.
\end{abstract}
\maketitle



\begin{question} \label{Qhhjklfghjfd}
Suppose that during a five-day period, beginning at midnight Monday morning, the depth of the water at the Edmond's Pier is a sinusoidal function of time.

Suppose also that a low tide of $7$ feet occurs at 1:00am Monday morning and that the following high tide of $23$ feet occurs at 6:45am that same morning.  

\begin{enumerate}
\item Sketch by hand a graph of the function
\[
   h = f(t)
\] 
that expresses the depth of the water (in feet) as a function of the number of hours since midnight, Monday morning.

\item Find an expression for the above function. Include a domain.

\item Find all times during the week when the water is $10$ feet deep. Give exact times. Do not use a calculutor.

\item Approximate the clock times (to the nearest minute) on Friday when the water is $10$ feet deep.

\end{enumerate}
\end{question}


\begin{question} \label{Q04365d9ggdgbgh}
Assume for this question that each month has 30 days and that the number of hours of daylight/day in Seattle is a sinusoidal function of time. Assume also that on June 21, Seattle gets a maximum of $16$ hours of daylight/day and that on December 21, Seattle gets a minimu of $8$ hours of daylight/day.

\begin{enumerate}
\item Find a function 
\[
   H = f(t) \, , \, 0\leq t\leq 12,
\]
that expresses the number of daylight hours/day in Seattle in terms of the number of months since June 21. Use the cosine function. Start by sketching a graph. Explain your reasoning.

\item Use your function to determine the number of hours of daylight/day that Seattle gets on March 1.

\item About how many more minutes of daylight/day do we get tomorrow than today? 

\item Use your function to determine the day(s) of the year when Seattle gets $14$ hours of daylight/day.

\end{enumerate}

\end{question}



\end{document}