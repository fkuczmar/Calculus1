\documentclass{ximera}
\title{Motion}

\newcommand{\pskip}{\vskip 0.1 in}

\begin{document}
\begin{abstract}
An introduction to motion.
\end{abstract}
\maketitle


\begin{question}  \label{Q:4sdfdsfdellk}

Play the slider $u$ in the activity below to see the motion of a balloon. Use the animation to sketch graphs of

(a) the altitude of the balloon as a function of time.

(b) the balloon's rate of ascent as a function of time.

(c) Activate the folders in Lines 13 and 18 to see how you did.

\begin{onlineOnly}
    \begin{center}
\desmos{amv52b9ljt}{900}{600}
\end{center}
\end{onlineOnly}

Desmos activity available at \href{https://www.desmos.com/calculator/amv52b9ljt}{151: Balloon}

\end{question}


\begin{question}  \label{QDfsdtegg}
The function
\[
     h = f(t) = 10 - \frac{1}{2}t^2 e^{-t/5} \, , \, 0\leq t \leq 40,
\]
expresses the altitude of a balloon (in thousands of feet) in terms of the number of hours since noon on August 31, 2023.

Use the graph of this function in Question 1 to first approximate answers to the following questions. Then use calculus and algebra to determine the exact times.

(a) When is the balloon at its minimum height? At its maximum height?

(b) When is the balloon ascending at the fastest rate? Descending at the fastest rate? 

To compute these times you will end up solving a quadratic equation that may be written in the form
\[
    t^2 - \answer{20}t + \answer{50} = 0.
\]
 
So the balloon is ascending at its fastest rate at time (give the exact times and then approximations to the nearest hundredth of an hour)
\[
    t = \answer{10 + \sqrt{50}} \sim \answer[tolerance=0.01]{17.07}
\]
hours past noon and descending at its fastest rate at time
\[
   t = \answer{10 - \sqrt{50}} \sim \answer[tolerance=0.01]{2.93}
\]
hours past noon.

\end{question}

\end{document}