\documentclass{ximera}
\title{Derivatives and the Shapes of Curves}

\newcommand{\pskip}{\vskip 0.1 in}

\begin{document}
\begin{abstract}
First and second derivatives and their implications.
\end{abstract}
\maketitle

\section{A Surge Function}
\begin{question}  \label{Q56hnnhfgh}
The function
\[
      C = f(t) = A e^{-kt} \, , \, t\geq 0 ,
\]
expresses the concentration of a drug (in mg/L) in terms of the number of hours since the drug was injected. Here $A$ and $k$ are postive constants.

\begin{enumerate}
\item What are the units of $A$? Of $k$? Explain how you know.



\item Find an expression for the time when the concentration is a maximum.

\begin{enumerate}
\item Justify your assertion without relying on a graph.

\item Check that your expression has the correct units.
\end{enumerate}

\item Find expressions for the times when
\begin{enumerate}
\item the concentration is increasing at its maximum rate.

\item the concentration is decreasing at its maximum rate.

\end{enumerate}
Justify your assertions without relying on a graph.


\item Determine all time intervals during which 

\begin{enumerate}
\item the concentration is increasing at an increasing rate.

\item the concentration is increasing at an decreasing rate.

\item the concentration is decreasing at an increasing rate.

\item the concentration is decreasing at an decreasing rate.

\end{enumerate}
Justify your assertions without relying on a graph.
\end{enumerate}
\end{question}

\section{Changes and Rates of Change, Relative Changes and Relative Rates of Change}

\begin{question}  \label{Qdfdgndfdfdfnhn}
The function $P=f(t)$, $0\leq t \leq 12$, expresses the population of a colony of bacteria (measured in the number of bacteria) in terms of the number of hours past noon.

Suppose that
\[
     \frac{dP}{dt}\Big|_{t=4} = 400,000.
\]

\begin{enumerate}
\item What are the units of this derivative?

\item Explain the meaning of the derivative without using the language of small changes.

\item Explain the meaning of the derivative using the language of small changes by completing the following sentence.

Between 4:00pm and 4:01pm the population ...
\end{enumerate}

\end{question}


\begin{question}  \label{Qghldfdfgng}
The function $P=f(t)$, $0\leq t \leq 12$, expresses the population of a colony of bacteria (measured in the number of bacteria) in terms of the number of hours past noon.

Suppose that
\[
   \frac{d}{dt} \left(  \ln P \right)\Big|_{t=4} = \frac{d}{dt} \left(  \ln (f(t)) \right)\Big|_{t=4}  = \frac{3}{10} .
\]

\begin{enumerate}
\item What are the units of this derivative?

\item Explain the meaning of the derivative without using the language of small changes.

\item Explain the meaning of the derivative using the language of small changes by completing the following sentence.

Between 4:00pm and 4:01pm the population ...
\end{enumerate}
\end{question}


\begin{question} \label{Q9dfgnmnmcxcv}
The function $P=f(t)$, $0\leq t \leq 12$, expresses the population of a colony of bacteria (measured in the number of bacteria) in terms of the number of hours past noon.

Suppose that
\begin{equation}
   \frac{d^2 P}{dt^2} \Big|_{t=4} =  -6,000 .  %\label{Eq:SecondD2}
\end{equation}


\begin{enumerate}
\item What are the units of this derivative?

\item Explain the meaning of the derivative without using the language of small changes.

\item Explain the meaning of the derivative using the language of small changes by completing the following sentence.

Between 4:00pm and 4:01pm  ...

\item Suppose also that
\[
   \frac{dP}{dt}\Big|_{t=4} = 400,000 .
\]
Explain the meaning of the second derivative above \emph{without} using the language of small changes by completing the following sentence.

At 4:01pm  ...

\end{enumerate}
\end{question}



\begin{question}  \label{Qghldfdfgng}
The function $P=f(t)$, $0\leq t \leq 12$, expresses the population of a colony of bacteria (measured in the number of bacteria) in terms of the number of hours past noon.

Suppose that
\begin{equation}
   \frac{d^2 \left(  \ln P \right)}{dt^2} \Big|_{t=4} =  -\frac{3}{25} .  %\label{Eq:SecondD}
\end{equation}


\begin{enumerate}
\item What are the units of this derivative?

\item Explain the meaning of the derivative without using the language of small changes.

\item Explain the meaning of the derivative using the language of small changes by completing the following sentence.

Between 4:00pm and 4:01pm  ...

\item Suppose also that
\[
   \frac{d}{dt} \left(  \ln P \right)\Big|_{t=4} = \frac{d}{dt} \left(  \ln (f(t)) \right)\Big|_{t=4}  = \frac{3}{10} .
\]
Explain the meaning of the second derivative above \emph{without} using the language of small changes by completing the following sentence.

At 4:01pm  ...

\end{enumerate}
\end{question}




\begin{question}  \label{Q54356dfdfgng}
The function $P=f(t)$, $0\leq t \leq 12$, expresses the population of a colony of bacteria (measured in the number of bacteria) in terms of the number of hours past noon.

Suppose that
\begin{equation}
   \frac{d}{dt} \left( \ln \Big|  \frac{d}{dt} \left( \ln P  \right) \Big| \right) \Big|_{t=4} =  -\frac{3}{5} .  %\label{Eq:SecondD33}
\end{equation}


\begin{enumerate}
\item What are the units of this derivative?

\item Explain the meaning of the derivative without using the language of small changes.

\item Explain the meaning of the derivative using the language of small changes by completing the following sentence.

Between 4:00pm and 4:01pm  ...

\item Suppose also that
\[
   \frac{d}{dt} \left(  \ln P \right)\Big|_{t=4} = \frac{d}{dt} \left(  \ln (f(t)) \right)\Big|_{t=4}  = \frac{3}{10} .
\]
Explain the meaning of the second derivative above \emph{without} using the language of small changes by completing the following sentence.

At 4:01pm  ...

\end{enumerate}
\end{question}



\end{document}
