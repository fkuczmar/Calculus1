\documentclass{ximera}
\title{Final, Math 151}

\newcommand{\pskip}{\vskip 0.1 in}

\begin{document}
\begin{abstract}
Directions for our final.
\end{abstract}
\maketitle

\section{Directions}

Here are the directions for our final.

\begin{itemize}
\item Use only the material of our class. In particular, \emph{do not use vectors or other ideas that were not part of our class.}

\item Use the Leibniz differential notation for derivatives and their evaluations. Do \emph{not} use the prime notation.

\item Use the Leibniz notation to show \emph{all} steps when using the the chain, product, and quotient rules. See the next section \emph{Examples of How to Show Work} for more details.

\item Define, in complete sentences and with units any variables or unknowns you introduce.

\item Include a domain for each function you introduce.

\item Include units for \emph{each} number in \emph{each} numerical computation.

\item Give explanations of your reasoning along with your solutions to each problem.

\item Show all your work.

\item Simplify each of your answers as much as possible.

\item Answer each word problem with a concluding sentence.

%\item Write on one side of the paper. Leave the back side blank.

\item Write LARGE and neatly.

\item Leave plenty of space.  %every other line blank. 

\item Work vertically. Do \emph{not} split a page into multiple columns.

\item No technology permitted. Put away all cell phones and calculators. An open cell phone will result in an automatic score of $0$ for the exam.

\item A few derivatives:

\[
 \frac{d}{dx}\left( \arctan x \right) = \frac{1}{1+x^2} 
\]

\[
 \frac{d}{dx}\left( \arcsin x \right) = \frac{1}{\sqrt{1-x^2}} 
\]

\[
 \frac{d}{dx}\left( \arccos x \right) = -\frac{1}{\sqrt{1-x^2}} 
\]

\end{itemize}

\section{Examples of How to Show Work}

Here are some examples of how to show your work when taking derivatives. These examples will \emph{not} be included with the exam.

\begin{example} \label{Ex:POPOPODDD}
The chain rule (do \emph{not} use the quotient rule for a derivative like this one):

\begin{align*}
 \frac{d}{d\theta} \left(  \frac{4}{1+\tan\theta}   \right) &= 4 \cdot\frac{d}{d\theta} \left( ({1+\tan\theta} )^{-1}  \right) \\
               &= -4 (1+\tan\theta)^{-2} \cdot \frac{d}{d\theta} \left( \tan\theta \right) \\
                &= -4 (1+\tan\theta)^{-2} (\sec^2\theta) .
\end{align*}

Or, if you prefer, you could make the $u$-substitution explicit by letting
\[
 y = 4(1+\tan\theta)^{-1}
\]
and 
\[
   u = 1+\tan\theta .
\]
Then
\[
 y = u^{-1}
\]
and by the chain rule
\begin{align*}
\frac{dy}{d\theta} &= \frac{dy}{du} \cdot \frac{du}{d\theta} \\
                           &= \frac{d}{du}(u^{-1})\cdot \frac{d}{d\theta}(\tan\theta)  \\
                           &= (-u^{-2})(\sec^2\theta) \\
                            &= -4 (1+\tan\theta)^{-2} (\sec^2\theta) .
    \end{align*}

\end{example}

\begin{example} \label{E8787dfg4t4tre}
The quotient rule with the chain rule:

\begin{align*}
\frac{d}{dw}\left( \frac{w^2}{1+ e^{3w}} \right) &= \frac{(1+e^{3w})\frac{d}{dw}(w^2) - w^2\frac{d}{dw}(1+e^{3w})}{(1+e^{3w})^2}  \\
                 &= \frac{(1+e^{3w})(2w) - w^2 e^{3w}\frac{d}{dw}(3w)}{(1+e^{3w})^2}  \\
                 &= \frac{(1+e^{3w})(2w) - 3w^2 e^{3w}}{(1+e^{3w})^2} .
\end{align*}
\end{example}

\end{document}