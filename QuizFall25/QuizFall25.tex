\documentclass{ximera}
\title{Quiz Solutions Fall 2025}

\newcommand{\pskip}{\vskip 0.1 in}

\begin{document}
\begin{abstract}
Quiz solutions.
\end{abstract}
\maketitle

\section{Quiz1B}

\begin{question}  \label{QOERrfrefr}
The table below shows your gas mileage at different readings of the trip ododometer reading (in miles) during a stretch of a 200-mile car trip.

\pskip

\begin{tabular}{|c|c|}
\hline
Odometer Reading (miles) & Gas Mileage (miles/gallon) \\
\hline
23 & 23 \\  \hline
35 & 25 \\ \hline
40 & 30 \\ \hline
50 & 32 \\ \hline
%53 & 40 \\ \hline
\end{tabular}

\pskip \pskip

Assume the gas mileage is an increasing function of the odometer reading and find the best possible (ie. the least) upper bound for the number of gallons of gas your car burned between odometer readings 23 miles and 50 miles. Explain your reasoning thoroughly. Your concluding sentence should start with ``Between odometer readings 23 miles and 50 miles the car burned less than ...''
\end{question}

\begin{explanation}
We begin by considering the constant case. Here that means assuming your gas mileage is constant. Suppose for example our car gets $25$ miles/gallon over a 50 mile trip. Then over these 50 miles our car burns
\[
   \frac{50 \text{ miles}}{25 \text{ miles/gal}} = 2\text{ gallons}
\]
of gas.

Now we would burn the most gas when our car gets the worst (ie. least) gas mileage. So to find the least possible upper bound we should assume the gas mileage is constant over each interval of distance and equal to its minimum over that interval. Since the gas  mileage is assumed to increase throughout the trip, this means we should choose the left endpoint of each interval.

So between  odometer readings 23 miles and 50 miles the car burned less than
\begin{align*}
& \frac{(35-23)\text{ miles}}{23\text{ miles/gal}} + \frac{(40-35)\text{ miles}}{25\text{ miles/gal}} +  \frac{(50-40)\text{ miles}}{30\text{ miles/gal}}          < 1.06 \text{ gallons} 
\end{align*}
of gas.
\end{explanation}


\section{Quiz 2A}

\begin{question} \label{QKDFEFEFE}
The table below shows the inclination angle of a mountain trail at several elevations along the trail as it ascends to Snow Lake. The inclination angle is the angle the trail makes with the horizontal.

\pskip

\begin{tabular}{|c|c|}
\hline
Inclination Angle (radians) & Elevation (feet) \\
\hline
0.25 & 2300 \\  \hline
0.2  & 2340 \\ \hline
0.18 & 2400 \\ \hline
%50 & 32 \\ \hline
%53 & 40 \\ \hline
\end{tabular}

\pskip \pskip


Assume the inclination angle is a decreasing function of altitude and find the best possible (ie. the greatest) lower bound for the length of the trail between elevations 2300 feet and 2400 feet. Explain your reasoning thoroughly.


\begin{explanation}
We begin by considering the constant case. Here that means supposing the inclination angle is constant. Call the angle $\theta$ and consider a portion of the trail with length $\Delta s$ miles and change in elevation $\Delta h$ miles. Then (draw a picture) since
\[
   \sin \theta = \frac{\Delta h}{\Delta s} ,
\]
we know that
\[
  \Delta s = \frac{\Delta h}{\sin\theta}.
\]

Now the steeper the trail, the shorter the distance it takes to gain a given change in elevation. So to find a lower bound for the trail length, over each interval we should assume the trail has a constant inclination angle equal to the greatest angle in that interval. Since we've assumed the inclination angle to be a decreasing function over above section of trail, we should choose the left endpoints in our Riemann sum.

So between elevations $2300$ feet and $2400$ feet the length of the trail is at least
\[
\frac{(2340 - 2300)\text{ ft}}{\sin(0.25)} + \frac{(2400 - 2340)\text{ ft}}{\sin(0.20)}  > 463.68 \text{ ft}.
\]

\end{explanation}

\end{question}


\section{Quiz 2B}
\begin{question} \label{QldfFDLRE}
The function
\[
      \theta = g(s) = \frac{s}{2} - \frac{s^3}{12} \, , \, 0\leq s \leq 3 , 
\]
expresses the inclination angle (in radians) of the trail to Nada Lake in terms of the distance from the trailhead (measured along the trail in miles). The inclination angle is the angle the trail makes with the horizontal. It is positive (negative) when the trail slopes upward (downward) in the direction away from the trailhead.

The trail is at an elevation of $3200$ feet $1.6$ miles from the trailhead.

Use summation notation for an expression with $n$ equal intervals of distance to find a lower bound for the trail's elevation (in feet) at Nada Lake (3 miles from the trailhead).

Use the graph of the function $\theta=g(s)$ shown below to help with your explanation.

\begin{onlineOnly}
    \begin{center}
\desmos{nzhemrajow}{450}{600}  
\end{center}
\end{onlineOnly}

\href{https://www.desmos.com/calculator/nzhemrajow}{152: Quiz 3A}

\begin{explanation}
We begin by considering the constant case. Here that means supposing the inclination angle is constant. Call the angle $\theta$ and consider a portion of the trail with length $\Delta s$ miles and change in elevation $\Delta h$ miles. Then (draw a picture) since
\[
   \sin \theta = \frac{\Delta h}{\Delta s} ,
\]
we know that
\[
  \Delta h = \Delta s \sin\theta.
\]

Now the less steep the trail, the less the elevation gain over a given distance along the trail. So to find a lower bound for the elevation of Nada Lake, over each interval we should assume the trail has a constant inclination angle equal to the smallest angle in that interval. Since the inclination angle is a decreasing function over the trail from mile-marker 1.6 to Nada lake, we should choose the right endpoints in our Riemann sum.

With $n$-equal subintervals over the trail distance
\[
 \Delta s = (3 - 1.6) \text{ miles} = 1.4 \text{ miles},
\]
each subinterval has length
\[
 \frac{\Delta s}{n} \text{ miles} = \frac{1.4}{n} \text{ miles} .
\]

So the change in elevation from the $1.6$-mile marker to Nada Lake is at least
\[
  \frac{1.4}{n}\sum_{i=1}^{n} \sin\left( g\left( 1.6 + \frac{1.4}{n}i \right) \right) 
\]
miles. 

Since the elevation of the trail is 3200 feet at the 1.6 mile-marker and there are $5280$ feet in one mile, the elevation of Nada Lake is at least
\[
    3200 + 5280 \left(  \frac{1.4}{n}  \right)\sum_{i=1}^{n} \sin\left( g\left( 1.6 + \frac{1.4}{n}i \right) \right)
\]
feet.

\end{explanation}


\end{question}




\section{Quiz 3A}

\begin{question} \label{QLferr3r}
The function
\[
 r =g(t) = 12 -2 |t - 11| , 0 \leq t \leq 20,
\]
expresses a balloon’s rate of ascent (in meters/min) in terms of the number of minutes past noon.

\begin{enumerate}
\item Use geometry, not the Fundamental Theorem, to evaluate the integral
\[
  \int_{20}^3 g(t) \, dt.
\]
Include a graph of the function to help with your explanation. Shade the region of integration and indicate the direction of integration.

\item Interpret the meaning of the integral, in its form above, in this particular scenario.
\end{enumerate}

\begin{explanation}

\begin{enumerate}

\item The key to graphing the function is to realize $g(t)$ has its maximum value when $2|t-11|$ is as small as possible. This happens when $t=11$. 

\begin{onlineOnly}
    \begin{center}
\desmos{wioqqjwyt9}{450}{600}  
\end{center}
\end{onlineOnly}

\href{https://www.desmos.com/calculator/wioqqjwyt9}{152: Quiz 3A}

Adding the signed area of the three triangles and integrating from right to left gives
\begin{align*}
\int_{20}^3 g(t) \, dt   &= \frac{1}{2}(-3 \text{ min })(-6 \text{ meters/min}) + \frac{1}{2}(-12 \text{ min })(12 \text{ meters/min})  +  \\
                            &= \frac{1}{2}(-2 \text{ min })(-4 \text{ meters/min})    \\
                                &= - 59 \text{ ft}.
\end{align*}

\item The meaning of the integral is that going backward in time from 12:20pm to 12:03pm the balloon loses 59 feet of altitude. More simply stated, \emph{the balloon is 59 feet lower at 12:03pm than it is at 12:20pm.}

\end{enumerate}

\end{explanation}

\end{question}



\section{Quiz 3B}

\begin{question} \label{QKDFefggg}
Evaluate the definite integral
\[
  \int_1^4 \frac{2-\sqrt{t}}{t^2} \, dt.
\]

\begin{explanation}
The key is to rewrite the integrand as
\[
  \frac{2-\sqrt{t}}{t^2} = 2t^{-2} - t^{-3/2} .
\]
That way we use the Fundamental Theorem and undo the power rule. Here's the computation.

\begin{align*}
\int_1^4 \frac{2-\sqrt{t}}{t^2} \, dt  &= \int_1^4 (  2t^{-2} - t^{-3/2}  )\, dt  \\
                                                      &= \left(  -2t^{-1} + 2t^{-1/2}   \right)\Big|_1^4 \\
                                                      &= (-\frac{1}{2} + 2)  -(-1 + 2) \\
                                                       &=\frac{1}{2} .
\end{align*}

\end{explanation}

\end{question}


\section{Quiz 4A}
This was almost identical to Quiz 3B.


\section{Quiz 4B}

\begin{question} \label{QLDFefe555441}

Use the graph of the function
\[
  y = \int_2^x f(t)\, dt
\]
shown below to answer the following questions. Explain your reasoning thoroughly in complete sentences.


\begin{onlineOnly}
    \begin{center}
\desmos{zoyrvkqbv8}{450}{600}  
\end{center}
\end{onlineOnly}

\href{https://www.desmos.com/calculator/zoyrvkqbv8}{152: Quiz 4B}

\begin{enumerate}

\item Solve the inequality $f(x)>0$. Explain your reasoning.

\item For what value of $x\in[0,5]$ is $f(x)$ a minimum? Explain your reasoning.

\item Find the minimum value of $f(x)$, $x\in [0,5]$. Explain your reasoning and show your work.
\end{enumerate}


\begin{explanation}

The key point to answering these questions is to recognize that
\[
 \frac{d}{dx} \left(   \int_2^x f(t)\, dt \right) = f(t).
\]

\begin{enumerate}
\item Solving the inequality $f(x)>0$ means finding the values of $x$ where the function graphed above is increasing. The solution set is therefore
\[
  \{ 0 < x <1 \text{ or } 4 < x \leq 5 \}.
\] 

\item The minimum value of $f(x)$ occurs where the function graphed above is decreasing at the fastest rate. This occurs at approximate $x=3$.

\item The minimum value of $f(x)$ is the slope of the tangent line to the above graph at $x=3$. Using the approximate coordinates of the points $(3,-1.6)$ and $(3.2, -2)$ on the graph, the minimum value is approximately
\[
 \frac{\Delta y}{\Delta x} \sim \frac{-0.4}{0.2} = -2.
\]  

\end{enumerate}


We can make better sense of this problem by adding some context. So we'll suppose the function $f(t)$ expresses the rate of ascent (in ft/sec) of a balloon in terms of the number of seconds past noon. Then (changing the variable from $x$ to $t$) the function 
\[
  y = g_1(t) = \int_2^t f(x)\, dx ,
\]
expresses the balloon's change in height (measured in feet) from time $t=2$ to time $t$ seonds past noon. Adding the balloon's height at time $t=2$, say $h_0$ feet, gives us the height (in feet)
\[
 h = g_2(t) = h_0 + \int_2^t f(x)\, dx
\]
at time $t$ seconds past noon.

Then since the functions $g_1$ and $g_2$ differ only by a constant, the derivatives 
\[
dh/dt = dy/dt 
\]
give the balloon's rate of ascent $f(t)$ function. 

So the questions are asking us to determine
\begin{enumerate}
\item when the balloon is ascending

\item when the balloon is descending at the fastest rate

\item the fastest rate of descent, expressed as a negative rate.


\end{enumerate}
\end{explanation}
\end{question}

\section{Quiz 5A}

\begin{question} \label{QOererr3rr3}
\begin{enumerate}
\item Evaluate the definite integral
\[
      \int_1^2 \frac{5x}{1+9x^2}\, dx .
\]
Use substitution as in the homework. Do not skip steps. No credit without using substitution. 

\end{enumerate}

\begin{explanation}
It helps to write the integral as
\[
  \int_1^2 \frac{5x}{1+9x^2}\, dx = \int_1^2 (1+9x^2)^{-1}5x\, dx .
\]

Then the integrand is a product of two functions,
\[
        (1+9x^2)^{-1}
\]
and
\[
    5x .
\]
The first, $(1+9x^2)^{-1}$, is a composition of two functions. And the second $5x$ is a scalar multiple of the derivative
\[
 \frac{d}{dx}\left( 1+9x^2  \right) = 18x
\]
of the inside function of the composition.

This tells us that the substitution
\[
 u = 1 + 9x^2
\]
will undo the chain rule.

So with $u=1+9x^2$,  we have
\[
    du = 18x \, dx
\]
and 
\[
     5x\, dx = \frac{5}{18} \, du.
\]

When $x=1$,
\[
      u = 1+9(1)^2 = 10
\]
and when $x=2$,
\[
     u = 1+9(2)^2 = 37.
\]

So
\begin{align*}
\int_1^2 \frac{5x}{1+9x^2}\, dx  &= \int_1^2 (1+9x^2)^{-1} (5x \, dx)  \\
                                                  &= \int_{10}^{37} u^{-1} \left(  \frac{5}{18}\, du \right) \\
                                                  &= \frac{5}{18}\int_{10}^{37} u^{-1} \, du \\
                                                  & = \frac{5}{18}\ln \left| u \right| \Big|_{u=10}^{u=37} \\
                                                  &=\frac{5}{18} \left( \ln|37| - \ln|10| \right) \\
                                                  &= \frac{5}{18} \ln \left( 3.7 \right) .
\end{align*}

\end{explanation}
\end{question}

\section{Quiz 5B}


\begin{question}  \label{Q9ierER3P}
\item Evaluate the definite integral
\[
      \int_0^1 \frac{8e^{-3t}}{\sqrt{9-4e^{-3t}}}\, dt .
\]

\begin{explanation}
It helps to write the integral as
\[
  \int_0^1 \frac{8e^{-3t}}{\sqrt{9-4e^{-3t}}}\, dt =  \int_0^1  8e^{-3t} \left(  9-4e^{-3t}\right)^{-1/2} dt .
\]

Then the integrand is a product of two functions,
\[
        \left(  9-4e^{-3t}\right)^{-1/2}
\]
and
\[
    8e^{-3t} .
\]
The first, $(  9-4e^{-3t})^{-1/2}$, is a composition of two functions. And the second, $8e^{-3t}$, is a scalar multiple of the derivative
\[
 \frac{d}{dt}\left(  9-4e^{-3t}  \right) = 12e^{-3t}
\]
of the inside function of the composition.

This tells us that the substitution
\[
 u =  9-4e^{-3t}
\]
will undo the chain rule.

So with $u= 9-4e^{-3t}$,  we have
\[
    du = 12e^{-3t} \, dt
\]
and 
\[
     8 e^{-3t}\, dt = \frac{2}{3} \, du.
\]

When $t=0$,
\[
      u = 9-4e^0 = 5
\]
and when $t=1$,
\[
     u = 9-4e^{-3}.
\]

So
\begin{align*}
\int_0^1 \frac{8e^{-3t}}{\sqrt{9-4e^{-3t}}}\, dt  &= \int_5^{ 9-4e^{-3}} ( 9-4e^{-3t})^{-1/2} (8e^{-3t} \, dt)  \\
                                                  &= \int_5^{ 9-4e^{-3}} u^{-1/2} \left(  \frac{2}{3}\, du \right) \\
                                                  &= \frac{2}{3}\int_5^{ 9-4e^{-3}} u^{-1/2} \, du \\
                                                  & = \frac{2}{3}(2)(u^{1/2}) \Big|_{u=5}^{u=9-4e^{-3}} \\
                                                  &=\frac{4}{3} \left( \sqrt{9-4e^{-3}} - \sqrt{5} \right). \\
\end{align*}

\end{explanation}
\end{question}


\section{Quiz 6A}

\begin{question} \label{QLfdferr33rr}
A parabola with its vertex at the origin and symmetric about the $y$-axis passes through the point $P(-3,6)$. 

Find the area of the region bounded by the parabola, the tangent line to the parabola at $P$, and the $x$-axis.

Do this by slicing the region perpendicular to the $y$-axis.

\begin{explanation}
Since the parabola has its vertex at the origin and issymmetric about the $y$-axis, its equation is of the form
\[
     y = cx^2
\]
for some constant $c\neq 0$. 

Since the point $(-3,6)$ lies on the parabola
\[
    6 = c(-3)^2
\]
and 
\[
 c = 2/3.
\]
So the parabola has equation
\[
 y = \frac{2}{3}x^2 .
\]

To find an equation of the tangent line at $(-3,6)$ we first compute the slope by evaluating the derivative $dy/dx$ at $x=-3$. The slope is
\begin{align*}
  \frac{dy}{dx}\Big|_{x=-3} & = \frac{d}{dx}\left(\frac{2}{3}x^2\right)\Big|_{x=-3} \\ 
                                         &= \frac{4}{3}x \Big|_{x=-3} \\
                                         & = -4 .
\end{align*}

So an equation of the tangent line at $(-3,6)$ is 
\[
    y  = 6 -4 (x+3) 
\]
or 
\[
 y = -6 - 4x .
\]

Now slice the region perpendicular to the $y$-axis to get the differential rectangle $AB$. The endpoints share the same $y$-coordinate. Call it $y$. 

The left endpoint $A$ lies on the tangent line with equation
\[
 y = -6 - 4x
\]
and has $x$-coordinate
\[
     x = - \frac{6+y}{4} .
\]
So $A$ has coordinates
\[
        \left(   - \frac{6+y}{4} , y  \right) .
\]

The right endpoint $B$ lies on the parabola
\[
       y = \frac{2}{3}x^2.
\]
So 
\[
     x = \pm \sqrt{\frac{3}{2}y} .
\]
But since the region is in the second quadrant, the $x$-coordinate of $B$ is negative. So $B$ has $x$-coordinate
\[
      x = - \sqrt{\frac{3}{2}y}
\]
and coordinates
\[
     \left( - \sqrt{\frac{3}{2}y} , y\right) .
\]

So the differential rectangle $AB$ has width $dy$, length
\[
   L = - \sqrt{\frac{3}{2}y} + - \frac{6+y}{4}
\]
and differential area
\[
  dA = L \, dy = \left( - \sqrt{\frac{3}{2}y} +  \frac{6+y}{4}  \right)  \, dy .
\]

The area of the region bounded is then the sum
\[
     \int_{0}^6 \left( - \sqrt{\frac{3}{2}y} +  \frac{6+y}{4}  \right)  \, dy .
\]
of the differential areas. 

Using the fundamental theorem, the area is 
\begin{align*}
   \int_{0}^6 \left( - \sqrt{\frac{3}{2}y} +  \frac{6+y}{4}  \right)  \, dy &= \left(-\frac{2}{3}\sqrt{\frac{3}{2}}y^{3/2} + \frac{(6+y)^2}{4}  \right)\Big|_{y=-3}^{y=0} \\
                         & = 3/2 .
\end{align*}

\end{explanation}



\end{question}


\section{Quiz 6B}

\begin{question} \label{QPfere3}
\item A region is bounded by the coordinate axes, the line $x=-3$ and the curve
\[
  y = \frac{70}{4x-15} .
\]


\begin{enumerate}
\item Sketch a reasonably accurate graph of the curve and shade the region.

\item Write an integral that gives the area of the region.

\item Use substitution to evalaute the integral and compute the area of the region.
\end{enumerate}


\begin{explanation}
\begin{enumerate}
\item The curve
\[
 y = f(x)= \frac{70}{4x-15} 
\]
is a linear transformation of the curve $y=1/x$ with a vertical asymptote at $x=15/4$ and no $x$-intercept. So to graph the function over the interval $x\in [-3,0]$ we need only plot the points
\[
     (0,f(0)) = (1,-14/3)
\]
and
\[
   (-3, f(-3)) =(-3,  -70/27) .
\]

This tells us the function is negative over the interval $x\in[-3,0]$. So a differential rectangle perpendicular to the $x$-axis has upper endpoint $A(x,0)$ and lower endpoint $B(x,70/(4x-15)$. So with a width $dx$ and length
\[
    L = 0 - \frac{70}{4x-15} = \frac{-70}{4x-15},
\]
the rectangle has differential area %of the differential rectangle is 
\[
   dA = L\, dx = \frac{-70}{4x-15} \, dx .
\]

\item The area of the region is the sum
\[
    A =\int_{-3}^0 \frac{-70}{4x-15} \, dx
\]
of the differential areas $dA$.

\item To evalaute the above integral, we make the substitution
\[
    u = 4x-15.
\]
Then 
\[
    du = 4x \, dx
\]
and
\[
  dx = \frac{1}{4} \, du .
\]

When $x=-3$,
\[
 u = 4(-3)-15 = -27
\]
and when $x=0$,
\[
  u  = 4(0) - 15 = -15.
\]

So the area of the region is
\begin{align*}
\int_{-3}^0 \frac{-70}{4x-15} \, dx  &= - \int_{-27}^{-15} \frac{70}{u} \, \left( \frac{1}{4} \, du\right)  \\
                                                   &= -\frac{35}{2} \ln |u|\Big|_{u=-27}^{u=-15} \\
                                                   &= \frac{35}{2} \left( \ln|-27| - \ln|-15| \right) \\
                                                   &= \frac{35}{2} \ln (27/15) .
\end{align*}

\end{enumerate}
\end{explanation}
\end{question}


\section{Quiz8A}

\begin{question} \label{Q9dfrkDF}
The base of a solid is the parabolic region below with base $b = 2a$ and height $h$. Cross-sections perpendicular to the $y$-axis
are rectangles half as tall (in the direction perpendicular to the base) as they are wide.


 
\begin{onlineOnly}
    \begin{center}
\desmos{mbxrsmdho8}{900}{600}
\end{center}
\end{onlineOnly}

\href{https://www.desmos.com/calculator/mbxrsmdho8}{152: Parabola 23}

Express the volume of the solid in terms of b and h. Explain your reasoning.


\begin{explanation}
Let the right endpoint $B$ of the dashed segment above have coordinates $(x,y)$. Then the left endpoint $A$ has coordinates $(-x,y)$.

The rectangular cross-section above segment $\overline{AB}$ has width $2x$, height $x$, and area
\[
       A = 2x^2.
\]
So correpsonding slice with differential width $dy$ has differential volume
\[
 dV = A\, dy = 2x^2 \, dy.
\] 

To express $dV$ in terms of $y$, we need an equation of the parabola. This equation has the form $y=cx^2$ and since the parabola passes through the point $(a,h)$,
\[
           h = ca^2
\]
and
\begin{align*}
   c   & = \frac{h}{a^2} \\
        &= \frac{h}{(b/2)^2} \\
        &= \frac{4h}{b^2} .
\end{align*}

So the parabola has equation
\[
    y = \frac{4h}{b^2}x^2 .
\]
Then 
\[
  2x^2  = \frac{2b^2}{h} y
\]
and 
\begin{align*}
   dV = &= 2x^2 \, dy \\
           &= \frac{2b^2}{h} y \, dy.
\end{align*}

The volume of the solid is the sum of these differential volumes as $y$ runs from $y=0$ to $y=h$. So the volume is
\begin{align*}
           V  & = \int_0^h  \frac{2b^2}{h} y \, dy \\
               &= \frac{b^2}{h}y^2 \Big|_{y=0}^{y=h}  \\
               & = b^2 h.
\end{align*}

\end{explanation}
\end{question}

\section{Quiz 8B}

\begin{question} \label{QUFeev3333}

Use integration to express the volume of the red cone in terms of the area $A$ of its base and its height $h$. 

Do this as  follows:

\begin{enumerate}
\item Establish a one-dimensional coordinate system as shown below.

\item Use similar cones to express the area of the cross-section at coordinate $w$ in terms of $w$, $h$, and $A$.

\item Find an expression for the differential volume of the slice.

\item Integrate to find the volume of the cone.

\begin{onlineOnly}
    \begin{center}
\desmosThreeD{6pzlieihca}{900}{600}
\end{center}
\end{onlineOnly}

\href{https://www.desmos.com/3d/6pzlieihca}{152: Cone}

\end{enumerate}

\begin{explanation}
Let $w$ be the distance from the vertex to the slicing plane and let $B$ be the area of the cross-section.  Then since the small and large cones have the same shape
\[
     \frac{B}{A} = \left(  \frac{w}{h} \right) ^2.
\]
So
\[
   B = A \left(  \frac{w}{h} \right) ^2
\]
and the slice with thickness differential $dw$ has differential volume
\[
    dV = A \left(  \frac{w}{h} \right) ^2 \, dw.
\]
Sum these volumes to get the volume of the cone,
\begin{align*}
    V &= A \int_0^h \left(  \frac{w}{h} \right) ^2 \, dw \\
       &= \frac{A}{h^2} \int_0^h w^2 \, dw  \\
       &= \frac{1}{3} Ah .
\end{align*}
\end{explanation}

\end{question}



\end{document}