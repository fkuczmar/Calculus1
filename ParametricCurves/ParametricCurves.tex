\documentclass{ximera}
\title{Parametrically-defined curves and Area}

\newcommand{\pskip}{\vskip 0.1 in}

\begin{document}
\begin{abstract}
Curves defined parametrically and signed area.
\end{abstract}
\maketitle

\section{The Basics}

In a sense there is nothing new here. Making a $u$-substitution is equivalent to reparameterizing the graph of a function. 

To see why and to understand what $u$-substitution does geometrically, it helps to think about making a \emph{reverse} $u$-substitution. This means starting with a simple integral and making it more complicated with a substitution. Here's an example.

\begin{example} \label{EKEMrerMMEF34}
We'll start with the integral
\[
      I = \int_0^\pi \sin u \, du
\]
and make the substitution 
\[
     u = t^2 .
\]
This substitution really reparameterizes the curve
\[
   y = f(u) = \sin u \, , \, 0\leq u \leq \pi
\] 
from the standard parameterization
\[
   (x,y) = (u,\cos u) \, , \, 0\leq u \leq \pi
\]
to the new parameterization
\[
   (x,y) = (t^2,\cos (t^2) \, , \, 0\leq t \leq \sqrt{\pi}.
\]

The substitution maps the rectangle with height $y=\sin u$ and width $du$ to the rectangle with the same height $y=\sin(t^2)$ and width
\[
   du = 2t \,dt .
\]
The effect is to transform a partition of the interval of integration, originally $u\in [0,\pi]$ with equal subintervals into a partition of the new interval of integration $t\in [0,\sqrt{2\pi}]$ with \emph{unequal} subintervals $du=2t\, dt$. You can see this transformation by dragging the slider $w$ in Line 2 of the worksheeet below from $w=0$ to $w=1$.



\begin{onlineOnly}
    \begin{center}
\desmos{toiwzzqkva}{450}{600}  
\end{center}
\end{onlineOnly}

\href{https://www.desmos.com/calculator/toiwzzqkva}{152: Sub 2}

\end{example}


\section{The Area of a Circle}

Computing the area of a circle of radius $a$ with integration can be a challenging problem when approached in the standard way.

Starting with an equation
\[
    \sqrt{x^2 + y^2} = a
\]
of a circle centerd at the origin, we'll find the area of the region bounded by the semicircle
\[
   y = f(x) = \sqrt{a^2 - x^2}
\]
and the $x$-axis. On the one hand, the area of a differential rectangle having width $dx$ and endpoints $(x,0)$ and $(x,y) = (x,f(x))$ is
\[
    dA = y\, dx = \sqrt{a^2-x^2} \, dx .
\] 
So the area of the semicircle is
\[
      A = 2\int_0^a \sqrt{a^2-x^2} \, dx .
\]

Now we're faced with the problem of finding an anti-derivative of $\sqrt{a^2 - x^2}$. The standard method is to use a trigonmetric substitution. But this really amounts parameterizing the circle. 

So let's start over and parameterize the circle $x^2 + y^2 = a^2$ in terms of the polar angle $\theta$ of the vector $\overrightarrow{OP}$ from the origin to a point $P$ on the circle with coordinates $(x,y)$. 

From the \emph{definitions} of the cosine and sine functions we have
\[
   x = a \cos \theta \text{ and } y = a\sin \theta \, , \, 0\leq \theta \leq 2\pi .
\]

We'll slice the upper-semicircle perpendicular to the $x$-axis as before. Now since $0\leq \theta \leq \pi$ for the upper semicircle,
\[
      dx = d(a\cos\theta) = -a\sin\theta \leq 0,
\]
the area of the differential rectangle is
\begin{align*}
  dA & = - y \, dx \\
       &= -(a\sin\theta) d (a\cos\theta) \\
       &=  -(a\sin\theta) (-a \sin\theta\, d\theta)\\
       &= a^2\sin^2\theta \, d\theta .
\end{align*}

Geometrically, the effect of this parameterization is to transform the original parametrization 
\[
    (x,y) = (x , \sqrt{a^2-x^2}) \, , \, -a\leq x \leq a
\]
of the upper-semicircle


\begin{onlineOnly}
    \begin{center}
\desmos{eu8hz9nmgz}{450}{600}  
\end{center}
\end{onlineOnly}

\href{https://www.desmos.com/calculator/eu8hz9nmgz}{152: Circle Area}


\end{document}
