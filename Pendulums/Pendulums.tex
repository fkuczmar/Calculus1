\documentclass{ximera}
\title{The Cycloidal Pendulum}

\newcommand{\pskip}{\vskip 0.1 in}

\begin{document}
\begin{abstract}
Pendulums.
\end{abstract}
\maketitle

\section{The Cycloidal Track}

\begin{center}  
\youtube{jTN8UY_9xV0}  
\end{center}

\href{https://www.youtube.com/watch?v=jTN8UY_9xV0}{Cycloid Track}

\begin{example}  \label{EWxer3tr34r}

There are several ways to show that the cycloid is a \emph{tautochrone}. The most insightful uses vectors and we'll do this next quarter. But for  now, here's maybe the second best approach.

We'll start with a paramterization 
\[
      (x,y) = \left( a\theta + a \sin\theta , a - a\cos\theta \right) \, , \, -\pi\leq \theta \leq  \pi ,
\]
of the cycloid shown in the rectangular coordinate system below, where the parameter $a$ has units of meters.

\begin{onlineOnly}
    \begin{center}
\desmos{x6qb4zs2rq}{900}{600}
\end{center}
\end{onlineOnly}

\href{https://www.desmos.com/calculator/x6qb4zs2rq}{152: Cycloidal Pendulum 1}

We'll start by computing the time for a mass to reach the bottom of the cycloid, assuming the mass is released from rest at the top of the cycloid.

For this, we need just a bit of physics. Supposing the ball to slide along the cycloid without friction, we know the ball's mechanical energy 
\[ 
   E = mgh + \frac{1}{2} mv^2
\]
is constant. Here $m$ (in kg) is the ball's mass, $h$ (in meters) its height, and $v$ (in m/sec) its speed.  The constant $g\sim 9.8\text{ (m/sec)/sec)}$ is the magnitude of acceleration due the force of gravity near the surface of the earth.

We'll measure the gravitational potential energy $mgh$ from the $y$-axis, so that $h=y$. 

Assuming the ball is released from the top of the cycloid at parameter $\theta=\pi$, the constant mechanical energy of the ball (in terms of $m$, $g$, and $a$) is 
\[
  E = \answer{2mga} .
\]

So when the mass has height $y$ meters,
\[
      mgy +  \frac{1}{2} mv^2 = \answer{2mga}
\]
and since $v\geq 0$,
\[
   v = \answer{\sqrt{g(4a - y )}} .
\]

Now to compute the time of descent, we first analyze the constant case, the one where speed is constant. Then to travel $\Delta s$ meters at a consant speed of $v$ meters/sec takes
\[
   \Delta t = \answer{\frac{\Delta s}{v}} \text{ sec}.
\]

So to compute the descent time from the release point $(a\pi, 2a)$ to the low point $(0,0)$, we'll sum the differential times
\[
   dt = \answer{\frac{ds}{v}} \text{ sec}
\]
for the mass to travel the differential distances $ds$ meters along the cycloid.

In order to compute these differential times, we'll need to express the speed as a function of the distance $s$, measured in meters  along the cycloid from the release point $A(\pi a, 2a)$.

For this, we'll first express the differential arclength $s$ in terms of the parameter $\theta$ as 
\begin{align*}
   ds & = \sqrt{(dx)^2 + (dy)^2}    \\
       & = \sqrt{\left( d (a\theta + a \sin\theta)\right)^2 + ( d(a - a\cos\theta))^2}  \\
       & = \sqrt{\answer{(a + a\cos\theta)^2 + a^2\sin^2\theta}} \, d\theta ,
\end{align*}
where the last equality assumes $d\theta \geq 0$.

A little algebra shows 
\[
    ds = a \sqrt{2 + 2\cos\theta} \, d\theta .
\]

Now using the double angle formula
\[
  \cos \theta = 2\cos^2 (\theta/2)  - 1
\]
gives
\[
      ds = \answer{2a\cos(\theta/2)} \, d\theta,  \, , \, -\pi \leq \theta \leq \pi.
\]

Now we'll sum these differential distances to get the distance along the cycloid measured from $A$ (where $\theta=0$) to the point $P$ with parameter $\theta$. Then

\begin{align*}
    s  &= \answer{2a}\int_0^\theta \cos(\theta/2) \, d\theta \\
       & \  \answer{4a \sin(\theta/2)} \, , \, -\pi \leq \theta \leq \pi.
\end{align*}


\end{example}


\end{document}
