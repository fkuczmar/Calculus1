\documentclass{ximera}
\title{Free Fall}

\newcommand{\pskip}{\vskip 0.1 in}

\begin{document}
\begin{abstract}
Rocks in free fall.
\end{abstract}
\maketitle


\begin{question} \label{Q54rghgeyghhg}

Part (a) of the following question is an important takeaway from this chapter. It is not necessary to answer this question now.

A rock is dropped near the surface of a planet without an atmosphere. 

\begin{enumerate}
\item Through what fraction of its initial height does the rock fall during the last $1/10$ of the time it takes to hit the surface? Give a quick estimate.

\item What is the exact fraction? 
\end{enumerate}
\end{question}


\begin{exploration}  \label{Qvbbtrg43444}

Play the slider $u$ in Line 1 of the desmos worksheet below to watch two motions. One is a rock falling to the surface of Mars when dropped from a height of $190$ meters above the surface. The other is a constant-speed motion.

\begin{onlineOnly}
    \begin{center}
\desmos{j2ciuemii7}{450}{600}  
\end{center}
\end{onlineOnly}

\href{https://www.desmos.com/calculator/j2ciuemii7}{151: Free Fall 1}

\begin{enumerate}
\item Use the animation to sketch a graph of the function $s=g(t)$ that expresses the distance (in meters) of the rock from its starting point in terms of the number of seconds since the rock was released. Include the appropriate units and variable names on the axes.

\item Activate the folder \emph{Graph of distance function} in Line 17 above to see how you did.

\item Our aim now is to compute the (instantaneous) speed of the rock (in m/s) at time $t=u$ seconds. We'll assume that 
\[
      s = g(t) = 1.9t^2 \, , \, 0\leq t \leq 10.
\] 

Do this as follows.  

\begin{enumerate}
\item Find an expression that gives the rock's average speed between times $t=u$ and $t=v$.

\item Evaluate the appropriate limit to find the speed of the rock at time $t=u$ seconds. Click the Hint tab above for help.

\begin{hint}

The speed at time $t=u$ is
\begin{align*}
    \frac{d\answer{s}}{d\answer{t}} \Big|_{\answer{t=u}} &=      \lim_{\answer{v}\to \answer{u}} \frac{\answer{1.9v^2-1.9u^2}}{v-\answer{u}}   \\
                                                       & = \lim_{\answer{v}\to \answer{u}} 1.9(\answer{v+u}) \\
                                                       & = 3.8u .
 \end{align*}
\end{hint}

\item Use the result of part (ii) to find the rock's speed as it hits the ground. Assume the rock is dropped from a height of $190$ meters.

\item Compare the rock's speed as it hits the ground with its average speed during the entire time of its fall.

\item  Use the result of part (iv) to approximate the fraction of the total distance of $190$ meters the rock falls through during the last second of its fall? 

\item Compare the rock's speed at time $t=u$ with its average speed during the first $u$ seconds of its fall.

\item Use point-slope to find an equation of the tangent line to the curve $s=1.9t^2$ at the point $(u,1.9u^2)$. Enter this equation in Line 21 of the worksheet above.

The point-slope equation is
\[
     s = 1.9u^2 + \answer{3.8u} ( \answer{t} - \answer{u}) .
\]

\item Activate the folder \emph{Average speed} in Line 23 above. How is the slope of the tangent line related to the slope of the line $OP$?

\item Find the coordinates of the point where the tangent line intersects the horizontal axes. Enter these in Line 30 above.

The coordinates are
\[
  (t,s) = \left( \answer{\frac{u}{2}} , 0 \right) .
\]


\end{enumerate}


\end{enumerate}

\end{exploration}

\begin{question} \label{Qd90t90ette}
Play the slider $u$ in Line 2 of the desmos worksheet below to watch two motions. One shows a rock falling to the surface of a planet when it is dropped from rest near the surface. The other is a constant-speed motion.

\begin{onlineOnly}
    \begin{center}
\desmos{xx77pqxepy}{450}{600}  
\end{center}
\end{onlineOnly}

\href{https://www.desmos.com/calculator/xx77pqxepy}{151: Free Fall 2}

\begin{enumerate}

\item What does the animation suggest about the speed of the rock as it hits the surface compareted with its average speed during the time it takes to hit the ground?

\item What does the animation suggest about the fraction of its initial height the rock falls through during the last $1/20$ of the time it takes to hit the surface? Give a quick estimate.

\item Now suppose the rock is dropped from a height of $H$ meters and that it takes $T$ seconds to hit the surface. Assume also that the planet has no atmosphere so that the function expressing the rock's distance from its starting point (in meters) in terms of the number of seconds since it was dropped is of the form
\[
      s = f(t) = at^2
\]  
for some constant $a$.

\begin{enumerate}

\item Express the constant $a$ in terms of $H$ and $T$. 

\item What are the units of the constant $a$?

\item Explain the meaning of the constant $a$. Do \emph{not} use the word \emph{acceleration}.

\item Find the domain of the function $f$.

\item Use calculus to express the speed of the rock (in metes/sec) in terms of $H$ and $T$. Check that your expression has the correct units.

\item Compare the speed of the rock as it hits the ground with its average speed during the time it falls.

\item Through what \emph{exact} fraction of its initial height the rock fall through during the last $1/20$ of the time it takes to hit the surface? Compare this with your earlier estimate.

\end{enumerate}

\end{enumerate}


\end{question}




\end{document}