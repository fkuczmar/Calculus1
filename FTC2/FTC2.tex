\documentclass{ximera}
\title{The Fundamental Theorem of Calculus, Part 1}

\newcommand{\pskip}{\vskip 0.1 in}

\begin{document}
\begin{abstract}
FTC
\end{abstract}
\maketitle


\begin{question}\label{Q7yyweff}
The function $f(t)$, $0 \leq t \leq 12$, gives the rate (in ft/hour) at which the
depth of the water in a large tank is changing at time t hours past noon.
The water is 40 feet deep at 5:00pm.


\begin{onlineOnly}
    \begin{center}
\desmos{gnwlraiqhd}{450}{600}  
\end{center}
\end{onlineOnly}

\href{https://www.desmos.com/calculator/gnwlraiqhd}{152:Beetle 1}

%\includegraphics[width=3.5in]{WaterFlow.png}

%https://www.desmos.com/calculator/gnwlraiqhd

\begin{enumerate}
    \item The graph of the function $r=f(t)$ is shown above. Label the axes with
the appropriate variable names and units.

\item Find a function $h = g(t)$ that expresses the depth of the water (in feet) in terms of the number of hours past noon.

\item Sketch the graph of the function 
\[
   \frac{dh}{dt} = \frac{d}{dt}\left( g(t)\right).
\]
Label the axes with the appropriate variable names and units.

\item Find expressions that gives the minimum and maximum depths of water
in the tank. Use the graph to approximate these depths.

\item Sketch a graph of the function $h=g(t)$. Label the axes with the appropriate variable names and units.

\item Write an equation you would solve to determine the time(s) when the
water level is 10 feet lower than it is at 10:00pm. Use the graph to estimate
this time(s). Explain your reasoning.
\end{enumerate}


\end{question}

\begin{question} \label{Q9r3r3DEDE}

The function
\[
 v =f(t) \, , \,  0\leq  t \leq 6,
\]
expresses the speed (in inches/sec) of a beetle in terms of the number of seconds past noon.

\begin{enumerate}
\item Find a function $s = g(t)$ that expresses the distance (in inches) of the beetle to its destination in terms of the number of seconds past noon. The distance is measured along the route.

\item Find an expression for the derivative $ds/dt$. Give two answers to this question. First, an intuitive explanation that someone without any knowledge of calculus might understand. Then again, using the Fundamental Theorem.



\item Now Suppose
\[
   v =f(t) = \sqrt{6t-t^2} \, , \, 0\leq t \leq 6.
\]

\begin{onlineOnly}
    \begin{center}
\desmos{cr9oays9zo}{450}{600}  
\end{center}
\end{onlineOnly}

\href{https://www.desmos.com/calculator/cr9oays9zo}{152:Beetle Destination}

\begin{enumerate}

\item Use the graph above to make a reasonably accurate graph of the function $s = g(t)$. Explain your reasoning.

\item Use the geometry of the curve $v = f(x)$ to find an explicit expression
 (without the integral) for the function $s = g(t)$.

\item Differentiate the function $s = g(t)$ to confirm your expression is correct.

\end{enumerate}
\end{enumerate}

\end{question}



\begin{question}\label{QWEdrf4}
The function
\[
V =\int_0^h A(w)\, dw, 0\leq h \leq 6,
\]
expresses the volume (in $ft^3$) of water in a tank in terms of the water’s depth (measured in feet). The graph of the function $A$ is shown below.


\begin{onlineOnly}
    \begin{center}
\desmos{ae516hl7j6}{450}{600}  
\end{center}
\end{onlineOnly}

\href{https://www.desmos.com/calculator/ae516hl7j6}{152:Beetle 1}

%https://www.desmos.com/calculator/ae516hl7j6

%\includegraphics[width=3.5 in]{CrossSectionArea.png}


\begin{enumerate}

\item Is the volume $V$ above a function of $w$ or $h$? Explain.

\item What are the units of $A(w)$? How do you know?

\item Label the axes on the graph with their appropriate variable names and
units.



Suppose the function $h = p(t)$, $0 \leq t \leq 12$, expresses the depth of the water (in ft) in terms of the number of hours past noon.

\item Find a function $V = k(t)$ that expresses the volume of water in terms of
the number of hours past noon.

\item Find an expression for the rate of change in the volume of the water with respect to time. Explain this expression intuitively.

\item Now suppose
\[
p(t) = 3 + 2 \cos\left( \frac{\pi}{12}t \right) \,,\, 0 \leq t \leq 12.
\]
\begin{enumerate}
    \item Is the volume increasing or decreasing at 6 pm? At what rate?
    \item Is the volume increasing or decreasing when the tank holds $44\text{ ft}^3$ of
water? At what approximate rate?
   % \item Sketch the shape of the tank if it is a surface of revolution.

\end{enumerate}
  
\end{enumerate}
\end{question}



\section{Gas Consumption}


\begin{question} \label{QoODROOOD}

The function
\[
 r =g(s) \, , \, 0\leq s \leq 200 ,
\]
 expresses the rate, in gallons/mile, at which your car burns gas in terms of the distance (in miles) from your home, as measured along your route on a 200-mile road trip to Spokane. When you are 50 miles from home, your tank has 10 gallons of gas. The trip takes four hours, and the fuction
\[
 v =w(t) = 64 - 6(t-3)^2 \, , \, 0 \leq t \leq 4. 
 \]
expresses yours speed (in miles/hour) in terms of the number of hours past noon.

\begin{enumerate}

\item Find a function $G = f(s)$ that expresses the volume of gas (in gallons) your tank holds in terms of your distance from home. Include its domain.

\item Find a function $G = Q(t)$ that expresses the volume of gas in terms of the number of hours past noon. Include its
 domain.

\item Use the Fundamental Theorem to find an expression for the derivative $dG/dt$. Then explain this expression intuitively, in a way that a student without a knowledge of calculus could understand.

\item Use the graph of the function $r = g(s)$ shown below to find approximate answers to the following questions.

\begin{onlineOnly}
    \begin{center}
\desmos{tzmqpsznou}{450}{600}  
\end{center}
\end{onlineOnly}

\href{https://www.desmos.com/calculator/tzmqpsznou}{152: FTC1 Gas}


\begin{enumerate}
\item Label the axes above with the appropriate variable names and units.

\item How many gallons of gas are in your tank at 2:00pm?

\item At what rate, with respect to time, is your car burning gas when you are 62 miles from Spokane?

\item What is your gas mileage at 2:00pm? 

\item At what rate (with respect to time) is your gas mileage changing at 2:00pm?
\end{enumerate}
\end{enumerate}
\end{question}


\section{On Trails}

\begin{question} \label{QPPdreefr34}
The function
\[
  h = f(s)=5\sin\ \left(\frac{s}{5}\right)\, , \, 0\le s\leq 15,
\]
expresses the altitude (in meters) of a beetle's trail in terms of the distance from the start (measured along the trail in meters).

The function 
\[
     v = g(t) = \frac{t}{90} + \frac{t^2}{900} \, , \, 0\le t\leq 30,
\]
expresses the speed (in meters/min) of the beetle in terms of the number of minutes past noon (when it began its journey).

\begin{enumerate}

\item How far from the start is the beetle at 12:15pm?

\item Find the inclination angle of the trail beneath the beetle at 12:15pm.

\item How fast is the beetle crawling at 12:15pm?

\item Is the beetle gaining or losing altitude at 12:15pm? At what rate?

\item Express the rate at which the beetle's altitude is changing at 12:15pm in terms of some of the following derivatives. Include units for each derivative and explain their meanings.

\begin{enumerate}
    \item $\frac{dh}{ds}\Big|_{s=15}$
    \item $\frac{dh}{ds}\Big|_{s=g(15)}$
    \item $\frac{dh}{ds}\Big|_{s=2.5}$
    \item $\frac{dv}{dt}\Big|_{t=15}$
    \item $\frac{d}{dt} \int_0^t g(u)\,du$
\end{enumerate}
    
\end{enumerate}

\end{question}

\begin{question} \label{QLdkf3555}
The function
\[
  \theta = f(s) , \, 0\le s\leq 15,
\]
expresses the inclination angle (in radians) of a beetle's trail in terms of the distance from the start (measured along the trail in meters). The trail is $200$ meters above sea level at the start.

The function 
\[
     v = g(t) = 5\sin (t/15) \, , \, 0\le t\leq 30,
\]
expresses the speed (in meters/min) of the beetle in terms of the number of minutes past noon (when it began its journey). 

\begin{enumerate}
    \item Find an expression for the function
    \[
         h = w(t) \, , \, 0\le t\leq 30,
    \]
    that expresses the beetle's elevation in terms of the number of minutes past noon.

    \item Find an expression for the derivative $dh/dt$ and interpret its meaning.

    \item Compare this question  with the Question 5.

\end{enumerate}
\end{question}


\begin{question}  \label{Q9erKKe333}
The function
\[
   h = f(s) \, , \, 0\leq s \leq 60 ,
\]
expresses the elevation (in meters) of a mountain road in terms of the distance (in kilometers) from the start.

The function 
\[
    T = g(h) , 0\leq h \leq 5000 ,
\] 
expresses the temperature (in Celsius degrees) in terms of altitude (in meters) in the vicinity.

The graphs of the functions $f$ and $g$ are shown below.

\begin{onlineOnly}
    \begin{center}
\desmos{gcmfa2a6jz}{450}{600}  
\end{center}
\end{onlineOnly}

\href{https://www.desmos.com/calculator/gcmfa2a6jz}{152: Elevation and Temperature}

\begin{onlineOnly}
    \begin{center}
\desmos{ohqk3ak539}{450}{600}  
\end{center}
\end{onlineOnly}

\href{https://www.desmos.com/calculator/ohqk3ak539}{152: Elevation and Temperature 2}


\end{question}





\end{document}

