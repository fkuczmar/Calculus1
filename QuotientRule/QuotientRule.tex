\documentclass{ximera}
\title{The Quotient Rule}

\newcommand{\pskip}{\vskip 0.1 in}

\begin{document}
\begin{abstract}
An introduction to the quotient rule.
\end{abstract}
\maketitle

\section*{Discussion Questions}

\begin{question}  \label{QKDFmdeffef}
Consider the following two questions.

\begin{enumerate}
\item During the course of one year, the national debt of a country increases by $20\%$ and its population increases by $5\%$. Find the relative change in the per-capita (ie. per-person) share of the national debt over the one-year period.

\item At some instant the national debt of a country is increasing at the rate of $20\%$/yr and its population is increasing at the rate of $5\%$/yr. At what relative rate is the per-capita share of the national debt changing at this instant?
\end{enumerate}

\begin{enumerate}
\item Are these questions identical?

\item Answer the first question.

\item Use numerical methods to approximate the answer to the second question. \emph{Hint:} Approximate the relative changes in the national debt and population over a small time interval. Then use these to approximate the relative change and the relative average rate of change in the per-capita share of the debt over this time interval. Use the worksheet below to help.

\begin{onlineOnly}
    \begin{center}
\desmos{rk4aocmccz}{900}{600}
\end{center}
\end{onlineOnly}

Desmos activity available at \href{https://www.desmos.com/calculator/rk4aocmccz}{151: Relative Quotient Rule 1}

\item For another way to approach the second question, assume the debt and population functions increase at constant relative rates. Then find expressions for these functions and use these to answer the question.

\item What is your guess about the relative (instantaneous) rate of change in the quotient of two functions? How is it related to the relative rates of change of the functions?
\end{enumerate}
\end{question}

\begin{question}  \label{QPeferreson}
Use your guess for the relative rate of change in the quotient of two functions to find an expression for the derivative of the quotient of two functions.
\end{question}



\begin{question} \label{QODFoferREG}

\begin{enumerate}

\item At 9am on May 29, the balance in an account is increasing at the relative rate of $8\%$/yr. At the same time, the price of a stock in increasing at the rate of $5\%$/yr. Is the number of shares you can buy with the balance in the account increasing or decreasing at this time? At what relative rate? No computations. Just explain what you think.

\item How would your answer to part (a) change if instead the stock price were decreasing at the rate of $5\%$/yr?

\end{enumerate}
\end{question}


\begin{question} \label{QPFVmefER}
 Let $B = f(t)$ and $P=g(t)$ be functions that respectively express the balance (in dollars) in an account and the price (in dollars/share) of a stock in terms of the number of years past 9am on May 29. Let $S= h(t)$ be the number of shares of the stock you can buy with the balance in the account at time $t$ years past 9am on May 29.

\begin{enumerate}

\item Interpret the meanings of the following derivatives. Include units.

\begin{enumerate}
\item $\frac{d}{dt} \left( \ln (f(t)) \right) = \frac{d}{dt} \left( \ln B) \right)$
 
\item $\frac{d}{dt} \left( \ln (g(t)) \right) = \frac{d}{dt} \left( \ln P) \right)$

\item $\frac{d}{dt} \left( \ln (h(t)) \right) = \frac{d}{dt} \left( \ln S) \right)$

\end{enumerate}

item Express the derivative
\[
 \frac{d}{dt} \left( \ln S) \right) =  \frac{d}{dt} \left( \ln  \left( \frac{B}{P}\right) \right)
\]
in terms of the derivatives
\[
 \frac{d}{dt} \left( \ln B) \right)
\]
and
\[
   \frac{d}{dt} \left( \ln P) \right) .
\]

\end{enumerate}
\end{question}

\section*{The Relative Quotient Rule}
Had we never learned about relative changes and relative rates of change, the following questions would have almost forced these ideas upon us.

\begin{question} \label{Ex:Dfggt4ttggg}
(a) Suppose over a six-month perioid, the national debt of a small country increases from $\$4$ billion to $\$4.16$ billion while the population increases from $20$ million to $21$ million. Does the per-capita (ie. per-person) share of the national debt increase or decrease during this period? 

(b) At noon on July 1, the national debt of a small country with a population of $20$ million is $\$4$ billion.At that same instant the population is increasing at the rate of 1 million people/yr while the national debt is increasing at the rate of $\$0.16$ billion/yr. Is the per-capita share of the national debt increasing or decreasing at this instant? At what rate?

\begin{explanation}
(a) The question is really to determine which is greater, 
\[
       c_1 =  \frac{4 \times 10^9}{2 \times 10^7} \frac{\text{dollars}}{\text{person}}
\]
or
\[
      c_2 =   \frac{4.16 \times 10^9}{2.1 \times 10^7} \frac{\text{dollars}}{\text{person}} = \frac{4 \times 10^9 \times 1.04}{2 \times 10^7\times 1.05} \frac{\text{dollars}}{\text{person}} = \left( \frac{1.04}{1.05} \right)c_1 .
\]

And because the total debt increased by $4\%$ while the population increased by $5\%$, the per-capita share of the national debt decreased during the six-month period.

But the per-capita share of the debt did \emph{not} decrease by $1\%$, but rather by about (to the nearest thousandth of a percent) $\answer[tolerance=0.001]{0.952}\%$.

(b) For the instantaneous rate, let
\[
         Q = f(t)
\]
be the total national debt in billions of dollars at time $t$ years past noon on July 1 and
\[
     P = g(t)
\]
the population in millions at time $t$. Also, let
\[
       C = h(t) = \frac{f(t)}{g(t)}
\]
be the per-capita share of the debt in thousands of dollars/person at time $t$ years past noon on July 1. We wish to find the value of the derivative
\[
      \frac{dC}{dt} \Big|_{t=0} .
\]
Part (a) suggests we express the relative instantaneous rate of change
\[
     \frac{1}{C} \cdot \frac{dC}{dt} \Big|_{t=0}
\]
in terms of the relative rates
\[
    \frac{1}{Q} \cdot \frac{dQ}{dt}\Big|_{t=0} = \answer{0.04} 
\]
and
\[
   \frac{1}{P} \cdot \frac{dP}{dt}\Big|_{t=0} = \answer{0.05}
\]

Perhaps the easiest way to do this, although not entirely complete, is to assume that the population and national debt each grow exponentially. In that case, the relative growth rates of each are constant and 
\[
    Q = f(t) = \answer{4}e^{\answer{0.04t}}
\]
and 
\[
      P = g(t) = \answer{20e^{0.05t}} .
\]
This tells us that
\[
       C = \frac{f(t)}{g(t)} = \answer{0.2 e^{-0.01t}}
\]
and
\[
       \frac{1}{C} \cdot \frac{dC}{dt} \Big|_{t=0} = \answer{-0.01} 
\]
So at noon on July 1, we suspect that the per-capita share of the debt is decreasing at the relative rate of $\answer{1\%}$/yr
and at the absolute rate of $\$\answer{2}$/person/yr.

To be sure this is true even if the relative rates of change in the population and total debt are \emph{not} constant, we can be more general and compute the relative instantaneous rate of change in the per-capita share of the national debt as
%\[
%     C = \frac{f(t)}{g(t)} = \frac{Q}{P} .
%\]
%Then
%\[
%   \ln C = \ln \left( \frac{Q}{P}\right)  = \ln Q  - \ln P .
%\]
%So the relative instantaneous rate of change in the per-capita share of the national debt is
\begin{align*}
        \frac{1}{C} \cdot \frac{dC}{dt} &= \frac{d}{dt}\left( \ln \left(\frac{Q}{P}\right)  \right) \\
                                                      &= \frac{d}{dt} \left(  \answer{\ln Q} - \answer{\ln P}     \right) \\
                                                      &= \frac{d}{dt} \left( \answer{\ln Q} \right) - \frac{d}{dt} \left(  \answer{\ln P} \right)  \\
                                                      &= \frac{1}{Q} \cdot \answer{\frac{dQ}{dt}} - \frac{1}{P} \cdot \answer{\frac{dP}{dt}} .
\end{align*}


\end{explanation}
\end{question}


\begin{theorem}
(a) (The Relative Quotient Rule)
If $Q=f(t)$ and $P=g(t)$ are differentiable functions of $t$, then if $g(t)\neq 0$,
\[
     C = \frac{f(t)}{g(t)} = \frac{Q}{P}
\]
is a differentiable function of $t$ and
\[
\frac{1}{C} \cdot \frac{dC}{dt} = \frac{1}{Q} \cdot \frac{dQ}{dt} - \frac{1}{P} \cdot \frac{dP}{dt}.
\]
The relative rate of change in a quotient of two functions is equal to the difference in the relative rates of change in the functions.

(b) (The Quotient Rule) With the same hypotheses (and obtained by multiplying both side of the previous equation by $C=Q/P$), 
\[
       \frac{d}{dt}\left( \frac{Q}{P} \ \right) = \frac{1}{P} \cdot \frac{dQ}{dt} - \frac{Q}{P^2} \cdot \frac{dP}{dt} .
\]
\end{theorem}


\section*{The Tangent Fuction}

\begin{question}  \label{Ex:dsf9t5gg3w}
Compute the derivative
\[
     \frac{d}{d\theta} \left( \tan\theta  \right) 
\]

\begin{explanation}
We compute this derivative from scratch by letting 
\[
    y = \tan\theta = \frac{\sin\theta}{\answer{\cos\theta}} .
\]
Then
\[
   \ln |y| = \ln |\sin\theta| - \ln | \answer{\cos\theta}   |
\]
and
\[
      \frac{d}{d\theta} \left( \ln |y| \right) = \frac{d}{d\theta} \left( \ln |\sin\theta| - \ln | \answer{\cos\theta}| \right) .
\]
So
\begin{align*}
   \frac{1}{y} \cdot \frac{dy}{d\theta} &= \frac{1}{\answer{\sin\theta}} \cdot \frac{d}{d\theta}\left( \sin\theta \right) - \frac{1}{\answer{\cos\theta}} \cdot \frac{d}{d\theta}\left( \cos\theta \right) \\
                &= \cot\theta + \answer{\tan\theta} .
\end{align*}
Then mulitplying both sides by $y=\tan\theta$ gives
\begin{align*}
          \frac{d}{d\theta}(\tan \theta) &= 1 + \answer{\tan^2\theta} . 
\end{align*}

\end{explanation}
\end{question}

\begin{question}  \label{Qerdf4ghbhh}
(a) Use the graph of the function
\[
 y = f(\theta) = \tan (2\theta)
\]     
below to estimate the $y$-coordinates of all points on the curve where the tangent lines are parallel to the lines

(i)  $6x - y = 12$.

(ii) $6x + y = 12$ .

(b) Find the exact $y$-coordinates \emph{without} using a calculator.

\begin{onlineOnly}
    \begin{center}
\desmos{obz6ghw3ej}{900}{600}
\end{center}
\end{onlineOnly}

Desmos activity available at \href{https://www.desmos.com/calculator/obz6ghw3ej}{151: Tangent Graph}

\end{question}


\begin{question}  \label{Q:defr4gg4t}
You stand $50$ feet from the base of a tree and measure the angle of elevation to the top of the tree with an error of at most $\pm 2^\circ$. You then compute the height of the tree above eye level to be $100$ feet. 

\begin{onlineOnly}
    \begin{center}
\desmos{yjyghsoeog}{900}{600}
\end{center}
\end{onlineOnly}

Desmos activity available at \href{https://www.desmos.com/calculator/yjyghsoeog}{151: Angle of Elevation 1}

(a) Use the demonstration above to approimate your error in computing the tree's height.  

(b) Use the appropriate linear approximation to estimate your error in computing the tree's height. Compare this with your estimate from part (a). Do not use a calculator. Do this as follows:

\pskip 

\begin{itemize}

\item{Find a function
\[
    h = f(\theta) \, , \, 0<\theta < \pi/2 ,
\]
that expresses the computed height of the tree above eye level (measured in feet) in terms of the measured angle of elevation (in radians). Draw a picture to help with your explanation.

This function is
\[
   h = f(\theta) = \answer{50\tan\theta} \, , \, 0<\theta < \pi/2 .
\]
}

\item{Next find an expression for the derivative $dh/d\theta$ and evaluate the derivative
\[
     \frac{dh}{d\theta}\Big|_{h=100} .
\]
}

\item{Now we'll take the exact height of the tree above eye level to be $100$ feet and let
\[
     \Delta h = h -  \answer{100}
\]
be the error in the computed height and 
\[
   \Delta \theta = \theta - \answer{\arctan(2)}
\]
be the error in the measured angle.

Then if $\Delta\theta \sim 0$, 
\[
      \frac{dh}{d\theta}\Big|_{h=100} \sim \frac{\Delta \answer{h}}{\Delta \answer {\theta}}
\]
and so
\[
     \Delta h \sim \answer{250} \left( \Delta {\answer{\theta}} \right) .
\]

\item{I'll let you continue from here.}
 }
\end{itemize}
\end{question}


\section*{The Inverse Tangent Function}
\begin{question}  \label{Q:KKDbret434}
(a) Explain the meaning of the function
\[
        \theta = f(y) = \arctan y .
\]
In particular, what does it take as an input and what does it return as an output? Include the domain and range of the function as part of your explanation.

(b) Express the derivative
\[
     \frac{d}{d\theta} \left( \tan\theta  \right) = 1 + \tan^2\theta
\]
of the function 
\[
    y = \tan\theta
\]
in terms of $y$.

(c) Use part (b) to find an expression for the derivative
\[
   \frac{d\theta}{dy} = \frac{d}{dy} \left(  \arctan y \right) =  \frac{d}{dy} \left(  \tan^{-1} y \right) .
\]

(d) Evaluate the derivatives
\[
   \frac{d}{d\theta} \left( \tan\theta  \right)\Big|_{\theta = \pi/4}
\]
and
\[
  \frac{d}{dx} \left(  \arctan x \right)\Big|_{x=1} . 
\]
Comments?
\end{question}

\begin{question}  \label{Q45544fhL}
You stand $60$ feet from the base of a tree and measure the angle of elevation to the top of an $85$-foot tall tree. You then move $4$ feet closer to the tree and measure the same angle. Assume your eyes are five feet above the ground. 

\begin{onlineOnly}
    \begin{center}
\desmos{qqgvq3noah}{900}{600}
\end{center}
\end{onlineOnly}

Desmos activity available at \href{https://www.desmos.com/calculator/qqgvq3noah}{151: Tree 3}


(a) Drag the slider $u$ in the demonstration above to estimate the chnage in the angle of elevation after you move $4$ feet closer to the tree. Note that the angles of elevation are $\angle TEB$ and $\angle TFB$ and that $\angle ETF$ measures their difference.   Consecutive tick marks on the protractor are spaced at intervals of $0.01$ radians.

(b) Use derivatives to approximate the change in the angle of elevation and compare your approximation with your estimate in part (a) and with the actual change. Use a calculator with only addition, subtradction, multiplication, and division.

Go about this by first finding a function
\[
    \theta = f(s) \, , s>0 ,
\]
that expresses the angle of elevation to the top of the tree (measured in radians) in terms of your distance from the tree (measured in feet). Use the arctangent function in your expression. Do \emph{not} use the inverse cotangent function.

The function is 
\[
      \theta = f(s)  = \answer{\arctan(80/s)} \, , s>0 .
\]
Then continue in a manner similar to Question 4. 

\end{question}



\section*{Two Motions}
\begin{question}  \label{Q09dg000}
(a) Play the slider $s$ in the demonstration below to show the motion of a beetle crawling along the $y$-axis as it leaves behind tracks spaced at equal time intervals.

(b) Use the animation to sketch (by hand) a graph of the function
\[
     s = f(t)
\]
that expresses the position (in this case the $y$-coordinate) of the beetle as a function of time. Label the axes with the appropriate variable names and units. Then activate the folder in Line 2 to see how you did.

(c) Use the animation to sketch (by hand) a graph of the function
\[
     v= g(t)
\]
that expresses the beetle's velocity (in this case the rate of change, with respect to time, in the beetle's $y$-coordinate). Label the axes with the appropriate variable names and units. Then activate the folder in Line 7 to see how you did.


\begin{onlineOnly}
    \begin{center}
\desmos{srotstrdzm}{900}{600}
\end{center}
\end{onlineOnly}

Desmos activity available at \href{https://www.desmos.com/calculator/srotstrdzm}{151: Tangent Motion}

\end{question}


\begin{question}  \label{Qdf5555}
Repeat parts (a)-(c) of the previous question for the motion below.

\begin{onlineOnly}
    \begin{center}
\desmos{h6vq21lfql}{900}{600}
\end{center}
\end{onlineOnly}

Desmos activity available at \href{https://www.desmos.com/calculator/h6vq21lfql}{151: ArcTangent Motion}


\end{question}



\section*{Exercises}

\begin{question}  \label{Qdfgggtre}
(a) Use the website below to compute an accurate estimate of the current rate (in dollars/year) at which the U.S. national debt is changing. Explain your method. 

\href{https://www.usdebtclock.org/}{National Debt Clock}

(b) Use the website below to compute an accurate estimate (in people/yr) at which the U.S. population is currently increasing.

\href{https://www.census.gov/popclock/}{Popluation Clock}

(c) Use your estimates above and the current national debt and U.S. population to compute

   (i) the current rate of change in the per-capita share of the national debt.

    (ii) the current relative rate of change in the per-captia share of the national debt.

Include units in every number in each step of your computations.

\end{question}


\begin{question}  \label{Q:ds45t45ty}
The function 
\[
     C = f(t) = A t^4 e^{-kt} \, , \, t\geq 0, 
\]
expresses the concentration of a drug (measured in mg/L) in the bloodstream in terms of the number of hours since the drug was injected. Here $A$ and $k$ are postive constants. 

\begin{onlineOnly}
    \begin{center}
\desmos{lvsxu3wa9a}{900}{600}
\end{center}
\end{onlineOnly}

Desmos activity available at \href{https://www.desmos.com/calculator/lvsxu3wa9a}{151: Drug Concentration}

(a) What are the units of the constant $A$? How do you know?

(b) What are the units of $k$? How do you know?

(c) Use the graph of $C=f(t)$ above to sketch by hand a graph of the derivative $r=dC/dt$. Be sure to label the axes with the appropriate variable names and units. Do \emph{not} use technology. Explain your reasoning.

(d) Use calculus and algebra to find an expression in terms of $k$ for the time when the concentration is a maximum. Work in general. Do \emph{not} use a specific value of $k$. Check your work by following the directions, Lines 4 and 5, in the desmos demonstration. Label also the coordinates of the corresponding point on your hand-drawn graph of the derivative.

(e) Find expressions (in terms of $k$) for the times when the concentration is increasing and decreasing at the maximum rates. Check that your expressions have the correct units. Work in general. Do \emph{not} use a specific value of $k$. Check your work by following the directions, Lines 6-9, in the desmos demonstration. Label also the coordinates of the corresponding points on your hand-drawn graph of the derivative.


(f) Find an expression (in terms of $k$) for the relative rate at which the concentration is changing $t$ hours after the injection. Check that your expression has the correct units. Work in general. Do \emph{not} use a specific value of $k$. 

(g) Suppose the concentration is at its maximum five hours after injection and determine when the concetration is increasing at the rate of $50\%$/hr. Determine also when the concentration is decreasing at the rate $50\%$/hr.




(h) Observations?

\end{question}

\begin{question}  \label{Q:dfgt4tnhy}
The function 
\[
      P = f(t) = 5 -3t + t^2 \, , \, 0\leq t \leq 4 , 
\]
expresses the price in $\$$/share of a stock in terms of the number of hours past 9am.

(a) Use the graphs of the function $P=f(t)$ and the function $r=f^\prime(t)/f(t)$  to estimate when the stock price is increasing at the greatest relative rate.

(b) Use algebra to find the exact time when the stock price is increasing at the greatest relative rate.
\begin{hint}
What is the value of the derivative $dr/dt$ at this time? But start by finding an expression for the instantaneous relative rate of change in the stock price.
\end{hint}


\begin{onlineOnly}
    \begin{center}
\desmos{xuupp3srqv}{900}{600}
\end{center}
\end{onlineOnly}

Desmos activity available at \href{https://www.desmos.com/calculator/xuupp3srqv}{151: Stock Price 4}

\end{question}


\begin{question}  \label{Qdgvbjuhjyhu}
You jog once around a circular track of radius $r$ meters at the constant speed of speed of $v$ m/sec. A flagpole lies $b$ meters due east of the track's center.

(a) Find a function 
\[
   s = f(t) \, , 0\leq t \leq  \answer{2\pi r/v},
\]
that expresses your distance (in meters) to the flagpole in terms of the time (measured in seconds) since you started running. Assume you start at the point $A$ on the track nearest the flagpole. Explain your reasoning. Work with the general parameters $r$, $v$, and $b$, \emph{not} with any specific values for these parameters.


(b) Find an expression fro the time when your distance to the flagpole is increasing at the greatest rate. Try to give a geometric interpretation of your position at this time.

\begin{onlineOnly}
    \begin{center}
\desmos{bxofhvfbfs}{900}{600}
\end{center}
\end{onlineOnly}

Demonstration available at \href{https://www.desmos.com/calculator/bxofhvfbfs}{Math 151: Jogger 3}


\end{question}

\end{document}