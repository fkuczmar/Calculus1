\documentclass{ximera}
\title{The Quotient Rule}

\newcommand{\pskip}{\vskip 0.1 in}

\begin{document}
\begin{abstract}
An introduction to the quotient rule.
\end{abstract}
\maketitle


Had we never learned about relative changes and relative rates of change, the following questions would have almost forced these ideas upon us.

\begin{question} \label{Ex:Dfggt4ttggg}
(a) Suppose over a six-month perioid, the national debt of a small country increases from $\$4$ billion to $\$4.16$ billion while the population increases from $20$ million to $21$ million. Does the per-capita (ie. per-person) share of the national debt increase or decrease during this period? 

(b) At noon on July 1, the national debt of a small country with a population of $20$ million is $\$4$ billion.At that same instant the population is increasing at the rate of 1 million people/yr while the national debt is increasing at the rate of $\$0.16$ billion/yr. Is the per-capita share of the national debt increasing or decreasing at this instant? At what rate?

\begin{explanation}
(a) The question is really to determine which is greater, 
\[
       c_1 =  \frac{4 \times 10^9}{2 \times 10^6} \frac{\text{dollars}}{\text{person}}
\]
or
\[
      c_2 =   \frac{4.16 \times 10^9}{2.1 \times 10^6} \frac{\text{dollars}}{\text{person}} = \frac{4 \times 10^9 \times 1.04}{2 \times 10^6\times 1.05} \frac{\text{dollars}}{\text{person}} = \left( \frac{1.04}{1.05} \right)c_1 .
\]

And because the total debt increased by $4\%$ while the population increased by $5\%$, the per-capita share of the national debt decreased during the six-month period.

But the per-capita share of the debt did \emph{not} decrease by $1\%$, but rather by about (to the nearest thousandth of a percent) $\answer[tolerance=0.001]{0.952}\%$.

(b) For the instantaneous rate, let
\[
         Q = f(t)
\]
be the total national debt in billions of dollars at time $t$ years past noon on July 1 and
\[
     P = g(t)
\]
the population in millions at time $t$. Also, let
\[
       C = h(t) = \frac{f(t)}{g(t)}
\]
be the per-capita share of the debt in thousands of dollars/person at time $t$ years past noon on July 1. We wish to find the value of the derivative
\[
      \frac{dC}{dt} \Big|_{t=0} .
\]
Part (a) suggests we express the relative instantaneous rate of change
\[
     \frac{1}{C} \cdot \frac{dC}{dt} \Big|_{t=0}
\]
in terms of the relative rates
\[
    \frac{1}{Q} \cdot \frac{dQ}{dt}\Big|_{t=0} = \answer{0.04} 
\]
and
\[
   \frac{1}{P} \cdot \frac{dP}{dt}\Big|_{t=0} = \answer{0.05}
\]

Perhaps the easiest way to do this, although not entirely complete, is to assume that the population and national debt each grow exponentially. In that case, the relative growth rates of each are constant and 
\[
    Q = f(t) = \answer{4}e^{\answer{0.04t}}
\]
and 
\[
      P = g(t) = \answer{2e^{0.05t}} .
\]
This tells us that
\[
       C = \frac{f(t)}{g(t)} = \answer{2e^{-0.01t}}
\]
and
\[
        \frac{dC}{dt} \Big|_{t=0} = \answer{-0.01} 
\]
So at noon on July 1, we suspect that the per-capita share of the debt is decreasing at the relative rate of $\answer{1\%}$/yr
and at the absolute rate of $\$\answer{20}$/yr.

To be sure this is true even if the relative rates of change in the population and total debt are \emph{not} constant, we can be more general and compute the relative instantaneous rate of change in the per-capita share of the national debt as
%\[
%     C = \frac{f(t)}{g(t)} = \frac{Q}{P} .
%\]
%Then
%\[
%   \ln C = \ln \left( \frac{Q}{P}\right)  = \ln Q  - \ln P .
%\]
%So the relative instantaneous rate of change in the per-capita share of the national debt is
\begin{align*}
        \frac{1}{C} \cdot \frac{dC}{dt} &= \frac{d}{dt}\left( \ln \left(\frac{Q}{P}\right)  \right) \\
                                                      &= \frac{d}{dt} \left(  \answer{\ln Q} - \answer{\ln P}     \right) \\
                                                      &= \frac{d}{dt} \left( \answer{\ln Q} \right) - \frac{d}{dt} \left(  \answer{\ln P} \right)  \\
                                                      &= \frac{1}{Q} \cdot \answer{\frac{dQ}{dt}} - \frac{1}{P} \cdot \answer{\frac{dP}{dt}} .
\end{align*}


\end{explanation}
\end{question}


\begin{theorem}
(a) (The Relative Quotient Rule)
If $Q=f(t)$ and $P=g(t)$ are differentiable functions of $t$, then if $g(t)\neq 0$,
\[
     C = \frac{f(t)}{g(t)} = \frac{Q}{P}
\]
is a differentiable function of $t$ and
\[
\frac{1}{C} \cdot \frac{dC}{dt} = \frac{1}{Q} \cdot \frac{dQ}{dt} - \frac{1}{P} \cdot \frac{dP}{dt}.
\]
The relative rate of change in a quotient of two functions is equal to the difference in the relative rates of change in the functions.

(b) (The Quotient Rule) With the same hypotheses, 
\[
       \frac{d}{dt}\left( \frac{Q}{P} \ \right) = \frac{1}{P} \cdot \frac{dQ}{dt} - \frac{Q}{P^2} \cdot \frac{dP}{dt} .
\]
\end{theorem}

\begin{question}  \label{Ex:dsf9t5gg3w}
Compute the derivative
\[
     \frac{d}{d\theta} \left( \tan\theta  \right) 
\]

\begin{explanation}
We compute this derivative from scratch by letting 
\[
    y = \tan\theta = \frac{\sin\theta}{\answer{\cos\theta}} .
\]
Then
\[
   \ln |y| = \ln |\sin\theta| - \ln | \answer{\cos\theta}   |
\]
and
\[
      \frac{d}{d\theta} \left( \ln |y| \right) = \frac{d}{d\theta} \left( \ln |\sin\theta| - \ln | \answer{\cos\theta}| \right) .
\]
So
\begin{align*}
   \frac{1}{y} \cdot \frac{dy}{d\theta} &= \frac{1}{\answer{\sin\theta}} \cdot \frac{d}{d\theta}\left( \sin\theta \right) - \frac{1}{\answer{\cos\theta}} \cdot \frac{d}{d\theta}\left( \cos\theta \right) \\
                &= \cot\theta + \answer{\tan\theta} .
\end{align*}
Then mulitplying both sides by $y=\tan\theta$ gives
\begin{align*}
          \frac{d}{d\theta}(\tan \theta) &= 1 + \answer{\tan^2\theta} . 
\end{align*}

\end{explanation}
\end{question}

\begin{question}  \label{Qerdf4ghbhh}
Without using a calculator, find the $y$-coordinates of all points on the curve
\[
   y = \tan (2\theta)
\]
where the tangent line is parallel to the line

(a)  $6x - y = 12$.

(b) $6x + y = 12$ .
      
Draw a graph to help with your explanations.
\end{question}


\begin{question}  \label{Q:defr4gg4t}
You stand $50$ feet from the base of a tree and measure the angle of elevation to the top of the tree with an error of at most $\pm 2^\circ$. You then compute the height of the tree above eye level to be $100$ feet. 

Use the appropriate linear approximation to estimate your error in computing the tree's height. Do not use a calculator.
\end{question}


\begin{question}  \label{Q:dfgt4tnhy}
The function 
\[
      P = f(t) = 5 -3t + t^2 \, , \, 0\leq t \leq 4 , 
\]
expresses the price in $\$$/share of a stock in terms of the number of hours past 9am.

(a) Use the graphs of the function $P=f(t)$ and the function $r=f^\prime(t)/f(t)$  to estimate when the relative rate of change in the price of the stock is increasing.

(b) Use algebra to find the exact time interval during which the relative rate of change in the price of the stock is increasing.

\begin{onlineOnly}
    \begin{center}
\desmos{xuupp3srqv}{900}{600}
\end{center}
\end{onlineOnly}
\end{question}

Desmos activity available at \href{https://www.desmos.com/calculator/xuupp3srqv}{151: Stock Price 4}

\end{document}