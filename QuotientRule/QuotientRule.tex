\documentclass{ximera}
\title{The Quotient Rule}

\newcommand{\pskip}{\vskip 0.1 in}

\begin{document}
\begin{abstract}
An introduction to the quotient rule.
\end{abstract}
\maketitle

\section*{Discussion Questions}

\emph{Question A:} (a) At 9am on May 29, the balance in an account is increasing at the relative rate of $8\%$/yr. At the same time, the price of a stock in increasing at the rate of $5\%$/yr. Is the number of shares you can buy with the balance in the account increasing or decreasing at this time? At what relative rate? No compuations. Just explain what you think.

(b) How would your answer to part (a) change if instead the stock price were decreasing at the rate of $5\%$/yr?

\pskip \pskip

\emph{Question B:} Let $B = f(t)$ and $P=g(t)$ be functions that respectively express the balance (in dollars) in an account and the price (in dollars/share) of a stock in terms of the number of years past 9am on May 29. Let $S= h(t)$ be the number of shares of the stock you can buy with the balance in the account at time $t$ years past 9am on May 29.

(a) Interpret the meanings of the following derivatives. Include units.

(i) $\frac{d}{dt} \left( \ln (f(t)) \right) = \frac{d}{dt} \left( \ln B) \right)$
 
(ii) $\frac{d}{dt} \left( \ln (g(t)) \right) = \frac{d}{dt} \left( \ln P) \right)$

(iii) $\frac{d}{dt} \left( \ln (h(t)) \right) = \frac{d}{dt} \left( \ln S) \right)$

\pskip

(b) Express the derivative
\[
 \frac{d}{dt} \left( \ln S) \right) =  \frac{d}{dt} \left( \ln  \left( \frac{B}{S}\right) \right)
\]
in terms of the derivatives
\[
 \frac{d}{dt} \left( \ln B) \right)
\]
and
\[
   \frac{d}{dt} \left( \ln P) \right) .
\]


\section*{The Relative Quotient Rule}
Had we never learned about relative changes and relative rates of change, the following questions would have almost forced these ideas upon us.

\begin{question} \label{Ex:Dfggt4ttggg}
(a) Suppose over a six-month perioid, the national debt of a small country increases from $\$4$ billion to $\$4.16$ billion while the population increases from $20$ million to $21$ million. Does the per-capita (ie. per-person) share of the national debt increase or decrease during this period? 

(b) At noon on July 1, the national debt of a small country with a population of $20$ million is $\$4$ billion.At that same instant the population is increasing at the rate of 1 million people/yr while the national debt is increasing at the rate of $\$0.16$ billion/yr. Is the per-capita share of the national debt increasing or decreasing at this instant? At what rate?

\begin{explanation}
(a) The question is really to determine which is greater, 
\[
       c_1 =  \frac{4 \times 10^9}{2 \times 10^6} \frac{\text{dollars}}{\text{person}}
\]
or
\[
      c_2 =   \frac{4.16 \times 10^9}{2.1 \times 10^6} \frac{\text{dollars}}{\text{person}} = \frac{4 \times 10^9 \times 1.04}{2 \times 10^6\times 1.05} \frac{\text{dollars}}{\text{person}} = \left( \frac{1.04}{1.05} \right)c_1 .
\]

And because the total debt increased by $4\%$ while the population increased by $5\%$, the per-capita share of the national debt decreased during the six-month period.

But the per-capita share of the debt did \emph{not} decrease by $1\%$, but rather by about (to the nearest thousandth of a percent) $\answer[tolerance=0.001]{0.952}\%$.

(b) For the instantaneous rate, let
\[
         Q = f(t)
\]
be the total national debt in billions of dollars at time $t$ years past noon on July 1 and
\[
     P = g(t)
\]
the population in millions at time $t$. Also, let
\[
       C = h(t) = \frac{f(t)}{g(t)}
\]
be the per-capita share of the debt in thousands of dollars/person at time $t$ years past noon on July 1. We wish to find the value of the derivative
\[
      \frac{dC}{dt} \Big|_{t=0} .
\]
Part (a) suggests we express the relative instantaneous rate of change
\[
     \frac{1}{C} \cdot \frac{dC}{dt} \Big|_{t=0}
\]
in terms of the relative rates
\[
    \frac{1}{Q} \cdot \frac{dQ}{dt}\Big|_{t=0} = \answer{0.04} 
\]
and
\[
   \frac{1}{P} \cdot \frac{dP}{dt}\Big|_{t=0} = \answer{0.05}
\]

Perhaps the easiest way to do this, although not entirely complete, is to assume that the population and national debt each grow exponentially. In that case, the relative growth rates of each are constant and 
\[
    Q = f(t) = \answer{4}e^{\answer{0.04t}}
\]
and 
\[
      P = g(t) = \answer{2e^{0.05t}} .
\]
This tells us that
\[
       C = \frac{f(t)}{g(t)} = \answer{2e^{-0.01t}}
\]
and
\[
        \frac{dC}{dt} \Big|_{t=0} = \answer{-0.01} 
\]
So at noon on July 1, we suspect that the per-capita share of the debt is decreasing at the relative rate of $\answer{1\%}$/yr
and at the absolute rate of $\$\answer{20}$/yr.

To be sure this is true even if the relative rates of change in the population and total debt are \emph{not} constant, we can be more general and compute the relative instantaneous rate of change in the per-capita share of the national debt as
%\[
%     C = \frac{f(t)}{g(t)} = \frac{Q}{P} .
%\]
%Then
%\[
%   \ln C = \ln \left( \frac{Q}{P}\right)  = \ln Q  - \ln P .
%\]
%So the relative instantaneous rate of change in the per-capita share of the national debt is
\begin{align*}
        \frac{1}{C} \cdot \frac{dC}{dt} &= \frac{d}{dt}\left( \ln \left(\frac{Q}{P}\right)  \right) \\
                                                      &= \frac{d}{dt} \left(  \answer{\ln Q} - \answer{\ln P}     \right) \\
                                                      &= \frac{d}{dt} \left( \answer{\ln Q} \right) - \frac{d}{dt} \left(  \answer{\ln P} \right)  \\
                                                      &= \frac{1}{Q} \cdot \answer{\frac{dQ}{dt}} - \frac{1}{P} \cdot \answer{\frac{dP}{dt}} .
\end{align*}


\end{explanation}
\end{question}


\begin{theorem}
(a) (The Relative Quotient Rule)
If $Q=f(t)$ and $P=g(t)$ are differentiable functions of $t$, then if $g(t)\neq 0$,
\[
     C = \frac{f(t)}{g(t)} = \frac{Q}{P}
\]
is a differentiable function of $t$ and
\[
\frac{1}{C} \cdot \frac{dC}{dt} = \frac{1}{Q} \cdot \frac{dQ}{dt} - \frac{1}{P} \cdot \frac{dP}{dt}.
\]
The relative rate of change in a quotient of two functions is equal to the difference in the relative rates of change in the functions.

(b) (The Quotient Rule) With the same hypotheses (and obtained by multiplying both side of the previous equation by $C=Q/P$), 
\[
       \frac{d}{dt}\left( \frac{Q}{P} \ \right) = \frac{1}{P} \cdot \frac{dQ}{dt} - \frac{Q}{P^2} \cdot \frac{dP}{dt} .
\]
\end{theorem}


\section*{The Tangent Fuction}

\begin{question}  \label{Ex:dsf9t5gg3w}
Compute the derivative
\[
     \frac{d}{d\theta} \left( \tan\theta  \right) 
\]

\begin{explanation}
We compute this derivative from scratch by letting 
\[
    y = \tan\theta = \frac{\sin\theta}{\answer{\cos\theta}} .
\]
Then
\[
   \ln |y| = \ln |\sin\theta| - \ln | \answer{\cos\theta}   |
\]
and
\[
      \frac{d}{d\theta} \left( \ln |y| \right) = \frac{d}{d\theta} \left( \ln |\sin\theta| - \ln | \answer{\cos\theta}| \right) .
\]
So
\begin{align*}
   \frac{1}{y} \cdot \frac{dy}{d\theta} &= \frac{1}{\answer{\sin\theta}} \cdot \frac{d}{d\theta}\left( \sin\theta \right) - \frac{1}{\answer{\cos\theta}} \cdot \frac{d}{d\theta}\left( \cos\theta \right) \\
                &= \cot\theta + \answer{\tan\theta} .
\end{align*}
Then mulitplying both sides by $y=\tan\theta$ gives
\begin{align*}
          \frac{d}{d\theta}(\tan \theta) &= 1 + \answer{\tan^2\theta} . 
\end{align*}

\end{explanation}
\end{question}

\begin{question}  \label{Qerdf4ghbhh}
Without using a calculator, find the $y$-coordinates of all points on the curve
\[
   y = \tan (2\theta)
\]
where the tangent line is parallel to the line

(a)  $6x - y = 12$.

(b) $6x + y = 12$ .
      
Draw a graph to help with your explanations.
\end{question}


\begin{question}  \label{Q:defr4gg4t}
You stand $50$ feet from the base of a tree and measure the angle of elevation to the top of the tree with an error of at most $\pm 2^\circ$. You then compute the height of the tree above eye level to be $100$ feet. 

Use the appropriate linear approximation to estimate your error in computing the tree's height. Do not use a calculator.
\end{question}


\section*{The Inverse Tangent Function}
\begin{question}  \label{Q:KKDbret434}
(a) Express the derivative
\[
     \frac{d}{d\theta} \left( \tan\theta  \right) = 1 + \tan^2\theta
\]
of the function 
\[
    y = \tan\theta
\]
in terms of $y$.

(b) Use part (a) to find an expression for the derivative
\[
   \frac{d\theta}{dy} = \frac{d}{dy} \left(  \arctan y \right) =  \frac{d}{dy} \left(  \tan^{-1} y \right) .
\]

(c) Evaluate the derivatives
\[
   \frac{d}{d\theta} \left( \tan\theta  \right)\Big|_{\theta = \pi/4}
\]
and
\[
  \frac{d}{dx} \left(  \arctan x \right)\Big|_{x=1} . 
\]
Comments?
\end{question}

\begin{question}  \label{Q45544fhL}
You stand $50$ feet from the base of a tree and measure the angle of elevation to the top of a $105$-foot tall tree. You then move two feet closer to the tree and measure the same angle. Use derivatives to estimate the change in the angle of elevation and compare your estimate with the actual change. Assume your eyes are $5$ feet above the ground. Use a calculator with only addition, subtradction, multiplication, and division.

Do this by first finding an expression for a function
\[
    \theta = f(s) \, , s>0 ,
\]
that expresses the angle of elevation to the top of the tree (measured in radians) in terms of your distance from the tree (measured in feet). Use the arctangent function in your expression.

\end{question}



\section*{Two Motions}


\section*{Exercises}

\begin{question}  \label{Q:dfgt4tnhy}
The function 
\[
      P = f(t) = 5 -3t + t^2 \, , \, 0\leq t \leq 4 , 
\]
expresses the price in $\$$/share of a stock in terms of the number of hours past 9am.

(a) Use the graphs of the function $P=f(t)$ and the function $r=f^\prime(t)/f(t)$  to estimate when the relative rate of change in the price of the stock is increasing.

(b) Use algebra to find the exact time interval during which the relative rate of change in the price of the stock is increasing.

\begin{onlineOnly}
    \begin{center}
\desmos{xuupp3srqv}{900}{600}
\end{center}
\end{onlineOnly}

Desmos activity available at \href{https://www.desmos.com/calculator/xuupp3srqv}{151: Stock Price 4}

\end{question}


\begin{question}  \label{Qdgvbjuhjyhu}
You jog around a circular track of radius $r$ feet at the constant speed of speed of $v$ ft/sec. A flagpole lies $b$ feet due east of the track's center.

(a) Use the animation below (and nothing else) to sketch by hand a graph of the function 
\[
   s = f(\theta) \, , \theta \geq 0,
\]
that expresses your distance (in feet) to the flagpole in terms of your angle $\theta = \angle FOJ$ of rotation about the track's center, measured in radians from the time you start. Assume you start at the point $A$ on the track nearest the flagpole. Be sure to include variable names, units, and scales on your axes. Explain your reasoning. For this particular graph assume that $r=40$ and $b=96$ (be sure to adjust the sliders in the worksheet to have these values).

\begin{onlineOnly}
    \begin{center}
\desmos{4pndurvhdd}{900}{600}
\end{center}
\end{onlineOnly}


\end{question}

\end{document}