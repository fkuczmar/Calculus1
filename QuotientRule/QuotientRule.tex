\documentclass{ximera}
\title{The Quotient Rule}

\newcommand{\pskip}{\vskip 0.1 in}

\begin{document}
\begin{abstract}
An introduction to the quotient rule.
\end{abstract}
\maketitle


Had we never learned about relative changes and relative rates of change, the following questions would have almost forced these ideas upon us.

\begin{example} \label{Ex:Dfggt4ttggg}
(a) Suppose over a six-month perioid, the national debt of a small country increases from $\$4$ billion to $\$4.16$ billion while the population increases from $20$ million to $21$ million. Does the per-capita (ie. per-person) share of the national debt increase or decrease during this period? 

(b) At a certain instant the national debt of a small country with a population of $20$ million is $\$4$ billion.At that same instant the population is increasing at the rate of 1 million people/yr while the national debt is increasing at the rate of $\$0.16$ billion/yr. Is the per-capita share of the national debt increasing or decreasing at this instant? At what rate?

\begin{explanation}
(a) The question is really to determine which is greater, 
\[
       p_1 =  \frac{4 \times 10^9}{2 \times 10^6} \frac{\text{dollars}}{\text{person}}
\]
or
\[
      p_2 =   \frac{4.16 \times 10^9}{2.1 \times 10^6} \frac{\text{dollars}}{\text{person}} = \frac{4 \times 10^9 \times 1.04}{2 \times 10^6\times 1.05} \frac{\text{dollars}}{\text{person}} = \left( \frac{1.04}{1.05} \right)p_1 .
\]

And because the total debt increased by $4\%$ while the population increased by $5\%$, the per-capita share of the national debt decreased during the six-month period.

But the per-capita share of the debt did \emph{not} decrease by $1\%$, but rather by about (to the nearest thousandth of a percent) $\answer[tolerance=0.01]{0.952}\%$.

\end{explanation}

\end{example}

\end{document}