\documentclass{ximera}
\title{Circles, Spheres, Area, Volume}

\newcommand{\pskip}{\vskip 0.1 in}

\begin{document}
\begin{abstract}
Thoughts on the hoof of Archimedes, spheres, circles.
\end{abstract}
\maketitle

\section{The Area of a Circle}

\begin{onlineOnly}
    \begin{center}
\desmos{2pe3pmtjyv}{900}{600}
\end{center}
\end{onlineOnly}

\href{https://www.desmos.com/calculator/2pe3pmtjyv}{152: Unwrapping Semicircle}

\begin{onlineOnly}
    \begin{center}
\desmos{9a3oluopqk}{900}{600}
\end{center}
\end{onlineOnly}

\href{https://www.desmos.com/calculator/9a3oluopqk}{152: Unwrapping Circle}


\section{The Hoof of Archimedes}

A solid circular cylinder of radius $a$ and height $h$ is sliced by a plane passing through a diameter of the base and intersecting the top of the cylinder in exactly one point. The plane cuts the cylinder into two pieces. Our goal here is to find an expression in $h$ and $a$  for the volume of the lower (smaller) piece.

Here are three ways to find the volume.


\begin{onlineOnly}
    \begin{center}
\desmosThreeD{j4cadgw4ct}{900}{600}
\end{center}
\end{onlineOnly}

\href{https://www.desmos.com/3d/j4cadgw4ct}{152: Hoof of Archimedes 3}

The volume of the hoof turns out to be
\[
      V = \frac{2}{3}a^2 h.
\]

This looks like the volume of a cone and suggests a quicker way to find the volume.

\begin{onlineOnly}
    \begin{center}
\desmosThreeD{ddcsbxgbrl}{900}{600}
\end{center}
\end{onlineOnly}

\href{https://www.desmos.com/3d/ddcsbxgbrl}{152: Hoof of Archimedes Unwrapped}



\section{Archimedes' Bicylinder}
The hoof of Archimedes comes up in unexpected places. Here's one.

\begin{onlineOnly}
    \begin{center}
\desmosThreeD{wzzhod4ya6}{900}{600}
\end{center}
\end{onlineOnly}

\href{https://www.desmos.com/3d/wzzhod4ya6}{152:Bicylinder}



\begin{onlineOnly}
    \begin{center}
\desmosThreeD{phy3gfnhcj}{900}{600}
\end{center}
\end{onlineOnly}

\href{https://www.desmos.com/3d/phy3gfnhcj}{152: Bicylinder and Hoof}

The bicylinder has close ties to one of our solids, the one with the circular base and square cross-sections.

\begin{onlineOnly}
    \begin{center}
\desmosThreeD{cwcjqkjndw}{900}{600}
\end{center}
\end{onlineOnly}

\href{https://www.desmos.com/3d/cwcjqkjndw}{152: Square Cross-Sections to Bicylinder}



\section{Tori}

The hoof is also connected to tori.

The unwrapping preserves both surface area and volume.

\begin{onlineOnly}
    \begin{center}
\desmosThreeD{9gcd0sjhzr}{900}{600}
\end{center}
\end{onlineOnly}

\href{https://www.desmos.com/3d/9gcd0sjhzr}{152: Torus Unwrapped}



\section{The Sphere}

We can also unwrap a sphere, preserving both its volume and surface area.

\begin{onlineOnly}
    \begin{center}
\desmosThreeD{wlqzsdy2xp}{900}{600}
\end{center}
\end{onlineOnly}

\href{https://www.desmos.com/3d/wlqzsdy2xp}{152: Sphere Unwrapped}


A second unwrapping should take take this truncated cylinder to a cone. To be continued...












\end{document}
