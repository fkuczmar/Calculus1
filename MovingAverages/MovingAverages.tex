\documentclass{ximera}
\title{Moving Averages and their Rates of Change}

\newcommand{\pskip}{\vskip 0.1 in}

\begin{document}
\begin{abstract}
Moving averages and the quotient rule.
\end{abstract}
\maketitle


\begin{question} \label{QPderRERglO}
\begin{enumerate}
\item Your average for the first four exams in a class is $90\%$. How would a score of $100\%$ on the fifth exam \emph{change} your average?

\item Your average for the first nine exams in a class is $90\%$. How would a score of $100\%$ on the tenth exam change your average?

\item Over the first four hours of a ten-hour car trip, your average speed was $90$ km/hour. During the fifth hour you drove at a constant speed of $100$ km/hour. Compare your average speeds over the first four hours and over the first five hours of your trip.

\item  Over the first nine hour of a ten-hour car trip, your average speed was $90$ km/hour. During the tenth hour you drove at a constant speed of $100$ km/hour. Compare your average speeds over the first nine hours and over the entire ten hours of your trip.

\item A computer display keeps track of your average speed since the start of your trip. Suppose four hours into a ten-hour car trip, your trip odometer reading (set to zero at the start of your trip) reads $360$km and your speedometer reads $100$ km/hour. Find the rate of change, with respect to time (measured in hours), in your average speed as shown on the display at this instant.

\item A computer display keeps track of your average speed since the start of your trip. Suppose nine hours into a ten-hour car trip, your trip odometer reading (set to zero at the start of your trip) reads $810$km and your speedometer reads $100$ km/hour. Find the rate of change, with respect to time (measured in hours), in your average speed as shown on the display at this instant.

\end{enumerate}
\end{question}


\end{document}