\documentclass{ximera}
\title{Trails and Hanging Chaings, Part 1}

\newcommand{\pskip}{\vskip 0.1 in}

\begin{document}
\begin{abstract}
Trails and catenaries.
\end{abstract}
\maketitle


\section{Angles of Inclination}

\begin{question}  \label{QPdiftgerdf}
Find the measure of the acute (between $0$ and $\pi/2$) angle the tangent line to the curve
\[
     y = \ln | \sec \theta|
\]
at the point $(2\pi/7, \ln |\sec (2\pi/7)|)$ makes with the $x$-axis. Give an exact answer without using a calculator.



\begin{onlineOnly}
    \begin{center}
\desmos{yyyj3pjmk2}{450}{600}  
\end{center}
\end{onlineOnly}

\href{https://www.desmos.com/calculator/yyyj3pjmk2}{152: Log Secant}
\end{question}


\begin{question}  \label{QJKDF3erdfsmdf}
Find the measure of the acute angle the tangent line to the curve
\[
    y = 4 \tan(x/20)
\]
makes with the $x$-axis at the points with $y$-coordinate $y=3$. Give an exact answer, fully simplified, without using a calculator.

\begin{onlineOnly}
    \begin{center}
\desmos{df7hjgabzb}{450}{600}  
\end{center}
\end{onlineOnly}

\href{https://www.desmos.com/calculator/df7hjgabzb}{152: Angle of Inclination}
\end{question}



\section{A Climber's Trail}

\begin{question} \label{QOIIdfsdftt444}
The function
\[
 h = f(s) =100+ 250s-10s^3 \, , \, 0\leq s \leq 5
\]
expresses the altitude (in meters) of a climber's trail in terms of the distance from the trailhead (in miles). Find the exact angle the trail makes with the horizontal one mile from the trailhead.

\begin{onlineOnly}
    \begin{center}
\desmos{ljcugfqmee}{450}{600}  
\end{center}
\end{onlineOnly}

\href{https://www.desmos.com/calculator/ljcugfqmee}{152: Hiking Trail 54}

\end{question}


\section{A Hanging Chain}

\begin{question} \label{QJFefexxxwe}
Which of the following curves has the same shape as the curve
\[
    y = \sin\theta?
\]
Select all that apply.

\begin{selectAll}
\choice{$y=2\sin\theta$}
\choice{$y=2\sin (2\theta)$}
\choice[correct]{$y=2\sin (\theta/2)$}
\choice{$y=0.2\sin (0.2\theta)$}
\choice[correct]{$y=0.2\sin (5\theta)$}
\end{selectAll}

Note: Two curves have the same shape if you can scale a photograph of one to get an exact copy of the other.

\end{question}

\begin{question} \label{QOdfertr3g}
A chain
\begin{onlineOnly}
    \begin{center}
\desmos{tj3dz2cnf0}{450}{600}  
\end{center}
\end{onlineOnly}

\href{https://www.desmos.com/calculator/tj3dz2cnf0}{152: Hanging Chain 1}


\begin{enumerate} 

\item To see that the dashed (orange) chain in the worksheet above is \emph{not a parabola} drag the slider $c$ in Line 2.

\item It turns out that a chain with uniform density hanging in a uniform graviatational field assumes the shape of the hyperbolic cosine function $y=\cosh x$ (we'll see why later in the course). All that's needed to describe a particular chain is to scale the size of the graph by some factor. Test this by dragging the slider $a$ in Line 6 to make the graph of the function
\[
     f(x) = -a + a\cosh(x/a)
\]
match the highlighted (orange) chain below.
\end{enumerate}

\end{question}



\end{document}