\documentclass{ximera}
\title{Trails and Hanging Chains, Part 1}

\newcommand{\pskip}{\vskip 0.1 in}

\begin{document}
\begin{abstract}
Trails and catenaries.
\end{abstract}
\maketitle


\section{Angles of Inclination}

\begin{question}  \label{QPdiftgerdf}
Find the measure of the acute (between $0$ and $\pi/2$) angle the tangent line to the curve
\[
     y = \ln | \sec \theta|
\]
at the point $(2\pi/7, \ln |\sec (2\pi/7)|)$ makes with the $x$-axis. Give an exact answer without using a calculator.



\begin{onlineOnly}
    \begin{center}
\desmos{yyyj3pjmk2}{450}{600}  
\end{center}
\end{onlineOnly}

\href{https://www.desmos.com/calculator/yyyj3pjmk2}{152: Log Secant}
\end{question}


\begin{question}  \label{QJKDF3erdfsmdf}
Find the measure of the acute angle the tangent lines to the curve
\[
    y = 4 \tan(x/20)
\]
make with the $x$-axis at the points with $y$-coordinate $y=3$. Give an exact answer, fully simplified, without using a calculator.

\begin{onlineOnly}
    \begin{center}
\desmos{df7hjgabzb}{450}{600}  
\end{center}
\end{onlineOnly}

\href{https://www.desmos.com/calculator/df7hjgabzb}{152: Angle of Inclination}


\begin{explanation}
Since we are given the $y$-coordinate of the point of tangency, we should express the derivative of our function not in terms of its input, but in terms of its output.

As a start, note that if $z=\tan\theta$, then
\begin{align*}
     \frac{dz}{d\theta} &= \frac{d}{d\theta}\left( \tan\theta  \right) \\
                                &= \sec^2\theta \\
                                 &= 1 + \tan^2\theta \\
                                 &= 1 + z^2 . 
\end{align*}

For our function $y=4\tan(x/20)$, let $u=x/20$. Then
\[
  y = 4 \tan u      
\]
and by the chain rule
\begin{align*}
         \frac{dy}{dx} &= \frac{dy}{du} \cdot \frac{du}{dx} \\
                                    &= \frac{d}{du} (4\tan u) \cdot \frac{d}{dx}\left( \frac{1}{20}x \right) \\
                                    &= 4 (1+\tan^2 u) \left( \frac{1}{20} \right) \\
                                    &=\frac{1+(y/4)^2}{5} .
\end{align*}

So the tangent lines at the points on the curve $y=4\tan(x/20)$ with $y$-coordinate $y=3$ have slope
\begin{align*}
  \frac{dy}{dx}\Big|_{y=3} &= \left( \frac{1+(y/4)^2}{5}\right) \Big|_{y=3} \\
                                       &= 5/16.
\end{align*}
These tangent lines therefore cut the $x$-axis at the acute angle $\phi =\arctan(5/16)$.

\end{explanation}


\end{question}



\section{A Climber's Trail}

\begin{question}  \label{QPPDFrLMDECC}
A loading ramp of length $L$ meters runs from the ground to a truck. The ramp makes an angle of $\phi$ radians with the ground. 

Find an expression for the function 
\[
      h = f(s) \, , \, 0\leq s \leq L ,
\]
that expresses the height of the ramp (in meters) in terms of the distance along the ramp, measured in meters from the bottom of the ramp.
\end{question}

\begin{question} \label{QOIIdfsdftt444}
The function
\[
 h = f(s) =100+ 250s-10s^3 \, , \, 0\leq s \leq 5
\]
expresses the altitude (in meters) of a climber's trail in terms of the distance from the trailhead (in km). 

\begin{enumerate}

\item Evaluate the derivative
\[
   \frac{dh}{ds}\Big|_{s=1} .
\]
Include units.

\item Interpret the meaning of the above derivative using the language of small changes for a \emph{specific} small change.

\item Find the exact angle the trail makes with the horizontal one kilometer from the trailhead.

\end{enumerate}


\begin{onlineOnly}
    \begin{center}
\desmos{ljcugfqmee}{450}{600}  
\end{center}
\end{onlineOnly}

\href{https://www.desmos.com/calculator/ljcugfqmee}{152: Hiking Trail 54}

\begin{explanation}
\begin{enumerate}
\item The derivative is
\begin{align*}
    \frac{dh}{ds} &= \frac{d}{ds}(  100+ 250s-10s^3 ) \\
                         &= 250 - 30s^2 \\
\end{align*}
and
\[
  \frac{dh}{ds}\Big|_{s=1}  = 220 \text{ meters/km} .
\]

\item The derivative relates a small change
\[
   \Delta s = s - 1
\]
in distance along the trail from kilometer marker $s=1$ and the corresponding small change
\[
    \Delta h = h - f(1) = h - 340
\]
in altitude (measured in meters). 

More specifically, if $\Delta s\sim 0$, then
\begin{align*}
  \Delta h &\sim \frac{dh}{ds}\Big|_{s=1} \cdot \Delta s \\
               &= (220 \text{ meters/km})(\Delta s \text{ km})  \\
                &= 220 \Delta s \text{ meters}.
\end{align*}
So for example, if we walk $1$ meter (ie. $0.001$ km) farther along the trail from kilometer marker $s=1$ km, then our altitude increases by approximately
\[
     \Delta h \sim 220(0.001)\text{ meters} = 0.2 \text{ meters} .
\]

\item The description of the derivative's meaning tells us how to compute the trail's inclination angle at kilometer marker $s=1$ km. Since we gain altitude at the rate of
\[
     220 \text{ meters/km} = 0.22 \text{ meters in elevation/meter walked}   
\]
one kilometer from the trailhead, the trail looks locally like a ramp inclined at the angle
\[
  \phi = \arcsin(0.22) \text{ rad}
\]
above the horizontal there.

\pskip

{\bf Remark:} It's important to distinguish between the graph of the function $h=f(s)$ that expresses altitude in terms of distance along the trail and the trail profile. The latter is like a picture of the actual trail, assuming the trail stays confined to a vertical plane. For most trails, the two are indistinguishable. The worksheet below, for example, shows our original graph $h=f(s)$ along with the corresponding trail profile (in red).

\begin{onlineOnly}
    \begin{center}
\desmos{e5kuf1je8p}{450}{600}  
\end{center}
\end{onlineOnly}

\href{https://www.desmos.com/calculator/e5kuf1je8p}{152: Hiking Trail 54B}

But for steeper trails, like one an ant might climb, there is a clear difference between the two.

\begin{onlineOnly}
    \begin{center}
\desmos{lktn53d7ju}{450}{600}  
\end{center}
\end{onlineOnly}

\href{https://www.desmos.com/calculator/lktn53d7ju}{152: Hiking Trail 54C}



\end{enumerate}
\end{explanation}


\end{question}


\section{A Hanging Chain}

\begin{question} \label{QJFefexxxwe}
Which of the following curves have the same shape as the curve
\[
    y = \sin\theta?
\]
Select all that apply.

\begin{selectAll}
\choice{$y=2\sin\theta$}
\choice{$y=2\sin (2\theta)$}
\choice[correct]{$y=2\sin (\theta/2)$}
\choice{$y=0.2\sin (0.2\theta)$}
\choice[correct]{$y=0.2\sin (5\theta)$}
\end{selectAll}

Note: Two curves have the same shape if you can scale a photograph of one to get an exact copy of the other.

\end{question}

\begin{question} \label{iifkdeyq3y}
A chain of uniform density hanging under its own weight in a uniform gravitational field takes the shape of a the curve $y=\cosh x$ (we'll see why later in the course).


\begin{onlineOnly}
    \begin{center}
\desmos{iifkdeyq3y}{450}{600}  
\end{center}
\end{onlineOnly}

\href{https://www.desmos.com/calculator/iifkdeyq3y}{152: Hanging Chain 2}


\begin{enumerate} 


\item All that's needed to describe a particular chain is to scale the size of the curve $y=\cosh x$ by some factor. Test this by dragging the slider $a$ in Line 6 to make the graph of the function
\[
     f(x) = -a + a\cosh(x/a)
\]
match the highlighted (orange) chain below.

\item What are the units of $a$? How do you know?

\item Using the value of $a$ from part (a), find the angle the chain makes with the horizontal at the points two meters above the low point of the chain. Give an exact expression, fully simplified, without using a calculator.
\end{enumerate}

See the chapter

\href{https://ximera.osu.edu/calcone/Calculus1/HyperbolicTrig/HyperbolicTrig}{Hyperbolic Trigonometry}

of our class notes for information on the hyperbolic trig functions.

\end{question}



\end{document}