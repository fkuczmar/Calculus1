\documentclass{ximera}
\title{Differentiable Functions, Part 2}

\newcommand{\pskip}{\vskip 0.1 in}

\begin{document}
\begin{abstract}
A more formal approach to defining what it means for a function to be differentiable.
\end{abstract}
\maketitle


\begin{exploration}   \label{Expdsfdsftttehh030}
The aim of this problem is to use numerical methods to approximate and algebra to evaluate the derivative
\[
   \frac{d}{dx}\left( x^3 \right) \Big|_{x=a}.
\]

\begin{onlineOnly}
    \begin{center}
\desmos{rto22qzlvm}{450}{600}  
\end{center}
\end{onlineOnly}

\href{https://www.desmos.com/calculator/rto22qzlvm}{151: Cubing Function 3}


The idea to compute the derivative algebraically is this: Fix the point $P(a,a^3)$ on the graph of $y=f(x)=x^3$. Then choose a variable point $Q$ on the graph, different from $P$, with coordinates $(v,v^3)$. When $Q$ is sufficiently close to $P$, the line $PQ$ approximates the curve $y=x^3$ near $P(a,a)$ and the slope of this line is above derivative. 

\begin{enumerate}
\item Our first step is to find an expression for the slope of line $PQ$ as a function of $v$. The slope is
\begin{align*}
        m(v) &= \frac{\Delta y}{\Delta v}  \\
                & = \frac{f(v) - f(a)}{\answer{v-a}} \\
                & = \frac{\answer{v^3-a^3}}{v-a} \\
\end{align*}
all assuming $v\neq \answer{a}$.


\item The next step is to factor the numerator $v^3-a^2$. Since 
\[
     (v^3 - a^3)\Big|_{v=a} = a^3 - a^3 = 0 ,
\]
we know that $\answer{v-a}$ is a factor of $v^3-a^3$. To simplify the quotient
\[
  \frac{v^3-a^3}{v-a} \, , \, v\neq a,
\]
we could use long division or factor the difference of two cubes:
\[
   A^3 - B^3 = (A-B)(A^2 + AB + B^2).
\]
Either way, the result is that
\[
     v^3 - a^3 = (v-a)(\answer{v^2 + va + a^2}).
\]
So
\[
   \frac{v^3-a^3}{v-a} = \answer{v^2+va+a^2} \, , \, v\neq a.
\]
Putting this all together, we get
\begin{align*}
\frac{d}{dx}\left( x^3 \right)\Big|_{x=a}  & = \lim_{v\to a} \frac{v^3-a^3}{v-a} \\
                                                              &= \lim_{v\to a} \frac{(v-a)(v^2+va+a^2)}{v-a} \\
                                                             & = \lim_{v\to a}(\answer{v^2 + va + a^2}) \\
                                                            & = a^2 + \answer{a^2} + \answer{a} \\
                                                            & = \answer{3a^2} .
\end{align*}
\end{enumerate}

\
\end{exploration}




\end{document}