\documentclass{ximera}
\title{Differentiable Functions, Part 2}

\newcommand{\pskip}{\vskip 0.1 in}

\begin{document}
\begin{abstract}
Using limits to compute the derivative of a function at a general input.
\end{abstract}
\maketitle

\section{Using Limits to Compute Derivatives}

\begin{exploration}   \label{Expdsfdsftttehh030}
In part (b) of Exploration 7 from the previous chapter where we computed

\[
   \frac{d}{dx}\left( x^3 \right) \Big|_{x=1}.
\]

Now we'll use limits to evaluate the derivative
\[
   \frac{d}{dx}\left( x^3 \right) \Big|_{x=a}.
\]

In other words, we'll compute the derivative of the function $f(x)=x^3$ at a general input $x=a$. Think about zooming in on the graph of $y=x^3$ sufficiently close to the point $(a,a^3)$ so that the graph looks like a straight line. We'll compute the slope of that line.

The algebra here is nearly identical to what we did earlier. You should compare the two computations. 

\begin{onlineOnly}
    \begin{center}
\desmos{8eiffwbgt5}{450}{600}  
\end{center}
\end{onlineOnly}

\href{https://www.desmos.com/calculator/8eiffwbgt5}{151: Cubing Function 3}


The idea to compute the derivative algebraically is this: Fix the point $P(a,a^3)$ on the graph of $y=f(x)=x^3$. Then choose a variable point $Q$ on the graph, different from $P$, with coordinates $(v,v^3)$. When $Q$ is sufficiently close to $P$, the line $PQ$ approximates the curve $y=x^3$ near $P(a,a)$ and the slope of this line is above derivative. 

\begin{enumerate}
\item Our first step is to find an expression for the slope of line $PQ$ as a function of $v$. The slope is
\begin{align*}
        m(v) &= \frac{\Delta y}{\Delta v}  \\
                & = \frac{f(v) - f(a)}{\answer{v-a}} \\
                & = \frac{\answer{v^3-a^3}}{v-a} \\
\end{align*}
all assuming $v\neq \answer{a}$.


\item The next step is to factor the numerator $v^3-a^2$. Since 
\[
     (v^3 - a^3)\Big|_{v=a} = a^3 - a^3 = 0 ,
\]
we know that $\answer{v-a}$ is a factor of $v^3-a^3$. To simplify the quotient
\[
  \frac{v^3-a^3}{v-a} \, , \, v\neq a,
\]
we could use long division or factor the difference of two cubes:
\[
   A^3 - B^3 = (A-B)(A^2 + AB + B^2).
\]
Either way, the result is that
\[
     v^3 - a^3 = (v-a)(\answer{v^2 + va + a^2}).
\]
So
\[
   \frac{v^3-a^3}{v-a} = \answer{v^2+va+a^2} \, , \, v\neq a.
\]
Putting this all together, we get
\begin{align*}
\frac{d}{dx}\left( x^3 \right)\Big|_{x=a}  & = \lim_{v\to a} \frac{v^3-a^3}{v-a} \\
                                                              &= \lim_{v\to a} \frac{(v-a)(v^2+va+a^2)}{v-a} \\
                                                             & = \lim_{v\to a}(\answer{v^2 + va + a^2}) \\
                                                            & = a^2 + \answer{a^2} + \answer{a} \\
                                                            & = \answer{3a^2} .
\end{align*}

\item More simply put, we just write the derivative
\[
   \frac{d}{dx}\left( x^3  \right) = \answer{3x^2}
\]
as a function of $x$.

\item We can check our result using the desmos worksheet above. Do this as follows:

\begin{enumerate}

\item Input the correct expression for the derivative $d(x^3)/dx$ on Line 17.

\item For a differentiable function $f(x)$ and values of $h$ near zero, we can approximate the derivative $d(f(x))/dx$ as the slope
\[
  \frac{d}{dx}\left( f(x)  \right) = \frac{f(x+h) - f(x-h)}{\answer{2h}}
\]
of the line through the points $(x-h, f(x-h))$ and $(x+h, f(x+h))$.

Now activate the folder \emph{Approximation to the derivaitve function} on Line 19. Then drag the slider $h$ on Line 21 to get a better approximation to the derivative.

\begin{freeResponse}
Describe what happens to the above approximation as $h \to 0$.
\end{freeResponse}

\end{enumerate}
\end{enumerate}
\end{exploration}


\begin{exercise}  \label{Exerg4t5677878}
\begin{enumerate}
\item Follow the method of Exploration \ref{Expdsfdsftttehh030} for the function $f(x)=x^4$ to compute the derivative
\[
    \frac{d}{dx}\left( x^4 \right) \Big|_{x=a}.
\]

\item Modify the desmos worksheet below for the function $f(x)=x^4$.

\href{https://www.desmos.com/calculator/8eiffwbgt5}{151: Cubing Function 3}

Then work through part (d) of Exploration  \ref{Expdsfdsftttehh030} for $f(x)=x^4$. Describe also what happens to the approximation in part (d) as $h\to 0$. Include a few screenshots to help with your description.

\end{enumerate}
\end{exercise}

\begin{exercise}
Repeat all parts of Exercise \label{Exerg4t5677878} for the function $f(x)=1/x^4$.
\end{exercise}


\section{Tangent Lines}

\section{A Geometric Interpretation}

We just saw that
\[ 
       \frac{d}{dx} \left( x^3 \right)\Big|_{x=a} = 4a^3.
\]

Here's a way to think about what this says geometrically.

\begin{onlineOnly}
    \begin{center}
\desmos{bwhvxlapr1}{450}{600}  
\end{center}
\end{onlineOnly}

\href{https://www.desmos.com/calculator/bwhvxlapr1}{Derivative Geometric Meaning}

bwhvxlapr1

\end{document}