\documentclass{ximera}
\title{Riemann Sums with Geometric Partitions}

\newcommand{\pskip}{\vskip 0.1 in}

\begin{document}
\begin{abstract}
Using Riemann sums with geometric partitions to evaluate definite integrals.
\end{abstract}
\maketitle



\section{Integrating $1/x$}

Before using Riemann sums and limits to find an expression for the integral
\[
    g(u) = \int_1^u \frac{1}{x}\, dx ,
\]
let's first compare the integrals %think geometrically. We'll assume $x>1$ and interpret the integral as an area.
\begin{equation}
      \int_1^2 \frac{1}{x}\, dx  \label{EqIntegral1}
\end{equation}
and 
\begin{equation}
      \int_3^6 \frac{1}{x}\, dx.   \label{EqIntegral2}
\end{equation}

We'll do this geometrially, without a computation.

The idea is to first describe a compositions of transformations that takes the curve $y=1/x$ to itself and the interval $[1,2]$ to the interval $[3,6]$. Then we'll look at how this composition affects are

\begin{exploration}  \label{Ex:LdkDERer4}

\begin{enumerate}
\item Drag the slider $v$ in Line 2 of the worksheet below from $v=1$ to $v=3$ and describe what happens to the graph of $y=1/x$.

\begin{freeResponse}
\end{freeResponse}

\item What does the transformation do the area of the shaded rectangle (or any rectangle)?

It mutiplies the area by a factor of $\answer{3}$.

\item Express a equation of the new curve (dotted) in terms of $v$.

An equation is
\[
 y = \answer{v/x}.
\]


\item Next drag the slider $w$ in Line 4 of the worksheet from $w=1$ to $w=3$ and describe what happens to the graph of $y=v/x$.

\begin{freeResponse}
\end{freeResponse}

\item What does the transformation do the area of the shaded rectangle (or any rectangle)?

It mutiplies the area by a factor of $\answer{1/3}$.

\item What is your conclusion about the relationship between the integrals (\ref{EqIntegral1}) and (\ref{EqIntegral2}) above? Explain your reasoning. 

\begin{freeResponse}
\end{freeResponse}



\end{enumerate}

\begin{onlineOnly}
    \begin{center}
\desmos{al95rowj4h}{450}{600}  
\end{center}
\end{onlineOnly}

\href{https://www.desmos.com/calculator/al95rowj4h}{152: Log 1 Transformations}


https://www.desmos.com/calculator/al95rowj4h


\end{exploration}



\section{Computing}

To evaluate
\[
   \int_1^2 \frac{1}{x}\, dx
\]
with a limit of Riemann sums, we will \emph{not} partition the interval $[1,2]$ of integration into subintervals of equal length. Instead, we'll use the partition
\[
   x_i = 2^{i/n}
\]


\begin{onlineOnly}
    \begin{center}
\desmos{rbcbxdrvcp}{450}{600}  
\end{center}
\end{onlineOnly}

\href{https://www.desmos.com/calculator/rbcbxdrvcp}{152: Log 1}

\end{document}