\documentclass{ximera}
\title{Arclength of Curves in the Plane}

\newcommand{\pskip}{\vskip 0.1 in}

\begin{document}
\begin{abstract}
Arclength.
\end{abstract}
\maketitle

\section{The Ideas}

We'll speak of our curves as trails and assume they lie in a vertical plane (they don't twist around a mountain like a real trail usually does).

The constant case for computing the length of a trail is when the trail makes a constant angle $\phi$ with the horizontal. This means that the trail  is a line segment and looks just like a ramp.

In this case, if the endpoints of the ramp have coordinates $(x_1, y_1)$ and $(x_2, y_2)$, the ramp has length
\[
   \Delta s = \sqrt{(\Delta x)^2 + (\Delta y)^2} ,
\]
where
\[
  \Delta x = x_2 - x_1
\]
and
\[
 \Delta y = y_2 - y_1 .
\]

It can be helpful to express the length in terms of the inclination angle $\phi$ of the ramp. Here we have two possibilities assuming $\Delta s \geq 0$. They are
\[
     \Delta s = \left| \Delta x  \right| \answer{\sec\phi}
\] 
or
\[
    \Delta s = \left| \Delta y  \right| \answer{\csc\phi} .
\]

For a general trail, where the inclination angle is not necessarily constant, the Pythagorean theorem tells us the differential arclength is
\[
  ds = \sqrt{(dx)^2+(dy)^2} .
\]
We can write this in terms of the inclination angle $\phi$ (that now can vary along the curve) as
\[
  ds = \sec\phi \, dx
\]
or
\[
 ds = \csc\phi \, dy .
\]

When the equation of a trial is expressed as a function $y=f(x)$, we usually write
\begin{align*}
   ds  &= \sqrt{(dx)^2+(dy)^2} \\
         &= \sqrt{1+\left( \frac{dy}{dx} \right)^2}\,  |dx| \\
        &= \sqrt{1+\left( \frac{dy}{dx} \right)^2}\,  dx
\end{align*}
if $dx>0$.

I like to think of 
\[
  \sqrt{1+\left( \frac{dy}{dx} \right)^2} \geq 1,
\]
as a dimensionless scaling factor that converts a differential change $dx$ along the horizontal to a differential arclength along the trial.

\begin{question} \label{QoERer3439}
\begin{enumerate}
\item Express the derivative $dy/dx$ in terms of the inclination angle $\phi$.

\item Express the scaling factor $\sqrt{1+\left( \frac{dy}{dx} \right)^2}$ in terms of $\phi$.
\end{enumerate}
\end{question}



Finally, the length of the trail between points $(a,b)$ and $(c,d)$ (with $c>a$) is the sum 
\[
   \int_a^c \sqrt{1+\left( \frac{dy}{dx} \right)^2}\,  dx
\]
of these differential arclengths.


\section{Circles}

\begin{example} \label{Ex:Ld3214v}
Let $a>0$ be a constant with units of meters.

\begin{enumerate}
\item Find a function
\[
     s = f(w) , -a \leq w \leq a , 
\] 
that expresses the distance from a point $(w,f(w))$ on the semicircle
\[
    y = f(x) = \sqrt{a^2 - x^2}
\] 
to the point $(a,0)$. Measure the distance along the semicircle.

\begin{onlineOnly}
    \begin{center}
\desmos{rmeuvbnjds}{900}{600}
\end{center}
\end{onlineOnly}

\href{https://www.desmos.com/calculator/rmeuvbnjds}{152: ArcLength Semicircle}



\end{enumerate}

\end{example}

\section{Parameterizing a Trail by Arclength}


\section{Trail Profiles}


\end{document}