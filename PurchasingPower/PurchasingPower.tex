\documentclass{ximera}
\title{Purchasing Power}

\newcommand{\pskip}{\vskip 0.1 in}

\begin{document}
\begin{abstract}
How a small change in price affects the quantity of an item you can buy.
\end{abstract}
\maketitle


\section{Purchasing Power}

\begin{question} \label{Qdgvvbrtghgygrert}
This problem investigates how a small change in the price of apples would affect the number of pounds we can buy with $\$10$.

\begin{enumerate}

\item We'll start by finding a function
\[
       n = f(p) \, , \, p>0 ,
\]
that expresses the number of pounds of apples we can buy with $\$10$ in terms of the price (measured in dollars/pound).

The function is
\[
       n = f(p) = \answer{10/p} ,\, p>0.
\]

\item Would you expect the derivative $dn/dp$ to be postive, negative, or zero? Explain.

\item Use limits to find an expression for the derivative
\[
    \frac{dn}{dp} \Big|_{p=a}.
\]

The derivative is (click the Hint tab above for help)

\begin{hint}
\begin{align*}
          \frac{dn}{dp} \Big|_{p=a} &= \lim_{\answer{p\to a}} \frac{1}{p-a} \left( f(p)-f(a) \right)  \\
                                                 &=  \lim_{\answer{p\to a}} \frac{1}{p-a} \left(\frac{10}{p} - \frac{10}{a} \right)  \\
                                                 &=  \lim_{\answer{p\to a}} \frac{10}{p-a} \left(\frac{\answer{a-p}}{pa} \right)  \\
                                                  &=  \lim_{\answer{p\to a}} \frac{\answer{-10}}{pa} \\
                                                  & = \answer{\frac{-10}{a^2}} .
\end{align*}
\end{hint}

\item Evaluate the derivative 
\[
         \frac{dn}{dp} \Big|_{p=2} . 
\]

\begin{enumerate}
\item What are the units of the above derivative?

\item What do you get by ``simplifying'' the units? Explain how simpifying the units of this derivative gives you insight into its meaning.

\end{enumerate}

\item Interpret the meaning of the above derivative. Click the arrow to the lower right for help.

\begin{expandable}

For a way to interpret the meaning of the derivative 
\[
         \frac{dn}{dp} \Big|_{p=2} . 
\]
we can ask ourselves what happens to the number of pounds of apples we can buy with $\$10$ if the price changes by a small amount from $\$2$/lb. For this, let $\Delta p$ be a small change in price from $\$2$/lb (measured in dollars/lb) and let $\Delta n$ be the corresponding change in the number of pounds we can buy with $\$10$.

For $\Delta p \sim 0$, we have
\[
    \answer{-2.5} =  \frac{dn}{dp} \Big|_{p=2} \sim \frac{\Delta n}{\Delta p} 
\]
and (type ``Delta'' for $\Delta$)
\[
     \Delta n \sim   \answer{-2.5 \Delta p}.
\]
This tells us that if the price increases from $\$2$/lb to say $\$2.10$/lb, then 
\[
   \Delta n \sim \left( \answer{-2.5} \frac{\text{lb}}{\$/\text{lb}} \right) \left( \answer{0.10} \frac{\$}{\text{lb}}  \right) =                        \answer{-0.25}\text{ lbs}.
\]
So if the price increases from $\$2$/lb to say $\$2.10$/lb, we can buy about $\answer{0.25}$ fewer pounds of apples with ten dollars. 

\end{expandable}

\item Next we'll use the worksheet below to visualize the approximate change
\begin{align*}
  \Delta n     & \sim  \left( \frac{dn}{dp}\Big|_{p=2} \right)\Delta p \\
                   &=  \left( -2.5\frac{\text{lb}}{\$/\text{lb}} \right) \left( 0.10 \frac{\$}{\text{lb}}  \right)  \\
                   &=   -0.25\text{ lbs}.
\end{align*}
in the number of pounds of apples we can buy with $\$10$ as the price increases from $\$2$/lb to $\$2.10$/lb. 



\begin{onlineOnly}
    \begin{center}
\desmos{qw7wislq0c}{450}{600}  
\end{center}
\end{onlineOnly}

\href{https://www.desmos.com/calculator/qw7wislq0c}{151: Apples}

\begin{enumerate}
\item To get started, find an equation (in point-slope form) of the tangent line to the curve $n=f(p)$ at the point $P$ with coordinates $(2,5)$. The tangent line has a slope equal to the derivative 
\[
   \frac{dn}{dp} \Big|_{p=2} .
\]
So its equation is 
\[
      n = 5 + \answer{-2.5(p-2)} .
\]

\item Enter your equation of the tangent line on Line 17 of the desmos worksheet.

\item Activate the folders \emph{tangent line} and \emph{linear approximation} on Lines 18 and 22. 

\item Explain why the difference in the $n$-coordinates of points $P$ and $R^\prime$ (ie. $n$-coordinate of $R^\prime$ minus $n$-coordinate of $P$) is equal to our approximation of $\Delta n$ above when $Q$ as coordinates $(2.10,f(2.10))$.

\item Drag Slider $v$ on Line 2 to make $Q$ approach $P$. What so  you think happens to the ratio of $\Delta n$ to our approximation of $\Delta n$ as $v\to 2$?
\end{enumerate}


\item We can get a better understanding of the changes 
\[
      \Delta p = 2.1 - 2 = 0.10 \text{ dollars/pound}
\]
and 
\[
   \Delta n = f(2.1)-f(2) \sim 0.25 \text { pounds}
\]
by thinking about relative instead of absolute change.

An increase in price from $\$2$/lb to $\$2.10$/lb is a relative change of
\[
    \frac{\Delta p}{p} = \frac{\$0.1/\text{lb}}{\$ 2/\text{lb}} = 0.05 = 5\% .
\]
This causes a relative change in the number of pounds we can buy of approximately
\[
    \frac{\Delta n}{n} \sim \frac{\answer{-0.25} \text{lbs}}{\answer{5}\text{lbs}} = \answer{-0.05} = \answer{-5}\%.
\]
 
\item We can get a the same result relating the two relative changes of the previous question by working in general. 

Suppose we increase the price of apples by $Q\%$, where $Q\sim 0$. What can we say about the relative change in the number of pounds of apples we can buy with $\$10$?

Well, if the change in price $\Delta p$ is near zero and $\Delta n$ is the corresponding change in the number of pounds we can buy, then 
\[
    \frac{\Delta n}{\Delta p}   \sim  \frac{dn}{dp} = \answer{-10/p^2}.
\]
So
\[
     \Delta n \sim \left( \answer{\frac{-10}{p^2}} \right) \Delta p.
\]
Dividing both sides by $n = 10/p$ tells us that
\[
        \frac{\Delta n}{n} \sim \left( \answer{\frac{-10}{p^2}} \right)\left( \frac{\Delta p}{\frac{10}{p}} \right) = -\frac{\Delta p}{\answer{p}} = -Q\%.
\]

So if we increase the price by $Q\%\sim 0$, then the number of apples we can buy decreases by approximately that same $Q\%$.

\item We can think about this relationship between the relative changes geometrically. To do this, remove your equation of the tangent line on Line 17 and activate the folder \emph{tangent line} in the worksheet above. Then drag slider $a$ on Line 4 to move point $P$. What do you notice about the ratio  $PB:PA$ of the distances $PA$ and $PB$ as $P$ moves? More on this later.

\end{enumerate}

\end{question}

\begin{question} \label{QGETGG3fg44}
Due to a printing error, the graph of the function
\[
      n = f(p)  \, , \, p>0 ,
\]
expressing the number of pounds of an item we can buy with $\$100$ in terms of the price (measured in $\$/\text{lb}$) is not shown below. All we can see is a point $A$ on the graph.

\begin{onlineOnly}
    \begin{center}
\desmos{y5nqs8fkvj}{450}{600}  
\end{center}
\end{onlineOnly}

\href{https://www.desmos.com/calculator/y5nqs8fkvj}{151: Printers Error 1}

The problem is to draw the tangent line to the graph at $A$ \emph{without} sketching the graph or doing any kind of computation. Click the arrow below for a hint.

\begin{expandable}
The key idea is to relate the slope of the tangent line at $A$ to the slope of the line $OA$ through the origin and $A$.


We know that
\[
      n = f(p)  = \frac{100}{p} \, , \, p>0 ,
\]
and
\[
 \frac{dn}{dp} =\answer{-100/p^2} .
\]

Now write this derivative in terms of both $n$ and $p$, 
\[
   \frac{dn}{dp} =\answer{-\frac{100}{p^2}} = \answer{-\frac{n}{p}} .
\]

Now relate this to the slope of line $OA$ and draw the tangent line at $A$.

\end{expandable}

\end{question}


\section{Weight in Space}

\begin{question}  \label{Q9df9tthhhg}

The weight of an object is the gravitational force that the earth exerts on the object's mass and varies with the object's height above the surface.

The function 
\[
  W = f(h) = \frac{k}{(h+4)^2} \, , \, h\geq 0,
\]
expresses the weight (in pounds) of an object in terms of its height (in thousands of miles) above the surface of the earth. We'll suppose the object weighs 200 pounds on the surface.

\begin{onlineOnly}
    \begin{center}
\desmos{zklsucctjp}{450}{600}  
\end{center}
\end{onlineOnly}

\href{https://www.desmos.com/calculator/zklsucctjp}{151: Weight in Space}


\begin{enumerate}
\item Find the value of the constant $k$. What are its units?

\item Find the average rate of change of $W$ with respect to $h$ over the interval between heights of $h$ and $v$ thousands of miles.

\item Use part (b) to find an expression for the derivative $dW/dh$.

\item Use the graph of the function $W=f(h)$ above to approximate the derivative
\[
      \frac{dW}{dh}\Big|_{h=4}.
\]
Include units

\item Evaluate the derivative
\[
      \frac{dW}{dh}\Big|_{h=4}.
\]

\item What are the units of the derivative above? Interpret it meaning.

\begin{enumerate}

\item At approximately what altitude does your weight decrease by $4$ pounds when your altitude increases by $200$ miles? Use your expression for the derivative $dW/dh$ to help. 

\item Compute the exact change in your weight over the interval you found in the previous question.
\end{enumerate}

\end{enumerate}

\end{question}

\section{Distance to the Horizon}

\begin{question} \label{Qdggbrgthghghb}
The function
\[
      s = f(h) = 1.22\sqrt{h} \, , \, 0\leq h \leq 10,000
\]
expresses the distance to the horizon (measured in miles) in terms of your alitude (measured in feet).

\begin{enumerate}
\item Find an expression for the average rate of change in the distance to the horizon with respect to altitude between altitudes $h$ feet and $v$ feet.

\item Use your expression from part (a) to find an expression for the derivative $ds/dh$.

\item Evaluate the derivative 
\[
       \frac{ds}{dh}\Big|_{h=25} .
\]

\item What are the units of the derivative above? Interpret its meaning.

\item Use the result of part (c) to approximate the distance to the horizon at an altitude of $24$ feet. Then compare this approximation with the actual distance.

\item Approximate the relative change in the distance to the horizon in terms of a small relative change in altitude.

\item Use your result from part (b) to approximate the altitude at which moving $10$ feet higher increase the distance to the horizon by $0.5$ miles.

\end{enumerate}
\end{question}




\section{Intensity of Sound}

\begin{question}  \label{Qdfdgt446666}
The \emph{intensity} of sound is measured in Watts per square meter and is a function of the distance from the source. 

Suppose for a jet taking off, this function is
\[
          I = f(r) = \frac{100}{r^2} \, , \, r\geq 10 ,
\]
where $r$ is the distance (in meters) from the jet.    %vacuum cleaner%  and  $k=10^{-4}\text{watts}\cdot m^2$ %$k=10^{-4}\text{watts}\cdot m^2$ is a constant.

\begin{enumerate}

\item Find an expression for the average rate of change of the sound intensity (in $\text{Watts}/m^2$) with respect to the distance from the source (in meters) between distances $r$ and $a$ meters.

\item Use your expression from part (a) to find an expression for the derivative 
\[
   dI/dr \Big|_{r=a} 
\]

\item Evaluate the derivative
\[
       dI/dr \Big|_{r=20} .
\]

\item What are the units of the derivative above? What is its meaning?

\item Approximate the relative change in the sound intensity in terms of a small relative change in the distance to the source.
\end{enumerate}

\end{question}

\section{A rock in Free Fall}

\begin{question}  \label{Qdfgg4t66}
A rock is dropped from a height of $100$ meters on the planet Krypton.

The function
\[
     v = f(h) \, , \, 0\leq h \leq 100 ,
\]
expresses the speed (in meters/sec) of the rock in terms of its height (in meters) above the surface.

\begin{enumerate}
\item Would you expect the derivative 
\[
  \frac{dv}{dh}\Big|_{h=36}
\]
to be postive or negative? Why?

\item Suppose
\[
     v = f(h)  = 4\sqrt{100-h} \, , \, 0\leq h \leq 100
\]
and find an expression for the average rate of change of the rock's speed (measured in meters/sec) with respect to its height (in meters) between heights of $v$ meters and $a$ meters.

\item Use your expression from part (b) to find an expression for the derivative 
\[
   dv/dh \Big|_{h=a}. 
\]

\item Evaluate the derivative
\[
       dv/dh \Big|_{h=64} .
\]

\item What are the units of the derivative above? What is its meaning?

\item ``Simplify the units of the derivative". What insight does this give you into its meaning?

\item Use part (e) to approximate the rocks's speed at a height of $61$ feet.


\end{enumerate}
\end{question}


\end{document}