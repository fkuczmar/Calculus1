\documentclass{ximera}
\title{Purchasing Power}

\newcommand{\pskip}{\vskip 0.1 in}

\begin{document}
\begin{abstract}
How a small change in price affects the quantity of an item you can buy.
\end{abstract}
\maketitle


\begin{question} \label{Qdgvvbrtghgygrert}
How would a small change in the price of apples affect the number of pounds you can buy with $\$10$?

\begin{enumerate}

\item Start by finding a function
\[
       n = f(p) \, , \, p>0 ,
\]
that expresses the number of pounds of apples you can buy with $\$10$ in terms of the price (measured in dollars/pound).

The function is
\[
       n = f(p) = \answer{10/p} ,\, p>0.
\]

\item Would you expect the derivative $dn/dp$ to be postive, negative, or zero? Explain.

\item Use limits to find an expression for the derivative
\[
    \frac{dn}{dp} \Big|_{p=a}.
\]

The derivative is (click the Hint tab above for help)

\begin{hint}
\begin{align*}
          \frac{dn}{dp} \Big|_{p=a} &= \lim_{\answer{p\to a}} \frac{1}{p-a} \left( f(p)-f(a) \right)  \\
                                                 &=  \lim_{\answer{p\to a}} \frac{1}{p-a} \left(\frac{10}{p} - \frac{10}{a} \right)  \\
                                                 &=  \lim_{\answer{p\to a}} \frac{10}{p-a} \left(\frac{\answer{a-p}}{pa} \right)  \\
                                                  &=  \lim_{\answer{p\to a}} \frac{\answer{-10}}{pa} \\
                                                  & = \answer{\frac{-10}{a^2}} .
\end{align*}
\end{hint}

\item Evaluate the derivative 
\[
         \frac{dn}{dp} \Big|_{p=2} . 
\]

\item What are the units of the above derivative?

\item Interpret the meaning of the above derivative. Click the arrow to the lower right for help.

\begin{expandable}

For a way to interpret the meaning of the derivative 
\[
         \frac{dn}{dp} \Big|_{p=2} . 
\]
we can ask ourselves what happens to the number of pounds of apples we can buy with $\$10$ if the price changes by a small amount from $\$2$/lb. For this, let $\Delta p$ be a small change in price from $\$2$/lb(measured in dollars/lb) and let $\Delta n$ be the corresponding change in the number of pounds we can buy with $\$10$.

For $\Delta p \sim 0$, we have
\[
    2.5 =  \frac{dn}{dp} \Big|_{p=2} \sim \frac{\Delta n}{\Delta p} 
\]
and
\[
     \Delta n \sim   \answer{2.5 \Delta p}.
\]
This tells us that if the price increases from $\$2$/lb to say $\$2.10$/lb, then
\[
   \Delta n \sim \answer{-0.25}.
\]
So if the price increases from $\$2$/lb to say $\$2.10$/lb, we can buy about $0.25$ fewer pounds of apples with ten dollars. 

\end{expandable}

\end{enumerate}



\end{question}


\end{document}