\documentclass{ximera}
\title{Purchasing Power}

\newcommand{\pskip}{\vskip 0.1 in}

\begin{document}
\begin{abstract}
How a small change in price affects the quantity of an item you can buy.
\end{abstract}
\maketitle


\section{Purchasing Power}

\begin{question} \label{Qdgvvbrtghgygrert}
This problem investigates how a small change in the price of apples would affect the number of pounds we can buy with $\$10$.

\begin{enumerate}

\item We'll start by finding a function
\[
       n = f(p) \, , \, p>0 ,
\]
that expresses the number of pounds of apples we can buy with $\$10$ in terms of the price (measured in dollars/pound).

The function is
\[
       n = f(p) = \answer{10/p} ,\, p>0.
\]

\item Would you expect the derivative $dn/dp$ to be postive, negative, or zero? Explain.

\item Use limits to find an expression for the derivative
\[
    \frac{dn}{dp} \Big|_{p=a}.
\]

The derivative is (click the Hint tab above for help)

\begin{hint}
\begin{align*}
          \frac{dn}{dp} \Big|_{p=a} &= \lim_{\answer{p\to a}} \frac{1}{p-a} \left( f(p)-f(a) \right)  \\
                                                 &=  \lim_{\answer{p\to a}} \frac{1}{p-a} \left(\frac{10}{p} - \frac{10}{a} \right)  \\
                                                 &=  \lim_{\answer{p\to a}} \frac{10}{p-a} \left(\frac{\answer{a-p}}{pa} \right)  \\
                                                  &=  \lim_{\answer{p\to a}} \frac{\answer{-10}}{pa} \\
                                                  & = \answer{\frac{-10}{a^2}} .
\end{align*}
\end{hint}

\item Evaluate the derivative 
\[
         \frac{dn}{dp} \Big|_{p=2} . 
\]

\item What are the units of the above derivative?

\item Interpret the meaning of the above derivative. Click the arrow to the lower right for help.

\begin{expandable}

For a way to interpret the meaning of the derivative 
\[
         \frac{dn}{dp} \Big|_{p=2} . 
\]
we can ask ourselves what happens to the number of pounds of apples we can buy with $\$10$ if the price changes by a small amount from $\$2$/lb. For this, let $\Delta p$ be a small change in price from $\$2$/lb (measured in dollars/lb) and let $\Delta n$ be the corresponding change in the number of pounds we can buy with $\$10$.

For $\Delta p \sim 0$, we have
\[
    \answer{-2.5} =  \frac{dn}{dp} \Big|_{p=2} \sim \frac{\Delta n}{\Delta p} 
\]
and (type ``Delta'' for $\Delta$)
\[
     \Delta n \sim   \answer{-2.5 \Delta p}.
\]
This tells us that if the price increases from $\$2$/lb to say $\$2.10$/lb, then 
\[
   \Delta n \sim \left( \answer{-2.5} \frac{\text{lb}}{\$/\text{lb}} \right) \left( \answer{0.10} \frac{\$}{\text{lb}}  \right) =                        \answer{-0.25}\text{ lbs}.
\]
So if the price increases from $\$2$/lb to say $\$2.10$/lb, we can buy about $\answer{0.25}$ fewer pounds of apples with ten dollars. 

\begin{onlineOnly}
    \begin{center}
\desmos{qw7wislq0c}{450}{600}  
\end{center}
\end{onlineOnly}

\href{https://www.desmos.com/calculator/qw7wislq0c}{151: Apples}


\end{expandable}

\item We can get a better understanding of these changes by thinking about relative instead of absolute change.

An increase in price from $\$2$/lb to $\$2.10$/lb is a relative change of
\[
    \frac{\Delta p}{p} = \frac{\$0.1/\text{lb}}{\$ 2/\text{lb}} = 0.05 = 5\% .
\]
This causes a relative change in the number of pounds we can buy of approximately
\[
    \frac{\Delta n}{n} \sim \frac{\answer{-0.25} \text{lbs}}{\answer{5}\text{lbs}} = \answer{-0.05} = \answer{-5}\%.
\]
 
\item We can get a the same result relating the two relative changes of the previous question by working in general. 

Suppose we increase the price of apples by $Q\%$, where $Q\sim 0$. What can we say about the relative change in the number of apples we can buy with $\$10$?

Well, if $\Delta p \sim 0$,
\[
    \frac{\Delta n}{\Delta p}   \sim  \frac{dn}{dp} = \answer{-10/p^2},
\]
and
\[
     \Delta n \sim \left( \answer{\frac{-10}{p^2}} \right) \Delta p.
\]
Dividing both sides by $n = 10/p$ tells us that
\[
        \frac{\Delta n}{n} \sim \left( \answer{\frac{-10}{p^2}} \right)\left( \frac{\Delta p}{\frac{10}{p}} \right) = -\frac{\Delta p}{\answer{p}} = -Q\%.
\]

So if we increase the price by $Q\%\sim 0$, then the number of apples we can buy decreases by approximately that same $Q\%$.

\end{enumerate}

\end{question}

\begin{question} \label{QGETGG3fg44}
Due to a printing error, the graph of the function
\[
      n = f(p)  \, , \, p>0 ,
\]
expressing the number of pounds of an item we can buy with $\$100$ in terms of the price (measured in $\$/\text{lb}$) is not shown below. All that we see is a point $A$ on the graph.

\begin{onlineOnly}
    \begin{center}
\desmos{wlr2dignce}{450}{600}  
\end{center}
\end{onlineOnly}

\href{https://www.desmos.com/calculator/wlr2dignce}{151: Printers Error 1}

The problem is to draw the tangent line to the graph at $A$ \emph{without} sketching the graph or doing any kind of computation. Click the arrow below for a hint.

\begin{expandable}
We know that
\[
      n = f(p)  = \frac{100}{p} \, , \, p>0 ,
\]
and
\[
 \frac{dn}{dp} =\answer{-100/p^2} .
\]

The idea is to write this derivative in terms of both $n$ and $p$, 
\[
   \frac{dn}{dp} =\answer{-\frac{100}{p^2}} = -\frac{n}{p} .
\]

Now interpret the ratio $n/p$ as the slope of a line in the graph above. Then use that interpretation to draw the tangent line at $A$.

\end{expandable}

\end{question}

\section{Intensity of Sound}

\begin{question}  \label{Qdfdgt446666}
The \emph{intensity} of sound is measured in Watts per square meter and is a function of the distance from the source. For a typical vacuum cleaner, this function is
\[
          I = f(r) = \frac{k}{r^2} \, , \, r\geq 1 ,
\]
where $r$ is the distance (in meters) from the vacuum cleaner and $k=10^{-4}\text{watts}\cdot m^2$ is a constant.

\begin{enumerate}
\item stop here
\end{enumerate}

\end{question}


\end{document}