\documentclass{ximera}
\title{Volumes of Solids in Three-Dimensions}

\newcommand{\pskip}{\vskip 0.1 in}

\begin{document}
\begin{abstract}
Volume.
\end{abstract}
\maketitle

\section{Square Cross Sections}

\begin{question} \label{QPkdf9fM}
The base of a solid is a disk of radius $a$ meters. Parallel cross-sections perpendicular to the base are squares. 

\begin{enumerate}
\item Find the volume of the solid.
\item Compare the solid's volume with the volume of the surrounding cylinder.
\end{enumerate}

\begin{onlineOnly}
    \begin{center}
\desmosThreeD{q5jo9tip5q}{900}{600}
\end{center}
\end{onlineOnly}

\href{https://www.desmos.com/3d/q5jo9tip5q}{152: Circular Base Square Cross Sections}



\begin{explanation}

\item It is not necessary to visualize the solid to compute its volume. Instead, the best approach is to sketch its base.

\begin{onlineOnly}
    \begin{center}
\desmos{mnjevsoffp}{900}{600}
\end{center}
\end{onlineOnly}

\href{ https://www.desmos.com/calculator/mnjevsoffp}{152: Solid Base 1}

We can image the square cross-sections coming out of the page, but the key is to start by computing the area of the cross-section. The square cross-section has side length $s=2y$ and area
\[ 
 A = s^2 = 4y^2 .
\]
The differential slice of the solid has width $dx$ and differential volume
\[
    dV = A \, dx = 4y^2 \, dx .
\]

Before integrating, we'll need to eliminate the mixed variables and express the differential volume in terms of $x$ and $dx$. To do this we need to use the fact that the base is bounded by the circle with equation
\[
      x^2 + y^2 = a^2
\]
in the above coordinate system. So
\[
      y^2 = a^2 - x^2
\]
and the differential slice has volume
\begin{align*}
  dV & = 4y^2 \, dx \\
       &= 4(a^2-x^2)\, dx .
\end{align*}

We add these differential volumes to get the volume of the solid,
\begin{align*}
   V &= \int_{-a}^a 4(a^2-x^2)\, dx \\
      &= 8 \int_0^a (a^2-x^2)\, dx \\
      &= 8 \left( a^2x - \frac{1}{3}x^3 \right) \Big|_{x=0}^{x=a} \\
      &= \frac{16}{3}a^3 .
\end{align*}


\item Activate the folder in Line 1 above to see the cylinder that surrounds the solid. 

This cylinder has volume
\[
 W = (2  a) (\pi a^2) = 2\pi a^3  \sim 6.28 a^3.
\]
So the solid occupies
\[
  \frac{V}{W} = \frac{16}{6\pi} \sim 84.9\%
\]
the volume of the cylinder.
\end{explanation}
\end{question}

\begin{question} \label{QLdfer3r}
\begin{enumerate}
\item Solve Question 1 if the cross-sections are semi-circles (with their diameters in the base) instead of squares.

\item Describe the solid in this case. 

\item What fraction of the cylinder's volume below is occupied by the tennis balls?

\begin{onlineOnly}
    \begin{center}
\desmos{5ip866uqox}{900}{600}
\end{center}
\end{onlineOnly}

\href{https://www.desmos.com/calculator/5ip866uqox}{152: Tennis Balls}

\end{enumerate}
\end{question}


\begin{question} \label{QoFDerf323f}
Let $a,b>0$ be constants.

The base of a solid is the region bounded by the ellipse
\[
    \frac{x^2}{a^2} + \frac{y^2}{b^2} = 1.
\]
Cross-sections perpendicular to the $x$-axis are squares. 

\begin{onlineOnly}
    \begin{center}
\desmosThreeD{zbcupsnbjv}{900}{600}
\end{center}
\end{onlineOnly}

\href{https://www.desmos.com/3d/zbcupsnbjv}{152: Elliptical Base Square Cross Sections}


\begin{enumerate}
\item Find the volume of the solid.

\item Check that your expression for the volume is dimensionally correct.

\item Compare the solid's volume with the volume of the surrounding cylinder.

\end{enumerate} 

\end{question}


\begin{question} \label{Q88rew3efds}

The base of a tent is a regular hexagon with side length $a$ meters. Three poles bent into semicircles run between opposite vertices. Cross sections parallel to the base are regular hexagons having their vertices on the poles.

\begin{onlineOnly}
    \begin{center}
\desmosThreeD{wczn2fml4l}{900}{600}
\end{center}
\end{onlineOnly}

\href{https://www.desmos.com/3d/wczn2fml4l}{152: Tent}

\begin{enumerate}
\item Find an expression for the volume of the tent.

\item Compare the volume with the volume of the surrounding cylinder.

\item The tent is partially filled with water that takes up half the tent's volume. What is the approximate depth of the water?

\end{enumerate}


\end{question}




\begin{question} \label{QLkdfmmdf}
The base of a solid is the parabolic region below with base $b=2a$ and height $h$. Cross sections perpendicular to the $y$-axis are rectangles half as tall as they are wide

\begin{onlineOnly}
    \begin{center}
\desmos{mbxrsmdho8}{900}{600}
\end{center}
\end{onlineOnly}

\href{ https://www.desmos.com/calculator/mbxrsmdho8}{152: Parabola 23}

\begin{onlineOnly}
    \begin{center}
\desmosThreeD{gdebhnchrz}{900}{600}
\end{center}
\end{onlineOnly}

\href{https://www.desmos.com/3d/gdebhnchrz}{152: Parabolic Base}


\begin{enumerate}

\item Find an expression for the volume of the solid. 

\item Check that your expression is dimensionally correct.

\item Compare the solid's volume with the volume of the surrounding cylinder.


\end{enumerate}
\end{question}

\section{The Hoof of Archimedes}

\begin{question} \label{QKDFMReR}
A solid circular cylinder of radius $a$ and height $h$ is sliced by a plane passing through a diameter of the base and intersecting the top of the cylinder in exactly one point. The plane cuts the cylinder into two pieces. Our goal here is to find an expression in $h$ and $a$ for the volume of the lower (smaller) piece.

\begin{onlineOnly}
    \begin{center}
\desmosThreeD{cnspcoiggo}{900}{600}
\end{center}
\end{onlineOnly}

\href{https://www.desmos.com/3d/cnspcoiggo}{152: Archimedes Hoof 2}

We'll determine the volume of the lower piece (the hoof of Archimedes) by taking slices perpendicular to the coordinate axes . Turn off the folder in Line 2 and activate the appropriate folder in Lines 5, 13, or 22 if you need help visualizing the cross sections.

Slice the solid 
\begin{enumerate}
\item perpendicular to the $x$-axis (activate the folder in Line 5),

\item perpendicular to the $y$-axis (activate the folder in Line 13), and

\item perpendicular to the $z$-axis (activate the folder in Line 22).
\end{enumerate} 

\begin{enumerate}
\item Check that your expression for the volume is dimensionally correct.

\item What fraction of the original cylinder does the hoof occupy? Activate the folder in Line 33 to see the cylinder.
\end{enumerate}


\end{question}

\section{Archimedes' Bicylinder}
Two solid right circular cylinders of radius $a$ cm intersect orthogonally as shown below (the cylinders' axes of symmetry also intersect). We'll determine the volume of the solid common to both (Archimedes' bicylinder).

\begin{onlineOnly}
    \begin{center}
\desmosThreeD{wzzhod4ya6}{900}{600}
\end{center}
\end{onlineOnly}

\href{https://www.desmos.com/3d/wzzhod4ya6}{152: Bicylinder}

\begin{enumerate}
\item Find an expression for the volume of the bicylinder and check that it is dimensionally correct.

\item Compare the volume with its surrounding cyinders to make sure your answer is reasonable.

\item Compare the volume of the bicylinder with the volume of the solid in Question 1 above. Then drag the slider $w_2$ in the worksheet below and describe what you see.

\begin{onlineOnly}
    \begin{center}
\desmosThreeD{cwcjqkjndw}{900}{600}
\end{center}
\end{onlineOnly}

\href{https://www.desmos.com/3d/cwcjqkjndw}{152: Square Cross-sections to Bicylinder}





\end{enumerate}

\end{document}