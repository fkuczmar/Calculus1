\documentclass{ximera}
\title{Volumes of Solids in Three-Dimensions}

\newcommand{\pskip}{\vskip 0.1 in}

\begin{document}
\begin{abstract}
Volume.
\end{abstract}
\maketitle

\section{Square Cross Sections}

\begin{question} \label{QPkdf9fM}
The base of a solid is a disk of radius $a$ meters. Parallel cross-sections perpendicular to the base are squares. 

\begin{enumerate}
\item Find the area of the solid.
\item Compare the solid's volume with the volume of the surrounding cylinder.
\end{enumerate}

\begin{onlineOnly}
    \begin{center}
\desmosThreeD{q5jo9tip5q}{900}{600}
\end{center}
\end{onlineOnly}

\href{https://www.desmos.com/3d/q5jo9tip5q}{152: Circular Base Square Cross Sections}



\begin{explanation}

\item It is not necessary to visualize the solid to compute its volume. Instead, the best approach is to sketch its base.

\begin{onlineOnly}
    \begin{center}
\desmos{mnjevsoffp}{900}{600}
\end{center}
\end{onlineOnly}

\href{ https://www.desmos.com/calculator/mnjevsoffp}{152: Solid Base 1}

We can image the square cross-sections coming out of the page, but the key is to start by computing the area of the cross-section. The square cross-section has side length $s=2y$ and area
\[ 
 A = s^2 = 4y^2 .
\]
The differential slice of the solid has width $dx$ and differential volume
\[
    dV = A \, dx = 4y^2 \, dx .
\]

Before integrating, we'll need to get rid of the mixed variables and express the differential volume in terms of $x$ and $dx$. To do this we need to use the fact that the base is bounded by the circle with equation
\[
      x^2 + y^2 = a^2
\]
in the above coordinate system. So
\[
      y^2 = a^2 - x^2
\]
and the differential slice has volume
\begin{align*}
  dV & = 4y^2 \, dx \\
       &= 4(a^2-x^2)\, dx .
\end{align*}

We add these differential volumes to get the volume of the solid,
\begin{align*}
   V &= \int_{-a}^a 4(a^2-x^2)\, dx \\
      &= 8 \int_0^a (a^2-x^2)\, dx \\
      &= 8 \left( a^2x - \frac{1}{3}x^3 \right) \Big|_{x=0}^{x=a} \\
      &= \frac{16}{3}a^3 .
\end{align*}


\item Activate the folder in Line 1 above to see the cylinder that surrounds the solid. 

This cylinder has volume
\[
 W = (2  a) (\pi a^2) = 2\pi a^3  \sim 6.28 a^3.
\]
So the solid occupies
\[
  \frac{V}{W} = \frac{16}{6\pi} \sim 84.9\%
\]
the volume of the cylinder.
\end{explanation}
\end{question}


\begin{question} \label{QoFDerf323f}
Let $a,b>0$ be constants.

The base of a solid is the region bounded by the ellipse
\[
    \frac{x^2}{a^2} + \frac{y^2}{b^2} = 1.
\]
Parallel cross-sections perpendicular to the base are squares. 

\begin{onlineOnly}
    \begin{center}
\desmosThreeD{zbcupsnbjv}{900}{600}
\end{center}
\end{onlineOnly}

\href{https://www.desmos.com/3d/zbcupsnbjv}{152: Elliptical Base Square Cross Sections}


\begin{enumerate}
\item Find the volume of the solid.

\item Check that your expression for the volume is dimensionally correct.

\item Compare the solid's volume with the volume of the surrounding cylinder.

\end{enumerate} 

\end{question}


\begin{question} \label{Q88rew3efds}

The base of a tent is a regular hexagon with side length $a$ meters. Three poles bent into semicircles run between opposite vertices. Cross sections parallel to the base are regular hexagons.

\begin{onlineOnly}
    \begin{center}
\desmosThreeD{wczn2fml4l}{900}{600}
\end{center}
\end{onlineOnly}

\href{https://www.desmos.com/3d/wczn2fml4l}{152: Tent}

\begin{enumerate}
\item Find an expression for the volume of the tent.

\item Compare the volume with the volume of the surrounding cylinder.

\item The tent is partially filled with water that takes up half the tent's volume. What is the approximate depth of the water?

\end{enumerate}


\end{question}




\begin{question} \label{QLkdfmmdf}
The base of a solid is the parabolic region below with base $b=2a$ and height $h$. Cross sections perpendicular to the $y$-axis are squares.



\begin{onlineOnly}
    \begin{center}
\desmos{mbxrsmdho8}{900}{600}
\end{center}
\end{onlineOnly}

\href{ https://www.desmos.com/calculator/mbxrsmdho8}{152: Parabola 23}
\begin{enumerate}

\item Find an expression for the volume of the solid. 

\item Check that your expression is dimensionally correct.

\item Compare the solid's volume with the volume of the surrounding cylinder.


\end{enumerate}
\end{question}



\end{document}