\documentclass{ximera}
\title{Differentiable Functions}

\newcommand{\pskip}{\vskip 0.1 in}

\begin{document}
\begin{abstract}
An introduction to what it means for a function to be differentiable.
\end{abstract}
\maketitle

\section{Differentiability}
Zoom in closer and closer near a point $(a,f(a))$ on the graph of a common function $f$ and you'll most likely notice that the graph looks more and more like the graph of a linear function (ie. like a non-vertical line). If so, we say the function $y=f(x)$ is \emph{differentiable} at $x=a$. And the \emph{derivative} of the function, evaluated at $x=a$, written as
\[
     \frac{dy}{dx}\Big|_{x=a},
\]
is just the slope of that line or the rate of change of the linear function.

\begin{exploration}  \label{Exp:34gt4trt}
Use the graph of the function $y=f(x)$ below to answer the following questions.

\begin{onlineOnly}
    \begin{center}
\desmos{ojdj4r3r9v}{450}{600}  
\end{center}
\end{onlineOnly}

\href{https://www.desmos.com/calculator/ojdj4r3r9v}{151: Diff0}

\begin{enumerate}
\item Zoom in toward the point $A$ to determine if the function $y=f(x)$ is differentiable at $x=1$. If so, approximate the derivative
\[
      \frac{dy}{dx}\Big|_{x=1} = \frac{d}{dx}\left( f(x) \right) \Big|_{x=1}.
\]
The derivative is
\[
       \frac{dy}{dx}\Big|_{x=1} = \answer{3} .
\]

\item Zoom in toward the point $B$ to determine if the function $y=f(x)$ is differentiable at $x=0$. If so, approximate the derivative
\[
      \frac{dy}{dx}\Big|_{x=1} = \frac{d}{dx}\left( f(x) \right) \Big|_{x=0}.
\]
The derivative is
\[
       \frac{dy}{dx}\Big|_{x=0} = \answer{-1} .
\]
\end{enumerate}


\end{exploration}


\begin{exploration}  \label{Exp9485rf5r}

Use the graph of the function 
\[
    y = f(x) = |x|
\]
below to evaluate the following derivatives if possible. Explain your reasoning.

\begin{onlineOnly}
    \begin{center}
\desmos{us2fruzbra}{450}{600}  
\end{center}
\end{onlineOnly}

\href{https://www.desmos.com/calculator/us2fruzbra}{151: Diff1}

\begin{enumerate}
\item 
\[
 \frac{d (|x|)}{dx}\Big|_{x=2} = \answer{1}
\]

\item 
\[
 \frac{d (|x|)}{dx}\Big|_{x=0} = 
\]

\item 
\[
 \frac{d (|x|)}{dx}\Big|_{x=-3} = \answer{-1}
\]
\end{enumerate}
\end{exploration}


\begin{exploration} \label{Exp4kfgk4g4}
Use the graph of the function $y=f(x)$ below to determine if the function is differentiable at $x=0$. Explain your reasoning.

\begin{onlineOnly}
    \begin{center}
\desmos{ov8qt938ot}{450}{600}  
\end{center}
\end{onlineOnly}

\href{https://www.desmos.com/calculator/ov8qt938ot}{151: Diff2}
\end{exploration}


\begin{exploration} \label{Exp4dfd44444g4}
\begin{enumerate}

\item Use the graph of the function $y=f(x)$ below to determine if the function is differentiable at $x=0.2$. If so, approximate the derivative
\[
      \frac{dy}{dx}\Big|_{x=0.2} = \frac{d}{dx}\left( f(x) \right) \Big|_{x=0.2}.
\]
Explain your reasoning.

\item Use the graph of the function $y=f(x)$ below to determine if the function is differentiable at $x=0$. If so, approximate the derivative
\[
      \frac{dy}{dx}\Big|_{x=0} = \frac{d}{dx}\left( f(x) \right) \Big|_{x=0}.
\]
Explain your reasoning.


\end{enumerate}

\begin{onlineOnly}
    \begin{center}
\desmos{tvwtbx9hco}{450}{600}  
\end{center}
\end{onlineOnly}

\href{https://www.desmos.com/calculator/tvwtbx9hco}{151: Not Differentiable}
\end{exploration}


\begin{exploration} \label{Expt666776665221}
Use the graph of $y=f(x)$ below to determine whether each of the following derivatives are negative, positive, or zero. Explain your reasoning.

\begin{enumerate}
\item 
\[
\frac{d}{dx}\left( f(x) \right)\Big|_{x=0.5}
\]

\item 
\[
\frac{d}{dx}\left( f(x) \right)\Big|_{x=1}
\]

\item 
\[
\frac{d}{dx}\left( f(x) \right)\Big|_{x=1.5}
\]

\item 
\[
\frac{d}{dx}\left( f(x) \right)\Big|_{x=2.29}
\]
\end{enumerate}

\begin{onlineOnly}
    \begin{center}
\desmos{ks2yui6ofs}{450}{600}  
\end{center}
\end{onlineOnly}

\href{https://www.desmos.com/calculator/ks2yui6ofs}{151:Diff 7}


\end{exploration}




\begin{question}  \label{Q64566y565454}
\begin{enumerate}
\item Summarize your understanding of the main ideas of this section.

\item What questions do you have about this section?
\end{enumerate}
\begin{freeResponse}
\end{freeResponse}
\end{question}


\section{Using Algebra to Compute Derivatives}

\begin{exploration}   \label{Exp9039000030}
The aim of this problem is to use numerical methods to approximate and algebra to evaluate the derivative
\[
   \frac{d}{dx}\left( x^3 \right) \Big|_{x=1}.
\]

\begin{enumerate}

\item First use the graph of the function $y=f(x)=x^3$ below to approximate or guess the value of the above derivative by zooming in on the appropriate point.
\[
      \frac{d}{dx}\left( x^3 \right)\Big|_{x=1} = \answer{3} .
\]


\begin{onlineOnly}
    \begin{center}
\desmos{6cqyorfc2f}{450}{600}  
\end{center}
\end{onlineOnly}

\href{https://www.desmos.com/calculator/6cqyorfc2f}{151: Cubing Function}

\item The idea to compute the derivative algebraically is this: Fix the point $P(1,1)$ on the graph of $y=f(x)=x^3$. Then choose a variable point $Q$ on the graph, different from $P$, with coordinates $(v,v^3)$. When $Q$ is sufficiently close to $P$, the line $PQ$ approximates the curve $y=x^3$ near $P(1,1)$ and the slope of this line approximates the derivative above. 

\begin{enumerate}
\item Our first step is to find an expression for the slope of line $PQ$ as a function of $v$. The slope is
\begin{align*}
        m(v) &= \frac{\Delta y}{\Delta v}  \\
                & = \frac{f(v) - f(1)}{\answer{v-1}} \\
                & = \frac{\answer{v^3-1}}{v-1} \\
\end{align*}
all assuming $v\neq \answer{1}$.

\item Now we'll use the slope (or average rate of change) function $m(x)$ to create a table of slopes for the lines $PQ$. Reveal the contents of the Table folder in Line 1 of the worksheet above by clicking the Right Arrow just to the left of ``Table''. 

\begin{enumerate}
\item Add a few entries to the table to get \emph{better} approximations to the above derivative. 

\item The slopes $m(v)$ should appear should appear to approach some number as $v$ approaches $1$. What is that number?

\item This suggests that 
\[
   \frac{d}{dx}\left( x^3 \right)\Big|_{x=1}  = \lim_{v\to 1} \frac{v^3-1}{v-1} = \answer{3} .
\] 
\end{enumerate}

\item Another approach to approximating the derivative is to graph the slope function
\[
    m(v) = \frac{\answer{v^3-1}}{v-1} \, , \, v\neq \answer{1}.
\]
Activate the Average Rate of Change folder on Line 11 by clicking the camera icon to the left of the line to see the graph of this function. Activate also the Table folder in Line 1.

\begin{enumerate}
\item Drag the slider $v$ on Line 4. Describe the relationship between the graph of the slope function and the line $PQ$. %What does the graph of $y=m(x)$ suggest about the slope function?

\item There is a hole in the graph of the function $y=m(x)$. Where is it? Why is it there?

\end{enumerate}

\item To verify the numerical and graphical evidence that %computation we'll use algebra to compute
\[
     \frac{d}{dx}\left( x^3 \right)\Big|_{x=1}  = \lim_{v\to 1} \frac{v^3-1}{v-1}  =\answer{3} ,
\]
we'll use algebra to evaluate the above limit. 

The idea is to factor the numerator $v^3-1$. Since 
\[
     (v^3 - 1)\Big|_{v=1} = 1^3 - 1 = 0 ,
\]
we know that $\answer{v-1}$ is a factor of $v^3-1$. To simplify the quotient
\[
  \frac{v^3-1}{v-1} \, , \, v\neq 1,
\]
we could use long division or factor the difference of two cubes:
\[
   a^3 - b^3 = (a-b)(a^2 + ab + b^2).
\]
Either way, the result is that
\[
     v^3 - 1 = (v-1)(\answer{v^2 + v + 1}).
\]
So
\[
   \frac{v^3-1}{v-1} = \answer{v^2+v+1} \, , \, v\neq 1.
\]
Putting this all together, we get
\begin{align*}
\frac{d}{dx}\left( x^3 \right)\Big|_{x=1}  & = \lim_{v\to 1} \frac{v^3-1}{v-1} \\
                                                              &= \lim_{v\to 1} \frac{(v-1)(v^2+v+1)}{v-1} \\
                                                             & = \lim_{v\to 1}(\answer{v^2 + v + 1}) \\
                                                            & = 1^2 + 1 + 1 \\
                                                            & = \answer{3} .
\end{align*}
\end{enumerate}

\end{enumerate}
\end{exploration}

\begin{exploration} \label{Exp4354567778}
Repeat all parts of Exploration \ref{Exp9039000030} for the following derivatives.

\begin{enumerate}
\item 
\[
      \frac{d}{dx}\left( x^3 \right)\Big|_{x=2} = \answer{12} .
\]

\item
\[
      \frac{d}{dx}\left( x^3 \right)\Big|_{x=a} = \answer{3a^2} .
\]
Use the demonstration below to check your work by dragging the sliders $v$ (Line 4) and $a$ (Line 16). Then activate the Derivative Function folder on Line 17.

\begin{onlineOnly}
    \begin{center}
\desmos{ewy7jqij6s}{450}{600}  
\end{center}
\end{onlineOnly}

\href{https://www.desmos.com/calculator/ewy7jqij6s}{151: Cubing Function 2}




\item
\[
      \frac{d}{dx}\left( \frac{1}{x} \right)\Big|_{x=1} = \answer{-1} .
\]

\item
\[
      \frac{d}{dx}\left(\frac{1}{x} \right)\Big|_{x=a} = \answer{-1/a^2} .
\]

\item
\[
      \frac{d}{dx}\left(\frac{1}{x^2} \right)\Big|_{x=a} = \answer{-2/a^3} .
\]

\item
\[
      \frac{d}{dx}\left(\frac{1}{1+x^2} \right)\Big|_{x=a} = \answer{-\frac{2a}{(1+a^2)^2}} .
\]

\end{enumerate}

\end{exploration}

\section{Applications}

\begin{question}  \label{Q4ghg5t4t4tr4}
The function
\[
      h = f(t)  \, , \, 0\leq t \leq 2.2 ,                     %= -2t^3+7t^2-8t+8 \, , \, 0\leq t \leq 2.2 ,
\]
expresses the height of a balloon (in hundreds of feet) in terms of the number of minutes past noon.

The graph of the function $h=f(t)$ is shown below.

\begin{onlineOnly}
    \begin{center}
\desmos{kovtjpsebu}{450}{600}  
\end{center}
\end{onlineOnly}

\href{https://www.desmos.com/calculator/kovtjpsebu}{151: Balloon}

\begin{enumerate}

\item Find an expression for the function $r=m(v)$ that gives the balloon's average rate of ascent (measured in hundreds of feet per minute) between time $v$ minutes past noon and 12:02 pm. Include also the function's domain.

\item How is the average rate of ascent function in part (a) related to the line $PQ$ in the demonstration above?

\item Use the graph of the function $h=f(t)$ above to sketch a rough graph of the function $r=m(v)$.

\item Activate the folder (avg. rate of change function) in Line 11 to check your graph of the function $r=m(v)$. 

\item Open the Table in Line 1 by clicking the \emph{right arrow} at the left of the line. 

\begin{enumerate}
\item What is the balloon's average rate of ascent between 12:02:00 pm and 12:02:06 pm?

\item Use the graph of the function $h=f(t)$ above to estimate the balloon's rate of ascent at 12:02pm by zooming in on the apprpriate point.

\item What does the table suggest about the balloon's rate of ascent at 12:02pm? Explain.
\end{enumerate}

\item Now suppose
\[
   f(t) = -2t^3+7t^2-8t+8 .
\]

\begin{enumerate}
\item Find a simplified expression for the function $r=m(v)$. Include the appropriate domain.

\item Use your simplified expression to compute the balloon's rate of ascent at 12:02pm.
\end{enumerate}



\end{enumerate}

\end{question}

\end{document}
