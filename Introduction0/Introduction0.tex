\documentclass{ximera}
\title{An Introduction to Differentiable Functions}

\newcommand{\pskip}{\vskip 0.1 in}

\begin{document}
\begin{abstract}
An introduction to what it means for a function to be differentiable.
\end{abstract}
\maketitle




Zoom in closer and closer near a point $(a,f(a))$ on the graph of a common function $f$ and you'll most likely notice that the graph looks more and more like the graph of a linear function (ie. like a non-vertical line). If so, we say the function $y=f(x)$ is \emph{differentiable} at $x=a$. And the \emph{derivative} of the function, evaluated at $x=a$, written as
\[
     \frac{dy}{dx}\Big|_{x=a},
\]
is just the slope of that line or the rate of change of the linear function.

When the independent variable is time, we can interpret the derivative as an instantaneous rate of change.

\begin{example}  \label{Ex:KJDmft4thghhgdf}
The graph of the function 
\[
   T = f(m) \, ,  \, 0\leq m \leq 60,
\]
expressing the temperature (in $^\circ$C) of a cup of coffee in terms of the number of minutes past noon is shown below.

\begin{onlineOnly}
    \begin{center}
\desmos{fuftb4mq0k}{450}{600}  
\end{center}
\end{onlineOnly}

\href{https://www.desmos.com/calculator/fuftb4mq0k}{151: Cooling Coffee}


\begin{enumerate}
\item Zoom in close enough to the point $P$ above to make the graph look like a straight line. Then approximate the slope of this line. Include units.

\item Use the resut of part (a) to interpret the value and the meaning of the derivative
\[
   \frac{dT}{dm} \Big|_{T=40} .
\]

\item Repeat parts (a) and (b) for other temperatures by draggind the slider $T_0$ in Line 2 of the worksheet above. Then see if you can find a relationship between the temperature $T$ and the rate $dT/dm$ at which the temperature of the coffee is changing. 

\end{enumerate}
\end{example}

\end{document}