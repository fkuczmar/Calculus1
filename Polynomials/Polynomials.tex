\documentclass{ximera}
\title{Derivatives of Polynomials}

\newcommand{\pskip}{\vskip 0.1 in}

\begin{document}
\begin{abstract}
Working with polynomials and their derivatives.
\end{abstract}
\maketitle

\section*{Differentiating Polynomials}

\begin{question}  \label{Qdst4hbbh}
(a) Use the algebra of limits to find an expression for the derivative 
\[
     \frac{d}{dx}(f(x)) =   \frac{d}{dx}\left( x^5 \right)
\]
of the function $f(x)=x^5$.

(b) Use the result of part (a) to find an expression for the derivative
\[
    \frac{d}{dx}(f^{-1}(x)) =   \frac{d}{dx}\left( x^{1/5} \right)
\]
of the function $f^{-1}(x) = x^{1/5}$. And use set-builder notation to state the domains of $f^{-1}(x)$ and its derivative. 

(c) Use the result of part (a) and the algebra of limits to find an expression for the derivative
\[
    \frac{d}{dx} \left( \frac{1}{f(x)} \right) =   \frac{d}{dx}\left( x^{-5} \right)
\]
of the function $1/f(x) = x^{-5}$.

(d) Use the results of parts (a)-(c) to find equations of the tangent lines to the three curves $y=f(x)$, $y=f^{-1}(x)$, and $y=1/f(x)$ at the point $(1,1)$. Graph the curves and their tangent line in Desmos to check your work.

\end{question}


\begin{question}  \label{Qdsferewg}
(a) Use the results of Question 1 to make a conjecture about the derivative
\[
     \frac{d}{dx}(f(x)) =   \frac{d}{dx}\left( x^n \right)
\]
of the function $f(x)=x^n$.

(b) What do you get for the derivative when $n=0$? When $n=1$? Are these results correct? Explain.

\end{question}


\begin{question}   \label{Qergh55}
(a) Suppose one giraffe is always twice as tall as another. What can you say about their growth rates at any instant?

(b) Suppose one giraffe is always two feet taller than another. What can you say about their growth rates at any instant?

(c) What do parts (a) and (b) suggest about how to compute the derivatives
\[
   \frac{d}{dx} \left( f(x) + b  \right) ,
\]
\[
   \frac{d}{dx} \left(a f(x) \right) ,
\]
and
\[
    \frac{d}{dx} \left(a f(x) + b  \right)
\]
for constants $a,b\in \mathbb{R}$?

(d) Make up your own scenario that suggests how to compute the derivative
\[
   \frac{d}{dx} \left( f(x) + g(x)  \right) 
\]
of the sum of two functions.
\end{question}

\begin{question}  \label{Qdsfsdfre}
Find an equation of the tangent line to the curve
\[
      y = -x^3 + 4x^2 -3x + 1
\] 
at the point $(2,3)$. Graph the curve and its tangent line on Desmos. 
\end{question}


\section*{Thinking about Parabolas}

\begin{question} \label{Qsdfsdfgg}
(a) Find an equation of the tangent line to the parabola $y=x^2$ at the point $(-3,9)$.

(b) Find an equation of the tangent line to the parabola perpendicular to the tangent line in part (a).

(c) Find the coordinates of the point where the lines in parts (a) and (b) intersect.

(d) Let ${\cal L}$ be the line through the points of tangency of the lines in parts (a) and (b). Find the coordinates of the point where ${\cal L}$ intersects the $y$-axis.

(e) Repeat parts (a)-(d) above for the tangent line to the parabola $y=x^2$ at the general point $(b,b^2)$. What do you notice? Enter your work in the Desmos activity below.

\pdfOnly{
Access Desmos interactives through the online version of this text at
 
\href{https://www.desmos.com/calculator/qe7mgnu5sv}.
}
 
\begin{onlineOnly}
    \begin{center}
\desmos{qe7mgnu5sv}{900}{600}
\end{center}
\end{onlineOnly}


\end{question}

\begin{question}  \label{Qerdfggg}
Let
\[
     y = f(x) = a x^2
\]
where $a\in \mathbb{R}$ is a constant, and the variables $x, y$ are measured in meters.

(a) What are the units of the constant $a$? How do you know?

(b) Answer part (e) of the previous question for this function. Modify the desmos activity in the previous question to check your work.
\end{question}


\begin{question}  \label{Qdfgbbvvd}
In the absense of air resistance, a rock released from rest near the surface of the earth falls $s=16t^2$ feet during the first $t$ seconds of its fall.

Compare the speed of the rock when it hits the ground with the average speed of the rock during the entire time interval of its fall.
\end{question}



\begin{question}  \label{Qwerdfggg}
(a) The demonstration below shows two normal lines to a parabola and their point of intersection $P$. What do you think happens to $P$ as point $B$ approaches $A$? Answer this question \emph{without} dragging the slider $b$.

(b) Now drag the slider $b$ near $a=2$ and observe what happens to point $P$. Were you correct? 


\pdfOnly{
Access Desmos interactives through the online version of this text at
 
\href{https://www.desmos.com/calculator/ybaivhc2tl}.
}
 
\begin{onlineOnly}
    \begin{center}
\desmos{ybaivhc2tl}{900}{600}
\end{center}
\end{onlineOnly}
Access this activity online at \href{https://www.desmos.com/calculator/ybaivhc2tl}{151: Normals to Parabola}

\pskip \pskip

The parabola has equation $y=x^2/4$, point $A$ has coordinates $(2,1)$, and point $B$ has coordinates $(b,b^2/4)$. 

(c) Find an equation of the normal lines to the parabola at $A$ and $B$.

(d) Use algebra to find an equation of the point $P$ where the normal lines intersect.

(e) The point $P$ approaches some point $Q$ as $B$ appraoches $A$. Use the algebra of limits to find the coordinates of $Q$.

(f) Find an equation of the circle centered at $Q$ through $A$. Do this by first using vector algebra to find the coordinates of the center of the circle.

(g) Repeat parts (c)-(g), replacing the point $A(2,1)$ with the point $A(a,a^2)$.


\end{question}


\section*{Applications}

\begin{question}  \label{Qersdfgg4t}
The function
\[
      h = f(t) = 5 -2t -\frac{t^2}{4} + t^3 \, , \, -1\leq t \leq 1.8 ,
\]
expresses the height (in thousands of feet) of a balloon in terms of the number of hours past noon.

(a) Find the balloon's average rate of ascent between 11:00am and 11:30am.

(b) Is the balloon rising or falling at 1:00pm? At what rate? Use the graph of the function $h=f(t)$ below to approximate the rate. Then compute the exact rate.

(c) When is the balloon descending at the rate of $1000$ ft/hour? Use the sliders $m$ anb $b$ below to approximate the time(s). Then compute the exact time(s).

(d) Use the graph below to approximate when the balloon is descending at the fastest rate. Approximate this rate from the graph. Then compute the exact time and rate.

(e) Use the graph below to approximate when the balloon is at its lowest point. Then compute the exact time.

(f) Use the graph below to approximate when the balloon is at its highest point between 11am and 1:36pm. Then compute the exact time.

(g) Use algebra to find all half-hour time intevals during which the balloon descends at an average rate of $500$ ft/hour.

\begin{onlineOnly}
    \begin{center}
\desmos{bvtukd0vlc}{900}{600}
\end{center}
\end{onlineOnly}


Access this activity online at \href{https://www.desmos.com/calculator/occgyjoawh}{151: Height of Balloon}

\end{question}

\begin{question} \label{Qdsfsadgt4e}
The function 
\[
   G = f(s) = \frac{11}{5} +\frac{1}{5000}\left( s^3-50s^2+300s \right) , \, 3\leq s \leq 28 ,
\]
expresses the number of gallons of gas in your car in terms of your distance from home. The distance is measured in miles along your route. 

\begin{onlineOnly}
    \begin{center}
\desmos{cphmgnrtm7}{900}{600}
\end{center}
\end{onlineOnly}

Desmos activity available at
\href{https://www.desmos.com/calculator/cphmgnrtm7}{151: Gas as a Function of Distance 2c}

(a) Use the graph of the function $f$ shown above to determine if you are driving toward or away from home. Explain your reasoning.

(b) Zoom in on the graph to approximate your gas mileage at the moment you are $20$ miles from home. Show a screenshot to help explain how you got your approximation. Then compute the exact gas mileage.

(c) Use the sliders $m$ and $b_1$ in the graph to approximate your distance from home at the moment your car gets $30$ miles/gallon. Show a screenshot to help explain how you got your approximation. Then compute the exact distances.

(d) Use the sliders $m$ and $b_1$ in the graph to approximate an interval beginning or ending when you are $20$ miles from home over which your average gas mileage is equal to you gas mileage at the moment you are $20$ miles from home. Show a screenshot to help explain how you got your approximation. Then compute the exact interval.

(e) Sensors on your car measure both the (instantaneous) gas mileage and the number of gallons of gas in your tank at each instant. A computer then uses these measurements to estimate the number of additional miles you can drive before running out of gas. Use this idea to find a function 
\[
  m =g(s) \, , \, 3\leq s \leq 28 ,
\]
that expresses the number of miles you can drive before running out of gas (assuming your gas mileage remains constant for the remainder of your trip) in terms of your distance from home. Explain your reasoning. 

\end{question}

\begin{question}  \label{Qdet5h5t5}
(a) Make up your own \emph{quadratic} function
\[
  v = f(G)  = aG^2 + bG +c,
\]
with $a$,$b$, and $c$ all  not equal to zero, that expresses the speed of a car (measured in miles/hour) in terms of its gas mileage (measured in miles/gallon). Be sure to include a domain. Explain why you think your function is reasonable.

(b) Compute the derivative $dv/dG$ and evaluate it at a spefic gas mileage. Include units.

(c) Evaluate the derivative $dv/dG$ at a specific gas mileage and its meaning. Include units in your explanation.

(d) Use your function from part (a) to find a function
\[
    r = h(G)
\]
that expresses the rate (in gal/hr) at which the car burns gas in terms of its gas mileage (in miles/gal).  Explain your logic thoroughly.

(e) Evaluate $h(G)$ at the same gas mileage, say $G_0$, you used in part (c). Compare the units of $h(G_0)$ and the derivative $dv/dG\Big|_{G=G_0}$. Are these two numbers related? What does this tell you about simplifying the units of a derivative?

(f) Use the ideas of this chapter (ie. derivatives of polynomials, and nothing beyond) to find an expression for the derivative
$dr/dG$ .

(g) What are the units of the derivative $dr/dG$? 

(h) Evalute the derivative $dr/dG$ at a specific gas mileage and explain its meaning. Include units in your explanation.

(i) Make up and answer your own question about the derivative $dr/dG$ at a specific gas mileage.

\end{question}


\section*{A Few More Problems}

\begin{question}  \label{Qdgt455t}
The function 
\[
      h = f(t) = kt^3 , t\geq 0 ,
\]
expresses the height of a balloon (in thousands of feet) in terms of the number of hours past noon. Here $k>0$ is a positive constant.

(a) What are the units of the constant $k$? Explain how you know.

(b) Find the balloon's rate of ascent (measured in thousands of ft/hr) at time $t=3$ hours past noon. Then find its average rate of ascent (measured in thousands of ft/hr) between noon and 3pm. How are these rates related? Note that both will be expressed in terms of $k$.

(c) Find a function
\[
     r = g(a)
\]
that expresses the balloon's rate of ascent (measured in thousands of ft/hr) at time $t=u$ hours past noon in terms of  its average rate of ascent (measured in thousands of ft/hr) over the time interval $t\in [0,u]$. Assume $u>0$.

(d) Interpret your result from part (b) geometrically on the graph of the function $f$. Follow the directions on Lines 4, 6, and 8 in the demonstration below to help explain your interpretation.

(e) Which of the following expresses the balloon's rate of ascent at time $t$ hours past noon in terms of $t$ and $h=f(t)$?
\begin{multipleChoice}
\choice{$\frac{h}{t}$}
\choice{$\frac{h}{3t}$}
\choice[correct]{$\frac{3h}{t}$}
\end{multipleChoice}
 

\begin{onlineOnly}
    \begin{center}
\desmos{0byjcy77yw}{900}{600}
\end{center}
\end{onlineOnly}

Desmos activity available at
\href{https://www.desmos.com/calculator/0byjcy77yw}{151: Subtangent 1}


\end{question}


\begin{question}  \label{Qdsfgbhn54}
The function 
\[
     v = f(G)
\]
expresses the speed of a car (in miles/hour) in terms of its gas mileage (in miles/gallon) for speeds between $55$ miles/hour and $70$ miles/hour. 

Suppose $f(30)=60$.

(a) Which of the following is more likely to be true?
\[
 \frac{dv}{dG}\Big|_{G=30} = 4 
\]
or
\[
   \frac{dv}{dG}\Big|_{G=30} = -4 ?
\]
Explain your reasoning.

(b) What are the units of the above derivative? Do not simplify the units and do not write ``per'' in place of ``/''.

(c) Explain the meaning of the derivative in part (a). It is \emph{not} enough to say ``the rate of change of something with respect to something else." Remember this class is all about small changes and your explanation should be about an approximate relationship between small changes in this setting.

(d) Simplify the units of the derivative. What does this suggest about its meaning?

(e) At what rate does the car burn gas (in gal/hour) at a speed of $60$ miles/hour?

(f) What does this problem suggest about simplifying the units of a derivative? 

\end{question}

\begin{question}  \label{Qdgfv4t5t}
The function 
\[
     v = f(G) , 20\leq G \leq 40,
\]
expresses the speed of a car (in miles/hour) in terms of its gas mileage (in miles/gallon). Ue the graph of the function $f$ to find approximate answers to the following questions. Change the position and slope of line $AB$ by dragging either the line or the points $A$ or $B$. Change the position of the tangent line by dragging the slider $G$.


(a) Label the axes with the appropriate variable names and units.

(b) At what speed does the car burn gas as the fastest rate?

\begin{hint}
One of questions (b), (c) is related to the tangent lines to the curve, the other is related to the lines through the origin and the points of the curve. 
\end{hint}

(c) At what speed does increasing the speed by $0.1$ miles/hour result in the greatest absolute change in the gas mileage? Approximate that change.

\begin{onlineOnly}
    \begin{center}
\geogebra{vjdf6x6z}{400}{300}
\end{center}
\end{onlineOnly}

Geogebra activity available at
\href{https://www.geogebra.org/classic/vjdf6x6z}{151: Gas Mileage}


\end{question}


\end{document}
  
