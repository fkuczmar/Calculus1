\documentclass{ximera}
\title{Derivatives of Polynomials}

\newcommand{\pskip}{\vskip 0.1 in}

\begin{document}
\begin{abstract}
Working with polynomials and their derivatives.
\end{abstract}
\maketitle

\section*{Differentiating Polynomials}

\begin{question}  \label{Qdst4hbbh}
(a) Use the algebra of limits to find an expression for the derivative 
\[
     \frac{d}{dx}(f(x)) =   \frac{d}{dx}\left( x^5 \right)
\]
of the function $f(x)=x^5$.

(b) Use the result of part (a) to find an expression for the derivative
\[
    \frac{d}{dx}(f^{-1}(x)) =   \frac{d}{dx}\left( x^{1/5} \right)
\]
of the function $f^{-1}(x) = x^{1/5}$. And use set-builder notation to state the domains of $f^{-1}(x)$ and its derivative. 

(c) Use the result of part (a) and the algebra of limits to find an expression for the derivative
\[
    \frac{d}{dx} \left( \frac{1}{f(x)} \right) =   \frac{d}{dx}\left( x^{-5} \right)
\]
of the function $1/f(x) = x^{-5}$.

(d) Use the results of parts (a)-(c) to find equations of the tangent lines to the three curves $y=f(x)$, $y=f^{-1}(x)$, and $y=1/f(x)$ at the point $(1,1)$. Graph the curves and their tangent line in Desmos to check your work.

\end{question}


\begin{question}  \label{Qdsferewg}
(a) Use the results of Question 1 to make a conjecture about the derivative
\[
     \frac{d}{dx}(f(x)) =   \frac{d}{dx}\left( x^n \right)
\]
of the function $f(x)=x^n$.

(b) What do you get for the derivative when $n=0$? When $n=1$? Are these results correct? Explain.

\end{question}


\begin{question}   \label{Qergh55}
(a) Suppose one giraffe is always twice as tall as another. What can you say about their growth rates at any instant?

(b) Suppose one giraffe is always two feet taller than another. What can you say about their growth rates at any instant?

(c) What do parts (a) and (b) suggest about how to compute the derivatives
\[
   \frac{d}{dx} \left( f(x) + b  \right) ,
\]
\[
   \frac{d}{dx} \left(a f(x) \right) ,
\]
and
\[
    \frac{d}{dx} \left(a f(x) + b  \right)
\]
for constants $a,b\in \mathbb{R}$?

(d) Make up your own scenario that suggests how to compute the derivative
\[
   \frac{d}{dx} \left( f(x) + g(x)  \right) 
\]
of the sum of two functions.
\end{question}

\begin{question}  \label{Qdsfsdfre}
Find an equation of the tangent line to the curve
\[
      y = -x^3 + 4x^2 -3x + 1
\] 
at the point $(2,3)$. Graph the curve and its tangent line on Desmos. 
\end{question}


\section*{Thinking about Parabolas}

\begin{question} \label{Qsdfsdfgg}
(a) Find an equation of the tangent line to the parabola $y=x^2$ at the point $(-3,9)$.

(b) Find an equation of the tangent line to the parabola perpendicular to the tangent line in part (a).

(c) Find the coordinates of the point where the lines in parts (a) and (b) intersect.

(d) Let ${\cal L}$ be the line through the points of tangency of the lines in parts (a) and (b). Find the coordinates of the point where ${\cal L}$ intersects the $y$-axis.

(e) Repeat parts (a)-(d) above for the tangent line to the parabola $y=x^2$ at the general point $(b,b^2)$. What do you notice? Enter your work in the Desmos activity below.

\pdfOnly{
Access Desmos interactives through the online version of this text at
 
\href{https://www.desmos.com/calculator/qe7mgnu5sv}.
}
 
\begin{onlineOnly}
    \begin{center}
\desmos{qe7mgnu5sv}{900}{600}
\end{center}
\end{onlineOnly}


\end{question}

\begin{question}  \label{Qerdfggg}
Let
\[
     y = f(x) = a x^2
\]
where $a\in \mathbb{R}$ is a constant, and the variables $x, y$ are measured in meters.

(a) What are the units of the constant $a$? How do you know?

(b) Answer part (e) of the previous question for this function. Modify the desmos activity in the previous question to check your work.
\end{question}


\begin{question}  \label{Qdfgbbvvd}
In the absense of air resistance, a rock released from rest near the surface of the earth falls $s=16t^2$ feet during the first $t$ seconds of its fall.

Compare the speed of the rock when it hits the ground with the average speed of the rock during the entire time interval of its fall.
\end{question}



\begin{question}  \label{Qwerdfggg}

\pdfOnly{
Access Desmos interactives through the online version of this text at
 
\href{https://www.desmos.com/calculator/ybaivhc2tl}.
}
 
\begin{onlineOnly}
    \begin{center}
\desmos{ybaivhc2tl}{900}{600}
\end{center}
\end{onlineOnly}

Access this activity online at \href{https://www.desmos.com/calculator/ybaivhc2tl}{151: Normals to Parabola}

\end{question}


\section*{Applications}



\end{document}
  
