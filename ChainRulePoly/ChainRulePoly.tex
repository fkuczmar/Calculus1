\documentclass{ximera}
\title{The Chain Rule and Polynomials}

\newcommand{\pskip}{\vskip 0.1 in}

\begin{document}
\begin{abstract}
Problems with polynomials and the chain rule.
\end{abstract}
\maketitle

\begin{question} \label{Q8dfDfrfrf}
The function
\[
      h = f(s) \, , \, 0\leq s \leq 3 ,
\]
expresses the altitude (in thousands of feet) of a trail in terms of the distance (in miles) from the trailhead.

The function
\[
    s = g(t) \, , \, 0\leq t \leq 4,
\]
expresses your distance from the trailhead (in miles) as you hike along the trail in terms of the number of hours past noon.

The functions $f$ and $g$ are graphed below.

\begin{onlineOnly}
    \begin{center}
\desmos{sm5rxml9zi}{900}{600}   sm5rxml9zi
\end{center}
\end{onlineOnly}

Worksheet available at \href{https://www.desmos.com/calculator/sm5rxml9zi}{151: Hiking Trail}  %x0ywksqpsw  akk25oy6sz


\begin{enumerate}

\item Use the graphs above to estimate the following. Include units and write each as a derivative using the Leibniz notation.

\begin{enumerate}

\item your elevation at 2:30pm.

\item your hiking speed at 2:30pm.

\item the steepness of your part of the trail at 2:30pm. Interpret the meaning of this in terms of small changes.

\item the rate (in feet/hour) at which you are gaining or losing altitude at 2:30pm. 
\end{enumerate}

\item Now suppose that
\[
    h = f(s) = -\frac{s^3}{3} + s^2 + s + 2 \, , \, 0\leq s\leq 3
\]
and
\[
    s = g(t) = 0.1 t^2 + 0.35t \, , \, 0\leq t \leq 4.
\]

\begin{enumerate}
\item Find the exact values of each of the derivatives in part (a).
\end{enumerate}
\end{enumerate}

\end{question}



\begin{question} \label{QLkmdfdsrr}
Due to a late-season frost in eastern Washington, the price of Cosmic Crisp apples is rising precipitously. The function 
\[
          P = f(t) \, , \, 0\leq t \leq 8 ,
\]
expresses the pirce (in dollars/pound) in terms of the number of hours past 9am on April 25, 2025.

\begin{enumerate}
\item Find an expression for the function
\[
       N = g(P)
\]
for the number of pounds of apples you can buy with $\$36$ in terms of the price (in dollars/pound).

\item Evaluate the derivative
\[
 \frac{dN}{dP}\Big|_{P=4} .
\]
Include units.

\item Interpret the meaning of the above derivative in terms of small changes.

\item Find an expression for the derivative $dN/dt$ in terms of $P$ and the derivative $dP/dt$.

\item Suppose 
\[
     P = f(t) = 2 + \frac{2}{5}t \, , \, 0\leq t \leq 8 ,
\]
and determine the rate of change (with respect to time) at 2pm in then number of pounds of apples you can buy with $\$36$.

\item Suppose (for an arbitrary differentiable price function) that at some instant the price is decreasing at the relative rate of $10\%$/hour. Is the number of pounds of apples you can buy with $\$36$ increasing or decreasing at this instant? At what relative rate?

\end{enumerate}
\end{question}


\begin{question}  \label{QPERdgfret}
The bottom end of a ten-foot ladder slides across a horizontal floor as its top end slides along a vertical wall.

\begin{enumerate}
\item Find a function
\[
   h = f(s) \, , \, 0\leq s \leq 10 ,
\]
that expresses the height of the top end above the floor (in feet) in terms of the distance of the bottom end from the wall (in feet).

\item Use the demonstration below to estimate the derivative
\[
    \frac{dh}{ds}\Big|_{s=8}.
\]
Include units. Then use calculus to find the exact value.

\begin{onlineOnly}
    \begin{center}
\desmos{jqwsnh33o1}{900}{600}
\end{center}
\end{onlineOnly}

Worksheet available at \href{https://www.desmos.com/calculator/jqwsnh33o1}{151: Ladder and Tree 23} 

\item Interpret the meaning of the above derivative in terms of small changes.

\item Suppose when the ladder's bottom end is eight feet from the wall, it is sliding away from the wall at a speed of $5$ ft/sec. Is the top end moving up or down the wall at this instant? At what speed?

\item Now suppose the function
\[
    s = g(t)= 10-\frac{2}{5}(t-5)^2\, , \, 0\leq t \leq 10 ,
\]
expresses the distance of the ladder's bottom end from the wall (in feet) in terms of the number of seconds past noon

\begin{enumerate}
\item Play 
\end{enumerate}


\end{enumerate}

\end{question}

\end{document}

