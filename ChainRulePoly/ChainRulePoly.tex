\documentclass{ximera}
\title{The Chain Rule and Polynomials}

\newcommand{\pskip}{\vskip 0.1 in}

\begin{document}
\begin{abstract}
Problems with polynomials and the chain rule.
\end{abstract}
\maketitle

\begin{question} \label{Q8dfDfrfrf}
The function
\[
      h = f(s) \, , \, 0\leq s \leq 3 ,
\]
expresses the altitude (in thousands of feet) of a trail in terms of the distance (in miles) from the trail head.

The function
\[
    s = g(t) \, , \, 0\leq t \leq 4,
\]
expresses your distance from the trailhead (in miles) as you hike along the trail in terms of the number of hours past noon.

The functions $f$ and $g$ are graphed below.

\begin{onlineOnly}
    \begin{center}
\desmos{x0ywksqpsw}{900}{600}
\end{center}
\end{onlineOnly}

Worksheet available at \href{https://www.desmos.com/calculator/x0ywksqpsw}{151: Hiking Trail}


\begin{enumerate}

\item Use the graphs above to estimate the following. Include units and write each as a derivative using the Leibniz notation.

\begin{enumerate}

\item your elevation at 2:30pm.

\item your hiking speed at 2:30pm.

\item the steepness of your part of the trail at 2:30pm. Interpret the meaning of this in terms of small changes.

\item the rate (in feet/hour) at which you are gaining or losing altitude at 2:30pm. 
\end{enumerate}

\end{enumerate}

\end{question}



\end{document}

