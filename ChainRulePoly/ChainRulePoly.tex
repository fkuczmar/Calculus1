\documentclass{ximera}
\title{The Chain Rule and Polynomials}

\newcommand{\pskip}{\vskip 0.1 in}

\begin{document}
\begin{abstract}
Problems with polynomials and the chain rule.
\end{abstract}
\maketitle

\begin{question} \label{Q8dfDfrfrf}
The function
\[
      h = f(s) \, , \, 0\leq s \leq 3 ,
\]
expresses the altitude (in thousands of feet) of a trail in terms of the distance (in miles) from the trailhead.

The function
\[
    s = g(t) \, , \, 0\leq t \leq 4,
\]
expresses your distance from the trailhead (in miles) as you hike along the trail in terms of the number of hours past noon.

The functions $f$ and $g$ are graphed below.

\begin{onlineOnly}
    \begin{center}
\desmos{sm5rxml9zi}{900}{600}   sm5rxml9zi
\end{center}
\end{onlineOnly}

Worksheet available at \href{https://www.desmos.com/calculator/sm5rxml9zi}{151: Hiking Trail}  %x0ywksqpsw  akk25oy6sz


\begin{enumerate}

\item Use the graphs above to estimate the following. Include units and write each as a derivative using the Leibniz notation.

\begin{enumerate}

\item your elevation at 2:30pm.

\item your hiking speed at 2:30pm.

\item the steepness of your part of the trail at 2:30pm. Interpret the meaning of this in terms of small changes.

\item the rate (in feet/hour) at which you are gaining or losing altitude at 2:30pm. 
\end{enumerate}

\item Now suppose that
\[
    h = f(s) = -\frac{s^3}{3} + s^2 + s + 2 \, , \, 0\leq s\leq 3
\]
and
\[
    s = g(t) = 0.1 t^2 + 0.35t \, , \, 0\leq t \leq 4.
\]

\begin{enumerate}
\item Find the exact values of each of the three derivatives in part (a). Use the desmos worksheet to save time by entering f'(?) and g'(?) in a blank line, where you need to determine the inputs to these derivatives.
\end{enumerate}
\end{enumerate}

\end{question}



\begin{question} \label{QLkmdfdsrr}
Due to a late-season frost in eastern Washington, the price of Cosmic Crisp apples is rising precipitously. The function 
\[
          P = f(t) \, , \, 0\leq t \leq 8 ,
\]
expresses the pirce (in dollars/pound) in terms of the number of hours past 9am on April 25, 2025.

\begin{enumerate}
\item Find an expression for the function
\[
       N = g(P)
\]
for the number of pounds of apples you can buy with $\$36$ in terms of the price (in dollars/pound).

\item Evaluate the derivative
\[
 \frac{dN}{dP}\Big|_{P=4} .
\]
Include units.

\item Interpret the meaning of the above derivative in terms of small changes.

\item Find an expression for the derivative $dN/dt$ in terms of $P$ and the derivative $dP/dt$.

\item Suppose 
\[
     P = f(t) = 2 + \frac{2}{5}t \, , \, 0\leq t \leq 8 ,
\]
and determine the rate of change (with respect to time) at 2pm in then number of pounds of apples you can buy with $\$36$.

\item Suppose (for an arbitrary differentiable price function) that at some instant the price is decreasing at the relative rate of $10\%$/hour. Is the number of pounds of apples you can buy with $\$36$ increasing or decreasing at this instant? At what relative rate?

\end{enumerate}
\end{question}


\begin{question}  \label{QPERdgfret}
The bottom end of a ten-foot ladder slides across a horizontal floor as its top end slides along a vertical wall.

\begin{enumerate}
\item Find a function
\[
   h = f(s) \, , \, 0\leq s \leq 10 ,
\]
that expresses the height of the top end above the floor (in feet) in terms of the distance of the bottom end from the wall (in feet).

\item Use the demonstration below to estimate the derivative
\[
    \frac{dh}{ds}\Big|_{s=8}.
\]
Include units. Then use calculus to find the exact value.

\begin{onlineOnly}
    \begin{center}
\desmos{jqwsnh33o1}{900}{600}
\end{center}
\end{onlineOnly}

Worksheet available at \href{https://www.desmos.com/calculator/jqwsnh33o1}{151: Ladder and Tree 23} 

\item Interpret the meaning of the above derivative in terms of small changes.

\item Suppose when the ladder's bottom end is eight feet from the wall, it is sliding away from the wall at a speed of $5$ ft/sec. Is the top end moving up or down the wall at this instant? At what speed?

\item Now suppose the function
\[
    s = g(t)= 10-\frac{2}{5}(t-5)^2\, , \, 0\leq t \leq 10 ,
\]
expresses the distance of the ladder's bottom end from the wall (in feet) in terms of the number of seconds past noon.

\begin{enumerate}
\item Play the slider in Line 2 in the animation below to watch the motion. Then sketch by hand a rough graph of the composition
\[
     h = f(g(t)) \, , \, 0\leq t \leq 10.
\]
Label the axes with the appropriate variable names and units. Then activate the folder in Line 43 to see how you did.

\begin{onlineOnly}
    \begin{center}
\desmos{duq8yqxozk}{900}{600}
\end{center}
\end{onlineOnly}

Worksheet available at \href{https://www.desmos.com/calculator/duq8yqxozk}{151: Ladder and Tree 25} 


\item Use the graph of $h=f(g(t))$ to sketch a graph of the function
\[
  r = \frac{dh}{dt} .
\]
Label the axes with the appropriate variable names and units. Then activate the folder in Line 47 to see how you did. Explain the meaning of this function.

\item What comments/questions do you have about the graph of $r=dh/dt$?

\item At time $t=7$ seconds past noon, is the top end of the ladder moving up or down? At what speed? Use calculus to find the exact speed.


\end{enumerate}
\end{enumerate}
\end{question}

\begin{question}  \label{QPDFefrdffd}
The function
\[
        g = f(v) \, , \, 20\leq v \leq 50 ,
\]
expresses the gas mileage (in miles/gal) of a car in terms of its speed (in miles/hour).

The function 
\[
    G = h(w) \, , \, 32 \leq w \leq 90 ,
\] 
expresses the gas mileage (in km/liter) of the same car in terms of its speed (in km/hour).

Assume there are exactly $1.6$ km in one mile and exactly $4$ liters in one gallon.

Suppose also that 
\[
   \frac{dg}{dv}\Big|_{v=40} = 0.3 .
\]

\begin{enumerate}
\item What are the units of the derivative above? Explain its meaning in terms of small changes.

\item Express the function $h$ in terms of the function $f$.

\item Use common sense to evaluate the derivative
\[
          \frac{dG}{dw}\Big|_{w=64}.
\]

\item Use common sense to express the derivative $dG/dw$ in terms of $dg/dv$. Then use the chain rule and check if you get the same result. 
\end{enumerate}
\end{question}

\begin{question} \label{Qdf8r3fgdsf}
The function
\[
  W =f(h)  = \frac{2400}{(h+4)^2}  \, , \,h \geq 0 ,
\]
expresses the weight (in pounds) of an astronaut in terms of her height (in thousands of miles above the surface of the earth).

\begin{enumerate}
\item What are the units of the constant $2400$? How do you know?

\item Suppose when her spaceship is $100$ miles above the earth's surface, the astronaut is moving directly toward the earth at a speed of $15,000$ miles/hour. At what rate (with respect to time) is her weight changing at this instant. Compute this rate as the product of two other rates. Include units for these rates and explain their meanings in terms of small changes.
\end{enumerate}

\end{question}

\begin{question}  \label{QOdfdfrDFGDF}
Suppose an oil spill in the Gulf of Mexico retains the shape of a perfect circle as it expands. At some moment the circle's radius is increasing at the relative rate of $5\%$/hour. At what relative rate is its area changing at this moment?
\end{question}


\section{Discussion Questions}

\begin{question}  \label{Q9dfFRFrfddf}
The function 
\[
   P = f(t) \, , \, 0\leq t \leq 7 ,
\]
expresses the price of a stock (in dollars/share) in terms of the number of hours past 9am. Its graph is shown below.


\begin{onlineOnly}
    \begin{center}
\desmos{0yv49behlw}{900}{600}
\end{center}
\end{onlineOnly}

Worksheet available at \href{https://www.desmos.com/calculator/0yv49behlw}{151: Stock Chain Rule} 

\begin{enumerate}
\item Is the number of shares you can buy with $\$200$ increasing or decreasing at 3pm? At what rate? Drag points $P$ and $Q$ in the graph to help approximate the rate. Explain your reasoning.

\item Suppose
\[
   P = f(t) = -\frac{1}{2}t^2 + 4t + 4 \, , \, 0\leq t \leq 7.
\]

\begin{enumerate}
\item Find the exact rate (with respect to time) at which the number of shares you can buy with $\$200$ is changing at 3pm. 

\item Find the relative rates of change (with respect to time) at 3pm in the stock price and the number of shares you can buy with $\$200$. Include units. Compare these rates.
\end{enumerate}
\end{enumerate}
\end{question}


\begin{question} \label{QLFMENEdefdes}
A rock is dropped from rest near the surface of some planet and allowed to fall until it hits the surface. The function
\[
  v = f(h) \, , \, 0\leq t \leq 100,
\]
expresses the speed of the rock (in meters/sec) in terms of its height above the surface (in meters).

\begin{enumerate}
\item Would you expect the derivative
\[
   \frac{dv}{dh}\Big|_{h=18}
\]
to be positive or negative? Explain your reasoning.

\item Suppose
\[
   v = g(h) = \frac{\sqrt{100-2h}}{3} \, , \, 0\leq h \leq 50.
\]
\begin{enumerate}

\item Use the graph of the function $y=\sqrt{x}$ and the sliders $u$ and $v$ below to help approximate the derivative in part (a).

\begin{onlineOnly}
    \begin{center}
\desmos{ilb4wptgxe}{900}{600}
\end{center}
\end{onlineOnly}

Worksheet available at \href{https://www.desmos.com/calculator/ilb4wptgxe}{151: Stock Chain Rule} 

\item Use the expression for the function $v=g(h)$ to find the exact value of the derivative in part (a).

\item Interpret the meaning of the derivative in part (a) in terms of small changes. 

\item Use part (iii) to approximate the rock's speed at a height of $17.7$ meters.


\end{enumerate}
\end{enumerate}

\begin{question} \label{QOERPERder}
Go back to Question 3, part (e) of this chapter and determine all possible angles the ladder makes with the ground when its top end is moving twice as fast as its bottom end.
\end{question}
\end{question}


\end{document}

