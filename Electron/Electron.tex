\documentclass{ximera}
\title{Electron in a Crossed Field}

\newcommand{\pskip}{\vskip 0.1 in}

\begin{document}
\begin{abstract}
Electron.
\end{abstract}
\maketitle

The force on an a charge $q$ with velocity ${\bf v}$ in a magnetic/electric field is given by
\[
     {\bf F} = q ({\bf E} + {\bf v}\times {\bf B} ) .
\]

The problem is to parameterize the motion of a charged particle when the two fields are assumed uniform and orthogonal. With the magnetic field a constant multiple of ${\bf k}$ and the electric field a constant multiple of ${\bf j}$, suppose the particle either starts from rest or has velocity ${\bf v}_0$ perpendicular to ${\bf k}$ at time $t=0$.

Not understanding any of the physics, the charge has acceleration
\[
    {\bf a} =  \frac{d {\bf v}}{dt} = k_1 {\bf j} + k_2 {\bf v}\times {\bf k} ,
\]
for some constants $k_1, k_2 \geq 0$.

We thought first about turning off the electric field


\begin{exploration} \label{Edgvb5rthh}

\begin{onlineOnly}
    \begin{center}
\desmos{jdgp5obovm}{900}{600}  %    pxsmo04nmg  8swp20zond
\end{center}
\end{onlineOnly}

\href{https://www.desmos.com/calculator/jdgp5obovm}{Electron 11}


\end{exploration}


\end{document}

