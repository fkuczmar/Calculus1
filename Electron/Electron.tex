\documentclass{ximera}
\title{Electron in a Crossed Field}

\newcommand{\pskip}{\vskip 0.1 in}

\begin{document}
\begin{abstract}
Electron.
\end{abstract}
\maketitle

The force on an a charge $q$ with velocity ${\bf v}$ in a crossed magnetic/electric field is given by
\[
     {\bf F} = q ({\bf E} + {\bf v}\times {\bf B} ) .
\]

Our problem is to parameterize the motion of a charged particle when the fields are uniform and mutually orthogonal. We suppose the magnetic and electric fields point in the respective directions ${\bf k}$ and ${\bf j}$, and that the charge has velocity ${\bf v}_0$ perpendicular to ${\bf k}$ at time $t=0$.

Then the charge, assumed to be positve, has acceleration
\[
    {\bf a} =  \frac{d {\bf v}}{dt} = k_1 {\bf j} + k_2 {\bf v}\times {\bf k} ,
\]
for some constants $k_1, k_2 \geq 0$, with respective units $\text{m}/\text{sec}^2$ and $\text{sec}^{-1}$..

With the electric field turned off ($k_1=0$), the acceleration and velocity vectors are perpendicular, so the charge moves with constant speed $|{\bf v}_0| = v_0$. But then because ${\bf v}$ is perpendicular to ${\bf k}$,
\[
      |{\bf a}| = k_2 |v| = k_2 v_0
\]
is also constant.

So we are looking for a plane motion having constant speed and an acceleration with constant magnitude. One choice would be uniform circular motion, where
\[
    |{\bf a}| = v_0 \omega ,
\]
where $\omega = d\theta/dt$ is the (constant) rotation rate of the pinned velocity vector. Then $k_2 = \omega$ and the trajectory has radius
\[
   r = \frac{v_0}{\omega} = \frac{v_0}{k_2} .
\]

To see that there are no other possible motions we could probably appeal to some uniqueness theorem of differential equations. But for a more geometric approach, consider what we know.

\begin{enumerate}
\item that the charge moves with constant speed and

\item that the acceleration vector rotates at a constant rate.
\end{enumerate}

\begin{question}  \label{Qdtg4th5yg5}
What do these conditions imply about the trajectory?

\begin{explanation}
Click the arrow to the lower right for the solution.

\begin{expandable}
Let $\theta$ be the angle from ${\bf i}$ to ${\bf v}$. Then the conditions imply that the trajectory has a constant radius of curvature 
\[
  r = \frac{|v|}{\Big|\frac{d\theta}{dt} \Big|} = \frac{v_0}{k_2} 
\]
and being a plane curve is thus a circle.
\end{expandable}
\end{explanation}

\end{question} 



\section{Newton's Law of Cooling}



\begin{exploration} \label{Edgvb5rthh}

\begin{onlineOnly}
    \begin{center}
\desmos{yebumxuwms}{900}{600}  %    pxsmo04nmg  8swp20zond
\end{center}
\end{onlineOnly}

\href{https://www.desmos.com/calculator/yebumxuwms}{Electron 11}


\end{exploration}


\end{document}

