\documentclass{ximera}
\title{Clock Hands}

\newcommand{\pskip}{\vskip 0.1 in}

\begin{document}
\begin{abstract}
A twist on a familiar problem.
\end{abstract}
\maketitle

%\section*{Introduction}

\begin{question}
The question is to determine the angle between the minute and hour hands of a clock when the distance between the tips of the hands is increasing at the fastest rate.

We'll assume the hour and minute hands have respective lengths $a$ cm and $b$ cm.


\begin{onlineOnly}
    \begin{center}
\desmos{efhex45zkr}{450}{600}  %qvk0mzy26u
\end{center}
\end{onlineOnly}

\href{https://www.desmos.com/calculator/efhex45zkr}{151: Hands of a Clock 3}

\begin{enumerate}

\item Play the slider $u$ in Line 2 above and reconcile the graph of the function $s=f(t)$ that expresses the distance between the hands (in cm) in terms of the nuumber of minutes past noon.

\item Change the lengths of the hands in Lines 4 and 6 to see how the graph of the function $f$ changes.

\item The distance function looks sinusoidal, but open the folder in Line 26 to see the graph of the derivative dispels that notion. 

\item Find an expression for the function 
\[
  c=h(\theta) \, , \, 0\leq \theta \leq 2\pi ,
\]
that expresses the distance between the tips of the hands in terms of the angle between the hands. Work in general with the parameters $a$ and $b$, \emph{not} with their specific values in the worksheet.

\item Find an expression for the derivative
\[
  \frac{dc}{d\theta} = \frac{d}{d\theta}\left( f(\theta) \right) .
\]
Include units.

\item Use your expression from part (e) to determine the angle between the hands when the distance between their tips is increasing at the fastest rate. There are three cases to consider.

\begin{enumerate}
\item the minute hand is longer than the hour hand ($b>a$)

\item the hour hand is longer ($a>b$)

\item the hands have the same length ($a=b$)
\end{enumerate}

\item What can you say about the angle between the shorter hand and the segment $AB$ when the distance is increasing at the fastest rate? 

\end{enumerate}

\end{question}

\section*{A Computational Solution}
%It simplifies things if we ignore time and work in the reference frame of one of the hands

\begin{question} \label{Q9dgDDGHGMMMB}
Determine the angle between the hour and minute hands of a clock when the distance between the tips of these hands is increasing at the fastest rate. Suppose the hour and minute hands have respective lengths $a$ and $b$ inches, with $a<b$. 

\begin{explanation}
We'll let $a$ and $b$ denote the respective lengths of the hour and minute hands, $\theta$ the angle (in radians) between the hands, and $c$ the distance. Since the hands each turn at a constant rate, our aim is to maximize the derivative
\[
        \frac{dc}{d\theta} = \frac{ab\sin \theta}{c} = \frac{ab \sin \theta}{\sqrt{a^2+b^2-2ab\cos\theta}}.
\]
Differentiating again and doing some algebra gives
\[
       \frac{d^2c}{d\theta^2} = \frac{ab}{c^3} \left( c^2\cos\theta - ab\sin^2\theta  \right).
\]
Then setting the second derivative equal to zero and substituting 
\[
      c^2 = a^2+b^2-2ab\cos\theta
\]
leads to the equation (symmetric in $a$ and $b$)
\[
     ab \cos^2\theta - (a^2 + b^2)\cos\theta + ab = 0
\]
or
\[
              (a \cos \theta - b) (b \cos \theta - a) = 0 .
\]
Assuming the minute hand is longer than the hour hand ($b>a$) and $0\leq \theta \leq 2\pi$, we conclude that the distance is increasing at the fastest rate when $\theta = \arccos(a/b)$ and decreasing at the fastest rate when $\theta  = 2\pi - \arccos(a/b)$.

\end{explanation}
\end{question}


\begin{question} \label{Qdgbhhtnnngnb}
\begin{enumerate}
\item What can you say if $a<b$? If $a=b$?

\item Supposing $a\neq b$, find an expression for the maximum rate at which the distance between the tips of the hands is increasing.
\end{enumerate}
\end{question}


\section*{A More Geometrical Solution}
We could have saved ourselves the trouble of taking the second derivative by using the law of sines in $\Delta ABC$. Then %to write the rate of change $dc/d\theta$ as
\[
    \Big| \frac{dc}{d\theta}  \Big| =   \frac{ab\sin C}{c}  = \frac{ab \sin B}{B} = a\sin B
\]
so that the rate of change is maximized when $B = \pi/2$ or when $\theta = \arccos(a/b)$ as above.

%To do this, note that when $0<\theta < \pi$, $ab\sin\theta = 2K$ is twice the area of $\Delta ABC$. So
%\[
%     \Big| \frac{dc}{d\theta} \Big| = \frac{2K}{c}  = h,
%\]
%where $h$ is the altitude from from $C$ to side $\overline{AB}$ in $\Delta ABC$ above. This distance is maximized when $\overline{CA}$ is perpendicular to $\overline{AB}$, giving the same result as before. 

%Or we could use the law of sines in $\Delta ABC$ to get
%\[
%   \frac{dc}{d\theta} = \frac{ab\sin \theta}{c} = ab\frac{sin B}{b} = a\sin B .
%\]
%So the rate is maximized when angle $B$ is as close to $\pi/2$ as possible.

\section*{A Solution with Vectors}

\end{document}
