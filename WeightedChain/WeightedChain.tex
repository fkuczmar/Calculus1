\documentclass{ximera}
\title{The Weighted Chain}

\newcommand{\pskip}{\vskip 0.1 in}

\begin{document}
\begin{abstract}
Lifting a suspended weight.
\end{abstract}
\maketitle


\begin{question}  \label{Q:LMMN89}
You attach a small weight to the center of a twenty-inch string, hold the ends of the string together and then pull your hands apart. Play the slide $a$ in Line 2 of the animation below to watch the motion. 
\begin{onlineOnly}
    \begin{center}
\desmos{jmqscra2if}{900}{600}
\end{center}
\end{onlineOnly}

Desmos activity available at \href{https://www.desmos.com/calculator/jmqscra2if}{151: Weighted Chain 22} 

\begin{enumerate}
\item Use the animation to sketch by hand a graph of the function
\[
 h = f(s) \, , \, 0\leq s \leq 10
\]
that expresses the distance (in inches) of the weight from its starting point in term of half the distance (in inches) between your hands (ie. the distance $MR$ above). Then activate the folder in Line 15 to see how you did.

\item Use the graph of the height function $h=f(s)$ to sketch by hand a rough graph of its derivative $r = dh/ds = f'(s)$. Be sure to label the axes with the appropriate variable names and units. Then activate the folder in Line 20 to see how you did.

\item Find an expression for the function $h=f(s)$.

\item Find an expression for the average rate of change of $h$ with respect to $s$ between $s=b$ and $s=w$.

\item Use the algebra of limits to find an expression for the derivative
\[
   \frac{dh}{ds}\Big|_{s=b} .
\]

\item Drag the slider $a$ in Line 2 to $a=9.5$ and use the animation to approximate the derivative
\[
  \frac{dh}{ds}\Big|_{s=9.5}
\]

\item Compute the exact value of the above derivative. Include units. Compare this with your approximation.

\item Explain the meaning of the above derivative in terms of small changes.


\end{enumerate}


\end{question}


\begin{question}  \label{Q:DFDFGGtg}
This question animates the first question. For example, we might move our hands at a constant speed of $4$ inches/sec and then find an expression for the speed of the weight at any instant. But we'll work more generally, and suppose we have a function
\[
   s = g(t) \, , \, 0\leq t \leq 5,
\]
that expresses the distance $MR$ (ie. half the distance between our hands), measured in inches, in terms of the number of seconds since we began the motion.

\begin{onlineOnly}
    \begin{center}
\desmos{0cmsododxk}{900}{600}
\end{center}
\end{onlineOnly}

Desmos activity available at \href{https://www.desmos.com/calculator/0cmsododxk}{151: Weighted Chain 23}

\end{question}


\end{document}