\documentclass{ximera}
\title{The Quotient and Product Rules}

\newcommand{\pskip}{\vskip 0.1 in}

\begin{document}
\begin{abstract}
Quotient and product rules.
\end{abstract}
\maketitle

THIS IS TEST

Here are examples of how to use the Leibniz notation correctly in computing the derivative of a product and quotient of functions.

\begin{example} \label{Edfhnhhthgg}

Find expressions for the derivatives



\begin{enumerate}

\item 
\[
   \frac{d}{dx} \left( (5x^3-2)(4 - 6x)  \right)
\]

\item 
\[
  \frac{d}{dt} \left( \frac{5t^3-2}{4 - 6t}  \right)
\]
\end{enumerate}

\begin{explanation}
\begin{enumerate}

\item 
\begin{align*}
\frac{d}{dx} \left( (5x^3-2)(4 - 6x)  \right)  &= (4-6x)\frac{d}{dx} \left( 5x^3-2 \right) + (5x^3-2)   \frac{d}{dx} \left( 4-6x \right)    \\ \\
              &= (4-6x) 15x^2+ (5x^3-2) (-6) .
\end{align*}

\item 

\begin{align*}
\frac{d}{dt} \left( \frac{5t^3-2}{4 - 6t}  \right)  &=\frac{1}{(4-6t)^2} \left( (4-6t) \frac{d}{dt} \left( 5t^3-2 \right)  - (5t^3-2)   \frac{d}{dt} \left( 4-6t \right)   \right)  \\ \\
              &= \frac{(4-6t) 15t^2 - (5t^3-2) (-6)}{(4-6t)^2} .
\end{align*}


\end{enumerate}

\end{explanation}

\end{example}

\section{Relative Rates of Change}

Suppose $P=f(t)$ and $Q=g(t)$ are differentiable functions of $t$ and $g(t)\leq 0$. Then the quotient and product rules are better written in the forms
\begin{equation}  \label{Eq:Quotient}
   \frac{d}{dt} \left( \frac{P}{Q} \right) =  \frac{P}{Q} \left(   \frac{1}{P}\cdot \frac {dP}{dt} -  \frac{1}{Q}\cdot \frac {dQ}{dt}      \right)
\end{equation}
and
\[
   \frac{d}{dt} \left( PQ \right) =  PQ \left(   \frac{1}{P}\cdot \frac {dP}{dt} + \frac{1}{Q}\cdot \frac {dQ}{dt}      \right) .
\]

\begin{question} \label{Qgbhere}
\begin{enumerate}
\item Verify the statements above.

\item What do they say about the relative rate of change in the quotient of two functions? In their product?
\end{enumerate}
\end{question}

\section{Exercises}

\begin{exercise}  \label{Qdfbv4trttg}
Between speeds of $45$ miles/hr and $75$ miles/hr, the function
\[
    G = f(v) \, , \, 45 \leq v \leq 75,
\]
expresses the gas mileage (in miles/gal) of a car in terms of its speed (in miles/hour).

Suppose $f(50)=25$ and 
\[
  \frac{dG}{dv}\Big|_{v=50} = 0.8.
\]

\begin{enumerate}
\item What are the units of the derivative above? What is the meaning? Explain in terms of small changes.

\item What are the simplified units of the derivative above? What insight do these units give you about the derivative's meaning?

\item Approximate the gas mileage at a speed of $48$ miles/hour.

\item Let 
\[
  r=g(v) \, , \, 45 \leq v \leq 75,
\]
be the function that expresses the rate (in gal/hr) at which the car burns gas in terms of its speed (in miles/hr).

\begin{enumerate}  
\item Use Equation (\ref{Eq:Quotient}) above to evaluate the derivative
\[
  \frac{dr}{dv}\Big|_{v=50} .
\]
Include units for all numbers.

\item What are the units of the derivative above? What is the meaning? Explain in terms of small changes.

\item What are the simplified units of the derivative above? What insight do these units give you about the derivative's meaning?

\item Approximate the rate (in gal/hr) at which the car burns gas at a speed of $48$ miles/hr.

\end{enumerate}

\end{enumerate}

\end{exercise}


\begin{exercise}  \label{EEdfbygyhnnjjhh}
At 9:00am on February 23, 2023, the price of oil was decreasing at the relative rate of $2\%$/hour. At what relative rate was the number of gallons of oil you could buy with $\$100,000$ changing at that time? Use calculus to justify your assertion. 
\end{exercise}

\begin{exercise}  \label{EEdvbrttyer}
The function 
\[
      P = f(t) = 5 -3t + t^2 \, , \, 0\leq t \leq 4 , 
\]
expresses the price in $\$$/share of a stock in terms of the number of hours past 9am.

(a) Use the graphs of the function $P=f(t)$ and the function $r=f^\prime(t)/f(t)$  to estimate when the stock price is increasing at the greatest relative rate.

(b) Use algebra to find the exact time when the stock price is increasing at the greatest relative rate.
\begin{hint}
What is the value of the derivative $dr/dt$ at this time? But start by finding an expression for the instantaneous relative rate of change in the stock price.
\end{hint}


\begin{onlineOnly}
    \begin{center}
\desmos{xuupp3srqv}{900}{600}
\end{center}
\end{onlineOnly}

Desmos activity available at \href{https://www.desmos.com/calculator/xuupp3srqv}{151: Stock Price 4}

\end{exercise}


\begin{exercise}  \label{EEdgvbfghjjjyu}

An $h$-foot tall giraffe walks directly toward a spotlight on the ground as the light casts its shadow on a vertical wall as illustrated below. The wall is $b$ feet from the light.

\begin{onlineOnly}
    \begin{center}
\desmos{2eiyjjpu9n}{900}{600}
\end{center}
\end{onlineOnly}

Desmos activity available at \href{https://www.desmos.com/calculator/2eiyjjpu9n}{151: Spotlight}

Suppose at a certain instant the giraffe is $w$ feet from the spotlight and is walking at a speed of $v$ ft/sec. 

\begin{enumerate}

\item Is the length of the giraffe's shadow increasing or decreasing at this instant? 

\item At what rate? 

\item Check that your answer in part (b) has the correct units.

\item Assume $v$ is constant and describe what happens to the rate in part (b) as the giraffe approaches the spotlight. How is your conclusion revealed in the animation?

\end{enumerate}

\end{exercise}


\begin{exercise}  \label{Ekhkhglcbbg}
\begin{enumerate}
\item On January 1, 2024 the national debt of a country was decreasing at the rate of $3\%$/yr and the population was increasing at the rate of $2\%$/yr.  Was the per-captita (ie. per person) share of the national debt increasing or decreasing at this time? At what relative rate?

\item During the year 2024 the national debt of a country decreased $3\%$ and the population increased $2\%$.  Did the per-captita share of the national debt increase or decrease during the year? By what percent?

\item Compare the two questions above and their answers.

\end{enumerate}



\end{exercise}


\end{document}