\documentclass{ximera}
\title{The Quotient and Product Rules}

\newcommand{\pskip}{\vskip 0.1 in}

\begin{document}
\begin{abstract}
Quotient and product rules.
\end{abstract}
\maketitle

Here are examples of how to use the Leibniz notation correctly in computing the derivative of a product and quotient of functions.

\begin{example} \label{Edfhnhhthgg}

Find expressions for the derivatives

\begin{enumerate}

\item 
\[
   \frac{d}{dx} \left( (5x^3-2)(4 - 6x)  \right)
\]

\item 
\[
  \frac{d}{dt} \left( \frac{5t^3-2}{4 - 6t}  \right)
\]
\end{enumerate}

\begin{explanation}
\begin{enumerate}

\item 
\begin{align*}
\frac{d}{dx} \left( (5x^3-2)(4 - 6x)  \right)  &= (4-6x)\frac{d}{dx} \left( 5x^3-2 \right) + (5x^3-2)   \frac{d}{dx} \left( 4-6x \right)    \\ \\
              &= (4-6x) 15x^2+ (5x^3-2) (-6) .
\end{align*}

\item 

\begin{align*}
\frac{d}{dt} \left( \frac{5t^3-2}{4 - 6t}  \right)  &=\frac{1}{(4-6t)^2} \left( (4-6t) \frac{d}{dt} \left( 5t^3-2 \right)  - (5t^3-2)   \frac{d}{dt} \left( 4-6t \right)   \right)  \\ \\
              &= \frac{(4-6t) 15t^2 - (5t^3-2) (-6)}{(4-6t)^2} .
\end{align*}


\end{enumerate}

\end{explanation}

\begin{question}  \label{Qdfbv4trttg}
blank for now.
\end{question}




\end{example}

\end{document}