\documentclass{ximera}
\title{Small Changes}

\newcommand{\pskip}{\vskip 0.1 in}

\begin{document}
\begin{abstract}
We explore how small changes to the input of a function change the output.
\end{abstract}
\maketitle

The main idea of differential calculus is that for some functions (called \emph{differentiable}) the change in the output is approximately proportional to the change in the input. The proportionality factor is called the derivative. In this chapter we explore this idea in a few scenarios.


\section{The Falling Ladder}


\begin{example}   \label{Ex435rsfeetr}

The top end of a ten-foot ladder leans against a vertical wall and the bottom end rests on the horizontal floor. We analyze how a small change in the distance between the wall and the bottom of the ladder affects the height of the ladder's top above the floor. 

\pdfOnly{
Access Desmos interactives through the online version of this text at
 
\href{https://www.desmos.com/calculator/dvyuifyyg4}.
}
 
\begin{onlineOnly}
    \begin{center}
\desmos{dvyuifyyg4}{900}{600}
\end{center}
\end{onlineOnly}

Desmos activity available at

\href{https://www.desmos.com/calculator/dvyuifyyg4}{151: Ladder 1B}

\pskip \pskip

\begin{question}  \label{Qdsfesarr4}

(a) The slider $s$ in Line 1 of the demonstration above controls the distance between the wall and the bottom of the ladder, measured in feet. Use the slider $s$ to describe \emph{qualitatively} how a small change in $s$ changes the height $h$ (also measured in feet) of the ladder's top end above the floor. 

\pskip

(i) Do the small changes have the same or opposite signs?

(ii) At what positions of the ladder does a small change in $s$ result in a comparatively greater change in $h$?

\pskip

(b) Now let's focus the particular position of the ladder where the bottom end is  $s=8$ feet from the wall. For this, turn on the ``one position" folder in Line 3.

\pskip

(i) Drag the slider $s$ close to $s=8$ and use the cordinates of the endpoints of the ladder to construct a table with five columns showing the values of $s$, $h$, $\Delta s$, $\Delta h$, and $\Delta h / \Delta s$. Include units in the heading of each column. The table should include seven rows, with $s=7.7, 7.8, \ldots 8.2, 8.3$.

(ii) Find a function
\[
  h = f(s) , 0\leq s \leq 10 ,
\]
that expresses the height of the top end (in feet) in terms of the distance of the bottom end from the wall (in feet). %Activate the folder ``graph of function" in Line 8.

(iii) Use your function $f$ to construct another table, like the one in part (i), with $s=7.9, 7.99, 7.999,8, 8.0001, 8.01, 8.1$. Do this by finding expressions for 
\[
    \Delta h = f(s) - f(8)
\]
and
\[
   r = g(s) = \frac{\Delta h}{\Delta s} = \frac{f(s) - f(8)}{s-8} ,
\]
both in terms of $s$.

(iv) Does  your table from part (iii) suggest that the ratios $\Delta h/ \Delta s$ approach some number as $\Delta s$ approaches zero? If so, what would is your guess for the exact value of this limit? What are its units?

(v) Activate the folder ``graph of function" on Line 8. How is the line $PQ$, through the fixed point $P(8,6)$ and the variable point $Q$ on the graph of the function $h=f(s)$ related to the ratios $\Delta h / \Delta s$?

(vi) Activate the folder ``graph: average rate of change function" on Line 18.


\end{question}


\end{example}


\section{Riding a Ferris Wheel}


Suppose  you ride a ferris wheel

\end{document}