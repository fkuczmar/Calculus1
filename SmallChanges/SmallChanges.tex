\documentclass{ximera}
\title{Small Changes}

\newcommand{\pskip}{\vskip 0.1 in}

\begin{document}
\begin{abstract}
We explore how small changes to the input of a function change the output.
\end{abstract}
\maketitle

The main idea of differential calculus is to approximate the change in the output of a function in terms of a small change in the input. For some functions, called \emph{differentiable}, the change in the output is approximately proportional to the (small) change in the input. The proportionality factor is called the derivative. In this chapter we explore this idea.


\section{Odometer Readings}

\begin{example} \label{Ex:lkdftr45}
The graph of the function
\[
    s = f(t) , 0\leq t \leq 2 ,
\]
that expresses the trip odometer reading (measured in miles) on your car in terms of the number of hours past noon during a two-hour trip is shown below.

\pdfOnly{
Access Desmos interactives through the online version of this text at
 
\href{https://www.desmos.com/calculator/tumgpu4n0w}.
}
 
\begin{onlineOnly}
    \begin{center}
\desmos{iiw69lr1bc}{900}{600}
\end{center}
\end{onlineOnly}

Desmos activity available at

\href{https://www.desmos.com/calculator/iiw69lr1bc}{151: Odometer}

\pskip \pskip

Our goal is to approximate the car's speed at 12:30pm in three ways:

\pskip

(1) geometrically, using the above graph as is.

(2) geometrically, by zooming in on the above graph.

(3) arithmetically, using the specific expression for the function $f$.

%(3) algebraically, also using the specific expression for $f$.

\pskip

(a) Start by using the graph above to describe how the speed of the car varies over the two-hour period. Explain your reasoning. Then play the Slider $u$ in Line 2 and use the animation of the motion to check if your description was accurate. Explain.

(b) Set the slider in Line 6 to $n=20$. Use \emph{only} the graph (with $u=2$) to create a table with five columns showing the values of $t$, $s$ (approximate this), $\Delta t = t-0.5$, $\Delta s=f(t)-f(0.5)$, and $\Delta s / \Delta t$. Include units in the heading of each column. The table should include five rows, with $t= 0.3, 0.5, \ldots 0.7$. 

\begin{center}
  \begin{tabular}{ | c| c | c | c | c |}
    \hline
    $t$ (hours) & $s = f(t)$ (miles) & $\Delta t = t-0.5$ (hrs)  & $\Delta s = f(t) - f(0.5)$ (miles) & $\Delta s/\Delta t$ (miles/hr)  \\ \hline
    $0.3$ &  &  &   &  \\ \hline
    $0.4$ &  &  &   &  \\ \hline
    $0.5$  &   &  &  &   \\ \hline
    $0.6$ &  & &   &  \\ \hline
    $0.7$ &  &  &   &  \\ \hline
    \hline
  \end{tabular}
\end{center}

(c) Explain the meaning of the fifth column in the table of part (b). What do the entries in this column suggest about the speed of the car at 12:30pm?

(d) Now we'll use the fact that
\[
   s = f(t) = 48t^2 - 16t^3, 0 \leq t \leq 2 ,
\]
to construct another table like the one in part (b). Do this by finding expressions for 
\[
    \Delta s = f(t) - f(0.5)
\]
and
\[
   r = g(t) = \frac{\Delta s}{\Delta t} = \frac{f(t) - f(0.5)}{t-0.5} ,
\]
both in terms of $t$ (and \emph{not} $\Delta t$). Use these functions to fill in the missing entries in the table below.

\begin{center}
  \begin{tabular}{ | c| c | c | c | c |}
    \hline
    $t$ (hours) & $s = f(t)$ (miles) & $\Delta t = t-0.5$ (hrs)  & $\Delta s = f(t) - f(0.5)$ (miles) & $\Delta s/\Delta t$ (miles/hr)  \\ \hline
    $0.49$ &  &  &   &  \\ \hline
    $0.499$ &  &  &   &  \\ \hline
   $0.4999$ &  &  &   &  \\ \hline
    $0.5$  &   &  &  &   \\ \hline
    $0.5001$ &  & &   &  \\ \hline
    $0.501$ &  & &   &  \\ \hline
    $0.51$ &  &  &   &  \\ \hline
    \hline
  \end{tabular}
\end{center}


(e) Does  your table from part (d) suggest that the ratios $\Delta s/ \Delta t$ approach some number as $t\to 0.5$? If so, what would be your guess for the exact value of this number? What are its units? What is its meaning?

(f) Activate the folder ``Graph of average speed" on Line 9. 

(i) Use the graph to check some your entries in the fifth column of your table from part (d). Explain.

(ii) How is the line $PQ$, through the fixed point $P(0.5,10)$ and the variable point $Q(t,s)$ on the graph of the function $r=g(t)$ related to the ratio $\Delta s / \Delta t$?

(iii) Describe what happens to the line $PQ$ as point $Q$ approaches point $P$.

(g) For a quicker way to approximate car's speed at 12:30pm, zoom in sufficiently close to point $P$ in the graph above to make the graph of $s=f(t)$ look like a line. Use the coordinates of point $P$ and a second point in the window far away from $P$ to estimate the car's speed at 12:30pm. Explain your method.

(h) Summarize your conclusions by comparing your three estimates for the car's speed at 12:15pm. Which estimate do you think is most accurate? Least accurate?

\end{example}


\begin{example} \label{Exsdf4r55ytyh}
This is a continuation of the previous example where we'll algebraically compute the exact speed of the car at 12:30pm, using the odometer function
\[
     s = f(t) = 48t^2 - 16t^3, 0\leq t \leq 2.
\]

The idea is to first find an algebraic expression for the car's average speed between time $t$ and time $t=0.5$ hours past noon. Then we'll evaluate the limit of this average speed as $t\to 0.5$ to find the (instantaneous) speed at 12:30pm.

\pskip


\begin{question} \label{Q:43fggte}
First we'll find the average speed between time $t$ and time $t=0.5$. 

(a) Explain in general how to compute a car's average speed over some time interval. What do you need to know? What is the computation? Make up your own specific example.

(b) Now for our particular odometer function above, the average speed $v_{\text{avg}}(t)$, measured in miles/hour, between time $t$ and time $t=0.5$ is
\begin{align*}
   v_{\text{avg}}(t) &= \frac{f(t) - f(0.5)}{t - 0.5} \\
                        &= \frac{48t^2 - 16t^3 - 10 }{t-0.5} \\
                        &= \frac{(2t-1)(\answer{-8t^2 + 20t + 10})}{t-0.5} \\
                        & = \answer{-16t^2+40t + 20} \text{ if } t\neq \answer{0.5}.
\end{align*}

The key step in the computation above is in the third line. How did we know $2t-1$ was a factor of 
\[
   f(t) - f(0.5) = 48t^2 - 16t^3 - 10 ?
\]  
The reason is that $t=0.5$ is a root of the polynomial $f(t)-f(0.5)$ and therefore $t - 0.5$ is a factor. And so 
\[
   2(t-0.5) = 2t-1
\]
is also a factor. Then we can use long division to find the quotient.

(c) Show the steps in the long division.

(d) The final step in computing the car's speed $v$ (in miles/hour) at 12:30pm is to evaluate the limit of these average speeds as $t\to 0.5$. We get
\begin{align*}
    v &= \lim_{t\to 0.5} (\answer{-16t^2+40t + 20}) \\
       & = \answer{36} .
\end{align*}


(e) Here's another way to simplify the average speed in part (b). Fill in the missing steps.
\begin{align*}
   v_{\text{avg}}(t) &= \frac{f(t) - f(0.5)}{t - 0.5} \\
                        &= \frac{(48t^2 - 16t^3) - (48(0.5)^2 -16(0.5)^3)}{t-0.5} \\
                        &= \frac{(48t^2 - 48(0.5)^2) - (16t^3 -16(0.5)^3) }{t-0.5} \\
                        &= \frac{48(t^2 - (0.5)^2) - 16(t^3 -(0.5)^3) }{t-0.5} \\
                       &= \frac{48(t-0.5)(\answer{t+0.5})) - 16(t-0.5)(\answer{t^2 + 0.5 t + 0.25}) }{t-0.5} \\
                       &=  48(\answer{t+0.5})) - 16(\answer{t^2 + 0.5 t + 0.25})  \text{ if } t\neq \answer{0.5}.
\end{align*}

(f) Use the above expression for the average speed function to compute the (instantaneous) speed of the car at 12:30pm by evaluating the approriate limit.

(g) Sketch \emph{by hand} a graph of the average speed function $y=v_{avg}(t)$ over the appropriate domain. Be sure also to state this function's domain. 

\end{question}
\end{example}


\section{A Projectile}

\begin{example}   \label{Ex4erxde}


\pdfOnly{
Access Desmos interactives through the online version of this text at
 
\href{https://www.desmos.com/calculator/l4fknr0hpl}.
}
 
\begin{onlineOnly}
    \begin{center}
\desmos{l4fknr0hpl}{900}{600}
\end{center}
\end{onlineOnly}

Desmos activity available at

\href{https://www.desmos.com/calculator/l4fknr0hpl}{151: Projectile}

\pskip \pskip


\end{example}


\section{The Falling Ladder, Part 1}


\begin{example}   \label{Ex435rsfeetr}

The top end of a ten-foot ladder leans against a vertical wall and the bottom end rests on the horizontal floor. We analyze how a small change in the distance between the wall and the bottom of the ladder affects the height of the ladder's top above the floor. 

\pdfOnly{
Access Desmos interactives through the online version of this text at
 
\href{https://www.desmos.com/calculator/dvyuifyyg4}.
}
 
\begin{onlineOnly}
    \begin{center}
\desmos{dvyuifyyg4}{900}{600}
\end{center}
\end{onlineOnly}

Desmos activity available at

\href{https://www.desmos.com/calculator/dvyuifyyg4}{151: Ladder 1B}

\pskip \pskip

\begin{question}  \label{Qdsfesarr4}

(a) The slider $s$ in Line 1 of the demonstration above controls the distance between the wall and the bottom of the ladder, measured in feet. Use the slider $s$ to describe \emph{qualitatively} how a small change in $s$ changes the height $h$ (also measured in feet) of the ladder's top end above the floor. 

\pskip

(i) Do the small changes have the same or opposite signs?

(ii) At what positions of the ladder does a small change in $s$ result in a comparatively large change in $h$?

\pskip

(b) Now let's focus on the particular position of the ladder where the bottom end is  $s=8$ feet from the wall. For this, turn on the ``one position" folder in Line 3.

\pskip

(i) Drag the slider $s$ close to $s=8$ and use the cordinates of the endpoints of the ladder to construct a table with five columns showing the values of $s$, $h$, $\Delta s = s - 8$, $\Delta h = f(s)-f(8)$, and $\Delta h / \Delta s$. Include units in the heading of each column. The table should include seven rows, with $s=7.7, 7.8, \ldots 8.2, 8.3$. Here $h=f(s)$ is the function described in part (ii) below.

(ii) Find a function
\[
  h = f(s) , 0\leq s \leq 10 ,
\]
that expresses the height of the ladder's top end above the ground (in feet) in terms of the distance of its bottom end from the wall (in feet). %Activate the folder ``graph of function" in Line 8.

(iii) Use your function $f$ to construct another table, like the one in part (i), with $s=7.9, 7.99, 7.999,8, 8.0001, 8.01, 8.1$. Do this by finding expressions for 
\[
    \Delta h = f(s) - f(8)
\]
and
\[
   r = g(s) = \frac{\Delta h}{\Delta s} = \frac{f(s) - f(8)}{s-8} ,
\]
both in terms of $s$ (and \emph{not} $\Delta s$).

(iv) Does  your table from part (iii) suggest that the ratios $\Delta h/ \Delta s$ approach some number as $s\to 8$? If so, what would is your guess for the exact value of this limit? What are its units? What is its meaning?

(v) Activate the folder ``graph of function" on Line 8. How is the line $PQ$, through the fixed point $P(8,6)$ and the variable point $Q(s,h)$ on the graph of the function $h=f(s)$ related to the ratios $\Delta h / \Delta s$?

(vi) Change the bounds for $s$ in Line 1 to run between $s=7.9$ and $s=8.1$. Then activate the folder ``graph: average rate of change function" on Line 18. Move the slider $s$ and use the graph of the function $r=g(s)$ to check your computations in part (iii).

(vii) Use the result of part (iv) to write an approximation for the change in height
\[
   \Delta h = h-8
\]
in terms of the change
\[
  \Delta s = s - 8 .
\]  

(viii) The graph of the function $h=f(s)$ suggests another, geometric way to find the proportionality constant (of part (vii)) that relates $\Delta h$ to $\Delta s$. Explain how.

\end{question}

\end{example}

\section{The Falling Ladder, Part 2}


\begin{example}  \label{Ex324trertg}

A tree leans precariously with its trunk making an angle of $\phi = \pi/6$ radians with the ground. One end of a ten-foot ladder leans against the trunk, the other rests on the horizontal ground. We analyze how a small change in the distance between the bottom of the ladder and the base of the trunk changes the distance between the top of the ladder and the base of the trunk.


\pdfOnly{
Access Geogebra interactives through the online version of this text at
 
\href{https://www.geogebra.org/classic/qmke5y7x}.
}
 
\begin{onlineOnly}
    \begin{center}
\geogebra{qmke5y7x}{900}{600}
\end{center}
\end{onlineOnly}

We'll let $t$ be the distance between the top of the ladder and the base of the trunk (measured in feet) and $s$ the distance between the bottom of the ladder and the base of the trunk (also measured in feet).

The slider $\theta$ in the demonstration above controls the angle that the ladder makes with the ground, but this angle does not come into play in our problem.

\pskip

(a) Use the slider $\theta$ to describe qualitatively how a small change in $\color{green}{s}$ (the length of segment $GC$) changes $\color{pink}{t}$ (the length of segment $GB$):

(i) For what positions of the ladder do these small changes have the same signs? Opposite signs?

(ii) For what positions of the ladder does a small change in $s$ result in a comparatively large change in $t$?

\pskip

(b) Now let's focus on the particular position of the ladder when the bottom end $C$ is $16$ feet from the trunk's base and the top end $D$ is about $8$ feet \emph from the base \emph{as illustrated above}. Our first step is to find a function 
\[
   t = f(s) 
\]
that expresses $t$ in terms of $s$ for values of $s$ and $t$ near $s=16$ and $t=8$ respectively. 

To do this, first use the law of cosines to write an equation relating $s$ and $t$. Then \emph{complete the square} to solve this equation for $t$ in terms of $s$ to find the function $t = f(s)$. Keep in mind that when $s=16$, we must have $t\sim 8$. %Note that the inputs and outputs to this function are $


(c) Use your function from part (b) to find an expressions for
\[
  \Delta t = f(s) - f(16)
\]
and for the function
\[
  r = g(s) = \frac{\Delta t}{\Delta s} = \frac{f(s) - f(16)}{s-16} .
\]
Explain what the output of the function $g$ measures. What are its units?


(d) Use the results of part (c) to construct a table with five columns showing the values of $s$, $t$, $\Delta s = s-16$, $\Delta t =f(s)-f(16)$, and $\Delta t / \Delta s$. Include units in the heading of each column. The table should include seven rows, with $s=15.9,15.99, 15.999, 16, 16.001, 16.01, 16.1$.

(e) Does  your table from part (d) suggest that the ratio $\Delta t/ \Delta s$ approaches some number as $s \to 16$? If so, approximate the value of this number. What are its units?

(f) Check the box ``GraphofRelation'' in the demonstration above and explain how the line $EF$ is related to part (d).

(g) Use the result of part (e) to write an approximation for the change
\[
   \Delta t = f(s)- f(16)
\]
in terms of the change
\[
  \Delta s = s - 16  
\]  
for values of $s$ near $16$. Use this approximation to estimate the distance between the top of the ladder and the base of the trunk when the bottom of the ladder is $16.4$ feet from the trunk's base. Compare your approximation with the exact distance.

\end{example}





\section{Riding a Ferris Wheel}


Suppose  you ride a ferris wheel

\end{document}