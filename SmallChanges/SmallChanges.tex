\documentclass{ximera}
\title{Small Changes}

\newcommand{\pskip}{\vskip 0.1 in}

\begin{document}
\begin{abstract}
We explore how small changes to the input of a function change the output.
\end{abstract}
\maketitle

The main idea of differential calculus is to approximate the change in the output of a function in terms of a small change in the input. For some functions, called \emph{differentiable}, the change in the output is approximately proportional to the (small) change in the input. The proportionality factor is called the derivative. In this chapter we explore this idea.


\section{Odometer Reading}

\begin{example}
The graph of the function
\[
    s = f(t) , 0\leq t \leq 2 ,
\]
that expresses the trip odometer reading (measured in miles) on your car in terms of the number of hours past noon is shown below.

\pdfOnly{
Access Desmos interactives through the online version of this text at
 
\href{https://www.desmos.com/calculator/hgr4vmqvbj}.
}
 
\begin{onlineOnly}
    \begin{center}
\desmos{hgr4vmqvbj}{900}{600}
\end{center}
\end{onlineOnly}

Desmos activity available at

\href{https://www.desmos.com/calculator/hgr4vmqvbj}{151: Odometer}

\pskip \pskip

https://www.desmos.com/calculator/hgr4vmqvbj

Our goal is to approximate the car's speed at 12:15pm in three ways:

\pskip

(1) geometrically, using the above graph.

(2) arithmetically, using the specific expression for the function $f$.

(3) algebraically, also using the specific expression for $f$.

\pskip


\end{example}


\section{The Falling Ladder, Part 1}


\begin{example}   \label{Ex435rsfeetr}

The top end of a ten-foot ladder leans against a vertical wall and the bottom end rests on the horizontal floor. We analyze how a small change in the distance between the wall and the bottom of the ladder affects the height of the ladder's top above the floor. 

\pdfOnly{
Access Desmos interactives through the online version of this text at
 
\href{https://www.desmos.com/calculator/dvyuifyyg4}.
}
 
\begin{onlineOnly}
    \begin{center}
\desmos{dvyuifyyg4}{900}{600}
\end{center}
\end{onlineOnly}

Desmos activity available at

\href{https://www.desmos.com/calculator/dvyuifyyg4}{151: Ladder 1B}

\pskip \pskip

\begin{question}  \label{Qdsfesarr4}

(a) The slider $s$ in Line 1 of the demonstration above controls the distance between the wall and the bottom of the ladder, measured in feet. Use the slider $s$ to describe \emph{qualitatively} how a small change in $s$ changes the height $h$ (also measured in feet) of the ladder's top end above the floor. 

\pskip

(i) Do the small changes have the same or opposite signs?

(ii) At what positions of the ladder does a small change in $s$ result in a comparatively large change in $h$?

\pskip

(b) Now let's focus on the particular position of the ladder where the bottom end is  $s=8$ feet from the wall. For this, turn on the ``one position" folder in Line 3.

\pskip

(i) Drag the slider $s$ close to $s=8$ and use the cordinates of the endpoints of the ladder to construct a table with five columns showing the values of $s$, $h$, $\Delta s$, $\Delta h$, and $\Delta h / \Delta s$. Include units in the heading of each column. The table should include seven rows, with $s=7.7, 7.8, \ldots 8.2, 8.3$.

(ii) Find a function
\[
  h = f(s) , 0\leq s \leq 10 ,
\]
that expresses the height of the ladder's top end above the ground (in feet) in terms of the distance of its bottom end from the wall (in feet). %Activate the folder ``graph of function" in Line 8.

(iii) Use your function $f$ to construct another table, like the one in part (i), with $s=7.9, 7.99, 7.999,8, 8.0001, 8.01, 8.1$. Do this by finding expressions for 
\[
    \Delta h = f(s) - f(8)
\]
and
\[
   r = g(s) = \frac{\Delta h}{\Delta s} = \frac{f(s) - f(8)}{s-8} ,
\]
both in terms of $s$ (and \emph{not} $\Delta s$).

(iv) Does  your table from part (iii) suggest that the ratios $\Delta h/ \Delta s$ approach some number as $\Delta s$ approaches zero? If so, what would is your guess for the exact value of this limit? What are its units?

(v) Activate the folder ``graph of function" on Line 8. How is the line $PQ$, through the fixed point $P(8,6)$ and the variable point $Q(s,h)$ on the graph of the function $h=f(s)$ related to the ratios $\Delta h / \Delta s$?

(vi) Change the bounds for $s$ in Line 1 to run between $s=7.9$ and $s=8.1$. Then activate the folder ``graph: average rate of change function" on Line 18. Move the slider $s$ and use the graph of the function $r=g(s)$ to check your computations in part (iii).

(vii) Write an approximation for the change in height
\[
   \Delta h = h-8
\]
in terms of the change
\[
  \Delta s = s - 8 .
\]  

(viii) The graph of the function $h=f(s)$ suggests another, geometric way to find the proportionality constant that relates $\Delta h$ to $\Delta s$. Explain how.

\end{question}

\end{example}

\section{The Falling Ladder, Part 2}


\begin{example}  \label{Ex324trertg}

A tree leans precariously with its trunk making an angle of $\phi = \pi/6$ radians with the ground. One end of a ten-foot ladder leans against the trunk, the other rests on the horizontal ground. We analyze how a small change in the distance between the bottom of the ladder and the base of the trunk affects the distance between the top of the ladder and the base of the trunk.


\pdfOnly{
Access Geogebra interactives through the online version of this text at
 
\href{https://www.geogebra.org/classic/qmke5y7x}.
}
 
\begin{onlineOnly}
    \begin{center}
\geogebra{qmke5y7x}{900}{600}
\end{center}
\end{onlineOnly}

We'll let $s$ be the distance between the top of the ladder and the base of the trunk (measured in feet) and $g$ the distance between the bottom of the ladder and the base of the trunk (also measured in feet).

The slider $\theta$ in the demonstration above controls the angle that the ladder makes with the ground, but this angle does not come into play in our problem.

\pskip

(a) Use the slider $\theta$ to describe qualitatively how a small change in $g$ changes $s$:

(i) For what positions of the ladder do these small changes have the same signs? Opposite signs?

(ii) For what positions of the ladder does a small change in $g$ result in a comparatively large change in $s$?

\pskip

(b) Now let's focus on the particular position of the ladder when the bottom end $C$ is $16$ feet from the trunk's base and the top end $D$ is about $8$ feet \emph from the base \emph{as illustrated above}. Our first step is to find a function 
\[
    s = f(g) 
\]
that expresses $s$ in terms of $g$ for values of $g$ and $s$ near $g=16$ and $s=8$ respectively. 

To do this, first use the law of cosines to write an equation relating $s$ and $g$. Then \emph{complete the square} to solve this equation for $s$ in terms of $g$ to find the function $g = f(s)$. Keep in mind that when $g=16$, we must have $s\sim 8$. %Note that the inputs and outputs to this function are $


(c) Use your function from part (b) to find an expressions for
\[
  \Delta s = f(g) - f(16)
\]
and for the function
\[
  r = h(g) = \frac{\Delta s}{\Delta g} = \frac{f(g) - f(16)}{g-16} .
\]
Explain what the output of the function $h$ measures. What are its units?


(d) Use the results of part (c) to construct a table with five columns showing the values of $g$, $s$, $\Delta g$, $\Delta s$, and $\Delta s / \Delta g$. Include units in the heading of each column. The table should include seven rows, with $g=15.9,15.99, 15.999, 16, 16.001, 16.01, 16.1$.

(e) Does  your table from part (d) suggest that the ratio $\Delta s/ \Delta g$ approaches some number as $\Delta g \to 0$? If so, approximate the value of this number. What are its units?

(f) Check the box ``GraphofRelation'' in the demonstration above and explain how the line $EF$ is related to part (d).

(g) Write an approximation for the change
\[
   \Delta s = s- f(16)
\]
in terms of the change
\[
  \Delta g = g - 16 . 
\]  
Use this approximation to estimate the distance between the top of the ladder and the base of the trunk when the bottom of the ladder is $16.4$ feet from the trunk's base. Compare your approximation with the exact distance.

\end{example}





\section{Riding a Ferris Wheel}


Suppose  you ride a ferris wheel

\end{document}