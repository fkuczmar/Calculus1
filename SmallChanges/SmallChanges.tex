\documentclass{ximera}
\title{Small Changes}

\newcommand{\pskip}{\vskip 0.1 in}

\begin{document}
\begin{abstract}
How does a small change to the input of a function change the output?
\end{abstract}
\maketitle


\section{The Falling Ladder}


\begin{example}   \label{Ex435rsfeetr}

The top end of a ladder leans against a vertical wall and the bottom end rests on the horizontal floor. We wish to analyze how a small change in the distance between the wall and the bottom of the ladder affects the height of the ladder's top above the floor. 

\pdfOnly{
Access Desmos interactives through the online version of this text at
 
\href{https://www.desmos.com/calculator/6at8gegn2b}.
}
 
\begin{onlineOnly}
    \begin{center}
\desmos{hpphciegnp}{900}{600}
\end{center}
\end{onlineOnly}

Desmos activity available at

\href{https://www.desmos.com/calculator/6at8gegn2b}{151: Ladder 1B}

\pskip \pskip

\begin{question}  \label{Qdsfesarr4}

(a) The slider $s$ in the demonstration above controls the distance between the wall and the bottom of the ladder, measured in feet. Use the slider $s$ to describe \emph{qualitatively} how a small change in $s$ changes the height $h$ of the ladder's top end above the floor.


\end{question}


\end{example}


\section{Riding a Ferris Wheel}


Suppose  you ride a ferris wheel

\end{document}