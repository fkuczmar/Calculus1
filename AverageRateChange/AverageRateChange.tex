\documentclass{ximera}
\title{Average Rates of Change}

\newcommand{\pskip}{\vskip 0.1 in}

\begin{document}
\begin{abstract}
Graphing average rate of change functions.
\end{abstract}
\maketitle


With time as the independent variable, the derivative is most naturally interpreted as an instantaneous rate of change. We'll start by reviewing average rates of change.

\begin{question}  \label{Q4ghg5t4t4tr4}
The function
\[
      h = f(t)  \, , \, 0\leq t \leq 2.2 ,                     %= -2t^3+7t^2-8t+8 \, , \, 0\leq t \leq 2.2 ,
\]
expresses the height of a balloon (in hundreds of feet) in terms of the number of minutes past noon.

The graph of the function $h=f(t)$ is shown below.

\begin{onlineOnly}
    \begin{center}
\desmos{yd4xm6x6ub}{450}{600}  
\end{center}
\end{onlineOnly}

\href{https://www.desmos.com/calculator/yd4xm6x6ub}{151: Balloon}

\begin{enumerate}

\item Let $a\in [0,2.2]$ be a constant. Interpret the meaning of the function
\[
   m(v) =  \frac{f(v)-f(a)}{v-a}
\]
in the context of this scenario. Include units. State the domain of the function as well.

\item Use the graph of the function $h=f(t)$ above and the slider $u$ to to sketch a rough graph of the function $r=m(v)$. Label the axes with the appropriate variable names and units.

\item Activate the folder (avg. rate of change function) in Line 11 to check your graph of the function $r=m(v)$. 

\item Click on the appopriate points of the curve $r=m(v)$ to find the balloon's average rate of ascent between 

\begin{enumerate}
\item 12:00pm and 12:02pm

\item 12:01pm and 12:02mp
\end{enumerate}

\item Zoom in close enough near the missing point on the curve $r=m(v)$ to approximate the balloon's rate of ascent at 12:02pm.

\item Open the Table in Line 1 by clicking the \emph{right arrow} at the left of the line. Use it to guess the instantaneous rate of ascent at 12:02pm.

\item Now suppose
\[
   f(t) = -2t^3+7t^2-8t+8 .
\]

\begin{enumerate}
\item Find a simplified expression for the function $r=m(v)$. Include the appropriate domain.

\item Use your simplified expression to compute the balloon's rate of ascent at 12:02pm.
\end{enumerate}

\end{enumerate}
\end{question}

\begin{question}  \label{Qdfsa4555}

\begin{enumerate}

\item Drag the slider $u$ from $u=0$ to $u=1$ in Line 2 below. Then use the graph/animation of the stretching function
\[
   H = f(L) = 0.1 +0.5L + 0.1(L-1)^3 \, , \, 0\leq L \leq 3 ,
\]
to approximate the local stretching factor at $L=2$. Include units. 

\begin{onlineOnly}
    \begin{center}
\desmos{boubpczsne}{450}{600}  
\end{center}
\end{onlineOnly}

\href{https://www.desmos.com/calculator/boubpczsne}{151: Rubber Band 12}

\item Use the graph of the same stretching function $H=f(L)$ show again below to graph (by hand) the function
\[
    r = m(v)
\]
that gives the average stretching factor of the portion of the band between lengths $L=v$ and $L=2$ meters. Label the axes with the appropriate variable names and units. Then activate the average rate of change folder in Line 11 to see how you did.

\begin{onlineOnly}
    \begin{center}
\desmos{vss7ofbwii}{450}{600}  
\end{center}
\end{onlineOnly}

\href{https://www.desmos.com/calculator/vss7ofbwii}{151: Rubber Band 12}

\item Use the graph of the average rate of change function and the Table in Line 1 to guess the local stretching factor at $L=2$.

\item Use the algebra of limits to compute the local stretching factor at $L=2$.


\end{enumerate}


\end{question}

\end{document}