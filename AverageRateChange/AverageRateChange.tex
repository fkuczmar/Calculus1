\documentclass{ximera}
\title{Average Rates of Change}

\newcommand{\pskip}{\vskip 0.1 in}

\begin{document}
\begin{abstract}
Graphing average rate of change functions.
\end{abstract}
\maketitle


With time as the independent variable, the derivative is most naturally interpreted as an instantaneous rate of change. Here's an example. %We'll start by reviewing average rates of change.

\begin{question}  \label{Q4ghg5df5yhhfg4tr4}
Due to a late-season frost in eastern Washington, the price of Cosmic Crisp apples is rising precipitously.

The function
\[
      N = f(t)  \, , \, 0\leq t \leq 8 ,                     %= -2t^3+7t^2-8t+8 \, , \, 0\leq t \leq 2.2 ,
\]
expresses the number of pounds of apples you can buy with $\$36$ in terms of the number of hours past 9am on April 23, 2005. 

The graph of the function $N=f(t)$ is shown below.

\begin{onlineOnly}
    \begin{center}
\desmos{njanpkrqex}{450}{600}  
\end{center}
\end{onlineOnly}

\href{https://www.desmos.com/calculator/njanpkrqex}{151: Apples Drought}

\begin{enumerate}

\item Let $b\in [0,8]$ be a constant. Interpret the meaning of the function
\[
   m(v) =  \frac{f(v)-f(b)}{v-b}
\]
in the context of this scenario. Include units. State the domain of the function as well.

\item Use the graph of the function $N=f(t)$ above and the slider $v$ on Line 4 to to sketch a rough graph of the function $r=m(v)$ when $b=5$. Label the axes with the appropriate variable names and units.

\item Activate the folder (avg. rate of change function) in Line 11 to check your graph of the function $r=m(v)$. 

\item Click on the appopriate points of the curve $r=m(v)$ to find the average rate of change (with respect to time) in the number of pounds of apples you can buy with $\$36$ between 

\begin{enumerate}
\item 11:00 am and 2:00 pm

\item 1:00 pm and 2:00 pm

Be sure to include units. Note that because of the different  scale on the vertical axis for the function $r=m(t)$, you will need to divide the $r$-coordinates by $10$.

\end{enumerate}

\item Zoom in close enough near the missing point on the curve $r=m(v)$ to approximate the rate of change at 2pm in the number of pounds of apples you can buy with $\$36$  at 2:00pm.

\item Open the Table in Line 1 by clicking the \emph{right arrow} at the left of the line. Use it to guess the rate of change (with respect to time) in the number of pounds of apples you can buy with $\$36$  at 2:00pm.

\item Now suppose between 9am and 5pm the price of apples increases at a constant rate. Suppose also that the price is $\$2$/lb at 9am and $\$4$/lb at 2pm. 

\begin{enumerate}
\item Find an expression for the function $N=f(t)$. %Click the arrow to the lower right for a hint.

\emph{Hint:} First find a function 
\[
     P = g(t), 0\leq t \leq 8,
\]
that expresses the price (in dollars/pound) in terms of the number of hours past 9am.


\item Find an expression for the function $r=m(v)$. Include the appropriate domain. %Click the arrow at the lower right for the solution.

\begin{explanation}
Working with the function
\begin{align*}
         N  &= f(t)  \\
             &= \frac{36}{2+\frac{2}{5}t}  \\
             &= \frac{90}{5+t} \, , \, 0\leq t \leq 8,
\end{align*}
we have
\begin{align*}
 m(v) &= \frac{f(v)-f(5)}{v-5} \\
         &= \frac{1}{v-5} \left(   \frac{90}{5+v} - 9   \right) \\
         &= \frac{1}{v-5} \left(   \frac{45-9v}{v+5}  \right) \\
         &= \frac{-9}{v+5} \text{ if } v\neq 5 .
\end{align*}

\end{explanation}

\item Use your expression for $m(v)$ and the algebra of limits to compute the rate of change (with respect to time) at 2pm in the number of pounds of apples you can buy with $\$36$. %Click on the arrow at the lower right for the solution.

\begin{explanation}
The question is asking us to compute the derivative
\[
  \frac{dN}{dt}\Big|_{t=5} = \lim_{v\to 5} \frac{f(v)-f(5)}{v-5} .
\]

Using the previous part, we get
\begin{align*}
\frac{dN}{dt}\Big|_{t=5} &= \lim_{v\to 5} \frac{f(v)-f(5)}{v-5} \\
                                    &= \lim_{v\to 5} \frac{-9}{v+5} \\
                                    &= \frac{-9}{5+5} \\
                                   &= -0.9 .
\end{align*}

\emph{Conclusion:} At 2pm we can buy 5 pounds of apples with $\$36$. And at 2pm the number of pounds of apples we can buy with $\$36$ is decreasing at the rate of $0.9$ lb/hour.
\end{explanation}

\item Compare at 2pm the relative rate of change in the number of pounds of apples you can buy with $\$36$ and the relative rate of change in the price. %Click on the arrow for a solution.

\begin{explanation}
To compute the \emph{relative} rate of change in the price at 2pm, we divide the absolute rate of change, $0.40(\$/\text{lb})/hr$, by the price $P=f(5) = \$4$/lb at that time. This tells us the price is increasing at the relative rate of 
\[
   \frac{1}{P} \cdot \frac{dP}{dt} \Big|_{t=5} = \frac{0.40(\$/\text{lb})/\text{hr}}{4(\$/\text{lb})} = 10\%/\text{hr}
\]
at 2pm.

At the same time, the number of pounds of apples we can buy with $\$36$ is changing at the relative rate of 
\[
  \frac{1}{N} \cdot \frac{dN}{dt} \Big|_{t=5} = \frac{-0.9 \text{ lb/hr}}{9\text{ lb}} = -10\%/\text{hr} .
\]
So at 2pm the number of pounds of apples we can buy with $\$36$ is decreasing at the relative rate of $10\%$/hr.
\end{explanation}

\begin{freeResponse}
Notice anything?
\end{freeResponse}


\end{enumerate}

\end{enumerate}
\end{question}


\begin{question} \label{Q09fdfgsgg}
This question is a continuation of Question 1. We'll work with the same function
\[
     N = f(t) = \frac{36}{2 + \frac{2}{5}t} \, , \, 0\leq t \leq 8
\]
and the expression
\[
 \frac{\Delta N}{\Delta t} = \frac{f(v)-f(b)}{v-b}
\]
for the average rate of change of $N$ (the number of pounds of apples we can buy with $\$36$) with respect to time over the time interval between $t=b$ and $t=v$ hours past 9am.

\begin{enumerate}
\item Use the algebra of limits and the average rate of change function above to compute the derivative
\[
  \frac{dN}{dt}\Big|_{t=b} .
\]

\item Use the result of part (a) to compare at 9am the relative rates of change in the price and the number of pounds of apples we can buy with $\$36$. 

\item Express the derivative from part (a) in terms of $N$ and 
\[
  P  2+ \frac{2}{5}t.
\] 
Use this equation to relate the relative rates of change in the price and the number of pounds of apples we can buy with $\$36$ at any instant. 


\end{enumerate}
\end{question}

\begin{question}  \label{Qopdfpdsgnbhtr}
We can often get more insight into a problem by taking away the numbers. Here we'll generalize the first two questions by supposing the (differentiable) function
\[
         P = g(t) , 0\leq t \leq 8 ,
\]
expresses the price (in dollars/pound) of Cosmic Crisp apples in terms of the number of hours past 9am. And we'll let
\[
          N = f(t) , 0\leq t \leq 8 ,
\]
be the function that expresses the number of pounds of apples we can buy with $\$100$ dollars in terms of he number of hours past 9am.

\begin{enumerate}
\item Find an expression for the function $N=f(t)$ in terms of the function $g(t)$.

\item Use your expression from part (a) to find an expression (in terms of the function $g$) for the average rate of change
\[
 \frac{\Delta N}{\Delta t} = \frac{f(v)-f(b)}{v-b} .
\]

\item Use your expression from part (b) to find an expression for the derivative
\[
  \frac{dN}{dt}\Big|_{t=b} 
\]
in terms of the function $P=g(t)$ and its derivative $dP/dt$.

\item Express the derivative from part (c) in terms of $N$, $P$, and $dP/dt$.

\item Use your equation from part (c) to compare the  relative rates of change in the price and the number of pounds of apples we can buy with $\$100$ at any instant. 
\end{enumerate}

\end{question}


\begin{question} \label{Q:987GHEjerf}
The function
\[
    G = f(s) = \frac{1}{2000}(s-100)^2 \, , \, 0\leq s \leq 80,
\]
expresses the number of gallons of gas in a car in terms of the trip odometer reading in miles. Its graph is shown below.

\begin{onlineOnly}
    \begin{center}
\desmos{igubjr4nw1}{450}{600}  
\end{center}
\end{onlineOnly}

\href{https://www.desmos.com/calculator/igubjr4nw1}{151: Gas Consumption B}

\begin{enumerate}
\item What are the units of the factor $1/2000$ in the function $f$? How do you know?

\item Find the average gas mileage (in mile/gal) between odometer readings $s=20$ and $s=60$ miles.

\item Zoom in on the appropriate point in the graph above to approximate the gas mileage (in miles/gal) at the instant the odometer reads $20$ miles.

\item Find an expression for the average gas mileage between odometer readings $s=b$ and $s=v$ miles. Simplify this expression.

\item Use the previous part to find an expression for the gas mileage at the instant the odometer reads $b$ miles.

\item What is the gas mileage at the instant the odometer reads $s=20$ miles.

\item At what odometer reading is the car getting $20$ miles/gal?

\end{enumerate}

\end{question}



\begin{question}  \label{Q4gdfetdsfsdfhfg4tr4}
The function
\[
      g = f(v)  \, , \, 10\leq v \leq 55 ,                     %= -2t^3+7t^2-8t+8 \, , \, 0\leq t \leq 2.2 ,
\]
expresses the rate (in gal/hr) at which a car burns gas in terms of its speed (in miles/hour).

The graph of the function $g=f(v)$ is shown below.

\begin{onlineOnly}
    \begin{center}
\desmos{3ubcc1x8kh}{450}{600}  
\end{center}
\end{onlineOnly}

\href{https://www.desmos.com/calculator/3ubcc1x8kh}{151: Rate of Gas Consumption}

\begin{enumerate}

\item Let $b\in [10,55]$ be a constant. Interpret the meaning of the function
\[
   m(w) =  \frac{f(w)-f(b)}{w-b}
\]
in the context of this scenario. Include units. State the domain of the function as well.

\item Use the graph of the function $g=f(v)$ above and the slider $w$ on Line 4 to to sketch a rough graph of the function $r=m(w)$ when $b=20$. Label the axes with the appropriate variable names and units.

\item Activate the folder (avg. rate of change function) in Line 12 to check your graph of the function $r=m(v)$. 

\item Click on the appopriate points of the curve $r=m(v)$ to find the average rate of change (with respect to speed) in the fuel consumption rate between speeds of  

\begin{enumerate}
\item 15 miles/hour and 20 miles/hour

\item 20 miles/hour and 30 miles/hour

Be sure to include units. Note that because of the different  scale on the vertical axis for the function $r=m(v)$, you will need to divide the $r$-coordinates by $10$.

\end{enumerate}

\item Zoom in close enough near the missing point on the curve $r=m(v)$ to approximate the rate of change (with respect to speed) in the rate of fuel consumption at a speed of $20$ miles/hr.

\item Open the Table in Line 1 by clicking the \emph{right arrow} at the left of the line. Use it to guess the rate of change (with respect to speed) in the  rate of fuel consumption at a speed of $20$ miles/hr.

\item Now suppose between speeds of $10$ miles/hour and $55$ miles/hour that the car's gas mileage is a linear function of its speed. Suppose also that the car gets $25$ miles/gal at at a speed of $20$ miles/hour and $35$ miles/gal at a speed of $40$ miles/hour.

\begin{enumerate}
\item Find an expression $G=k(v)$ for the function that expresses the car's gas mileage (in miles/gal) in terms of its speed (in miles/hr).

\item Find an expression for the function $g=f(v)$ that expresses the rate (in gal/hr) at which the car burns gas in terms of its speed (in miles/hour). Note: This is not a linear function.

\item Find an expression for the function $r=m(v)$. Include the appropriate domain.

\item Use your expression for $m(v)$ and the algebra of limits to compute the rate of change (with respect to speed) in the rate of fuel consumption at a speed of $20$ miles/hr.

\item Use the language of small changes to interpret the meaning of the derivative
\[
   \frac{dg}{dv}\Big|_{v=20} .
\]

\item Simplify the units of the derivative above. What do these units suggest about its meaning? Is this correct? Why or why not?

\end{enumerate}

\end{enumerate}
\end{question}















\begin{question}  \label{Q4ghg5t4t4tr4}
The function
\[
      h = f(t)  \, , \, 0\leq t \leq 2.2 ,                     %= -2t^3+7t^2-8t+8 \, , \, 0\leq t \leq 2.2 ,
\]
expresses the height of a balloon (in hundreds of feet) in terms of the number of minutes past noon.

The graph of the function $h=f(t)$ is shown below.

\begin{onlineOnly}
    \begin{center}
\desmos{yd4xm6x6ub}{450}{600}  
\end{center}
\end{onlineOnly}

\href{https://www.desmos.com/calculator/yd4xm6x6ub}{151: Balloon}

\begin{enumerate}

\item Let $b\in [0,2.2]$ be a constant. Interpret the meaning of the function
\[
   m(v) =  \frac{f(v)-f(b)}{v-b}
\]
in the context of this scenario. Include units. State the domain of the function as well.

\item Use the graph of the function $h=f(t)$ above and the slider $u$ to to sketch a rough graph of the function $r=m(v)$ when $b=2$. Label the axes with the appropriate variable names and units.

\item Activate the folder (avg. rate of change function) in Line 11 to check your graph of the function $r=m(v)$. 

\item Click on the appopriate points of the curve $r=m(v)$ to find the balloon's average rate of ascent between 

\begin{enumerate}
\item 12:00pm and 12:02pm

\item 12:01pm and 12:02pm
\end{enumerate}

\item Zoom in close enough near the missing point on the curve $r=m(v)$ to approximate the balloon's rate of ascent at 12:02pm.

\item Open the Table in Line 1 by clicking the \emph{right arrow} at the left of the line. Use it to guess the instantaneous rate of ascent at 12:02pm.

\item Now suppose
\[
   f(t) = -2t^3+7t^2-8t+8 \, , \ 0\leq t \leq 2.2.
\]

\begin{enumerate}
\item Find a simplified expression for the function $r=m(v)$. Include the appropriate domain.

\item Use your simplified expression to compute the balloon's rate of ascent at 12:02pm.
\end{enumerate}

\end{enumerate}
\end{question}

\begin{question}  \label{Qdfsa4555}

\begin{enumerate}

\item Drag the slider $u$ from $u=0$ to $u=1$ in Line 2 below. Then use the graph/animation of the stretching function
\[
   H = f(L) = 0.1 +0.5L + 0.1(L-1)^3 \, , \, 0\leq L \leq 3 ,
\]
to approximate the local stretching factor at $L=2$. Include units. 

\begin{onlineOnly}
    \begin{center}
\desmos{boubpczsne}{450}{600}  
\end{center}
\end{onlineOnly}

\href{https://www.desmos.com/calculator/boubpczsne}{151: Rubber Band 12}

\item Use the graph of the same stretching function $H=f(L)$ show again below to graph (by hand) the function
\[
    r = m(v)
\]
that gives the average stretching factor of the portion of the band between lengths $L=v$ and $L=2$ meters. Label the axes with the appropriate variable names and units. Then activate the average rate of change folder in Line 11 to see how you did.

\begin{onlineOnly}
    \begin{center}
\desmos{vss7ofbwii}{450}{600}  
\end{center}
\end{onlineOnly}

\href{https://www.desmos.com/calculator/vss7ofbwii}{151: Rubber Band 12}

\item Use the graph of the average rate of change function and the Table in Line 1 to guess the local stretching factor at $L=2$.

\item Use the algebra of limits to compute the local stretching factor at $L=2$.


\end{enumerate}


\end{question}

\end{document}