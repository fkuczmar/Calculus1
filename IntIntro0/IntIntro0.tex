\documentclass{ximera}
\title{Introduction to Integration, Part 1}

\newcommand{\pskip}{\vskip 0.1 in}

\begin{document}
\begin{abstract}
Introduction to integral calculus.
\end{abstract}
\maketitle

The \emph{differential} in Differential Calculus has the same root as \emph{difference}, and \emph{calculus} the same root as \emph{compute}. Differential calculus is all about \emph{computing differences}. Expressed more simply, it is about \emph{subtraction}.

\emph{Integral Calculus}, on the other hand, is about addition. \emph{Integral} shares the same root as \emph{integer} and comes from the Latin \emph{integrare}, to make whole. 

Addition and subtraction are inverse operations. They undo each other. Similarly, differentiation undoes integration. But we need to be more precise to claim that integration undoes differentiation. More about this later.

It would be a mistake to think integral calculus is primarily about finding anti-derivatives. While anti-differentiation plays a role in integration, this class is more about recongizing \emph{when} to integrate rather than \emph{how} to integrate. Many of your STEM classes use definite integrals, with an emphasis on setting up and interpreting integrals, not computing their values. 

It would also be a mistake to think this class is about using definite integrals to compute \emph{numbers}, like the distance travelled over a certain time interval or the period of a pendulum. Rather, the emphasis here will be on using definite integrals to create \emph{functions}. Expressing the position of a pendulum as a function of time, for example, gives a lot more information that just computing its period. 

Using definite integrals to write functions generalizes the point-slope equation of a line. You used this equation in differential calculus is to approximate a function in the neighborhood of a point. To capture the local nature of the approximation (ie. the approximation is usually accurate in just a small neighborhood of the point of tangency), it's the point-slope equation, \emph{not} slope-intercept, that works best. Here's an example that illustrates this idea.   % the main idea of differential calculus.

\begin{example}  \label{Ex:IUDFr3f3fgl}
Between 12:03pm and 1:00pm a balloon ascends at a constant rate of $73$ ft/min. The balloon is $1900$ feet high at 12:10pm.

\begin{enumerate}
\item Find an expression for the function 
\[
       h=g(t)\; , \; 0\leq t \leq 60, 
\]
that expresses the balloon's height (measured in feet) in terms of the number of minutes past noon.

\item Explain the logic behind your expression.

\begin{explanation}
The function is
\[
    h = g(t) = 1900 + 73(t-10) \, , \, 0\leq t \leq 60.
\]
The logic behind the expression is this. To find the height at time $t$ minutes past noon, we start with the height ($1900$ feet) at 12:10pm and add the change in height from 12:10pm to time $t$ minutes past noon. The change in height,
\[
   \Delta h = 73(t-10)
\]
is the product of the rate of ascent ($73$ ft/min) and the number of minutes
\[
  \Delta t = t - 10
\]
from 12:10pm to time $t$ minutes past noon.
\end{explanation}
\end{enumerate}
\end{example}

In the context of the last example, the main topic for this class is how to compute the height function \emph{if the rate of ascent is not constant.}

The logic behind the solution is almost identical to point-slope. Supppose we're given a function
\[
   r =f(t) \; , \; 0\leq t \leq 60, 
\]
that gives the balloon's rate of ascent (in ft/min) at a general time $t$ minutes past noon. To find the height at time $t$, we start with the height of the balloon at 12:10 pm and add to that the change in height from 12:10pm to time $t$. But because the rate of ascent is not necessarily constant, we'll almost certainly have (infinitely) many rates. Our job is to use these rates to compute, or at least approximate the change in height. The next example illustrates this idea.

\begin{example}  \label{Ex:98dfrghha}
Suppose we know the balloon is $1900$ feet high at 12:10pm and that we're given a graph of the function
\[
    r = f(t) \, , \, 0\leq t \leq  60 , 
\]
(assumed continuous) that expresses its rate of ascent (measured in ft/min) in terms of the number of minutes past noon. This graph is shown below.

\begin{onlineOnly}
    \begin{center}
\desmos{h6cworakdw}{450}{600}  
\end{center}
\end{onlineOnly}

\href{https://www.desmos.com/calculator/h6cworakdw}{152:Balloon Introduction 2}

Our problem is to approximate the balloon's height at 12:50pm.

We'll start by computing a rough approximation of the change in height from 12:10pm and 12:50pm. For this, we'll partition the $40$-minute interval into five equal subintervals of length 
\[
  \Delta t = \frac{40\text{ min}}{5} = 8 \text{ min}.
\]
 
Now we'll suppose that during each subinterval the balloon ascends at a constant rate. While this is not true, the rate certainly varies less over an $8$-minute interval than it does over the entire $40$-minute interval. So we should get a better appoximation  this way than if we assumed just one constant rate of ascent between 12:10pm and 12:50pm.

What rate we choose for each $8$-minute interval doesn't really matter as long as we choose the actual rate of ascent at some time during that interval. To make things simple, we'll choose the rate at the start of each subinterval. 

Then between 12:10pm and 12:18pm, the approximate change in height (measured in feet) is %the product 
\[
        f(10)(8) \sim (31 \text{ ft/min})(8 \text{ min}) = 248 \text{ ft}. 
\]
Between 12:18pm and 12:26pm, the approximate change in height (measured in feet) is
\[
        f(18)(8) \sim (12 \text{ ft/min})(8 \text{ min}) = 96 \text{ ft}. 
\]
And we continue similarly to approximate the changes in height (this time negative) over the remaining three 8-minute intervals. So from 12:10pm to 12:50pm the approximate change in height (in feet) is
\begin{align*}
  \Delta h_5 &= h - 1900  \\
               &\sim (f(10) + f(18) + f(26)+ f(34) + f(42))(8) \\
               &\sim -448 .
\end{align*}

\emph{Verify this computation by typing this expression in Line 38 of the worksheet above.}

\emph{This sum is on Line 7 of the worksheet.}

To see the previous computation geometrically, move the slider $a$ (the start of our time interval) in Line 2 above to $a=10$ and the slider $b$ (the end of our time interval) in Line 3 to $b=50$. Then the signed area of a shaded retangle measures the approximate change in height during that subintevral. And the approximate change in height from 12:10pm to 12:50pm is just the sum of these signed areas. The sign of the area and the rate of ascent are the same. So rectangles below the horizontal axis contribute a negative signed area (ie. a decrease in height), while those above make a positive contribution (ie. an increase in height).

Our first approximation to the balloon's height at 12:50pm is then 
\[
    (1900  - 448 )\text{ ft} \sim 1452 \text{ ft}.
\]

To get a better approximation, we partition the $40$-minute interval from 12:10pm to 12:50pm into more subintervals of equal length. To do this, drag the slider $n$ on Line 5 (the number of equal subintervals) to the right.

\begin{enumerate}
\item Describe what happens to the approximate change in height ($s_1$ in Line 5) and how the picture changes as $n$ grows without bound. 

\item Then approximate the exact change in height and interpret that change geometrically. 

\end{enumerate}
\begin{freeResponse}
\end{freeResponse}

\end{example} 

\noindent {\bf Takeaways:}

Here's some notation along with a summary of the main idea for the last example.
 
The exact height of the balloon at 12:50pm is 
\[
     1900 + \int_{10}^{50} f(t)\, dt .
\]
This expression starts with the height at 12:10pm (1900 feet) and adds to it the change in height
\[
   \int_{10}^{50} f(t)\, dt 
\]
from 12:10pm to 12:50pm. The integral sign $\int$ stands for the letter \emph{S} and means \emph{sum}. You can think of the product
\[
   f(t) \, dt
\]
as a small change in height. It multiplies the rate of ascent $f(t)$ and a small (differential) time interval $dt$ to give a small (differential) change in height. The integral then sums these changes to get the exact change in height.

More generally, the height of the balloon as a function of time (measured in minutes past noon) is
\[
  h = f(t) = 1900 + \int_{10}^{t} f(u)\, du
\]
and the logic is the same. 

Compare this with the expression (point-slope) from Example 1 for the height
\[
  h = 1900 + 73(t-10).
\]
\begin{freeResponse}
\end{freeResponse}

\end{document}