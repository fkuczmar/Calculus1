\documentclass{ximera}
\title{Introduction to Integration, Part 1}

\newcommand{\pskip}{\vskip 0.1 in}

\begin{document}
\begin{abstract}
Introduction to integral calculus.
\end{abstract}
\maketitle

The \emph{differential} in Differential Calculus has the same root as \emph{difference}, and \emph{calculus} the same root as \emph{compute}. Differential calculus is all about \emph{computing differences}. Expressed more simply, it is about \emph{subtraction}.

\emph{Integral Calculus}, on the other hand, is about addition. \emph{Integral} shares the same root as \emph{integer} and comes from the Latin \emph{integrare}, to make whole. 

Addition and subtraction are inverse operations. They undo each other. And in some sense (to be made precise later) integration
undoes differentiation. But it would be a mistake to think integral calculus is about finding anti-derivatives. While anti-differentiation plays a role, this class is more about recongizing \emph{when} to integrate rather than \emph{how} to integrate. Many of your STEM classes use definite integrals, with an emphasis on setting up and interpreting integrals, not computing their values. 

It would also be a mistake to think this class is about using definite integrals to compute \emph{numbers}, like the distance travelled over a certain time interval or the period of a pendulum. Rather, the emphasis here will be on using definite integrals to create \emph{functions}. Expressing the position of a pendulum as a function of time, for example, gives a lot more information that just computing its period. 

Using definite integrals to write functions generalizes the point-slope equation of a line. You used this equation in differential calculus is to approximate a function in the neighborhood of a point. To capture the local nature of the approximation (ie. the approximation is usually accurate in just a small neighborhood of the point of tangency), it's the point-slope equation, \emph{not} slope-intercept, that works best. The next two examples illustrate this idea.   % the main idea of differential calculus.

\begin{example}  \label{Ex:IUDFr3f3fgl}
Between 12:14pm and 1:00pm the temperature of a beaker of water increases at a constant rate of $\frac{33}{13}^\circ$C/sec. The temperature is $23^\circ$C at 12:23pm.

\begin{enumerate}
\item Find an expression for the function 
\[
       C=f(t)\; , \; 14\leq t \leq 60, 
\]
that expresses the temperature of the water (in Celsius degrees) in terms of the number of minutes past noon.

\item Explain the logic behind your expression.

\end{enumerate}
\end{example}



\begin{example}  \label{Ex:98dfrghha}

The function 
\[
    r = f(t) \, , \, 0\leq t \leq  60 , 
\]
expresses a balloon's rate of ascent (measured in ft/min) in terms of the number of minutes past noon. Its graph is shown below.

\begin{onlineOnly}
    \begin{center}
\desmos{tgi5yiuzab}{450}{600}  
\end{center}
\end{onlineOnly}

\href{https://www.desmos.com/calculator/tgi5yiuzab}{152:Balloon Introduction}

\begin{enumerate}

\item When is the balloon ascending? Descending? Give all times for each and explain your reasoning.

\item Do you think the balloon is higher at noon or at 1:00pm? Explain your reasoning. Try to prove your assertion using only arithmetic and simple reasoning.

\item Now suppose the balloon is $2500$ feet above the ground at 12:34pm. 

\begin{enumerate}

\item Use the graph to approximate the balloon's height $30$ seconds later. Explain your reasoning.

\item Is your estimate greater or less than the actual height? Explain your reasoning.

\end{enumerate}
\end{enumerate}
\end{example}

We can turn the previous question of differential calculus into an addition problem of integral calculus by estimating the balloon's height at a time farther removed from 12:34. 

\begin{example} \label{ExLKDrDEfRE9}

We'll assume as before that the balloon is $2500$ feet above the ground at 12:34pm, but now we'll approximate its height at 1:00pm.

\begin{onlineOnly}
    \begin{center}
\desmos{h6cworakdw}{450}{600}  
\end{center}
\end{onlineOnly}

\href{https://www.desmos.com/calculator/h6cworakdw}{152:Balloon Introduction 2}

We'll start by computing a rough approximation of the change in height between 12:34pm and 1:00pm. For this, we'll partition the $26$-minute interval into five equal subintervals of length 
\[
  \Delta t = \frac{26\text{ min}}{5} = 5.2 \text{ min}.
\]
 
Now we'll suppose that during each subinterval the balloon ascends at a constant rate. While this is not true, the rate certainly varies less over a $5.2$-minute interval than it does over the entire $26$-minute interval. So we should get a better appoximation  than if we assumed just one constant rate of ascent between 12:34pm and 1:00pm.

What rate we choose for each $5.2$-minute interval doesn't really matter as long as we choose the actual rate of ascent at some time during that interval. To make things simple, we'll choose the rate at the start of each subinterval as illustrated in the worksheet above. Then the signed area of a shaded retangle measures the approximate change in height during that subintevral. And the approximate change in height from 12:34pm to 1:00pm is just the sum of these areas.

\begin{enumerate}
\item Use the graph above to appoximate the change in height. Then approximate the height at 1:00pm.

\item Write an expression for the approximate change in height from 12:34pm to 1:00pm in terms of the unknown function $f$, and then an expression for the approximate height at 1:00pm.

\item Use summation notation to express the change in part (b). Type this expression in Line 5 of the worksheet above. Compare this value with your estimate from part (a).

The approximate change in height from 12:34pm to 1:00pm is
\[
  1.2 \sum_{i=0}^{\answer{4}} f(\answer{34 + 1.2i})  
\]
and the approximate height at 1:00pm is
\[
     \answer{2500} + 1.2 \sum_{i=0}^{\answer{4}} f(\answer{34 + 1.2i})  .
\]

\item Repeat parts (a)-(c) using the rate of ascent at the \emph{end} of each subinterval instead. Turn off the folder in Line 11 and activate the folder in Line 18 above. As a check, type your expression for the approximate height in Line 5 above.

The approximate change in height from 12:34pm to 1:00pm is
\[
  1.2 \sum_{i=1}^{\answer{5}} f(\answer{34 + 1.2i})  
\]
and the approximate height at 1:00pm is
\[
     \answer{2500} + 1.2 \sum_{i=1}^{\answer{5}} f(\answer{34 + 1.2i})  .
\]

\item Repeat parts (a)-(c) using the rate of ascent at the \emph{middle} of each subinterval instead. Turn off the folder in Line 18 and activate the folder in Line 24 above. As a check, type your expression for the approximate height in Line 5 above.

The approximate change in height from 12:34pm to 1:00pm is
\[
  1.2 \sum_{i=0}^{\answer{4}} f(\answer{34.6+ 1.2i})  
\]
and the approximate height at 1:00pm is
\[
     \answer{2500} + 1.2 \sum_{i=0}^{\answer{4}} f(\answer{34.6 + 1.2i})  .
\]

\item Compare your three estimates for the change in height. Which do you think is most accurate? Explain why.

\end{enumerate}
\end{example} 

\begin{example}  \label{Ex:IjdRJrehreDF}
\begin{enumerate}
\item  Drag the slider in Line 4 of Example 3 to $n=20$ and repeat parts (c) - (e) of Example 3 with $n=20$ equal subintervals. 

\item Try parts (c) - (e) of Example 3 in general, with $n$-equal subintervals.

Using left endpoints, the approximate height at 1:00pm is
\[
        f(60) \sim   \answer{2500} +  \answer{\frac{26}{n}} \sum_{i=0}^{\answer{n-1}} f\left(\answer{34+ \frac{26}{n}i}\right)  .
\]
With right endpoints,
\[
        f(60) \sim   \answer{2500} +  \answer{\frac{26}{n}} \sum_{i=1}^{\answer{n}} f\left(\answer{34+ \frac{26}{n}i}\right)  .
\]
And with midpoints,
\[
        f(60) \sim   \answer{2500} +  \answer{\frac{26}{n}} \sum_{i=0}^{\answer{n-1}} f\left(\answer{34.6+ \frac{26}{n}i}\right)  .
\]

\end{enumerate}
\end{example}


\end{document}