\documentclass{ximera}
\title{The Limits to Growth}

\newcommand{\pskip}{\vskip 0.1 in}


\begin{document}
\begin{abstract}
Depletion of the world's resources.
\end{abstract}
\maketitle

%\begin{image}  
%\includegraphics[width=\textwidth]{Graph1.png}  
%\end{image}

\emph{The Project on the Predicament of Mankind} was initiated in 1968 by a group of 30 individuals having a wide range of expertise with the intent of examining the complex problems troubling all nations at that time and that still persist today. Within a year thay had created a model of the five basic factors they identified as limiting growth  - population, agricultural production, natural resources, industrial production, and polution. The model  provided valuable insights into how these areas interact with each other to limit growth. The findings are summarized in the 1972 classic \emph{The Limits to Growth}, available at

\href{https://www.library.dartmouth.edu/digital/digital-collections/limits-growth}{Limits to Growth} .

Here we focus on their analysis of the depletion of the world's chromium reserves as illustrated in the figure below. %This model might be turned into a meaningful project for a Math 152 class.

\pdfOnly{
Access Desmos interactives through the online version of this text at
 
\href{https://www.desmos.com/calculator/9kqmbczco7}.
}
 
\begin{onlineOnly}
    \begin{center}
\desmos{9kqmbczco7}{900}{600}
\end{center}
\end{onlineOnly}




\pskip \pskip

Chromium is just one of 19 non-renewable natural resources analyzed. For each, the authors provide the following information.

\begin{itemize}
\item{The known global reserves in 1970; for chromium $7.75\times 10^8$ tons}

\item{The static index. This is the number of years it the global reserves would last at the current global consumption rate. The static index for chromium is $420$ years.}

\item{The projected rate of growth in the global consumption rate. For chromium, the nominal growth rate is $2.6\%$/yr, compounded continuously.}

\end{itemize}

For each resource, the authors compute the \emph{exponential index}. This is the number of years the known current reserves would last with consumption growing exponentially at the average rate of growth. They also compute the exponential index using five times the known reserves. 

\begin{question}   \label{Q34fratr5}
Let's work in general (like the authors) and let $r$ be the nominal annual growth rate (compounded continuously) in the global consumption rate and let $s$ be the static index. Our goal here is to express  the exponential index in terms of $r$ and $s$.  
%But more broadly, our aim is to reproduce the graphs in Figure 11 (Chromium Reserves) above and reproduced below. 



\pdfOnly{
Access Desmos interactives through the online version of this text at
 
\href{https://www.desmos.com/calculator/er2jq9vzeq}.
}
 
\begin{onlineOnly}
    \begin{center}
\desmos{er2jq9vzeq}{900}{600}
\end{center}
\end{onlineOnly}

Access this worksheet at

\href{https://www.desmos.com/calculator/er2jq9vzeq}{Limits to Growth 2}

\pskip \pskip

We'll let $R_0$ be the 1970 known global reserves (in units of hundreds of millions of tons). Use this parameter along with the parameters $r$ and $s$ as need be, to find expressions for the following:

\pskip 

\begin{enumerate}

\item The usage rate function 
\[
     u = f(t) , 0\leq t \leq 240,
\]
that expresses the usage rate (measured in $10^8$ tons/year) in terms of the number of years since 1970. Recall the assumption that the usage rate increases exponentially at the annual nominal rate of $r \text{ yr}^{-1}$, compounded continuously. This means that
\[
    u  = f(t) = u_0 e^{rt}, 0\leq t \leq 240
\]
where $u_0$ was the usage rate in 1970. All that you need to do is to express $u_0$ in terms of $R_0$ and the static index $s$.


The graph of this function does not really belong on the same coordinate system as the others. Why not?

\item The remaining reserves function
\[
    R = g(t) , 0\leq t \leq 240,
\]
that expresses the remaining global reserves (in hundreds of millions of tons) in terms of the number of years since 1970, assuming a usage rate that increases exponentially at the nominal rate of $r \text{ yr}^{-1}$, compounded continuously.

\item The exponential index.


%(e) the remaining reserves function
%\[
 %   R = h(t) , 0\leq t \leq 240,
%\]
%that expresses the remaining global reserves (in hundreds of millions of tons) in terms of the number of years since 1970, %assuming a usage rate that increases exponentially at the nominal rate of $r \text{ yr}^{-1}$, compounded continuously, and an initial reserve equal to five times the 1970 reserve.

%\pskip \pskip

%Next, find expressions (in terms of $R_0$, $r$, and $s$ as need be - do not use specific numbers) for the coordinates of the following points and explain the significance of each (if possible):

%(a) the point where the curve $R=g(t)$ intersects the $t$-axis.

%(b) the point where the curve $R=h(t)$ interesects the $t$-axis.

%(c) the point where the curve $u=f(t)$ intersects the line $R=R_0$.

%(d) the point where the curves $u=f(t)$ and $R=g(t)$ intersect.

%\pskip \pskip

\end{enumerate}

%Finally, drag the slider $u$ and observe that some points appear to collide. Do they actually collide? Can you find any significance to the apparent collision times?

\end{question}

\end{document}

