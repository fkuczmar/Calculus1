\documentclass{ximera}
\title{Short Quizzes Math 142}

\newcommand{\pskip}{\vskip 0.1 in}

\begin{document}
\begin{abstract}
Short Quizzes Math 142
\end{abstract}
\maketitle

\begin{question}   \label{Q45hbrgrgrhgr}
Let $Q$ be the point on the circle of radius $40$ meters that is $90$ meters from the point $A(40,0)$ (coordinates measured in meters). The distance is measured \emph{clockwise} around the circle from $A$ to $Q$.

Between 12:06pm and 1:00pm a beetle crawls counterclockwise around this circle at a constant speed of $5$ meters/min, passing the point $Q$ at 12:43pm. 

\begin{enumerate}

\item Find an expression for the function
\[
  \theta = f(t) \, , \, 6\leq t \leq 60,
\]
that expresses the polar angle (meausred in radians) of the beetle (more precisely of the vector giving the beetle's position relative to the origin) in terms of the number of minutes past noon being sure to do the following:

\begin{enumerate}
\item Explain your reasoning. This means to include a brief description of what you are computing for \emph{each} of your computations.

\item Include units for each number in each computation.

\item Include a graph (drawn by hand) of the function $\theta=f(t)$ to help with your explanation. 

\item Include a sketch (drawn by hand) of the circle on a coordinate system with appropriately labeled points and arclengths. Label the axes with the appropriate variable names and units.

\end{enumerate}

\item Find functions
\[
   x = g_1(t)  \, , \, 6\leq t \leq 60,
\]
and
\[
     y = g_2(t) \, , \, 6\leq t \leq 60,
\]
that express the coordinates (in meters) of the beetle in terms of the number of minute past noon. Include a brief explanation.

\item Input the correct coordinate functions on Lines 2 and 3 of the worksheet below. Then play the slider $u$ (another name for $t$, the number of minutes past noon) in Line 1 to see if your functions are correct. 

\begin{onlineOnly}
    \begin{center}
\desmos{lkrunhfgxi}{900}{600}
\end{center}
\end{onlineOnly}

\href{https://www.desmos.com/calculator/lkrunhfgxi}{142: Short Quiz 1}

\begin{enumerate}
\item Explain how you know that your functions are correct.

\item Include at least two screenshots of the animation at different times to help with your explanation.
\end{enumerate}
\end{enumerate}

\end{question}

\begin{question}  \label{Qdhftghhghnbnnh}
Solve the equation
\[
        9 - 3 \sin \theta = 11 .
\]

\begin{enumerate}
\item Explain your reasoning thoroughly, in complete sentences.

\item Include a picture as in class to help with your explanation.

\item End with a concluding sentence that expresses the solution as a set.
\end{enumerate}

\end{question}


\end{document}