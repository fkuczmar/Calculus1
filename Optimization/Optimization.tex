\documentclass{ximera}
\title{Optimization}

\newcommand{\pskip}{\vskip 0.1 in}

\begin{document}
\begin{abstract}
Optimization.
\end{abstract}
\maketitle


\begin{question} \label{QOepREPER}
Use calculus to find an expression for the $x$-coordinate of the turning point of the parabola
\[
    y = ax^2 + bx + c \, , \, c\neq 0.
\]
\end{question}

\begin{question} \label{Q545hhghnnggZ}
Determine the minimum and maximum values of the function
\[
     y = f(x) = x^3 - 9x \, , \, -3\leq x \leq 0 .
\]
\end{question}

\begin{question} \label{Q5hdfhgeyghhg}
The function
\begin{align*}
    G  &= f(v)    \\
         &=  -\frac{v^2}{20} + 5v -90 \, , \, 35\leq v \leq 70 ,
\end{align*}
graphed below expresses the gas mileage (in miles/gal) of a car in terms of its speed (in miles/hour).

\begin{onlineOnly}
    \begin{center}
\desmos{bcj0k1fymu}{450}{600}  
\end{center}
\end{onlineOnly}

\href{https://www.desmos.com/calculator/bcj0k1fymu}{151: Burning Gas 2}

\begin{enumerate}
\item Use the graph to determine the rate (in gal/hr) at which the car burns gas at a speed of $50$ miles/hour.

\item Use the slider $v$ to sketch by hand a graph of the function $r=g(v)$ that expresses the rate (in gal/hour) at which your car burns gas in terms of its speed (in miles/hour). Then activate the folder in Line 8 to seee how you did.

\item Drag the slider $v$ in Line 1 to approximate the speeds between $35$ miles/hour and $70$ miles/hour at which the car burns gas at the maximum and minimum rates (measured in gal/hr). Explain your reasoning.

\item Use calculus to determine the exact speeds in part (c). %Find a way that avoids using the quotient rule.

\end{enumerate}
\end{question}


\begin{question}  \label{Q:dfgtdftrtnhy}
The function 
\[
      P = f(t) = 5 -3t + t^2 \, , \, 0\leq t \leq 4 , 
\]
expresses the price in $\$$/share of a stock in terms of the number of hours past 9am.

(a) Use the graphs of the function $P=f(t)$ and the function $r=f^\prime(t)/f(t)$  to estimate when the stock price is increasing at the greatest relative rate.

(b) Use calculus and algebra to find the exact time when the stock price is increasing at the greatest relative rate.



\begin{onlineOnly}
    \begin{center}
\desmos{y78fnyy7s3}{900}{600}
\end{center}
\end{onlineOnly}

Desmos activity available at \href{https://www.desmos.com/calculator/y78fnyy7s3}{151: Stock Price 4}

\end{question}






\begin{question}  \label{Q0t0gogppdga}

A vertical wall $b$ feet high runs parallel to a tall building. The wall is $a$ feet from the building. A ladder reaching from the ground to the building rests on the top of the wall as shown below.

\begin{onlineOnly}
   \begin{center}
\desmos{4ak46ub8ay}{900}{600}
\end{center}
\end{onlineOnly}

\href{https://www.desmos.com/calculator/4ak46ub8ay}{151: Shortest Ladder}


\begin{enumerate}
\item Find a function that express the length of the ladder (measured in feet) in terms of the angle the ladder makes with the ground. Include an appropriate domain.

\item Use part (a) to express the length of the shortest such ladder in terms of $a$ and $b$. Justify your assertion. 

\item Check that your expression has the correct units.

\end{enumerate}

Work in general and \emph{not} with the particular values of $a$ and $b$ in the worksheet above.

\end{question} 


\begin{question}  \label{Q677jhgjhjfdbn}
The bottom and top edges of a painting are respectively $a$ and $b$ feet above eye level.

\begin{onlineOnly}
   \begin{center}
\desmos{dkqndsegod}{900}{600}
\end{center}
\end{onlineOnly}

\href{https://www.desmos.com/calculator/dkqndsegod}{151: Viewing Angle}


\begin{enumerate}
\item Find a function that expresses the viewing angle in terms of your distance from the painting. Include an appropriate domain.

\item How far from the paiting should you stand to maximimize the viewing angle marked above? Justify your assertion.

\item Check that your expression has the correct units.

\end{enumerate}
\end{question}

\end{document}
