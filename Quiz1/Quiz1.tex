\documentclass{ximera}
\title{Practice Quiz 1}

\newcommand{\pskip}{\vskip 0.1 in}

\begin{document}
\begin{abstract}
First practice quiz, Weeks 1-2
\end{abstract}
\maketitle

%\section{Directions}
\emph{Directions:}

\begin{enumerate}
\item Show all work.

\item Give brief explanations for each problem. Include these explanations in the flow of the solution.

\item Show all units in all computations.
\end{enumerate}


\section{Part 1}

\begin{question}  \label{Q4t5tgwertftggf}
Explain what it means intuitively for a function to be differentiable at a specific input in terms of the graph of that function.
\end{question}

\begin{question} \label{Qdggbrgthghghb}
The function
\[
      s = f(h) = 1.22\sqrt{h} \, , \, 0\leq h \leq 10,000 ,
\]
expresses the distance to the horizon (measured in miles) in terms of your alitude (measured in feet).

\begin{enumerate}
\item Find an expression for the average rate of change in the distance to the horizon with respect to altitude between altitudes $b$ feet and $v$ feet.

\item Use your expression from part (a) to find an expression for the derivative 
\[
    \frac{ds}{dh}\Big|_{h=b}.
\]

\item Evaluate the derivative 
\[
       \frac{ds}{dh}\Big|_{h=25} .
\]

\item What are the units of the derivative above? Interpret its meaning.

\item Use the result of part (c) to approximate the distance to the horizon at an altitude of $24$ feet. Then compare this approximation with the actual distance.

\item Approximate the relative change in the distance to the horizon in terms of a small relative change in altitude.

\item Use your result from part (b) to approximate the altitude at which moving $10$ feet higher increase the distance to the horizon by $0.5$ miles.

\end{enumerate}
\end{question}




\section{Intensity of Sound}

\begin{question}  \label{Qdfdgt446666}
The \emph{intensity} of sound is measured in Watts/(square meter) and is a function of the distance from the source. 

Suppose for a vacuum cleaner, this function is
\[
          I = f(r) = \frac{100}{r^2} \, , \, r\geq 0.5 ,
\]
where $r$ is the distance (in meters) from vacuum and $k$ is a constant. Suppose also that the intensity is $10^{-4}\text{watts}/m^2$ at a distance of $0.5$ meters form the vacuum.

\begin{enumerate}

\item Compute the value of the constant $k$. What are its units?

\item Find an expression for the average rate of change of the sound intensity (in $\text{Watts}/m^2$) with respect to the distance from the source (in meters) between distances $b$ and $r$ meters.

\item Use your expression from part (b) to find an expression for the derivative 
\[
   \frac{dI}{dr} \Big|_{r=b}  .
\]

\item Evaluate the derivative
\[
       \frac{dI}{dr} \Big|_{r=2} .
\]

\item What are the units of the derivative above? What is its meaning?

\item Approximate the relative change in the sound intensity in terms of a small relative change in the distance to the source.
\end{enumerate}

\end{question}



\end{document}