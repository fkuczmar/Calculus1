\documentclass{ximera}
\title{Practice Quiz 1}

\newcommand{\pskip}{\vskip 0.1 in}

\begin{document}
\begin{abstract}
First practice quiz, Weeks 1-2
\end{abstract}
\maketitle

%\section{Directions}
\emph{Directions:}

\begin{enumerate}
\item Show all work.

\item Give brief explanations for each problem. Include these explanations in the flow of the solution.

\item Show all units in all computations.
\end{enumerate}


\section{Part 1}

\begin{question}  \label{Q4t5tgwertftggf}
Explain what it means intuitively for a function to be differentiable at some input in terms of the graph of that function.
\end{question}

\begin{question} \label{Qdggbrgthghghb}
The function
\[
      s = f(h) = 1.22\sqrt{h} \, , \, 0\leq h \leq 10,000 ,
\]
expresses the distance to the horizon (measured in miles) in terms of your alitude (measured in feet).

\begin{enumerate}
\item Find an expression for the average rate of change in the distance to the horizon with respect to altitude between altitudes $b$ feet and $v$ feet.

\item Use your expression from part (a) to find an expression for the derivative 
\[
    \frac{ds}{dh}\Big|_{h=b}.
\]

\item Evaluate the derivative 
\[
       \frac{ds}{dh}\Big|_{h=25} .
\]

\item What are the units of the derivative above? Interpret its meaning.

\item Use the result of part (c) to approximate the distance to the horizon at an altitude of $24$ feet. Then compare this approximation with the actual distance.

\item Approximate the relative change in the distance to the horizon in terms of a small relative change in altitude.

\item Use your result from part (b) to approximate the altitude at which moving $10$ feet higher increase the distance to the horizon by $0.5$ miles.

\end{enumerate}
\end{question}



\begin{question}  \label{Qdfdgt446666}
The \emph{intensity} of sound is measured in Watts/(square meter) and is a function 
\[
          I = f(r) = \frac{100}{r^2} \, , \, r\geq 0.5 ,
\]
of the distance from the source. 

Suppose $r$ is measured in meters and that the intensity of the sound emitted by a vacuum cleaner is $10^{-4}\text{watts}/m^2$ at a distance of $0.5$ meters.

\begin{enumerate}

\item Compute the value of the constant $k$. What are its units?

\item Find an expression for the average rate of change of the sound intensity (in $\text{Watts}/m^2$) with respect to the distance from the source (in meters) between distances $b$ and $r$ meters.

\item Use your expression from part (b) to find an expression for the derivative 
\[
   \frac{dI}{dr} \Big|_{r=b}  .
\]

\item Evaluate the derivative
\[
       \frac{dI}{dr} \Big|_{r=2} .
\]

\item What are the units of the derivative above? What is its meaning?

\item Approximate the relative change in the sound intensity in terms of a small relative change in the distance to the source.
\end{enumerate}

\end{question}

\begin{question}
Between speeds of $70$ miles/hr and $84$ miles/hr, the gas mileage of a car (in miles/gal) is a one-to-one function $G=f(v)$ of its speed (in miles/hour). The car gets $10$ miles/gal at a speed of $80$ miles/hour.

\begin{enumerate}
\item Which of the following is more likely to be true? Explain your reasoning.

\begin{enumerate}
\item $\frac{dG}{dv}\Big|_{v=80} = 0.25$ or
\item $\frac{dG}{dv}\Big|_{v=80} = -0.25$
\end{enumerate}

\item What are the units of the correct derivative above? Explain its meaning.

\item Assuming the correct choice in part (b), evaluate the derivative
\[
   \frac{dv}{dG}\Big|_{G=10} .
\]

\item Simplify the units of the derivative in part (c). What does this tell you about its meaning?

\item At what rate (in gal/hour) does the car burn gas at a speed of $80$ miles/hour?


\end{enumerate}
\end{question}

\begin{question} \label{Qfrre4ehgfffff}
Between speeds of $55$ miles/hour and $70$ miles/hour the gas milesage of a car is a linear function of its speed. The car gets $40$ miles/gal at a speed of $55$ miles/hour and $30$ miles/gal at a speed of $70$ miles/hour.

\begin{enumerate}
\item Find a function 
\[
     r = f(v) \, , \, 55\leq v \leq 70,
\]
that express the rate (in gal/hr) at which the car burns gas in terms of its speed (in miles/hour). \emph{Note:} This function is \emph{not} linear

\item Find an expression for the average rate of change of $r$ with respect to $v$ between speeds of $b$ miles/hour and $w$ miles/hour.

\item Use your expression from part (b) to find an expression for the derivative
\[
  \frac{dr}{dv}\Big|_{v=b} .
\]

\item Evaluate the derivative
\[
  \frac{dr}{dv}\Big|_{v=60} .
\]

\item What are the units of the derivative above? Interpret its meaning.

\end{enumerate}



\end{question}

\end{document}