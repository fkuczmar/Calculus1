\documentclass{ximera}
\title{Derivatives of Trigonometric Functions}

\newcommand{\pskip}{\vskip 0.1 in}

\begin{document}
\begin{abstract}
Working with trigonometric functions and their derivatives.
\end{abstract}
\maketitle


\section{Exercises}

\begin{question}  \label{Qdfethnnjjjuu}
The function
\[
    h = f(t) = 1200 + 500\cos t + 300\sin t \, , \, 0\leq t \leq 10 ,
\]
expresses the altitude (in feet) of a balloon in terms of the number of hours past noon.

Find the minimum and maximum heights of the balloon between noon and 6pm without using a calculator.

\end{question}


\section{Visualizing Derivatives}
\begin{question}  \label{Qdfgtt5ttt55}

You ride a ferris wheel for one revolution and get off. Let 
\[
    h = f(s) \, , 0\leq s \leq ??,
\]
be the function that expresses your height above the ground (measured in feet) in terms of your distance traveled, measured (in feet) along your path from your starting point.

(a) Choose a radius for the ferris wheel that you think is reasonable and fill in the missing upper bound for the domain of $f$ above.

(b) Use the demonstration below to sketch by hand a graph of the function 
\[
     r = \frac{dh}{ds} = f^\prime(s) .
\] 
Do \emph{not} make any computations. Just use the demonstration below. Here are the key points to keep in mind:

\begin{itemize}
\item{Approximate the derivative $dh/ds$ by the ratio $\Delta h / \Delta s$}.

\item{The length of the red arclength (when the ferris wheel is on the way up) is the input $s$. }

\item{The length of the purple segment is the output $h$.}

\item{The lengths of the orange arc and orange segments are $\Delta s$.}

\item{The (signed) length of the green segment is  $\Delta h$.}

\end{itemize}




Explain your reasoning thoroughly. Be sure to include at least the following points:

\begin{itemize}
\item{The units of the input and output to the derivative}

\item{Scales on the vertical and horizontal axes}

\item{A discussion of how a small change in the input to the function $f$ changes the output at various positions along your ride. }

\item{A discussion of where a small change in the input to $f$ gives the greatest positive change in the output and a consideration of the ratios of these changes }

\item{A discussion of where a small change in the input to $f$ gives the negative change in the output with the greatest magnitude and a consideration of the ratios of these changes }

\item{A discussion of where a small change in the input to $f$ barely changes the output and a consideration of the ratios of these changes }

\end{itemize}

(c) How would your graph of the derivative $dh/ds$ change if you doubled the radius of the ferris wheel? Sketch the new graph.

\begin{onlineOnly}
    \begin{center}
\desmos{pxsmo04nmg}{900}{600}  %    pxsmo04nmg  8swp20zond
\end{center}
\end{onlineOnly}




\end{question}




\section{The Derivative of the Sine Function}


\begin{question}   \label{Qdgbdgh443}
\begin{onlineOnly}
    \begin{center}
\desmos{jcmcyrpndw}{900}{600}
\end{center}
\end{onlineOnly}

Desmos activity available at \href{https://www.desmos.com/calculator/jcmcyrpndw}{151: Derivative of Sine}
\end{question}


\begin{question} \label{Q:gllgbbdd}
(a) Describe a transformation that takes the graph of the function
\[
      y = f(\theta)  = \sin \theta
\]
to the graph of the function 
\[
   y = g(\theta) = \cos\theta.
\]

(b) Does that same transformation take the graph of the derivative of the sine function to the graph of the derivative of the cosine function? Do not use any particular facts about these functions to answer this question. Instead, give a general answer that would apply to all pairs of functions similarly related.

(c) Use your answer to part (b) and the fact that 
\[
      \frac{d}{d\theta} \left(  \sin\theta \right)  = \cos\theta
\]
to find an expression for the derivative
\[
   \frac{d}{d\theta} \left(  \cos\theta \right)
\]
of the cosine function.
\end{question}


\section{Applications 1}
\begin{question}  \label{Q5dgdgnbhyhy4}
The center of a ferris wheel with a radius of 50 feet is 60 feet above the ground. You ride the wheel for one revolution and get off.

(a) Find a function 
\[
 h = f(\theta) \, , \, 0\leq \theta \leq 2\pi ,
\] 
that expresses your height above the ground in terms of the rotation angle of the wheel, measured in radians. Use the \emph{cosine} function, \emph{not} the sine function.

(b) The wheel stops when you are $100$ feet above the ground and on the way up. It then starts again and turns through a small angle of $\Delta \theta$ radians before stopping again. Use the appropriate linear approximation to estimate the change $\Delta h$ in your height (measured in feet) as the wheel turned through the angle $\theta$.

\begin{onlineOnly}
    \begin{center}
\geogebra{tn75cq93}{900}{600}
\end{center}
\end{onlineOnly}

%https://www.geogebra.org/classic/tn75cq93

\end{question}


\section{Transformations of the Sine Function}
\begin{question}  \label{Q:gsdfgdfgfdsfd}
(a) Describe a transformation that takes the graph of the function
\[
   y  = f(\theta) =  \sin\theta
\]
to the graph of the function
\[
   y = g(\theta) = \sin (2\theta).
\]

(b) How does that same transformation affect the slope of a line?

(c) Use your answer to part (b) to find an expression for the derivative
\[
    g^\prime(\theta)  = \frac{d}{d\theta} \left(  \sin(2\theta) \right) .
\] 

\end{question}


\section{Applications 2}

\begin{question} \label{Q:324gg434}
The center of a ferris wheel with a radius of 50 feet is 60 feet above the ground. You travel at a constant speed of $5$ ft/sec as you ride the ferris wheel.

(a)  Find a function 
\[
 h = f(t) \, , t\geq 0
\] 
that expresses your height above the ground in terms of the number of seconds since you got on. Use the \emph{cosine} function, \emph{not} the sine function.

(b) Are you ascending or descending the second time you are 90 feet above the ground? At what rate? Use the methods of this class, not vectors, to answer this question.

(c) Find your height when you are descending at the rate of $4.8$ feet/sec. Give all possibilities. Do not use a calculator except to do arithmetic.

\end{question}


\begin{question}  \label{QDDFDF}
Nothing yet.
\end{question}


\end{document}
