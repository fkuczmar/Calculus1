\documentclass{ximera}
\title{Derivatives of Trigonometric Functions}

\newcommand{\pskip}{\vskip 0.1 in}

\begin{document}
\begin{abstract}
Working with trigonometric functions and their derivatives.
\end{abstract}
\maketitle


\section{Velocity}

Velocity is the rate of change of position with respect to time. Since position is a vector, so is velocity. The instantaneous velocity vector is the derivative (with respect to time) of the position vector. It (the velocity vector)  points in the direction of motion and is therefore tangent to the path. The length of the velocity vector is the speed.

Play the slider $u$ (time) below to watch a projectile motion in a uniform gravitational field.

\begin{onlineOnly}
    \begin{center}
\desmos{vdtsb2fc8j}{900}{600}  %    pxsmo04nmg  8swp20zond
\end{center}
\end{onlineOnly}

\href{https://www.desmos.com/calculator/vdtsb2fc8j}{151: Projectile Motion}


\begin {enumerate}
\item When is the object speeding up? Slowing down? How can you tell from looking at its velocity vector?

\item Suppose the position vector (measured in meters) at time $t$ seconds past launch is
\[
    \langle x,y \rangle  = \langle (v_0\cos\phi) t, (v_0\sin\phi)t - \frac{1}{2}gt^2 \rangle  \, , \, 0 \leq t \leq \frac{2v_0 \sin \phi}{g} .
\]
Here $v_0$ is the initial speed (in m/sec), $\phi$ is the launch angle relative to the horizontal, and $g$ is the magnitude of the gravitational acceleration (in $\text{m/sec}^2$). 

Find an expression for the velocity vector.

\end{enumerate}


\section{Derivatives of the Sine and Cosine Functions}

We just saw we can differentiate the position vector of a motion to find the velocity vector, at least if we know how to differentiate the component functions. There is one case where we can reverse this process. The next example shows how and gives a way to find the derivatives of the sine and cosine functions.
 

\begin{example} \label{ExPODfdsfLMV}
 
The idea is to look at the motion with position vector (in meters)
\[
   \overrightarrow{OP} = \langle \textcolor{blue}{1} \cos (\textcolor{red}{1} t) , \textcolor{blue}{1} \sin (\textcolor{red}{1} t) \rangle,
\]
expressed in terms of the number of seconds past noon. Play the slider $u$ (another name for $t$) in the worksheet below to watch the motion.


\begin{onlineOnly}
    \begin{center}
\desmos{ov0azsjucq}{900}{600}  %    pxsmo04nmg  8swp20zond
\end{center}
\end{onlineOnly}

\href{https://www.desmos.com/calculator/ov0azsjucq}{151: Circular Motion}

\begin{enumerate}
\item What are the units of the blue \textcolor{blue}{1} and the red \textcolor{red}{1} in the parameterization.

\item What is the speed of the motion?

\item Use the geometry of the circle to find the components of the velocity vector.

\item What does this tell us about the derivatives 
\[
     \frac{d}{dt} \left( \cos t \right)  \hskip 0.4 in \text{and}  \hskip 0.4 in \frac{d}{dt} \left( \sin t \right) .
\]

\item Use the same idea to compute the derivatives
\[
     \frac{d}{dt} \left( \cos (3t) \right)  \hskip 0.4 in \text{and}  \hskip 0.4 in \frac{d}{dt} \left( \sin(3t) \right) ?
\]
Then use the chain rule to compute the derivatives again.

\end{enumerate}
\end{example}


\section{Simple Harmonic Motion}


\begin{question}  \label{QKLKLggggghg}
Suppose for this problem that the earth is a ball with uniform density of radius 4000 miles. Now imagine drilling a straight tunnel through the earth from the North Pole to the South Pole. 

A subway car dropped from rest at the north pole falling through the tunnel would then oscillate in simple harmonic motion between the poles and return to the north pole every $84$ minutes. This means we can think of the car as being dragged along by a point moving around the earth at constant speed as illustrated below. 

\begin{onlineOnly}
    \begin{center}
\desmos{ij8dqowgza}{900}{600}
\end{center}
\end{onlineOnly}

Desmos activity available at \href{https://www.desmos.com/calculator/ij8dqowgza}{142: Simple Harmonic Motion}

\begin{enumerate}

\item Find a function 
\[
      s = f(t) \, , \, t\geq 0,
\]
that expresses the distance of the rock (measured in thousands of miles) form the South Pole in terms of the number of minutes since the rock was released. Assume the rock was dropped at noon on July 1, 2085.

\item Use your function and calculus to determine the car's speed as it passes the center of the earth. Then find the speed without calcululs.

\item Compare the average speed of the car during the time for one oscillation with its maximum speed.

\item Find the car's speed when it is $1/3$ of the way from the earth's center to the South Pole.


\end{enumerate}

\end{question}

\section{An Optimization Problem}

\begin{question}  \label{Qdfethnnjjjuu}
The function
\[
    h = f(t) = 1200 + 500\cos t + 300\sin t \, , \, 0\leq t \leq 10 ,
\]
expresses the altitude (in feet) of a balloon in terms of the number of hours past noon.

\begin{onlineOnly}
    \begin{center}
\desmos{f1ldi6yrek}{900}{600}  %    pxsmo04nmg  8swp20zond
\end{center}
\end{onlineOnly}

\href{https://www.desmos.com/calculator/f1ldi6yrek}{151: Trig 1}

\begin{enumerate}

\item Use calculus to find the minimum and maximum heights of the balloon between noon and 6pm. Do not use a calculator.

\item Find the balloon's maximum rate of ascent. No calculator.

\end{enumerate}

\end{question}


\section{Riding a Ferris Wheel}

\begin{question}  \label{Qd5t6dsfre6yuu}
The center of a ferris wheel with a radius of 50 feet is 60 feet above the ground. You ride the wheel for one revolution and get off.

\begin{enumerate}

\item Use the figure below to find a function 
\[
 h = f(\theta) \, , \, 0\leq \theta \leq 2\pi ,
\] 
that expresses your height above the ground in terms of the rotation angle of the wheel (measured in radians since the time you boarded). Use the \emph{cosine} function, \emph{not} the sine function.


\begin{onlineOnly}
    \begin{center}
\geogebra{tn75cq93}{900}{600}
\end{center}
\end{onlineOnly}


\item The function 
\[
    \theta = a(t) \, , \, 0\leq t \leq 120 ,
\]
expresses the rotation angle (in radians, measured since you boarded the wheel) in terms of the number of seconds since  you got on. Use the graph of this function below to approximate the your rate of ascent at time $t=40$ seconds. 

\begin{onlineOnly}
    \begin{center}
\geogebra{dbc9bdzeqm}{900}{600}
\end{center}
\end{onlineOnly}

\href{https://www.desmos.com/calculator/dbc9bdzeqm}{151: Ferris Wheel Angle}


\item Now suppose
\[
    \theta = a(t) = \frac{\pi}{2}\left(1- \cos \left( \frac{\pi}{120}t \right)    \right) \, , \, 0\leq t \leq 120 ,
\]
and compute your exact rate of ascent at time $t=40$ seconds. Compare this with your estimate.

\end{enumerate}
\end{question}



\section{Exercises}



\begin{question}  \label{Q:LKJMFJUFegvt4}
The function
\[
       h = f(t) = 800 - 200 \tan t \, , \, -1.3\leq t\leq 1.3,
\]
expresses the altitude (in feet) of a balloon in terms of the number of hours past noon.

\begin{onlineOnly}
    \begin{center}
\desmos{re6nqofgs0}{900}{600}  %    pxsmo04nmg  8swp20zond
\end{center}
\end{onlineOnly}

\href{https://www.desmos.com/calculator/re6nqofgs0}{151: Trig 2}

Find exact answers to the following questions without using a calculator.


\begin{enumerate}
\item When is the balloon descending at the rate of 150 ft/hour?

\item How high is the balloon when it is descending at the rate of $500$ ft/hour?

\item Express the balloon's rate of ascent (in feet/hour) in terms of its altitude.

\end{enumerate}

\end{question}


\begin{question}  \label{Qghghgfdgdsfg0900}

The function
\[
       h = f(t) = 11 - 7 \cos t\, , \, 0\leq t \leq 6 ,
\]
expresses the depth of the water (in feet) at the Edmonds Pier in terms of the number of \emph{two-hour periods} past noon.

\begin{onlineOnly}
    \begin{center}
\desmos{4bh7kimi7f}{900}{600}  %    pxsmo04nmg  8swp20zond
\end{center}
\end{onlineOnly}

\href{https://www.desmos.com/calculator/4bh7kimi7f}{151: Trig 3}

Find exact answers to the following questions without using a calculator.


\begin{enumerate}
\item Is the depth of the water increasing or decreasing at 2pm? At what rate?

\item At what rate is the depth of the water changing when the water is $6$ feet deep? 

\end{enumerate}
\end{question}


\begin{question}  \label{Qd5t65yy6yuu}
The center of a ferris wheel with a radius of 50 feet is 60 feet above the ground. You ride the wheel for one revolution and get off.

\begin{enumerate}

\item Use the figure below to find a function 
\[
 h = f(\theta) \, , \, 0\leq \theta \leq 2\pi ,
\] 
that expresses your height above the ground in terms of the rotation angle of the wheel (measured in radians since the time you boarded). Use the \emph{cosine} function, \emph{not} the sine function.


\begin{onlineOnly}
    \begin{center}
\geogebra{tn75cq93}{900}{600}
\end{center}
\end{onlineOnly}


\item The wheel stops when you are $100$ feet above the ground for the \emph{second} time. It then starts again and turns through a small angle of $\Delta \theta$ radians before stopping again.


\begin{onlineOnly}
    \begin{center}
\desmos{gi8obqmnav}{900}{600}
\end{center}
\end{onlineOnly}

\href{https://www.desmos.com/calculator/gi8obqmnav}{151: Ferris Wheel 34}


\begin{enumerate}

\item Zoom in on point $P$ above to approximate the change in your height (in feet) if $\Delta \theta = 0.1$.

\item Use a derivative to approximate the change in your height if $\Delta \theta = 0.1$.

\item Use a deriviative to approximate the change $\Delta h$ in your height (measured in feet) as the wheel turns through the small angle $\Delta \theta$ radians.

\end{enumerate}
\end{enumerate}
\end{question}





\section{Visualizing Derivatives}
\begin{question}  \label{Qdfgtt5ttt55}

You ride a ferris wheel for one revolution and get off. Let 
\[
    h = f(s) \, , 0\leq s \leq ??,
\]
be the function that expresses your height above the ground (measured in feet) in terms of your distance traveled, measured (in feet) along your path from your starting point.

(a) Choose a radius for the ferris wheel that you think is reasonable and fill in the missing upper bound for the domain of $f$ above.

(b) Use the demonstration below to sketch by hand a graph of the function 
\[
     r = \frac{dh}{ds} = f^\prime(s) .
\] 
Do \emph{not} make any computations. Just use the demonstration below. Here are the key points to keep in mind:

\begin{itemize}
\item{Approximate the derivative $dh/ds$ by the ratio $\Delta h / \Delta s$}.

\item{The length of the red arclength (when the ferris wheel is on the way up) is the input $s$. }

\item{The length of the purple segment is the output $h$.}

\item{The lengths of the orange arc and orange segments are $\Delta s$.}

\item{The (signed) length of the green segment is  $\Delta h$.}

\end{itemize}




Explain your reasoning thoroughly. Be sure to include at least the following points:

\begin{itemize}
\item{The units of the input and output to the derivative}

\item{Scales on the vertical and horizontal axes}

\item{A discussion of how a small change in the input to the function $f$ changes the output at various positions along your ride. }

\item{A discussion of where a small change in the input to $f$ gives the greatest positive change in the output and a consideration of the ratios of these changes }

\item{A discussion of where a small change in the input to $f$ gives the negative change in the output with the greatest magnitude and a consideration of the ratios of these changes }

\item{A discussion of where a small change in the input to $f$ barely changes the output and a consideration of the ratios of these changes }

\end{itemize}

(c) How would your graph of the derivative $dh/ds$ change if you doubled the radius of the ferris wheel? Sketch the new graph.

\begin{onlineOnly}
    \begin{center}
\desmos{pxsmo04nmg}{900}{600}  %    pxsmo04nmg  8swp20zond
\end{center}
\end{onlineOnly}




\end{question}




\section{The Derivative of the Sine Function}


\begin{question}   \label{Qdgbdgh443}
\begin{onlineOnly}
    \begin{center}
\desmos{jcmcyrpndw}{900}{600}
\end{center}
\end{onlineOnly}

Desmos activity available at \href{https://www.desmos.com/calculator/jcmcyrpndw}{151: Derivative of Sine}
\end{question}


\begin{question} \label{Q:gllgbbdd}
(a) Describe a transformation that takes the graph of the function
\[
      y = f(\theta)  = \sin \theta
\]
to the graph of the function 
\[
   y = g(\theta) = \cos\theta.
\]

(b) Does that same transformation take the graph of the derivative of the sine function to the graph of the derivative of the cosine function? Do not use any particular facts about these functions to answer this question. Instead, give a general answer that would apply to all pairs of functions similarly related.

(c) Use your answer to part (b) and the fact that 
\[
      \frac{d}{d\theta} \left(  \sin\theta \right)  = \cos\theta
\]
to find an expression for the derivative
\[
   \frac{d}{d\theta} \left(  \cos\theta \right)
\]
of the cosine function.
\end{question}


\section{Applications 1}
\begin{question}  \label{Q5dgdgnbhyhy4}
The center of a ferris wheel with a radius of 50 feet is 60 feet above the ground. You ride the wheel for one revolution and get off.

(a) Find a function 
\[
 h = f(\theta) \, , \, 0\leq \theta \leq 2\pi ,
\] 
that expresses your height above the ground in terms of the rotation angle of the wheel, measured in radians. Use the \emph{cosine} function, \emph{not} the sine function.

(b) The wheel stops when you are $100$ feet above the ground and on the way up. It then starts again and turns through a small angle of $\Delta \theta$ radians before stopping again. Use the appropriate linear approximation to estimate the change $\Delta h$ in your height (measured in feet) as the wheel turned through the angle $\theta$.

\begin{onlineOnly}
    \begin{center}
\geogebra{tn75cq93}{900}{600}
\end{center}
\end{onlineOnly}

%https://www.geogebra.org/classic/tn75cq93

\end{question}


\section{Transformations of the Sine Function}
\begin{question}  \label{Q:gsdfgdfgfdsfd}
(a) Describe a transformation that takes the graph of the function
\[
   y  = f(\theta) =  \sin\theta
\]
to the graph of the function
\[
   y = g(\theta) = \sin (2\theta).
\]

(b) How does that same transformation affect the slope of a line?

(c) Use your answer to part (b) to find an expression for the derivative
\[
    g^\prime(\theta)  = \frac{d}{d\theta} \left(  \sin(2\theta) \right) .
\] 

\end{question}


\section{Applications 2}

\begin{question} \label{Q:324gg434}
The center of a ferris wheel with a radius of 50 feet is 60 feet above the ground. You travel at a constant speed of $5$ ft/sec as you ride the ferris wheel.

(a)  Find a function 
\[
 h = f(t) \, , t\geq 0
\] 
that expresses your height above the ground in terms of the number of seconds since you got on. Use the \emph{cosine} function, \emph{not} the sine function.

(b) Are you ascending or descending the second time you are 90 feet above the ground? At what rate? Use the methods of this class, not vectors, to answer this question.

(c) Find your height when you are descending at the rate of $4.8$ feet/sec. Give all possibilities. Do not use a calculator except to do arithmetic.

\end{question}


\section{MATH 142}

\begin{question}  \label{QDDFDF}

The graph below shows the $x$-coordinate function of a beetle moving around a circle at a constant speed.

\begin{onlineOnly}
    \begin{center}
\desmos{qi6c9xbhnw}{900}{600}
\end{center}
\end{onlineOnly}

\href{https://www.desmos.com/calculator/qi6c9xbhnw}{142: Edmonds Pier 2}

Use the graph to answer the following questions. Be sure to include units.
\begin{enumerate}

\item Find the $x$-coordinate of the circle's center.

\item Find the radius of the circle.

\item Find the period of the motion. This the time it takes the beetle to make one revolution about the center of its circular path.

\item Find a time when the beetle's $x$-coordinate is a maximum.

\item Use (a)-(d) to find an expression $x=f(t)$ for the function that expresses the $x$-coordinate of the beetle (measured in feet) in terms of the number of minutes past noon. Include the domain.

\item Check your expression from part (e) by substituting the two times given in the graph.


\end{enumerate}


\end{question}


\end{document}
