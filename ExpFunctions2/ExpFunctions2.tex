\documentclass{ximera}
\title{Exponential Functions, Part 2}

\newcommand{\pskip}{\vskip 0.1 in}

\begin{document}
\begin{abstract}
More on exponential functions and their derivatives. 
\end{abstract}
\maketitle


\section{The Main Ideas}
\begin{question}
The function 
\[
   P =f(t) =3 (2^t) \, , \, 0\leq t \leq 6 ,
\]
expresses the population (in millions) of a colony of bacteria in terms of the number of hours past noon. Its graph is shown below.

\href{https://www.desmos.com/calculator/ri5r9suauk}{151: Exp Growth Subtangent}

 
\begin{onlineOnly}
    \begin{center}
\desmos{ri5r9suauk}{900}{600}
\end{center}
\end{onlineOnly}

\begin{enumerate}
\item Describe how the population grows.

\item Does the converse hold? That is, does your description imply the population function has the form given above?

\item If not, how can we describe precisely how the population grows?

\begin{enumerate}
\item Drag the slider $u$ in Line 2 and watch how the length of segment $AB$ changes. Watch also the values of the expressions in Lines 5 and 6. How are these values related to the length $AB$?

\item Drag the slider $h$ to $h=2$ and repeat part (i).

\item Drag the slider $h$ to be closer and closer to zero, each time repeating part (i). What do you notice? 

\end{enumerate}

\end{enumerate}

\end{question}




\begin{question}  \label{Qlgkfglbnbn}
What type of functions change at 
\begin{enumerate}
\item a constant rate?

\item a constant relative rate?
\end{enumerate}
\end{question}

\section{Exercises}

\begin{question} \label{Q677543345}
The function 
\[
     P = f(t) = 50 e^{\frac{t}{10}} \, , \, 0\leq t \leq 20 ,
\]
expresses the population (in millions) of a colony of bacteria in terms of the number of hours past  midnight. Graph shown below.

\begin{onlineOnly}
    \begin{center}
\desmos{kffjsoepmo}{900}{600}  %    pxsmo04nmg  8swp20zond
\end{center}
\end{onlineOnly}

\href{https://www.desmos.com/calculator/kffjsoepmo}{151: Exp Function 5}


\begin{enumerate}

\item Use the graph of the function $P=f(t)$ above to approximate the growth rate when there are $200$ million bacteria.

\item Find a function $r=g(P)$ that expresses the growth rate (measured in millions of bacteria/hour) in terms of the popuation (measured in millions). Include a domain.

\item Find the \emph{exact} growth rate when there are $200$ million bacteria. No calculator.

\item What is the population when it is increasing at the rate of $8$ million bacteria/hour?

\item Use the ideas of this question to describe how the population grows. \emph{Hint:} Think about the relative instantaneous growth rate.

\end{enumerate}
\end{question}


\begin{question}  \label{Q9deLFLDF}
Between noon and 6pm, a population of bacteria grows exponenially, increasing by $20\%$ every hour. There are 5 million bacteria at noon.

\begin{enumerate}
\item Find a function that expresses the population (in millions of bacteria) in terms of the number of hours past noon.

\item Find a function that expresses the growth rate (in millions of bacteria/hour) in terms of the population (in millions).

\item Find the population when it is increasing at the rate of $2$ million bacteria/hour.

\item What is the relative growth rate of the population?

\item Consider a second population that grow exponentially, increasing by $20\%$ every four hours. 

\begin{enumerate}
\item Use common sense to find its relative growth rate.
\item Use calculus to verify your answer from part (i) with a computation.
\end{enumerate}
\end{enumerate}

\end{question}

\begin{question}  \label{Qhhfhnnnhtytt543}
The function
\[
       N = f(t) = \frac{20}{1+5e^{-t/5}} \, , \, 0\leq t\leq 30,
\]
models the spread of a virus throughout a population. It takes as an input the number of months since January 1st and returns as an output the number of infected individuals (measured in millions). Graph shown below.

\begin{onlineOnly}
    \begin{center}
\desmos{ikis451wxu}{900}{600} 
\end{center}
\end{onlineOnly}

\href{https://www.desmos.com/calculator/ikis451wxu}{151: Logistic Model}

\begin{enumerate}
\item Use the sliders in the worksheet above to approximate the number of infected individuals when the virus is spreading at the rate of $800,000$ people/month.

\item Find a function 
\[
       r = g(N) 
\]
that expresses the infection rate (in millions of people/month) in terms of the population.

\item Use calculus and algebra to determine the number of infected individuals when the virus is spreading at the rate of $750,000$ people/month.

\item Use the sliders in the worksheet above to approximate the number of infected individuals when the virus is spreading at its fastest rate. Also approximate this rate.

\item Use calculus and algebra to determine the number of infected individuals when the virus is spreading at its fastest rate. Then find this exact rate.
\end{enumerate}


\end{question}


\begin{question}  \label{Qdghnntnmnmnmn}
Use the chain rule and your knowledge of the derivative $d(e^x)/dx$ to compute each of the following derivatives.

\begin{enumerate}
\item 
\[
   \frac{d}{dt}\left( 100 \cdot 2^t \right)
\]

\item 
\[
   \frac{d}{dt}\left( 100 \cdot 2^{t/5} \right)
\]
\end{enumerate}
\begin{explanation}

\begin{enumerate}
\item The key is to realize that 
\[
    2 = e^{\ln 2} .
\]
So we can write the function to differentiate as
\[
    P = 100 \cdot 2^t = 100 \left( e^{\ln 2} \right)^t = 100e^{t\ln 2} .
\]
Now we can make a $u$-substitution and use the chain rule. We let
\[
       u =t \ln 2 .
\]
Then
\[
      P = 100e^{t\ln 2} = 100e^u
\]
and
\begin{align*}
   \frac{dP}{dt} &= \frac{dP}{du} \cdot \frac{du}{dt} \\
                       &= \frac{d}{du} \left(100e^u\right) \cdot  \frac{d}{dt} \left(  t\ln 2 \right) \\
                        &= 100e^u (\ln 2) \\
                        &= P \ln 2 \\
                     &= (100 \cdot 2^t) \ln 2\\
                        &= (100 \ln 2)(2^t)
\end{align*}

\end{enumerate}
\end{explanation}

\end{question}

\begin{question}  \label{Qfyhyhgrrrreee}
A colony of bacteria has a population of $20$ million at noon and a population of $25$ million at 2pm. The population grows exponentially between noon and midnight.

\begin{enumerate}
\item Describe precisely how the population grows.

\item Use your description from part (a) to find a function 
\[
   P=f(t) \, , \, 0\leq t \leq 12,
\]
that expresses the population (in millions) in terms of the number of hours past noon. Do \emph{not} use $e$.

\item Find the relative instantaneous growth rate of the population.

\item Find a function $r=g(P)$ that expresses the (absolute) growth rate in terms of the population.

\item At what rate is the population changing when there are $30$ million bacteria?
\end{enumerate}

\begin{explanation}
\begin{enumerate}
\item The population increases by 
\[
  \frac{25\text{ million} - 20\text{ million}}{20\text{ million}} = 25\%
\]
every two hours. 

Better yet, at least for determining the function, is to say  that the two-hour growth factor is
\[
\frac{25\text{ million}}{20\text{ million}} = 1.25 .
\]
This means that every two hours the population gets multiplied by 1.25.

\item Since the two-hour growth factor is 1.25, the one-hour growth factor is $1.25^{1/2}$. So 
\[
       P = 20(1.25^{1/2})^t = 20(1.25)^{\frac{t}{2}} \, , \, 0\leq t \leq 12.
\]
 
\item The key to computing the derivative $dP/dt$ is to write the population function using base $e$. Since
\[
       1.25 = e^{\ln 1.25},
\]
\begin{align*}
      P &= 20(1.25)^{\frac{t}{2}}  \\
         &= 20 \left( e^{\ln 1.25} \right)^{\frac{t}{2}} \\
         &=  20 e^{t (\ln 1.25)/2}
\end{align*}


So the growth rate (in millions of bacteria/hour) is 
\begin{align*}
  \frac{dP}{dt} &= \frac{d}{dt} \left(  20(1.25)^{\frac{t}{2}} \right) \\
                     &= 20 \frac{d}{dt}\left( e^{t (\ln 1.25)/2}\right) \\
                     &= \left( 20 e^{t (\ln 1.25)/2}\right) \frac{d}{dt} \left(  t (\ln 1.25)/2 \right) \\
                     &= P (\ln 1.25)/2 .
\end{align*}
And the relative growth rate (units are $\text{hr}^{-1}$) is
\[
  \frac{1}{P} \cdot   \frac{dP}{dt} = \frac{1}{2}\ln 1.25.
\]

\item From part (c), the function
\[
   g(P) = \frac{dP}{dt} = P (\ln 1.25)/2 
\]
expresses the growth rate (in millions of bacteria/hour) in terms of the population.

\item When there are 30 million bacteria, the population is increasing at the rate of 
\[
       g(30) = (30 \text{ million bacteria}) \left(\frac{1}{2}\ln 1.25 /\text{hr}\right) = 15 \ln 1.25 \text{ million bacteria/hr}.
\]
\end{enumerate}
\end{explanation}

\end{question}


\begin{question}  \label{Qgfhhfhdffv}
Between ground level and an altitude of $10$ km, atmospheric pressure on the planet Krypton is an exponential function of altitude. The pressure is 100 kPa at ground level and 80 kPa at an altitude of $0.75$ km.

\begin{onlineOnly}
    \begin{center}
\desmos{j5h8kaj8xs}{900}{600}  %    pxsmo04nmg  8swp20zond
\end{center}
\end{onlineOnly}

\href{https://www.desmos.com/calculator/j5h8kaj8xs}{151: Atomspheric Pressure}

\begin{enumerate}
\item Describe exactly how the pressure decreases.

\item Use your description from part (a) to find a function
\[
         P = f(h) \, , \, 0\leq h \leq 10 ,
\]
that expresses the atmospheric pressure (in kPa) in terms of the altitude (in kilometers). Do \emph{not} use the number $e$.

\item Input your expression for $f$ in Line 2 of the worksheet above.

\item Evaluate the derivative 
\[
       \frac{dP}{dh}\Big|_{h=5}
\]
and interpret its meaning.

\item Find a function $r=g(P)$ that expresses the derivative $dP/dh$ in terms of $P$.

\item Use calculus to approximate the altitude at which the pressure is $0.2$ kPa less at an altitude $0.1$ km higher.

\end{enumerate}
  
\end{question}

\end{document}
