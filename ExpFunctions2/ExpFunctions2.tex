\documentclass{ximera}
\title{Exponential Functions, Part 2}

\newcommand{\pskip}{\vskip 0.1 in}

\begin{document}
\begin{abstract}
More on exponential functions and their derivatives.
\end{abstract}
\maketitle


\begin{question} \label{Q677543345}
The function 
\[
     P = f(t) = 50 e^{\frac{t}{10}} \, , \, 0\leq t \leq 20 ,
\]
expresses the population (in millions) of a colony of bacteria in terms of the number of hours past  midnight. Graph shown below.

\begin{onlineOnly}
    \begin{center}
\desmos{kffjsoepmo}{900}{600}  %    pxsmo04nmg  8swp20zond
\end{center}
\end{onlineOnly}

\href{https://www.desmos.com/calculator/kffjsoepmo}{151: Exp Function 5}


\begin{enumerate}

\item Use the graph of the function $P=f(t)$ above to approximate the growth rate when there are $60$ million bacteria.

\item Find the \emph{exact} growth rate when there are $60$ million bacteria. No calculator.

\item Find a function $r=g(P)$ that expresses the growth rate (measured in millions of bacteria/hour) in terms of the popuation (measured in millions). Include a domain.

\item What is the population when it is increasing at the rate of $10$ million bacteria/hour?

\end{enumerate}
\end{question}

\begin{question}  \label{Qgfhhfhdffv}
Between ground level and an altitude of $10$ km, atmospheric pressure on the planet Krypton is an exponential function of altitude. The pressure is 100 kPa at ground level and 80 kPa at an altitude of $0.75$ km.

\begin{onlineOnly}
    \begin{center}
\desmos{j5h8kaj8xs}{900}{600}  %    pxsmo04nmg  8swp20zond
\end{center}
\end{onlineOnly}

\href{https://www.desmos.com/calculator/j5h8kaj8xs}{151: Atomspheric Pressure}

\begin{enumerate}
\item Describe exactly how the pressure decreases.

\item Use your description from part (a) to find a function
\[
         P = f(h) \, , \, 0\leq h \leq 10 ,
\]
that expresses the atmospheric pressure (in kPa) in terms of the altitude (in kilometers). Do \emph{not} use the number $e$.

\item Input your expression for $f$ in Line 2 of the worksheet above.

\item Evaluate the derivative 
\[
       \frac{dP}{dh}\Big|_{h=5}
\]
and interpret its meaning.

\item Find a function $r=g(h)$ that expresses the derivative $dP/dh$ in terms of $P$.

\item Use calculus to approximate the altitude at which the pressure is $0.2$ kPa less at an altitude $0.1$ km higher.

\end{enumerate}
  
\end{question}

\end{document}
