\documentclass{ximera}
\title{Exponential Functions, Part 2}

\newcommand{\pskip}{\vskip 0.1 in}

\begin{document}
\begin{abstract}
More on exponential functions and their derivatives.
\end{abstract}
\maketitle

\begin{question}  \label{Qlgkfglbnbn}
What type of functions change at 
\begin{enumerate}
\item a constant rate?

\item a constant relative rate?
\end{enumerate}
\end{question}

\begin{question} \label{Q677543345}
The function 
\[
     P = f(t) = 50 e^{\frac{t}{10}} \, , \, 0\leq t \leq 20 ,
\]
expresses the population (in millions) of a colony of bacteria in terms of the number of hours past  midnight. Graph shown below.

\begin{onlineOnly}
    \begin{center}
\desmos{kffjsoepmo}{900}{600}  %    pxsmo04nmg  8swp20zond
\end{center}
\end{onlineOnly}

\href{https://www.desmos.com/calculator/kffjsoepmo}{151: Exp Function 5}


\begin{enumerate}

\item Use the graph of the function $P=f(t)$ above to approximate the growth rate when there are $200$ million bacteria.

\item Find the relative (instantaneous) growth rate of the population.

\item Find the \emph{exact} growth rate when there are $200$ million bacteria. No calculator.

\item Find a function $r=g(P)$ that expresses the growth rate (measured in millions of bacteria/hour) in terms of the popuation (measured in millions). Include a domain.

\item What is the population when it is increasing at the rate of $8$ million bacteria/hour?

\end{enumerate}
\end{question}

\begin{question}  \label{Qhhfhnnnhtytt543}
The function
\[
       N = f(t) = \frac{20}{1+5e^{-t/5}} \, , \, 0\leq t\leq 30,
\]
models the spread of a virus throughout a population. It takes as an input the number of months since January 1st and returns as an output the number of infected individuals (measured in millions). Graph shown below.

\begin{onlineOnly}
    \begin{center}
\desmos{ikis451wxu}{900}{600} 
\end{center}
\end{onlineOnly}

\href{https://www.desmos.com/calculator/ikis451wxu}{151: Logistic Model}

\begin{enumerate}
\item Use the sliders in the worksheet above to approximate the number of infected individuals when the virus is spreading at the rate of $800,000$ people/month.

\item Find a function 
\[
       r = g(N) 
\]
that expresses the infection rate (in millions of people/month) in terms of the population.

\item Use calculus and algebra to determine the number of infected individuals when the virus is spreading at the rate of $750,000$ people/month.

\item Use the sliders in the worksheet above to approximate the number of infected individuals when the virus is spreading at its fastest rate. Also approximate this rate.

\item Use calculus and algebra to determine the number of infected individuals when the virus is spreading at its fastest rate. Then find this exact rate.
\end{enumerate}


\end{question}


\begin{question}  \label{Qdghnntnmnmnmn}
Use the chain rule and your knowledge of the derivative $d(e^x)/dx$ to compute each of the following derivatives.

\begin{enumerate}
\item 
\[
   \frac{d}{dt}\left( 100 \cdot 2^t \right)
\]

\item 
\[
   \frac{d}{dt}\left( 100 \cdot 2^{t/5} \right)
\]

\end{enumerate}
\end{question}

\begin{question}  \label{Qfyhyhgrrrreee}
A colony of bacteria has a population of $20$ million at noon and a population of $25$ million at 2pm. The population grows exponentially between noon and midnight.

\begin{enumerate}
\item Describe precisely how the population grows.

\item Use your description from part (a) to find a function 
\[
   P=f(t) \, , \, 0\leq t \leq 12,
\]
that expresses the population (in millions) in terms of the number of hours past noon. Do \emph{not} use $e$.

\item Find the relative instantaneous growth rate of the population.
\end{enumerate}
\end{question}


\begin{question}  \label{Qgfhhfhdffv}
Between ground level and an altitude of $10$ km, atmospheric pressure on the planet Krypton is an exponential function of altitude. The pressure is 100 kPa at ground level and 80 kPa at an altitude of $0.75$ km.

\begin{onlineOnly}
    \begin{center}
\desmos{j5h8kaj8xs}{900}{600}  %    pxsmo04nmg  8swp20zond
\end{center}
\end{onlineOnly}

\href{https://www.desmos.com/calculator/j5h8kaj8xs}{151: Atomspheric Pressure}

\begin{enumerate}
\item Describe exactly how the pressure decreases.

\item Use your description from part (a) to find a function
\[
         P = f(h) \, , \, 0\leq h \leq 10 ,
\]
that expresses the atmospheric pressure (in kPa) in terms of the altitude (in kilometers). Do \emph{not} use the number $e$.

\item Input your expression for $f$ in Line 2 of the worksheet above.

\item Evaluate the derivative 
\[
       \frac{dP}{dh}\Big|_{h=5}
\]
and interpret its meaning.

\item Find a function $r=g(P)$ that expresses the derivative $dP/dh$ in terms of $P$.

\item Use calculus to approximate the altitude at which the pressure is $0.2$ kPa less at an altitude $0.1$ km higher.

\end{enumerate}
  
\end{question}

\end{document}
